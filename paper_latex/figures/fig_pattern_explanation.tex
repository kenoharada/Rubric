\colorlet{pcMatch}{red!75!black}
\begin{figure*}[t]
\centering
\begin{tcolorbox}[
colback=white,
colframe=black!25,
title=Pattern Explanations with Random Matched Snippets,
fonttitle=\bfseries\small,
fontupper=\scriptsize,
boxsep=1pt,left=2pt,right=2pt,top=2pt,bottom=2pt]
\textbf{1. Conditional Gating}\par \textit{What this pattern captures:} Condition-based branching rules (if / when / unless / provided that) that explicitly guide rater decisions under specific circumstances. Refined rubrics tend to add many conditional gates to reduce ambiguity in borderline situations.\par \textit{Typical cues:} if, when, unless, provided that\par \textit{Example 1:} ``... .g., "@PERCENT1 of students," "@PERSON1 says...")-even \textcolor{pcMatch}{\textbf{if}} embedded in awkward phrasing or grammatical errors. - ...'' \par \textit{Example 2:} ``... istance between each grade should be considered equal. \textcolor{pcMatch}{\textbf{When}} scoring, prioritize the quality of critical thinking a ...'' \par\vspace{1.2mm}\par \textbf{2. Boundary / Tie-Break}\par \textit{What this pattern captures:} Rules for resolving borderline cases between adjacent score bands. Includes tie-break procedures, explicit threshold cutoffs, and 'N vs N' comparisons (e.g., '3 vs 4'). Refined rubrics often add detailed boundary-resolution instructions to improve inter-rater agreement.\par \textit{Typical cues:} tie-break, borderline, threshold, N vs N, between adjacent\par \textit{Example 1:} ``... chanics (E) first: B determines maximum possible band ( \textcolor{pcMatch}{\textbf{3 vs 4}} vs 5/6), then apply E penalties. A and C then refine p ...'' \par \textit{Example 2:} ``... eaching 6 if the argument fails to cohere. 6. Holistic \textcolor{pcMatch}{\textbf{tie-break}} ers (final arbitration): - If features point to diff ...'' \par\vspace{1.2mm}\par \textbf{3. Stepwise Workflow}\par \textit{What this pattern captures:} Ordered step-by-step procedures (Step 1, Step 2...) or checklists that structure the scoring process into a reproducible workflow. Optimization tends to transform free-form scoring guidance into structured, sequential procedures for raters to follow.\par \textit{Typical cues:} step N, checklist, workflow, procedure, in order\par \textit{Example 1:} ``... e for tie-breaking and possible downward adjustment in \textcolor{pcMatch}{\textbf{Step 4}} . - If E = Minor: no reduction. Step 3 - Assess Cent ...'' \par \textit{Example 2:} ``... elopment. - Borderline handling (refined with explicit \textcolor{pcMatch}{\textbf{checklist}} and tie-breakers): When between adjacent scores, ask t ...'' \par\vspace{1.2mm}\par \textbf{4. Quantitative Threshold}\par \textit{What this pattern captures:} Numeric cutoffs and quantified criteria (e.g., 'at least 2 facts', '\textasciitilde{}30\% severe errors', '3 reasons') that replace vague qualitative descriptions with concrete numbers. Optimization frequently introduces numeric thresholds where the original rubric used imprecise terms like 'some' or 'several'.\par \textit{Typical cues:} at least, at most, <=, >=, N reasons/examples/sentences, N\%\par \textit{Example 1:} ``... and specific details/examples. Typical threshold: - \textcolor{pcMatch}{\textbf{At least}} 2 distinct reasons each with at least min-elab AND at ...'' \par \textit{Example 2:} ``... n. - Development depth requirement: expect roughly 2- \textcolor{pcMatch}{\textbf{3 sentences}} of elaboration per reason in a typical short essay (re ...'' \par\vspace{1.2mm}\par \textbf{5. Score Cap / Demotion}\par \textit{What this pattern captures:} Hard constraints that cap the maximum achievable score or forcibly demote ratings when specific conditions are unmet. Examples: 'cannot receive 4 or higher', 'do not award 5', 'downgrade to 2'. Refined rubrics add these guards to prevent systematic over-scoring of essays that superficially appear competent.\par \textit{Typical cues:} cannot be Score, must not receive, do not award, downgrade, demotion\par \textit{Example 1:} ``... id-idea and thereby leaves core reasoning undeveloped, \textcolor{pcMatch}{\textbf{downgrade}} one band. - Exceptions: If an essay otherwise meets ...'' \par \textit{Example 2:} ``... Score 4-even with numerous grammatical errors. Do not \textcolor{pcMatch}{\textbf{downgrade}} for language if the argument's logic and evidence are ...'' \par\vspace{1.2mm}\par \textbf{6. Concrete Exemplification}\par \textit{What this pattern captures:} Detects rubric text that uses illustrative examples (e.g., for example, for instance) to clarify scoring criteria. Refined rubrics frequently replace abstract descriptions with example-rich explanations, making this a strong indicator of rubric practicality improvement.\par \textit{Typical cues:} e.g., for example, for instance\par \textit{Example 1:} ``... dominate the text; some attempt at audience awareness ( \textcolor{pcMatch}{\textbf{e.g.}} , letter format) is present but ineffective. - Ideas ar ...'' \par \textit{Example 2:} ``... te (briefly) which specific requirements were missing ( \textcolor{pcMatch}{\textbf{e.g.}} , "only one reason developed; second reason absent," "e ...'' 
\end{tcolorbox}
\caption{Overview of rubric-refinement patterns and representative rubric snippets. For each pattern, we provide a short interpretation and randomly sampled matched spans from refined rubrics; highlighted words indicate the cue expressions that triggered each pattern. }
\label{fig:pattern_explanation}
\end{figure*}
