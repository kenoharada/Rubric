\colorlet{rpBgFallback}{gray!20}
\colorlet{rpBgIfRules}{yellow!35}
\colorlet{rpBgTieBreakerBoundary}{orange!30}
\colorlet{rpBgStepwiseProcess}{green!30}
\colorlet{rpBgQuantitativeThresholds}{cyan!28}
\colorlet{rpBgScoreCapDemotion}{red!24}
\colorlet{rpBgEvidenceCountSafeguard}{blue!22}
\colorlet{rpBgConcreteExemplification}{teal!28}
\providecommand{\rpHl}[2]{\begingroup\setlength{\fboxsep}{0.3pt}\colorbox{#1}{\strut #2}\endgroup}
\providecommand{\rpLegendItem}[2]{\begingroup\setlength{\fboxsep}{1.2pt}\colorbox{#1}{\strut #2}\endgroup}
\begin{figure*}[t]
\centering
\begin{tcolorbox}[colback=white,colframe=black!25,title=Pattern Legend,fonttitle=\bfseries\small,fontupper=\scriptsize,boxsep=1pt,left=2pt,right=2pt,top=2pt,bottom=2pt]
\begin{tabular}{@{}p{0.32\linewidth}p{0.32\linewidth}p{0.32\linewidth}@{}}
\rpLegendItem{rpBgIfRules}{\textbf{Conditional Gating} (n=29)} & \rpLegendItem{rpBgTieBreakerBoundary}{\textbf{Boundary / Tie-Break} (n=6)} & \rpLegendItem{rpBgStepwiseProcess}{\textbf{Stepwise Workflow} (n=3)} \\
\rpLegendItem{rpBgQuantitativeThresholds}{\textbf{Quantitative Threshold} (n=12)} & \rpLegendItem{rpBgScoreCapDemotion}{\textbf{Score Cap / Demotion} (n=1)} & \rpLegendItem{rpBgEvidenceCountSafeguard}{\textbf{Evidence Count Safeguard} (n=6)} \\
\rpLegendItem{rpBgConcreteExemplification}{\textbf{Concrete Exemplification} (n=13)} &  &  \\
\end{tabular}
\end{tcolorbox}
\vspace{1mm}
\begin{tcolorbox}[colback=white,colframe=black!25,title=Refined Rubric (Pattern-Highlighted),fonttitle=\bfseries\small,fontupper=\scriptsize]
\ttfamily
General note about anonymization:\par - Do NOT deduct points for named-entity placeholders (PERSON, LOCATION, NUM, PERCENT, etc.). Treat them as neutral substitutions for real details; evaluate the presence, clarity, and specificity of ideas rather than the literal labels. Count placeholders as specific details \rpHl{rpBgIfRules}{when} the writer clearly intends a concrete fact, example, or \rpHl{rpBgConcreteExemplification}{anecdote}.\par \par Core principles (apply holistically, \rpHl{rpBgStepwiseProcess}{in order} of priority):\par 1. Clarity of position (most important).\par ... [8 lines omitted] ...\par   b. Minimal elaboration (general explanation, vague example) -> "min-elab".\par   c. Specific elaboration/example or clear personal \rpHl{rpBgConcreteExemplification}{anecdote}/data -> "specific".\par - Treat personal anecdotes and clearly-intended placeholders-as-evidence as valid "specific".\par - Do not \rpHl{rpBgEvidenceCountSafeguard}{double-count} repeated restatements of the same reason. Different examples that support the same reason count as strengthening that one reason (do not convert them into separate reasons \rpHl{rpBgIfRules}{unless} they support a genuinely different claim).\par - \rpHl{rpBgIfRules}{When} in doubt about whether two supports are distinct reasons or sub-points of the same reason, prefer to count them as the same reason \rpHl{rpBgIfRules}{unless} they address different effects, audiences, or mechanisms.\par \par ... [1 lines omitted] ...\par - Position: May state a position or may be off-task; little or no purposeful response to the prompt.\par - Development: Few or no reasons; \rpHl{rpBgIfRules}{if} present they are list-only or irrelevant. No meaningful examples or elaboration.\par - Organization \& coherence: Fragmented, chaotic, or extremely hard to follow; may be one or two disjointed sentences.\par - Language: Grammar and usage may prevent comprehension.\par - Use \rpHl{rpBgIfRules}{when} the essay essentially fails to form an argument or provide any supporting content.\par \par ... [4 lines omitted] ...\par - Language: Frequent errors that sometimes impede comprehension.\par - Use \rpHl{rpBgIfRules}{when} the response is more than a sentence or two but lacks development, explanation, and clear structure.\par \par ... [2 lines omitted] ...\par - Development: Provides basic reasons with minimal elaboration. Typical patterns:\par   - 1-2 distinct reasons where \rpHl{rpBgQuantitativeThresholds}{at least} one has min-elab; or\par   - 2+ reasons but most are list-only or repetitive restatements.\par   - May include one brief \rpHl{rpBgConcreteExemplification}{specific example} or \rpHl{rpBgConcreteExemplification}{anecdote}, but it is not well developed or persuasive.\par - Organization \& coherence: Some sense of organization (intro, body, conclusion or paragraphing) though progression may be weak; limited transitions.\par - Language: Errors are frequent but overall meaning is still understandable.\par - Use \rpHl{rpBgIfRules}{when} the essay demonstrates a clear stance and rudimentary structure with limited support and few specific, distinct examples.\par \par ... [1 lines omitted] ...\par - Position: Clearly stated position throughout.\par - Development: Offers adequately elaborated reasons with a mix of general and specific details/examples. Typical \rpHl{rpBgTieBreakerBoundary}{threshold}:\par   - \rpHl{rpBgQuantitativeThresholds}{At least} 2 distinct reasons each with \rpHl{rpBgQuantitativeThresholds}{at least} min-elab AND \rpHl{rpBgQuantitativeThresholds}{at least} one clear \rpHl{rpBgConcreteExemplification}{specific example} supporting any one of the reasons; OR\par   - 2-3 distinct reasons with mostly min-elab development and \rpHl{rpBgQuantitativeThresholds}{at least} one specific that meaningfully strengthens the argument.\par - Organization \& coherence: Satisfactory organization with clear paragraphing and some transitions; readers can follow the argument.\par - Language: Noticeable errors may be present but do not substantially obscure meaning.\par - \rpHl{rpBgTieBreakerBoundary}{Tie-break}er to promote consistency: \rpHl{rpBgIfRules}{If} an essay has 2+ distinct reasons each with min-elab and \rpHl{rpBgQuantitativeThresholds}{at least} one clearly functioning specific (including placeholders or brief anecdotes), prefer 4 over 3-even \rpHl{rpBgIfRules}{when} language is weak.\par \par ... [2 lines omitted] ...\par - Development: Stronger elaboration than 4. Typical patterns (choose the rule that fits):\par   - 3+ distinct reasons with \rpHl{rpBgQuantitativeThresholds}{at least} two being "specific" examples/anecdotes; OR\par   - 2 distinct reasons each with robust, specific elaboration and varied supporting details; OR\par ... [2 lines omitted] ...\par - Language: Generally fluent; errors present but not distracting.\par - Important constraint to reduce over-scoring: \rpHl{rpBgScoreCapDemotion}{Do NOT award} a 5 \rpHl{rpBgIfRules}{if} the "specific" supports are repetitive restatements or the same example reused to pad counts. Multiple \rpHl{rpBgConcreteExemplification}{specific example}s must be distinct in content or context (different data points, different anecdotes, different illustrative scenarios).\par \par ... [3 lines omitted] ...\par   - The essay presents a nuanced/complex stance (acknowledges trade-offs, limits, or a qualified position) AND offers multiple specific, relevant examples and explanation; OR\par   - The essay contains 3+ distinct reasons each supported by clear, separate "specific" examples (not merely repeated restatements), combined with cohesive organization and some synthesis (linking reasons, explaining implications) even \rpHl{rpBgIfRules}{if} explicit counterargument is brief or implicit.\par - Organization \& coherence: Strong, logical organization with clear, effective transitions and paragraphing; the argument builds cohesively.\par ... [1 lines omitted] ...\par - Audience awareness: Heightened-persuasive techniques tailored to the intended audience.\par - Use 6 \rpHl{rpBgIfRules}{when} the essay demonstrates either clear analytic depth (counterargument, trade-offs, synthesis) OR very strong breadth and specificity of development (three distinct, well-supported reasons) plus clear organization.\par \par ... [2 lines omitted] ...\par    - Use the three-tier label (list-only / min-elab / specific) per reason.\par    - 2+ adequately elaborated reasons (min-elab) with \rpHl{rpBgQuantitativeThresholds}{at least} one specific -> lean 4.\par    - 3+ distinct reasons with 2+ \rpHl{rpBgConcreteExemplification}{specific example}s/anecdotes -> lean 5; \rpHl{rpBgIfRules}{if} those 3+ specifics are present and organization is strong, consider 6 (see 6's alternate path).\par    - 2 strongly specific, well-connected reasons with varied evidence or a clear rebuttal -> can justify 5.\par ... [1 lines omitted] ...\par 2. Favor development over surface fluency:\par    - Frequent mechanical errors should lower the fluency descriptor but should not automatically move a piece from 4/5/6 down to 2/3 \rpHl{rpBgIfRules}{if} the essay contains clear organization and multiple \rpHl{rpBgConcreteExemplification}{specific example}s.\par    - However, severe breakdowns in grammar that impede comprehension of key supports should lower the score.\par 3. Treat personal anecdotes and placeholder-based "studies" as valid evidence:\par    - \rpHl{rpBgIfRules}{If} the writer provides a personal story or clearly intended study/example (even with placeholders), count it as a "specific" example for development-\rpHl{rpBgIfRules}{unless} the placeholder is so vague that it does not function as evidence.\par 4. Distinguish \rpHl{rpBgEvidenceCountSafeguard}{repetition} vs. distinct evidence:\par    - \rpHl{rpBgEvidenceCountSafeguard}{Repetition} or \rpHl{rpBgEvidenceCountSafeguard}{rephras}ing of the same example should not be counted as multiple specifics.\par    - Different contexts or different \rpHl{rpBgConcreteExemplification}{concrete example}s (even \rpHl{rpBgIfRules}{if} they support the same general reason) strengthen that reason but \rpHl{rpBgEvidenceCountSafeguard}{do not count} as additional distinct reasons.\par    - To move from 4->5 using the "3+ reasons" route, ensure the third reason is independent (addresses a different effect or mechanism) and has a \rpHl{rpBgConcreteExemplification}{specific example}.\par 5. Use organization to resolve close calls:\par    - Clear intro/body/conclusion and logical paragraphing can raise a \rpHl{rpBgTieBreakerBoundary}{borderline} 3 to a 4 even \rpHl{rpBgIfRules}{when} details are modest.\par    - Conversely, poor organization can keep a richly supported response from reaching 6 \rpHl{rpBgIfRules}{if} the argument fails to cohere.\par 6. Holistic \rpHl{rpBgTieBreakerBoundary}{tie-break}ers (final arbitration):\par    - \rpHl{rpBgIfRules}{If} features point to different scores, prioritize \rpHl{rpBgStepwiseProcess}{in order}: (a) specificity \& number of supporting details (b) clarity of position (c) organization/cohesion.\par    - \rpHl{rpBgIfRules}{When} in doubt between 4 and 5, count \rpHl{rpBgConcreteExemplification}{specific example}s carefully-require distinctiveness and substantive support. \rpHl{rpBgIfRules}{If} you count 2 full- strength specifics (distinct content) and either a third reason or second reason with strong specifics, prefer 5.\par    - \rpHl{rpBgIfRules}{When} in doubt between 5 and 6, require either explicit nuance/counterargument OR 3+ distinct reasons each with clear specifics AND strong organization for 6.\par 7. Examples of boundary judgments (updated heuristics):\par    - Several distinct reasons but only list-like, no examples -> Score 2.\par    - Clear stance, 1-\rpHl{rpBgQuantitativeThresholds}{2 reasons} with brief examples or anecdotes; some organization -> Score 3.\par    - Clear stance, 2+ distinct reasons each with \rpHl{rpBgQuantitativeThresholds}{at least} some elaboration and \rpHl{rpBgQuantitativeThresholds}{at least} one \rpHl{rpBgConcreteExemplification}{specific example} -> Score 4.\par    - Clear stance, either (a) 3+ reasons with mostly specific, relevant examples (distinct and non-repetitive) OR (b) \rpHl{rpBgQuantitativeThresholds}{2 reasons} each with substantial specific elaboration and persuasive flow -> Score 5.\par    - Nuanced stance, thorough analysis, counterargument, or 3+ distinct, well-supported reasons with cohesive synthesis -> Score 6.\par \par Practical scoring \rpHl{rpBgStepwiseProcess}{checklist} (use \rpHl{rpBgIfRules}{when} assigning a score):\par - Is the position clear and consistent? (Yes -> continue; No -> lean 1-2)\par ... [2 lines omitted] ...\par - Count specifics: how many distinct "specific" supports? (Are they different in content/context or repetitive?)\par - Evaluate organization: clear paragraphs and transitions? (Yes raises \rpHl{rpBgTieBreakerBoundary}{borderline} 3->4)\par - Is there counterargument, synthesis, or analytic depth? (Yes -> consider 6)\par - Check language: are errors limiting comprehension? (\rpHl{rpBgIfRules}{If} comprehension fails, lower to 1-2; otherwise, do not heavily penalize development)\par - Apply \rpHl{rpBgTieBreakerBoundary}{tie-break}er rules above (prioritize specificity, then clarity, then organization).\par \par Rationale summary for raters:\par - Increase scores \rpHl{rpBgIfRules}{when} multiple distinct, \rpHl{rpBgConcreteExemplification}{specific example}s or anecdotes are present even \rpHl{rpBgIfRules}{if} the essay is marred by grammatical errors or placeholders.\par - Do not let surface-level \rpHl{rpBgEvidenceCountSafeguard}{repetition} or weak phrasing obscure counting of distinct supports-explicitly count and label supports.\par - Reserve the top score (6) for essays that either add analytical depth (counterargument, synthesis) OR supply broad, distinct, and specific development across 3+ independent reasons with cohesive organization.\par - Be stricter about distinctiveness of specifics \rpHl{rpBgIfRules}{when} moving between 4, 5, and 6-require different content/context for each counted specific support.
\end{tcolorbox}
\caption{Background colors mark text spans in the refined rubric that match each pattern. The legend above shows the pattern-to-color mapping and match counts. When multiple patterns overlap on the same span, only one highlight is retained to keep the visualization readable. The rubric panel shows excerpts around matched lines (±1 line context); omitted stretches are marked explicitly. (asap\_1, openai\_gpt-5-mini, base\_expert\_True\_train100\_iteration5\_top3\_bs4-8-12\_mc4).}
\label{fig:rubric_pattern_bg_asap_1_openai_gpt_5_mini_base_expert_True_train100_iteration5_top3_bs4_8_12_mc4}
\end{figure*}
