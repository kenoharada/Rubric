\colorlet{rpBgFallback}{gray!20}
\colorlet{rpBgIfRules}{yellow!35}
\colorlet{rpBgTieBreakerBoundary}{orange!30}
\colorlet{rpBgStepwiseProcess}{green!30}
\colorlet{rpBgQuantitativeThresholds}{cyan!28}
\colorlet{rpBgScoreCapDemotion}{red!24}
\colorlet{rpBgEvidenceCountSafeguard}{blue!22}
\colorlet{rpBgConcreteExemplification}{teal!28}
\providecommand{\rpHl}[2]{\begingroup\setlength{\fboxsep}{0.3pt}\colorbox{#1}{\strut #2}\endgroup}
\providecommand{\rpLegendItem}[2]{\begingroup\setlength{\fboxsep}{1.2pt}\colorbox{#1}{\strut #2}\endgroup}
\begin{figure*}[t]
\centering
\begin{tcolorbox}[colback=white,colframe=black!25,title=Pattern Legend,fonttitle=\bfseries\small,fontupper=\scriptsize,boxsep=1pt,left=2pt,right=2pt,top=2pt,bottom=2pt]
\begin{tabular}{@{}p{0.32\linewidth}p{0.32\linewidth}p{0.32\linewidth}@{}}
\rpLegendItem{rpBgIfRules}{\textbf{Conditional Gating} (n=9)} & \rpLegendItem{rpBgTieBreakerBoundary}{\textbf{Boundary / Tie-Break} (n=2)} & \rpLegendItem{rpBgStepwiseProcess}{\textbf{Stepwise Workflow} (n=1)} \\
\rpLegendItem{rpBgQuantitativeThresholds}{\textbf{Quantitative Threshold} (n=2)} & \rpLegendItem{rpBgEvidenceCountSafeguard}{\textbf{Evidence Count Safeguard} (n=1)} & \rpLegendItem{rpBgConcreteExemplification}{\textbf{Concrete Exemplification} (n=15)} \\
\end{tabular}
\end{tcolorbox}
\vspace{1mm}
\begin{tcolorbox}[colback=white,colframe=black!25,title=Refined Rubric (Pattern-Highlighted),fonttitle=\bfseries\small,fontupper=\scriptsize]
\ttfamily
Score Point 1: An undeveloped response that may take a position but offers no more than very minimal support. Typical elements:\par - Contains only one or two extremely vague sentences.\par - Is so fragmented that it is nearly impossible to discern a stance.\par - Shows no awareness of the specific prompt or audience.\par Note: \rpHl{rpBgIfRules}{If} a response provides any discernible reasons or specific details, even with severe mechanical errors, move to \rpHl{rpBgQuantitativeThresholds}{at least} a Score Point 2.\par \par Score Point 2: An under-developed response that takes a position but provides little support. Typical elements:\par - Contains only one general reason or a very short, unelaborated list of ideas.\par - Lists items (\rpHl{rpBgConcreteExemplification}{e.g.}, "play games, go on Facebook") without any explanation of why they are good or how they work.\par - Shows little or no evidence of structured organization.\par - Highly simplistic or redundant language.\par - Limited attempt to address the reader or the specific prompt requirements.\par \par Score Point 3: A minimally-developed response that takes a position and provides a rudimentary level of support. Typical elements:\par - Provides a list of reasons with very brief elaboration (usually only one sentence per point).\par - Offers more general than specific details. Even \rpHl{rpBgIfRules}{if} proper nouns or specific platforms (\rpHl{rpBgConcreteExemplification}{e.g.}, "Facebook," "Colombia") are mentioned, \rpHl{rpBgIfRules}{if} the sentence following them provides no further development of the idea, the response remains a 3.\par - Basic "listing" organization (\rpHl{rpBgConcreteExemplification}{e.g.}, "First, Second, Finally") with little internal development.\par - Characterized by highly simplistic sentence structures and vocabulary.\par - Contains frequent errors that may occasionally impede meaning.\par \par Score Point 4: A somewhat-developed response that takes a position and provides adequate support. Typical elements:\par - Addresses several reasons with a mix of general and specific details.\par - Support often feels formulaic or relies heavily on the ideas/phrasing provided in the prompt (\rpHl{rpBgConcreteExemplification}{e.g.}, simply expanding on "exercise" and "nature" without unique hypothetical scenarios).\par - Contains anecdotes or examples that are present but lack multi-layered development or "mechanics" (\rpHl{rpBgConcreteExemplification}{e.g.}, mentioning a surgery simulation or a personal fall but not exploring the broader implications).\par - Shows satisfactory organization with a clear intro, body, and conclusion.\par - May contain significant mechanical, spelling, or syntax errors; however, the argument is logically coherent.\par - Language is functional but lacks fluency. Even \rpHl{rpBgIfRules}{if} the essay introduces an original third point (like safety or jobs), \rpHl{rpBgIfRules}{if} the prose is clumsy and the elaboration is limited to 2-\rpHl{rpBgQuantitativeThresholds}{3 sentences} per point, it should remain a 4.\par \par Score Point 5: A developed response that takes a clear position and provides moderately persuasive support. Typical elements:\par - Has well-elaborated reasons with specific, original details that explore "mechanics" (\rpHl{rpBgConcreteExemplification}{e.g.}, describing the physical sensation of "getting fat," or the specific way a webcam creates "human interaction" compared to a phone).\par - Moves beyond the prompt's suggestions by significantly expanding on how technology functions in personal or professional lives (\rpHl{rpBgConcreteExemplification}{e.g.}, how a nurse uses videos for \rpHl{rpBgStepwiseProcess}{procedure}s or the emotional impact of staying in touch with a friend who moved).\par - Demonstrates a consistent persuasive tone and a clear attempt to engage the reader personally.\par - While it may use a formulaic structure (\rpHl{rpBgConcreteExemplification}{e.g.}, "My first reason... My last reason..."), the depth of the elaboration and the use of original scenarios justify the 5. Depth of content overrides repetitive transitions at this level.\par - Addresses the counter-argument or looks at the issue from multiple perspectives.\par \par Score Point 6: A well-developed response that takes a clear and thoughtful position and provides persuasive, in-depth support. Typical elements:\par - Fully elaborated reasons with numerous specific, \rpHl{rpBgConcreteExemplification}{concrete detail}s or "multi-layered" anecdotes.\par - Goes significantly beyond explaining "why" to explore the "implications" and "consequences" of the position (\rpHl{rpBgConcreteExemplification}{e.g.}, connecting computer use to global warming via power plants, or the specific danger of losing survival skills like lighting matches during a natural disaster).\par - Exhibits sophisticated organization or creative framing (\rpHl{rpBgConcreteExemplification}{e.g.}, using a powerful opening quote or a rhetorical "hook").\par - Shows a heightened awareness of audience through a compelling persuasive voice or a strong emotional hook that connects with the reader's daily life.\par - Fluency is high; while minor mechanical errors may exist, the rhetorical variety, complexity of thought, and "voice" are strong enough to carry authority.\par \par Note on Scoring: \par - Evaluators must prioritize the depth and specificity of content over technical accuracy.\par - "In-depth development" (Score 5 and 6) is defined by the student's ability to visualize a scenario for the reader or explain a cause-and-effect chain.\par - A \rpHl{rpBgTieBreakerBoundary}{4 vs. 5} Distinction: \rpHl{rpBgIfRules}{If} an essay provides \rpHl{rpBgConcreteExemplification}{specific example}s (like surgery simulations or medical stats) but fails to explain the *ripple effects* or *human impact* of those examples, it is a 4. \rpHl{rpBgIfRules}{If} it explores the "how" and "why" behind the examples (\rpHl{rpBgConcreteExemplification}{e.g.}, the emotional relief of a webcam or the specific process of getting fit), it is a 5.\par - A \rpHl{rpBgTieBreakerBoundary}{5 vs. 6} Distinction: A 6 must explore broader societal or philosophical "implications" (\rpHl{rpBgConcreteExemplification}{e.g.}, global warming, natural disaster survival, the future of the environment) or use highly creative framing and sophisticated voice.\par - Heavy \rpHl{rpBgEvidenceCountSafeguard}{repetition} in sentence structure (\rpHl{rpBgConcreteExemplification}{e.g.}, "One reason is... Another reason is...") caps an essay at 4 ONLY \rpHl{rpBgIfRules}{IF} the content is also basic/formulaic. \rpHl{rpBgIfRules}{If} the content within those paragraphs is deep and explores implications, it should be moved to a 5.\par - Proper nouns or platform names do not automatically grant a higher score; they must be accompanied by an explanation of *how* or *why* that specific item supports the argument.\par - Anonymization markers (like @CAPS, @LOCATION) should be treated as the intended words/details.
\end{tcolorbox}
\caption{Refined rubric of Gemini 3 Flash on ASAP. Background colors mark text spans in the refined rubric that match each pattern. The legend above shows the pattern-to-color mapping and match counts.}
\label{fig:rubric_pattern_bg_asap_1_google_gemini_3_flash_preview_base_expert_True_train100_iteration5_top3_bs4_8_12_mc4}
\end{figure*}
