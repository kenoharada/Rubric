\colorlet{rpBgFallback}{gray!20}
\colorlet{rpBgIfRules}{yellow!35}
\colorlet{rpBgTieBreakerBoundary}{orange!30}
\colorlet{rpBgStepwiseProcess}{green!30}
\colorlet{rpBgQuantitativeThresholds}{cyan!28}
\colorlet{rpBgScoreCapDemotion}{red!24}
\colorlet{rpBgEvidenceCountSafeguard}{blue!22}
\colorlet{rpBgConcreteExemplification}{teal!28}
\providecommand{\rpHl}[2]{\begingroup\setlength{\fboxsep}{0.3pt}\colorbox{#1}{\strut #2}\endgroup}
\begin{figure*}[t]
\centering
\begin{tcolorbox}[colback=white,colframe=black!25,title=Matched Pattern Legend,fonttitle=\bfseries\small,fontupper=\scriptsize,boxsep=1pt,left=2pt,right=2pt,top=2pt,bottom=2pt]
\textcolor{black}{\fcolorbox{black!20}{rpBgIfRules}{\rule{0.9em}{0.7em}}} \textbf{Conditional Gating} (Rule Structure; n=5)\par \textcolor{black}{\fcolorbox{black!20}{rpBgQuantitativeThresholds}{\rule{0.9em}{0.7em}}} \textbf{Quantitative Threshold} (Rule Structure; n=1)\par \textcolor{black}{\fcolorbox{black!20}{rpBgConcreteExemplification}{\rule{0.9em}{0.7em}}} \textbf{Concrete Exemplification} (Evidence Handling; n=4)
\end{tcolorbox}
\vspace{1mm}
\begin{tcolorbox}[colback=white,colframe=black!25,title=Refined Rubric (Pattern-Highlighted),fonttitle=\bfseries\small,fontupper=\scriptsize]
\ttfamily
After reading each essay and completing the analytical rating form, assign a holistic score based on the rubric below. For the following evaluations you will need to use a grading scale between 1 (minimum) and 6 (maximum). The distance between each grade should be considered equal. \rpHl{rpBgIfRules}{When} scoring, prioritize the quality of critical thinking and the student's ability to use evidence over surface-level mechanical errors, \rpHl{rpBgIfRules}{unless} those errors obscure meaning.\par \par SCORE OF 6: Outstanding Mastery. An essay in this category demonstrates clear and consistent mastery. It effectively and insightfully develops a point of view, showing sophisticated critical thinking (\rpHl{rpBgConcreteExemplification}{e.g.}, analyzing tone, diction, or complex contradictions). It uses clearly appropriate examples and evidence from the source text to support its position; it is well-organized, focused, and demonstrates clear coherence and smooth progression of ideas. It exhibits skillful use of language, varied vocabulary, and meaningful variety in sentence structure.\par \par SCORE OF 5: Strong Mastery. An essay in this category demonstrates reasonably consistent mastery, though it will have occasional errors or lapses in quality. The essay effectively develops a point of view and demonstrates strong critical thinking; it generally uses appropriate evidence from the source text; it is well-organized and focused, demonstrating coherence and progression; it exhibits facility in language, using appropriate vocabulary and varied sentence structure.\par \par SCORE OF 4: Adequate Mastery. An essay in this category demonstrates adequate mastery. The essay develops a clear point of view and demonstrates competent critical thinking by connecting the text to a broader argument, synthesizing several parts of the text to support a theme, or by analyzing the author's methods (\rpHl{rpBgConcreteExemplification}{e.g.}, how the author uses facts or examples to persuade). \rpHl{rpBgIfRules}{If} the essay moves beyond a chronological retelling to a thematic organization-even \rpHl{rpBgIfRules}{if} it contains frequent errors in grammar and mechanics-it should receive a 4. The student must demonstrate an understanding of the text's construction rather than just its content.\par \par SCORE OF 3: Developing Mastery. An essay in this category demonstrates developing mastery and is marked by one or more of the following: it develops a point of view but does so inconsistently; it shows some attempt at analysis but is limited in depth or relies more on paraphrasing than critical evaluation. A 3 may identify the author's argument and provide evidence but fails to explain *how* or *why* the evidence supports the argument in a meaningful way. It may follow the text's chronology too closely (\rpHl{rpBgConcreteExemplification}{e.g.}, "In paragraph 1... in paragraph 2...") but must offer \rpHl{rpBgQuantitativeThresholds}{at least} some original interpretation or evaluation of the ideas to remain in this category.\par \par SCORE OF 2: Little Mastery. An essay in this category demonstrates little mastery and is flawed by one or more of the following: it develops a vague, simplistic, or seriously limited point of view; it relies almost exclusively on listing facts or providing a sequential summary of the text. Common traits of a 2 include substituting conversational fillers, repetitive rhetorical questions (\rpHl{rpBgConcreteExemplification}{e.g.}, "Don't you want to find out?"), or a simple "I agree" for an actual argument. Even \rpHl{rpBgIfRules}{if} the writing is relatively clear and the structure is logical, an essay that is essentially a summary with a brief personal opinion tacked on belongs in this category.\par \par SCORE OF 1: Very Little or No Mastery. An essay in this category is severely flawed. It develops no viable point of view or provides no original evaluation. Essays that consist primarily of a list of facts, a sequential summary of the text's contents, or rely almost entirely on paraphrased/copied text without an original argumentative framework must be scored a 1. This includes essays that are organized and clear but function only as a report of what the text said. Pervasive errors that persistently interfere with meaning also belong in this category.
\end{tcolorbox}
\caption{Pattern-highlighted view of the refined rubric. Colored background spans indicate phrases matched by pattern rules; the legend reports pattern names and match counts. (ASAP2, google\_gemini-3-flash-preview, base\_expert\_True\_train100\_iteration5\_top3\_bs4-8-12\_mc4).}
\label{fig:rubric_pattern_bg_ASAP2_google_gemini_3_flash_preview_base_expert_True_train100_iteration5_top3_bs4_8_12_mc4}
\end{figure*}
