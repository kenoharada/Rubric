\colorlet{rpTypeRule}{red!80!black}
\colorlet{rpTypeEvidence}{blue!80!black}
\colorlet{rpTypeWriting}{teal!80!black}
\begin{figure*}[t]
\centering
\begin{tcolorbox}[colback=white,colframe=black!25,title=Pattern Type Guide,fonttitle=\bfseries\small,fontupper=\scriptsize,boxsep=1pt,left=2pt,right=2pt,top=2pt,bottom=2pt]
\textcolor{rpTypeRule}{\textbf{Rule Structure}}: Explicit decision logic for scoring: conditional branches, boundary tie-breakers, stepwise workflows, and numeric thresholds.\par \textcolor{rpTypeEvidence}{\textbf{Evidence Handling}}: How evidence is validated and counted: specific-example requirements, repetition/non-double-count rules, and cap rules for weak evidence.\par \textcolor{rpTypeWriting}{\textbf{Writing Quality}}: Language-quality criteria affecting score bands: organization/coherence/transition quality and grammar/mechanics severity.
\end{tcolorbox}
\vspace{1mm}
\begin{tcolorbox}[colback=white,colframe=black!25,title=Detailed Pattern Notes,fonttitle=\bfseries\small,fontupper=\scriptsize,boxsep=1pt,left=2pt,right=2pt,top=2pt,bottom=2pt]
\textcolor{rpTypeRule}{\textbf{Rule Structure}}:\par \quad \textcolor{rpTypeRule}{\textbf{Conditional Gating}} [n=44] Captures explicit condition-based rules that switch decisions only when a stated condition is met. Typical cues: if, when.\par \quad \textcolor{rpTypeRule}{\textbf{Boundary / Tie-Break Guidance}} [n=13] Marks criteria used to resolve borderline cases between adjacent score bands (e.g., 4 vs 5). Typical cues: tie-break, borderline, boundary, threshold, 4 vs 5.\par \quad \textcolor{rpTypeRule}{\textbf{Stepwise Rating Workflow}} [n=4] Detects ordered procedures and checklists that standardize how raters walk through scoring decisions. Typical cues: step, checklist, workflow, procedure, first/second/third.\par \quad \textcolor{rpTypeRule}{\textbf{Quantitative Threshold}} [n=29] Marks numeric cutoffs used for consistent decisions (minimum/maximum counts, percentages, explicit count thresholds). Typical cues: at least, at most, <=, >=, \%, N reasons/examples/sentences/words.\par \textcolor{rpTypeEvidence}{\textbf{Evidence Handling}}:\par \quad \textcolor{rpTypeEvidence}{\textbf{Specific Evidence Requirement}} [n=17] Highlights demands for concrete examples and explicit evidence links instead of generic assertions. Typical cues: for example, e.g., specific example, illustration, anecdote, evidence.\par \quad \textcolor{rpTypeEvidence}{\textbf{Off-Topic / Summary Cap}} [n=1] Identifies cap rules that restrict scores when responses are off-topic, irrelevant, or dominated by summary-only content. Typical cues: off-topic, irrelevant, digression, summary-only, cap.\par \quad \textcolor{rpTypeEvidence}{\textbf{Repetition Non-Count Rule}} [n=10] Captures rules that treat repetition/restatement as non-distinct support and prevent double-counting. Typical cues: repetition, restatement, double-count, do not double-count.\par \textcolor{rpTypeWriting}{\textbf{Writing Quality}}:\par \quad \textcolor{rpTypeWriting}{\textbf{Organization / Coherence Signal}} [n=14] Detects explicit references to discourse structure and logical flow as scoring criteria. Typical cues: organization, coherence, logical flow, transition.\par \quad \textcolor{rpTypeWriting}{\textbf{Grammar / Mechanics Signal}} [n=3] Detects references to language-form quality, especially grammar, spelling, punctuation, and mechanics. Typical cues: grammar, mechanics, spelling, punctuation.
\end{tcolorbox}
\vspace{1mm}
\begin{tcolorbox}[colback=white,colframe=black!25,title=Optimized Rubric (Pattern-Highlighted),fonttitle=\bfseries\small,fontupper=\scriptsize]
\ttfamily
Revised Scoring Guideline (Version 5 - tightened \textcolor{rpTypeRule}{\textbf{threshold}}s and quantitative guidance to reduce rater disagreement)\par ... [3 lines omitted] ...\par - Use the concrete indicators below (development, \textcolor{rpTypeWriting}{\textbf{organization}}/\textcolor{rpTypeWriting}{\textbf{coherence}}, language accuracy and clarity) to place the essay into one of three score bands.\par - Priority rule \textcolor{rpTypeRule}{\textbf{when}} indicators conflict: prioritize intelligibility and development. \textcolor{rpTypeRule}{\textbf{If}} meaning is frequently or repeatedly obscured, lower the score even \textcolor{rpTypeRule}{\textbf{if}} some \textcolor{rpTypeWriting}{\textbf{organization}} or vocabulary are present.\par ... [1 lines omitted] ...\par - Count both quantity and quality of supporting detail. Long essays with \textcolor{rpTypeEvidence}{\textbf{repetition}}, vagueness, or circular \textcolor{rpTypeEvidence}{\textbf{restatement}} of the same point are not sufficient for higher scores.\par ... [3 lines omitted] ...\par Score 3 - "Accomplished response" (use this only \textcolor{rpTypeRule}{\textbf{when}} the essay clearly meets most or all items below)\par ... [2 lines omitted] ...\par   - Provides \textcolor{rpTypeRule}{\textbf{at least}} two distinct supporting points/reasons. Distinct = different aspects or lines of reasoning (\textcolor{rpTypeEvidence}{\textbf{e.g.}}, cause vs. effect, two separate causes, two separate outcomes). \textcolor{rpTypeEvidence}{\textbf{Repetition}} or rewording of the same reason does not count as a \textcolor{rpTypeRule}{\textbf{second}} reason.\par   - For each supporting point, provides specific, relevant elaboration: at minimum 1) an explicit reason sentence, and 2) \textcolor{rpTypeRule}{\textbf{at least}} one additional sentence that explains relevance, consequence, or provides a concrete example tied to the reason.\par   - Example-linking requirement (strict): each example or \textcolor{rpTypeEvidence}{\textbf{anecdote}} must include an explicit, readable link back to the reason or claim. Acceptable links include short phrases such as "this shows that," "because," "therefore," "so," or a brief clause that ties the example to the reason. \textcolor{rpTypeRule}{\textbf{If}} the example is anecdotal or personal, it must include \textcolor{rpTypeRule}{\textbf{at least}} one concrete detail (time, place, specific action or outcome) and an explicit tie-back explaining how it supports the reason.\par   - Development depth requirement: expect roughly 2-\textcolor{rpTypeRule}{\textbf{3 sentences}} of elaboration per reason in a typical short essay (reason + explanation/link ± brief example). Single-sentence reasons with no explicit tie-back do not qualify for Score 3.\par - \textcolor{rpTypeWriting}{\textbf{Organization}} and \textcolor{rpTypeWriting}{\textbf{coherence}}\par   - Logical progression of ideas with paragraphing and \textcolor{rpTypeWriting}{\textbf{transition}}s; relationships among ideas are easy to trace.\par   - Has an adequately coherent introduction and conclusion. A brief or truncated conclusion is acceptable only \textcolor{rpTypeRule}{\textbf{if}} all required development (two distinct reasons, explicit tie-back, adequate elaboration) is already present in the body.\par ... [3 lines omitted] ...\par - Intelligibility \textcolor{rpTypeRule}{\textbf{threshold}} (quantified)\par ... [1 lines omitted] ...\par   - Quantitative guidance: in a typical short essay (approx200-\textcolor{rpTypeRule}{\textbf{300 words}}), the reader should need to re-read or infer meaning for no more than \textcolor{rpTypeRule}{\textbf{3 sentences}}, and these should constitute no more than \textasciitilde{}10\textcolor{rpTypeRule}{\textbf{\%}} of total sentences. \textcolor{rpTypeRule}{\textbf{If}} inference is required on more than \textcolor{rpTypeRule}{\textbf{3 sentences}} or >10\textcolor{rpTypeRule}{\textbf{\%}} of sentences, the essay cannot be Score 3.\par ... [1 lines omitted] ...\par Score 2 - "Limited to partial response" (use \textcolor{rpTypeRule}{\textbf{when}} development and/or language prevent full accomplishment)\par ... [4 lines omitted] ...\par   - Personal \textcolor{rpTypeEvidence}{\textbf{anecdote}}s that lack concrete detail (\textcolor{rpTypeRule}{\textbf{when}}, where, what specifically happened) and/or lack explicit linkage to the claim should be treated as weak \textcolor{rpTypeEvidence}{\textbf{evidence}} and typically count only as partial support.\par   - Clarified \textcolor{rpTypeRule}{\textbf{threshold}}: To assign Score 2, the essay should show at minimum one clearly articulated reason with \textcolor{rpTypeRule}{\textbf{at least}} one sentence of elaboration beyond the reason statement (explanation, consequence, or a linked example). \textcolor{rpTypeRule}{\textbf{If}} the essay presents only assertions or general statements with no elaboration beyond repeating the claim, treat as Score 1.\par - \textcolor{rpTypeWriting}{\textbf{Organization}} and \textcolor{rpTypeWriting}{\textbf{coherence}}\par   - Basic \textcolor{rpTypeWriting}{\textbf{organization}} exists, but connections among ideas may be occasionally unclear or abrupt; paragraphs/\textcolor{rpTypeWriting}{\textbf{transition}}s may be weak.\par ... [4 lines omitted] ...\par - Intelligibility \textcolor{rpTypeRule}{\textbf{threshold}}\par ... [1 lines omitted] ...\par   - Quantified guidance: inference/re-reading needed on more than \textcolor{rpTypeRule}{\textbf{3 sentences}} but not pervasive across the essay, or inference is needed occasionally but does not regularly interrupt comprehension.\par ... [1 lines omitted] ...\par Score 1 - "Weak or minimal response" (use \textcolor{rpTypeRule}{\textbf{when}} the response is minimal, disorganized, or frequently unintelligible)\par ... [2 lines omitted] ...\par   - Very limited development: few or no relevant reasons, explanations, or examples; ideas are superficial, largely undeveloped, repetitive, or \textcolor{rpTypeEvidence}{\textbf{off-topic}}.\par ... [1 lines omitted] ...\par   - Clarified \textcolor{rpTypeRule}{\textbf{threshold}}: \textcolor{rpTypeRule}{\textbf{If}} the essay contains only assertions, a stated opinion without any discernible supporting reason beyond \textcolor{rpTypeEvidence}{\textbf{repetition}}, or only one undeveloped reason with no elaboration, assign Score 1.\par - \textcolor{rpTypeWriting}{\textbf{Organization}} and \textcolor{rpTypeWriting}{\textbf{coherence}}\par   - \textcolor{rpTypeWriting}{\textbf{Organization}} is inadequate or absent. Ideas may be disjointed, digressive, or lack cohesion.\par   - \textcolor{rpTypeWriting}{\textbf{Transition}}s are missing and the reader cannot follow a clear line of reasoning without substantial effort.\par ... [1 lines omitted] ...\par   - Frequent and/or serious errors in sentence structure, \textcolor{rpTypeWriting}{\textbf{grammar}}, word form, or word choice that often obscure meaning.\par   - Recurring error patterns that regularly make propositions unclear warrant Score 1 even \textcolor{rpTypeRule}{\textbf{if}} a position is stated.\par - Intelligibility \textcolor{rpTypeRule}{\textbf{threshold}}\par ... [1 lines omitted] ...\par   - Quantitative guidance: inference or re-reading is required on a substantial portion of sentences (\textcolor{rpTypeEvidence}{\textbf{e.g.}}, >25\textcolor{rpTypeRule}{\textbf{\%}} of sentences or more than \textasciitilde{}\textcolor{rpTypeRule}{\textbf{8 sentences}} in a typical short essay), or multiple sentences are unintelligible.\par ... [2 lines omitted] ...\par - Specificity and example-quality rule (strengthened): For Score 3, require \textcolor{rpTypeRule}{\textbf{at least}} two distinct supporting points, and for each point require:\par ... [1 lines omitted] ...\par     2) \textcolor{rpTypeRule}{\textbf{at least}} one explicit linking sentence or clause tying any example to the reason, and\par     3) either a concrete example with specific detail or a clear explanation of consequence/relevance. Vague \textcolor{rpTypeEvidence}{\textbf{anecdote}}s or generic statements without specifics count as weak \textcolor{rpTypeEvidence}{\textbf{evidence}}.\par - Example-linking rule (clarified \& enforced): Award development credit for an example only \textcolor{rpTypeRule}{\textbf{when}} the writer explicitly connects the example to the reason/claim. \textcolor{rpTypeRule}{\textbf{If}} a reason is supported only by an implicit \textcolor{rpTypeEvidence}{\textbf{anecdote}} or generic example with no explicit tie-back, treat it as partial support (Score 2).\par - Density-of-\textcolor{rpTypeEvidence}{\textbf{evidence}} rule (strengthened): \textcolor{rpTypeRule}{\textbf{If}} an essay is long but provides few real supporting details (lots of \textcolor{rpTypeEvidence}{\textbf{repetition}}, vague generalities, or circular \textcolor{rpTypeEvidence}{\textbf{restatement}}), prefer Score 1 or 2. \textcolor{rpTypeEvidence}{\textbf{Repetition}} of the same reason in multiple paragraphs does not constitute multiplicity of reasons.\par ... [2 lines omitted] ...\par   - \textcolor{rpTypeRule}{\textbf{If}} errors sometimes obscure meaning (reader must infer), the essay cannot be Score 3; assign Score 2 or Score 1 depending on frequency and severity.\par   - To reduce subjective disagreement about "sometimes" vs "occasional," use the quantitative guidance above (Score 3: inference needed on <=\textcolor{rpTypeRule}{\textbf{3 sentences}} and <=10\textcolor{rpTypeRule}{\textbf{\%}} of sentences; Score 2: inference needed on >3 but not pervasive; Score 1: inference pervasive).\par   - \textcolor{rpTypeRule}{\textbf{If}} there is a recurring error pattern that regularly interferes with comprehension (many sentences where \textcolor{rpTypeWriting}{\textbf{grammar}} or word choice make meaning unclear), lower to Score 2 or Score 1. Frequent re-reading required = Score 1.\par   - Frequent surface errors that do not obscure core claims but substantially reduce clarity and fluency (many mis\textcolor{rpTypeWriting}{\textbf{spelling}}s, wrong word choices, sentence fragments) are grounds for Score 2 rather than Score 3. \textcolor{rpTypeRule}{\textbf{If}} such errors are pervasive across the essay and make main points shaky or effortful to recover, use Score 1.\par ... [1 lines omitted] ...\par   - \textcolor{rpTypeRule}{\textbf{If}} an essay terminates abruptly or ends mid-idea and thereby leaves core reasoning undeveloped, downgrade one band.\par   - Exceptions: \textcolor{rpTypeRule}{\textbf{If}} an essay otherwise meets Score 3 except for a truncated concluding sentence but both reasons are fully developed with explicit links in the body, maintain Score 3.\par   - \textcolor{rpTypeRule}{\textbf{If}} truncation leaves one required elaboration missing (\textcolor{rpTypeEvidence}{\textbf{e.g.}}, one reason only partially developed or an example not tied back), downgrade one band (usually Score 3 -> Score 2; Score 2 -> Score 1) rather than automatically to the lowest band.\par   - \textcolor{rpTypeRule}{\textbf{If}} truncation plus language errors make recovery of the missing elaboration difficult, consider further downgrade to the lowest band appropriate (Score 1).\par - \textcolor{rpTypeEvidence}{\textbf{Repetition}} vs. multiplicity rule (re-emphasized): Repeating the same reason with slightly different wording is not equivalent to providing multiple distinct reasons. Score 3 requires distinct reasons; repetitive \textcolor{rpTypeEvidence}{\textbf{restatement}} counts as one reason and should limit the score.\par - Personal \textcolor{rpTypeEvidence}{\textbf{anecdote}} handling: Personal experience can count as \textcolor{rpTypeEvidence}{\textbf{evidence}} but is often weaker than a concrete, \textcolor{rpTypeEvidence}{\textbf{specific example}} that is linked analytically to the claim. \textcolor{rpTypeRule}{\textbf{When}} a personal \textcolor{rpTypeEvidence}{\textbf{anecdote}} lacks detail or an explicit analytic tie-back, treat it as partial support (Score 2 or 1 depending on other factors).\par - Minimum-development \textcolor{rpTypeRule}{\textbf{threshold}}s (practical guidance for raters):\par   - To award Score 3: require \textcolor{rpTypeRule}{\textbf{at least}} two distinct reasons, each with explicit relevant elaboration and an explicit tie-back sentence/clause. Typical short essays should show \textasciitilde{}2-\textcolor{rpTypeRule}{\textbf{3 sentences}} per reason (reason + explanation/link + optional brief example). \textcolor{rpTypeRule}{\textbf{If}} either reason lacks this minimal elaboration and explicit tie-back, prefer Score 2.\par   - To award Score 2: the essay should show a clear position and \textcolor{rpTypeRule}{\textbf{at least}} one discernible supporting reason with some attempt at elaboration (one full sentence beyond the reason), or two weak reasons with thin support. \textcolor{rpTypeRule}{\textbf{If}} there is only assertion without elaboration or only one undeveloped reason, assign Score 1.\par ... [1 lines omitted] ...\par - \textcolor{rpTypeRule}{\textbf{Borderline}} handling (refined with explicit \textcolor{rpTypeRule}{\textbf{checklist}} and \textcolor{rpTypeRule}{\textbf{tie-break}}ers): \textcolor{rpTypeRule}{\textbf{When}} between adjacent scores, ask these questions in order:\par   1) Are there \textcolor{rpTypeRule}{\textbf{at least}} two distinct reasons? \textcolor{rpTypeRule}{\textbf{If}} not -> cannot be Score 3.\par   2) For each reason present, is there an explicit tie-back linking any example/explanation to that reason? \textcolor{rpTypeRule}{\textbf{If}} a reason lacks an explicit tie-back -> cannot be Score 3.\par   3) For each reason, is there \textcolor{rpTypeRule}{\textbf{at least}} one sentence of development beyond the reason statement (explanation, consequence, or concrete detail)? \textcolor{rpTypeRule}{\textbf{If}} not for one or more reasons -> cannot be Score 3.\par   4) Does error frequency/severity force the reader to infer or re-read often? Use quantitative \textcolor{rpTypeRule}{\textbf{threshold}}s: \textcolor{rpTypeRule}{\textbf{if}} inference is needed on <=\textcolor{rpTypeRule}{\textbf{3 sentences}} and <=10\textcolor{rpTypeRule}{\textbf{\%}} of sentences -> may still be Score 3; \textcolor{rpTypeRule}{\textbf{if}} inference is needed on >\textcolor{rpTypeRule}{\textbf{3 sentences}} but not pervasive -> prefer Score 2; \textcolor{rpTypeRule}{\textbf{if}} inference is pervasive -> Score 1.\par   5) Is the essay truncated or abruptly ended such that required elaboration is missing? \textcolor{rpTypeRule}{\textbf{If}} yes -> downgrade one band (usually to the adjacent lower band).\par   Use these ordered checks as decisive \textcolor{rpTypeRule}{\textbf{tie-break}}ers rather than holistic instinct \textcolor{rpTypeRule}{\textbf{when}} uncertain.\par ... [1 lines omitted] ...\par   - Do not award Score 3 solely because an essay contains two named examples; each example must be explicitly connected to a distinct reason and show relevant elaboration. \textcolor{rpTypeRule}{\textbf{If}} examples are undeveloped or linkage is implicit, prefer Score 2.\par ... [1 lines omitted] ...\par   - Penalize essays that appear long but are largely \textcolor{rpTypeEvidence}{\textbf{repetition}} or vague generalities; length is not \textcolor{rpTypeEvidence}{\textbf{evidence}} of sufficient development.\par   - \textcolor{rpTypeRule}{\textbf{If}} many surface errors exist but content is broadly understandable, Score 2 is expected unless development clearly meets the Score 3 \textcolor{rpTypeRule}{\textbf{threshold}}s. \textcolor{rpTypeRule}{\textbf{If}} errors make recovery effortful or the reasoning shaky, use Score 1.\par   - Personal experience statements like "I have realized" or "from my experience" without specifics and an explicit tie-back are weak \textcolor{rpTypeEvidence}{\textbf{evidence}}-treat them as partial support.\par   - Use the quantitative inference \textcolor{rpTypeRule}{\textbf{threshold}}s to reduce subjective disagreement about whether errors "sometimes" obscure meaning.\par - Practical quick \textcolor{rpTypeRule}{\textbf{checklist}} before choosing a score\par   - Is there a clear position? \textcolor{rpTypeRule}{\textbf{If}} no -> Score 1.\par   - How many distinct supporting reasons are given? 0 -> Score 1; 1 (with weak or no elaboration) -> Score 1; 1 with \textcolor{rpTypeRule}{\textbf{at least}} one sentence of elaboration beyond the reason -> Score 2; 2+ with explicit tie-back for each and reasonable elaboration (\textasciitilde{}\textcolor{rpTypeRule}{\textbf{2 sentences}} per reason) -> consider Score 3.\par   - For each reason, is there a concrete example/explanation and an explicit link to the claim? \textcolor{rpTypeRule}{\textbf{If}} not for one or more reasons -> do not assign Score 3.\par   - Are errors frequent enough that the reader must re-read or infer meaning often? \textcolor{rpTypeRule}{\textbf{If}} inference needed on <=\textcolor{rpTypeRule}{\textbf{3 sentences}}/<=10\textcolor{rpTypeRule}{\textbf{\%}} -> Score 3 possible; \textcolor{rpTypeRule}{\textbf{if}} inference needed on >3 but not pervasive -> Score 2; \textcolor{rpTypeRule}{\textbf{if}} inference pervasive or many unintelligible sentences -> Score 1.\par   - Is the ending abruptly cut such that one or more required elaborations are missing? \textcolor{rpTypeRule}{\textbf{If}} yes -> downgrade one band (usually to the adjacent lower band).\par - Documentation: For any Score 1 or 2 rating where the essay contains some apparent development, raters must annotate (briefly) which specific requirements were missing (\textcolor{rpTypeEvidence}{\textbf{e.g.}}, "only one reason developed; \textcolor{rpTypeRule}{\textbf{second}} reason absent," "examples lack explicit tie-backs," or "frequent error pattern obscures meaning") to aid later calibration.
\end{tcolorbox}
\caption{Pattern-focused view of the optimized rubric (ets3, openai\_gpt-5-mini, base\_expert\_True\_train100\_iteration5\_top3\_bs4-8-12\_mc4). Colored bold spans indicate regex-matched rubric cues. Color types: \textcolor{rpTypeRule}{\textbf{Rule Structure}} (Explicit decision logic for scoring: conditional branches, boundary tie-breakers, stepwise workflows, and numeric thresholds.); \textcolor{rpTypeEvidence}{\textbf{Evidence Handling}} (How evidence is validated and counted: specific-example requirements, repetition/non-double-count rules, and cap rules for weak evidence.); \textcolor{rpTypeWriting}{\textbf{Writing Quality}} (Language-quality criteria affecting score bands: organization/coherence/transition quality and grammar/mechanics severity.). Matched pattern categories: Conditional Gating (n=44); Boundary / Tie-Break Guidance (n=13); Stepwise Rating Workflow (n=4); Specific Evidence Requirement (n=17); Off-Topic / Summary Cap (n=1); Organization / Coherence Signal (n=14); Grammar / Mechanics Signal (n=3); Repetition Non-Count Rule (n=10); Quantitative Threshold (n=29).}
\label{fig:rubric_pattern_ets3_openai_gpt_5_mini_base_expert_True_train100_iteration5_top3_bs4_8_12_mc4}
\end{figure*}
