\colorlet{rpTypeRule}{red!80!black}
\colorlet{rpTypeEvidence}{blue!80!black}
\colorlet{rpTypeWriting}{teal!80!black}
\begin{figure*}[t]
\centering
\begin{tcolorbox}[colback=white,colframe=black!25,title=Pattern Legend,fonttitle=\bfseries\small,fontupper=\scriptsize,boxsep=1pt,left=2pt,right=2pt,top=2pt,bottom=2pt]
\textcolor{rpTypeRule}{\textbf{Rule Structure}} (if/threshold/stepwise guidance) \quad \textcolor{rpTypeEvidence}{\textbf{Evidence Handling}} (examples, repetition, and caps) \quad \textcolor{rpTypeWriting}{\textbf{Writing Quality}} (organization and grammar/mechanics)
\end{tcolorbox}
\vspace{2mm}
\begin{minipage}[t]{0.485\textwidth}
\begin{tcolorbox}[colback=white,colframe=black!25,title=Initial Rubric,fonttitle=\bfseries\small,fontupper=\scriptsize,breakable]
\ttfamily
- is generally well organized and well developed, using appropriate and sufficient explanations, exemplifications, and/or details\par - displays unity, progression, and \textcolor{rpTypeWriting}{coherence}, though it may contain occasional redundancy, \textcolor{rpTypeEvidence}{digression}, or unclear connections\par - displays facility in the use of language, demonstrating syntactic variety and range of vocabulary, though it will probably have occasional noticeable minor errors in structure, word form, or use of idiomatic language that do not interfere with meaning\par ... [3 lines omitted] ...\par - addresses the topic and task using somewhat developed explanations, exemplifications, and/or details\par - displays unity, progression, and \textcolor{rpTypeWriting}{coherence}, though connection of ideas may be occasionally obscured\par - may demonstrate inconsistent facility in sentence formation and word choice that may result in lack of clarity and occasionally obscure meaning\par ... [4 lines omitted] ...\par - limited development in response to the topic and task\par - inadequate \textcolor{rpTypeWriting}{organization} or connection of ideas\par - inappropriate or insufficient exemplifications, explanations, or details to support or illustrate generalizations in response to the task
\end{tcolorbox}
\end{minipage}
\hfill
\begin{minipage}[t]{0.485\textwidth}
\begin{tcolorbox}[colback=white,colframe=black!25,title=Optimized Rubric,fonttitle=\bfseries\small,fontupper=\scriptsize,breakable]
\ttfamily
Revised Scoring Guideline (Version 5 - tightened \textcolor{rpTypeRule}{threshold}s and quantitative guidance to reduce rater disagreement)\par \par ... [1 lines omitted] ...\par - Read the entire essay for overall communicative effectiveness (how clearly the writer expresses and supports a position).\par - Use the concrete indicators below (development, \textcolor{rpTypeWriting}{organization}/\textcolor{rpTypeWriting}{coherence}, language accuracy and clarity) to place the essay into one of three score bands.\par - Priority rule \textcolor{rpTypeRule}{when} indicators conflict: prioritize intelligibility and development. \textcolor{rpTypeRule}{If} meaning is frequently or repeatedly obscured, lower the score even \textcolor{rpTypeRule}{if} some \textcolor{rpTypeWriting}{organization} or vocabulary are present.\par - Apply demotion rules strictly: any essay whose meaning is often or repeatedly obscured by errors must not receive Score 3. Any essay with only minimal, token, repetitive, or fragmentary development must not receive Score 3. Essays that end abruptly such that one or more required elaborations are missing should be downgraded (see Incomplete/abrupt ending rule).\par - Count both quantity and quality of supporting detail. Long essays with \textcolor{rpTypeEvidence}{repetition}, vagueness, or circular \textcolor{rpTypeEvidence}{restatement} of the same point are not sufficient for higher scores.\par \par ... [1 lines omitted] ...\par \par Score 3 - "Accomplished response" (use this only \textcolor{rpTypeRule}{when} the essay clearly meets most or all items below)\par - Task response and development\par   - Clearly and directly addresses the prompt and establishes a clear, sustained position or central idea.\par   - Provides \textcolor{rpTypeRule}{at least} two distinct supporting points/reasons. Distinct = different aspects or lines of reasoning (\textcolor{rpTypeEvidence}{e.g.}, cause vs. effect, two separate causes, two separate outcomes). \textcolor{rpTypeEvidence}{Repetition} or rewording of the same reason does not count as a \textcolor{rpTypeRule}{second} reason.\par   - For each supporting point, provides specific, relevant elaboration: at minimum 1) an explicit reason sentence, and 2) \textcolor{rpTypeRule}{at least} one additional sentence that explains relevance, consequence, or provides a concrete example tied to the reason.\par   - Example-linking requirement (strict): each example or \textcolor{rpTypeEvidence}{anecdote} must include an explicit, readable link back to the reason or claim. Acceptable links include short phrases such as "this shows that," "because," "therefore," "so," or a brief clause that ties the example to the reason. \textcolor{rpTypeRule}{If} the example is anecdotal or personal, it must include \textcolor{rpTypeRule}{at least} one concrete detail (time, place, specific action or outcome) and an explicit tie-back explaining how it supports the reason.\par   - Development depth requirement: expect roughly 2-\textcolor{rpTypeRule}{3 sentences} of elaboration per reason in a typical short essay (reason + explanation/link ± brief example). Single-sentence reasons with no explicit tie-back do not qualify for Score 3.\par - \textcolor{rpTypeWriting}{Organization} and \textcolor{rpTypeWriting}{coherence}\par   - Logical progression of ideas with paragraphing and \textcolor{rpTypeWriting}{transition}s; relationships among ideas are easy to trace.\par   - Has an adequately coherent introduction and conclusion. A brief or truncated conclusion is acceptable only \textcolor{rpTypeRule}{if} all required development (two distinct reasons, explicit tie-back, adequate elaboration) is already present in the body.\par - Language accuracy and range\par ... [1 lines omitted] ...\par   - No recurring error patterns that regularly obscure sense.\par - Intelligibility \textcolor{rpTypeRule}{threshold} (quantified)\par   - Overall meaning is easy to understand; the reader rarely needs to infer missing information.\par   - Quantitative guidance: in a typical short essay (approx200-\textcolor{rpTypeRule}{300 words}), the reader should need to re-read or infer meaning for no more than \textcolor{rpTypeRule}{3 sentences}, and these should constitute no more than \textasciitilde{}10\textcolor{rpTypeRule}{\%} of total sentences. \textcolor{rpTypeRule}{If} inference is required on more than \textcolor{rpTypeRule}{3 sentences} or >10\textcolor{rpTypeRule}{\%} of sentences, the essay cannot be Score 3.\par \par Score 2 - "Limited to partial response" (use \textcolor{rpTypeRule}{when} development and/or language prevent full accomplishment)\par - Task response and development\par ... [2 lines omitted] ...\par   - Essays that include examples with only implicit links (no explicit tie-back) are Score 2 unless other signs of clear, specific development compensate strongly.\par   - Personal \textcolor{rpTypeEvidence}{anecdote}s that lack concrete detail (\textcolor{rpTypeRule}{when}, where, what specifically happened) and/or lack explicit linkage to the claim should be treated as weak \textcolor{rpTypeEvidence}{evidence} and typically count only as partial support.\par   - Clarified \textcolor{rpTypeRule}{threshold}: To assign Score 2, the essay should show at minimum one clearly articulated reason with \textcolor{rpTypeRule}{at least} one sentence of elaboration beyond the reason statement (explanation, consequence, or a linked example). \textcolor{rpTypeRule}{If} the essay presents only assertions or general statements with no elaboration beyond repeating the claim, treat as Score 1.\par - \textcolor{rpTypeWriting}{Organization} and \textcolor{rpTypeWriting}{coherence}\par   - Basic \textcolor{rpTypeWriting}{organization} exists, but connections among ideas may be occasionally unclear or abrupt; paragraphs/\textcolor{rpTypeWriting}{transition}s may be weak.\par   - Reader can follow the argument with some effort; occasional rereading or inference is needed to understand links.\par ... [2 lines omitted] ...\par   - Demonstrates inconsistent control of sentence formation and word choice. Errors are frequent enough to be noticeable and sometimes obscure meaning, but the essay remains broadly recoverable.\par - Intelligibility \textcolor{rpTypeRule}{threshold}\par   - Meaning is generally recoverable; the reader sometimes must infer or re-read portions to understand intent.\par   - Quantified guidance: inference/re-reading needed on more than \textcolor{rpTypeRule}{3 sentences} but not pervasive across the essay, or inference is needed occasionally but does not regularly interrupt comprehension.\par \par Score 1 - "Weak or minimal response" (use \textcolor{rpTypeRule}{when} the response is minimal, disorganized, or frequently unintelligible)\par - Task response and development\par   - Response is minimal, only loosely related to the prompt, lacks a clear position, or presents only fragmented ideas.\par   - Very limited development: few or no relevant reasons, explanations, or examples; ideas are superficial, largely undeveloped, repetitive, or \textcolor{rpTypeEvidence}{off-topic}.\par   - Essays that end abruptly with an unfinished thought or literally cut off mid-sentence/paragraph leaving core claims undeveloped should be Score 1.\par   - Clarified \textcolor{rpTypeRule}{threshold}: \textcolor{rpTypeRule}{If} the essay contains only assertions, a stated opinion without any discernible supporting reason beyond \textcolor{rpTypeEvidence}{repetition}, or only one undeveloped reason with no elaboration, assign Score 1.\par - \textcolor{rpTypeWriting}{Organization} and \textcolor{rpTypeWriting}{coherence}\par   - \textcolor{rpTypeWriting}{Organization} is inadequate or absent. Ideas may be disjointed, digressive, or lack cohesion.\par   - \textcolor{rpTypeWriting}{Transition}s are missing and the reader cannot follow a clear line of reasoning without substantial effort.\par - Language accuracy and range\par   - Frequent and/or serious errors in sentence structure, \textcolor{rpTypeWriting}{grammar}, word form, or word choice that often obscure meaning.\par   - Recurring error patterns that regularly make propositions unclear warrant Score 1 even \textcolor{rpTypeRule}{if} a position is stated.\par - Intelligibility \textcolor{rpTypeRule}{threshold}\par   - Meaning is often unclear or frequently obscured; the reader has significant difficulty recovering the writer's intended message.\par   - Quantitative guidance: inference or re-reading is required on a substantial portion of sentences (\textcolor{rpTypeEvidence}{e.g.}, >25\textcolor{rpTypeRule}{\%} of sentences or more than \textasciitilde{}\textcolor{rpTypeRule}{8 sentences} in a typical short essay), or multiple sentences are unintelligible.\par \par Additional clarifying and procedural rules (to reduce disagreement)\par - Specificity and example-quality rule (strengthened): For Score 3, require \textcolor{rpTypeRule}{at least} two distinct supporting points, and for each point require:\par     1) an explicit reason sentence,\par     2) \textcolor{rpTypeRule}{at least} one explicit linking sentence or clause tying any example to the reason, and\par     3) either a concrete example with specific detail or a clear explanation of consequence/relevance. Vague \textcolor{rpTypeEvidence}{anecdote}s or generic statements without specifics count as weak \textcolor{rpTypeEvidence}{evidence}.\par - Example-linking rule (clarified \& enforced): Award development credit for an example only \textcolor{rpTypeRule}{when} the writer explicitly connects the example to the reason/claim. \textcolor{rpTypeRule}{If} a reason is supported only by an implicit \textcolor{rpTypeEvidence}{anecdote} or generic example with no explicit tie-back, treat it as partial support (Score 2).\par - Density-of-\textcolor{rpTypeEvidence}{evidence} rule (strengthened): \textcolor{rpTypeRule}{If} an essay is long but provides few real supporting details (lots of \textcolor{rpTypeEvidence}{repetition}, vague generalities, or circular \textcolor{rpTypeEvidence}{restatement}), prefer Score 1 or 2. \textcolor{rpTypeEvidence}{Repetition} of the same reason in multiple paragraphs does not constitute multiplicity of reasons.\par - Error-severity and consistency/demotion rules (quantified \& clarified):\par   - Occasional minor errors that never obscure the main idea should not lower a Score 3 rating.\par   - \textcolor{rpTypeRule}{If} errors sometimes obscure meaning (reader must infer), the essay cannot be Score 3; assign Score 2 or Score 1 depending on frequency and severity.\par   - To reduce subjective disagreement about "sometimes" vs "occasional," use the quantitative guidance above (Score 3: inference needed on <=\textcolor{rpTypeRule}{3 sentences} and <=10\textcolor{rpTypeRule}{\%} of sentences; Score 2: inference needed on >3 but not pervasive; Score 1: inference pervasive).\par   - \textcolor{rpTypeRule}{If} there is a recurring error pattern that regularly interferes with comprehension (many sentences where \textcolor{rpTypeWriting}{grammar} or word choice make meaning unclear), lower to Score 2 or Score 1. Frequent re-reading required = Score 1.\par   - Frequent surface errors that do not obscure core claims but substantially reduce clarity and fluency (many mis\textcolor{rpTypeWriting}{spelling}s, wrong word choices, sentence fragments) are grounds for Score 2 rather than Score 3. \textcolor{rpTypeRule}{If} such errors are pervasive across the essay and make main points shaky or effortful to recover, use Score 1.\par - Incomplete/abrupt ending rule (refined and made more consistent):\par   - \textcolor{rpTypeRule}{If} an essay terminates abruptly or ends mid-idea and thereby leaves core reasoning undeveloped, downgrade one band.\par   - Exceptions: \textcolor{rpTypeRule}{If} an essay otherwise meets Score 3 except for a truncated concluding sentence but both reasons are fully developed with explicit links in the body, maintain Score 3.\par   - \textcolor{rpTypeRule}{If} truncation leaves one required elaboration missing (\textcolor{rpTypeEvidence}{e.g.}, one reason only partially developed or an example not tied back), downgrade one band (usually Score 3 -> Score 2; Score 2 -> Score 1) rather than automatically to the lowest band.\par   - \textcolor{rpTypeRule}{If} truncation plus language errors make recovery of the missing elaboration difficult, consider further downgrade to the lowest band appropriate (Score 1).\par - \textcolor{rpTypeEvidence}{Repetition} vs. multiplicity rule (re-emphasized): Repeating the same reason with slightly different wording is not equivalent to providing multiple distinct reasons. Score 3 requires distinct reasons; repetitive \textcolor{rpTypeEvidence}{restatement} counts as one reason and should limit the score.\par - Personal \textcolor{rpTypeEvidence}{anecdote} handling: Personal experience can count as \textcolor{rpTypeEvidence}{evidence} but is often weaker than a concrete, \textcolor{rpTypeEvidence}{specific example} that is linked analytically to the claim. \textcolor{rpTypeRule}{When} a personal \textcolor{rpTypeEvidence}{anecdote} lacks detail or an explicit analytic tie-back, treat it as partial support (Score 2 or 1 depending on other factors).\par - Minimum-development \textcolor{rpTypeRule}{threshold}s (practical guidance for raters):\par   - To award Score 3: require \textcolor{rpTypeRule}{at least} two distinct reasons, each with explicit relevant elaboration and an explicit tie-back sentence/clause. Typical short essays should show \textasciitilde{}2-\textcolor{rpTypeRule}{3 sentences} per reason (reason + explanation/link + optional brief example). \textcolor{rpTypeRule}{If} either reason lacks this minimal elaboration and explicit tie-back, prefer Score 2.\par   - To award Score 2: the essay should show a clear position and \textcolor{rpTypeRule}{at least} one discernible supporting reason with some attempt at elaboration (one full sentence beyond the reason), or two weak reasons with thin support. \textcolor{rpTypeRule}{If} there is only assertion without elaboration or only one undeveloped reason, assign Score 1.\par   - To award Score 1: the essay has no clear position OR only isolated, undeveloped statements, or frequent unintelligibility caused by pervasive errors or severe underdevelopment.\par - \textcolor{rpTypeRule}{Borderline} handling (refined with explicit \textcolor{rpTypeRule}{checklist} and \textcolor{rpTypeRule}{tie-break}ers): \textcolor{rpTypeRule}{When} between adjacent scores, ask these questions in order:\par   1) Are there \textcolor{rpTypeRule}{at least} two distinct reasons? \textcolor{rpTypeRule}{If} not -> cannot be Score 3.\par   2) For each reason present, is there an explicit tie-back linking any example/explanation to that reason? \textcolor{rpTypeRule}{If} a reason lacks an explicit tie-back -> cannot be Score 3.\par   3) For each reason, is there \textcolor{rpTypeRule}{at least} one sentence of development beyond the reason statement (explanation, consequence, or concrete detail)? \textcolor{rpTypeRule}{If} not for one or more reasons -> cannot be Score 3.\par   4) Does error frequency/severity force the reader to infer or re-read often? Use quantitative \textcolor{rpTypeRule}{threshold}s: \textcolor{rpTypeRule}{if} inference is needed on <=\textcolor{rpTypeRule}{3 sentences} and <=10\textcolor{rpTypeRule}{\%} of sentences -> may still be Score 3; \textcolor{rpTypeRule}{if} inference is needed on >\textcolor{rpTypeRule}{3 sentences} but not pervasive -> prefer Score 2; \textcolor{rpTypeRule}{if} inference is pervasive -> Score 1.\par   5) Is the essay truncated or abruptly ended such that required elaboration is missing? \textcolor{rpTypeRule}{If} yes -> downgrade one band (usually to the adjacent lower band).\par   Use these ordered checks as decisive \textcolor{rpTypeRule}{tie-break}ers rather than holistic instinct \textcolor{rpTypeRule}{when} uncertain.\par - Calibration notes addressing common rater pitfalls (explicit guidance derived from mismatch cases):\par   - Do not award Score 3 solely because an essay contains two named examples; each example must be explicitly connected to a distinct reason and show relevant elaboration. \textcolor{rpTypeRule}{If} examples are undeveloped or linkage is implicit, prefer Score 2.\par   - Do not let a coherent introduction and some understandable sentences override severe underdevelopment or pervasive errors. A readable intro is not a substitute for full development.\par   - Penalize essays that appear long but are largely \textcolor{rpTypeEvidence}{repetition} or vague generalities; length is not \textcolor{rpTypeEvidence}{evidence} of sufficient development.\par   - \textcolor{rpTypeRule}{If} many surface errors exist but content is broadly understandable, Score 2 is expected unless development clearly meets the Score 3 \textcolor{rpTypeRule}{threshold}s. \textcolor{rpTypeRule}{If} errors make recovery effortful or the reasoning shaky, use Score 1.\par   - Personal experience statements like "I have realized" or "from my experience" without specifics and an explicit tie-back are weak \textcolor{rpTypeEvidence}{evidence}-treat them as partial support.\par   - Use the quantitative inference \textcolor{rpTypeRule}{threshold}s to reduce subjective disagreement about whether errors "sometimes" obscure meaning.\par - Practical quick \textcolor{rpTypeRule}{checklist} before choosing a score\par   - Is there a clear position? \textcolor{rpTypeRule}{If} no -> Score 1.\par   - How many distinct supporting reasons are given? 0 -> Score 1; 1 (with weak or no elaboration) -> Score 1; 1 with \textcolor{rpTypeRule}{at least} one sentence of elaboration beyond the reason -> Score 2; 2+ with explicit tie-back for each and reasonable elaboration (\textasciitilde{}\textcolor{rpTypeRule}{2 sentences} per reason) -> consider Score 3.\par   - For each reason, is there a concrete example/explanation and an explicit link to the claim? \textcolor{rpTypeRule}{If} not for one or more reasons -> do not assign Score 3.\par   - Are errors frequent enough that the reader must re-read or infer meaning often? \textcolor{rpTypeRule}{If} inference needed on <=\textcolor{rpTypeRule}{3 sentences}/<=10\textcolor{rpTypeRule}{\%} -> Score 3 possible; \textcolor{rpTypeRule}{if} inference needed on >3 but not pervasive -> Score 2; \textcolor{rpTypeRule}{if} inference pervasive or many unintelligible sentences -> Score 1.\par   - Is the ending abruptly cut such that one or more required elaborations are missing? \textcolor{rpTypeRule}{If} yes -> downgrade one band (usually to the adjacent lower band).\par - Documentation: For any Score 1 or 2 rating where the essay contains some apparent development, raters must annotate (briefly) which specific requirements were missing (\textcolor{rpTypeEvidence}{e.g.}, "only one reason developed; \textcolor{rpTypeRule}{second} reason absent," "examples lack explicit tie-backs," or "frequent error pattern obscures meaning") to aid later calibration.
\end{tcolorbox}
\end{minipage}
\caption{Pattern-highlighted rubric comparison (ets3, openai\_gpt-5-mini, base\_expert\_True\_train100\_iteration5\_top3\_bs4-8-12\_mc4). Matched spans are color-coded by regex pattern. Color types: \textcolor{rpTypeRule}{\textbf{Rule Structure}} (if/threshold/stepwise guidance); \textcolor{rpTypeEvidence}{\textbf{Evidence Handling}} (examples, repetition, and caps); \textcolor{rpTypeWriting}{\textbf{Writing Quality}} (organization and grammar/mechanics).}
\label{fig:rubric_pattern_ets3_openai_gpt_5_mini_base_expert_True_train100_iteration5_top3_bs4_8_12_mc4}
\end{figure*}
