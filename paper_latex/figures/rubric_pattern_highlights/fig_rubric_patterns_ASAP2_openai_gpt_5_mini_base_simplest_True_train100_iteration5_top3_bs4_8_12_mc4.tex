\colorlet{rpTypeRule}{red!80!black}
\colorlet{rpTypeEvidence}{blue!80!black}
\colorlet{rpTypeWriting}{teal!80!black}
\begin{figure*}[t]
\centering
\begin{tcolorbox}[colback=white,colframe=black!25,title=Pattern Type Guide,fonttitle=\bfseries\small,fontupper=\scriptsize,boxsep=1pt,left=2pt,right=2pt,top=2pt,bottom=2pt]
\textcolor{rpTypeRule}{\textbf{Rule Structure}}: Explicit decision logic for scoring: conditional branches, boundary tie-breakers, stepwise workflows, and numeric thresholds.\par \textcolor{rpTypeEvidence}{\textbf{Evidence Handling}}: How evidence is validated and counted: specific-example requirements, repetition/non-double-count rules, and cap rules for weak evidence.\par \textcolor{rpTypeWriting}{\textbf{Writing Quality}}: Language-quality criteria affecting score bands: organization/coherence/transition quality and grammar/mechanics severity.
\end{tcolorbox}
\vspace{1mm}
\begin{tcolorbox}[colback=white,colframe=black!25,title=Detailed Pattern Notes,fonttitle=\bfseries\small,fontupper=\scriptsize,boxsep=1pt,left=2pt,right=2pt,top=2pt,bottom=2pt]
\textcolor{rpTypeRule}{\textbf{Rule Structure}}:\par \quad \textcolor{rpTypeRule}{\textbf{Conditional Gating}} [n=20] Captures explicit condition-based rules that switch decisions only when a stated condition is met. Typical cues: if, when.\par \quad \textcolor{rpTypeRule}{\textbf{Boundary / Tie-Break Guidance}} [n=17] Marks criteria used to resolve borderline cases between adjacent score bands (e.g., 4 vs 5). Typical cues: tie-break, borderline, boundary, threshold, 4 vs 5.\par \quad \textcolor{rpTypeRule}{\textbf{Stepwise Rating Workflow}} [n=6] Detects ordered procedures and checklists that standardize how raters walk through scoring decisions. Typical cues: step, checklist, workflow, procedure, first/second/third.\par \quad \textcolor{rpTypeRule}{\textbf{Quantitative Threshold}} [n=11] Marks numeric cutoffs used for consistent decisions (minimum/maximum counts, percentages, explicit count thresholds). Typical cues: at least, at most, <=, >=, \%, N reasons/examples/sentences/words.\par \textcolor{rpTypeEvidence}{\textbf{Evidence Handling}}:\par \quad \textcolor{rpTypeEvidence}{\textbf{Specific Evidence Requirement}} [n=34] Highlights demands for concrete examples and explicit evidence links instead of generic assertions. Typical cues: for example, e.g., specific example, illustration, anecdote, evidence.\par \quad \textcolor{rpTypeEvidence}{\textbf{Off-Topic / Summary Cap}} [n=4] Identifies cap rules that restrict scores when responses are off-topic, irrelevant, or dominated by summary-only content. Typical cues: off-topic, irrelevant, digression, summary-only, cap.\par \quad \textcolor{rpTypeEvidence}{\textbf{Repetition Non-Count Rule}} [n=1] Captures rules that treat repetition/restatement as non-distinct support and prevent double-counting. Typical cues: repetition, restatement, double-count, do not double-count.\par \textcolor{rpTypeWriting}{\textbf{Writing Quality}}:\par \quad \textcolor{rpTypeWriting}{\textbf{Organization / Coherence Signal}} [n=15] Detects explicit references to discourse structure and logical flow as scoring criteria. Typical cues: organization, coherence, logical flow, transition.\par \quad \textcolor{rpTypeWriting}{\textbf{Grammar / Mechanics Signal}} [n=2] Detects references to language-form quality, especially grammar, spelling, punctuation, and mechanics. Typical cues: grammar, mechanics, spelling, punctuation.
\end{tcolorbox}
\vspace{1mm}
\begin{tcolorbox}[colback=white,colframe=black!25,title=Optimized Rubric (Pattern-Highlighted),fonttitle=\bfseries\small,fontupper=\scriptsize]
\ttfamily
- Assign a single holistic score (1-6) reflecting overall performance across five dimensions: Comprehension \& Analysis, Development \& \textcolor{rpTypeWriting}{\textbf{Organization}}, Use of \textcolor{rpTypeEvidence}{\textbf{Evidence}}, Clarity \& Control of Language, and Overall Communicative Effectiveness.\par - For every score, provide a brief written justification naming the primary driver(s) (\textcolor{rpTypeEvidence}{\textbf{e.g.}}, "lowered to 2 due to pervasive sentence‑level errors" or "capped at 4 because \textcolor{rpTypeEvidence}{\textbf{evidence}} is quoted but not explained").\par ... [2 lines omitted] ...\par - To separate 1-2 from 3-6: place heavier weight on Clarity \& Control of Language (apply the strengthened Mechanical‑error rule \textcolor{rpTypeRule}{\textbf{first}}) and on Comprehension \& Analysis (including Factual‑misunderstanding rule).\par - To separate 3-4 from 5-6: place heavier weight on Use of \textcolor{rpTypeEvidence}{\textbf{Evidence}} and Development \& \textcolor{rpTypeWriting}{\textbf{Organization}} (especially \textcolor{rpTypeEvidence}{\textbf{evidence}} integration and explicit explanation).\par - To separate adjacent mid-scores (3 vs 4, \textcolor{rpTypeRule}{\textbf{4 vs 5}}): apply the three tie‑questions (see Conservatism in midpoints) but with concrete \textcolor{rpTypeRule}{\textbf{threshold}}s (see "Operational \textcolor{rpTypeRule}{\textbf{threshold}}s" below).\par ... [1 lines omitted] ...\par Operational \textcolor{rpTypeRule}{\textbf{threshold}}s and clarified decision rules\par ... [1 lines omitted] ...\par   - "Pervasive" errors = \textcolor{rpTypeWriting}{\textbf{spelling}}/word-choice/\textcolor{rpTypeWriting}{\textbf{punctuation}} or sentence‑level breakdowns in more than \textasciitilde{}25\textcolor{rpTypeRule}{\textbf{\%}} of sentences, OR repeated sentence fragments/run-ons that force reader to infer basic claims repeatedly.\par   - Assign 1 \textcolor{rpTypeRule}{\textbf{when}} meaning is largely or frequently incomprehensible even to a careful reader (multiple sentences/paragraphs where core claims cannot be reliably extracted).\par   - Assign 2 \textcolor{rpTypeRule}{\textbf{when}} pervasive errors often impede comprehension and the reader must repeatedly infer basic claims (meaning extractable only with repeated effort).\par   - Assign 3 \textcolor{rpTypeRule}{\textbf{when}} mechanical errors are frequent but the reader can generally extract the intended meaning without repeated inference (errors are distracting but not obstructive).\par   - Assign 4-6 only \textcolor{rpTypeRule}{\textbf{when}} meaning is consistently extractable and mechanical issues do not repeatedly force inference. (4 may have noticeable errors; 5-6 should have minor/no errors.)\par   - Raters must count error frequency roughly: \textcolor{rpTypeRule}{\textbf{if}} \textasciitilde{}1 in \textcolor{rpTypeRule}{\textbf{4 sentences}} or more contains a serious mechanical problem that impacts understanding, consider 2 rather than 3.\par ... [3 lines omitted] ...\par   - One significant factual inaccuracy that contradicts a core claim of the source (\textcolor{rpTypeEvidence}{\textbf{e.g.}}, saying "no spacecraft have ever landed on Venus" \textcolor{rpTypeRule}{\textbf{when}} the passage or known facts indicate otherwise) -> lower Comprehension \& Analysis \textcolor{rpTypeRule}{\textbf{at least}} one score band (\textcolor{rpTypeEvidence}{\textbf{e.g.}}, 4->3, 3->2).\par ... [3 lines omitted] ...\par - \textcolor{rpTypeEvidence}{\textbf{Evidence}}‑integration rule (strengthened):\par   - To award 5 or 6, the essay must (a) include relevant \textcolor{rpTypeEvidence}{\textbf{evidence}} for each major claim and (b) explicitly interpret or explain how each key piece of \textcolor{rpTypeEvidence}{\textbf{evidence}} supports the claim (linking, not just quoting).\par   - Essays that predominantly summarize the passage or string quotations with little explanation should \textcolor{rpTypeEvidence}{\textbf{cap}} at 4.\par ... [2 lines omitted] ...\par - Conservatism in midpoints (3 vs 4 and \textcolor{rpTypeRule}{\textbf{4 vs 5}}) - concrete tie questions and \textcolor{rpTypeRule}{\textbf{threshold}}s:\par   For each pair (3 vs 4 and \textcolor{rpTypeRule}{\textbf{4 vs 5}}), apply these three questions and score "yes" \textcolor{rpTypeRule}{\textbf{when}} the criterion is met clearly:\par ... [1 lines omitted] ...\par   (b) Is \textcolor{rpTypeEvidence}{\textbf{evidence}} explicitly linked to that claim (i.e., each major supporting point has \textcolor{rpTypeEvidence}{\textbf{evidence}} followed by explanation of how it supports the thesis)? (Yes = explicit linkage for most major claims; No = mostly paraphrase/quotation without explanation.)\par   (c) Does \textcolor{rpTypeWriting}{\textbf{organization}} show purposeful progression with clear paragraphing and \textcolor{rpTypeWriting}{\textbf{transition}}s (not mere \textcolor{rpTypeEvidence}{\textbf{repetition}} of passage sequence)? (Yes = purposeful development; No = repetitive summary or disjointed paragraphs.)\par   - For 3 vs 4: \textcolor{rpTypeRule}{\textbf{if}} \textcolor{rpTypeRule}{\textbf{at least}} two of three are "Yes", favor 4; otherwise favor 3. But \textcolor{rpTypeRule}{\textbf{if}} mechanical errors meet the "pervasive" \textcolor{rpTypeRule}{\textbf{threshold}} for 2, assign 2 instead.\par   - For \textcolor{rpTypeRule}{\textbf{4 vs 5}}: require all three "Yes" to favor 5. \textcolor{rpTypeRule}{\textbf{If}} only two are "Yes", remain at 4 unless \textcolor{rpTypeEvidence}{\textbf{evidence}} integration is strongly analytic and interpretive (in which case consider 5).\par ... [5 lines omitted] ...\par - Development \& \textcolor{rpTypeWriting}{\textbf{Organization}}: Logical, purposeful progression; distinct paragraphs and smooth \textcolor{rpTypeWriting}{\textbf{transition}}s.\par - Use of \textcolor{rpTypeEvidence}{\textbf{Evidence}}: \textcolor{rpTypeEvidence}{\textbf{Evidence}} is consistently relevant, fully integrated, and each key piece is explicitly explained in support of claims.\par ... [1 lines omitted] ...\par - Signals to award 6: All tie questions met; \textcolor{rpTypeEvidence}{\textbf{evidence}} exemplifies interpretation rather than summary; only isolated minor errors \textcolor{rpTypeRule}{\textbf{at most}}.\par ... [3 lines omitted] ...\par - Development \& \textcolor{rpTypeWriting}{\textbf{Organization}}: Coherent and substantive development; minor lapses possible.\par - Use of \textcolor{rpTypeEvidence}{\textbf{Evidence}}: \textcolor{rpTypeEvidence}{\textbf{Evidence}} usually integrated and explained; most major claims linked to textual support.\par - Clarity \& Control: Mechanical errors may be present but do not impede comprehension; error frequency well below "pervasive" (\textcolor{rpTypeEvidence}{\textbf{e.g.}}, <25\textcolor{rpTypeRule}{\textbf{\%}} of sentences affected).\par - Signals to award 5: All three tie questions largely met; explanation of \textcolor{rpTypeEvidence}{\textbf{evidence}} is analytic and not just paraphrase.\par ... [3 lines omitted] ...\par - Development \& \textcolor{rpTypeWriting}{\textbf{Organization}}: \textcolor{rpTypeWriting}{\textbf{Organization}} apparent but may be repetitive or underdeveloped.\par - Use of \textcolor{rpTypeEvidence}{\textbf{Evidence}}: Uses relevant \textcolor{rpTypeEvidence}{\textbf{evidence}}, but integration/explanation is superficial; heavy reliance on summary or quotes without interpretation.\par - Clarity \& Control: Noticeable errors and awkward phrasing but meaning is generally extractable without repeated inference (errors below "pervasive" \textcolor{rpTypeRule}{\textbf{threshold}}).\par - Signals to award 4: Typically answers \textcolor{rpTypeRule}{\textbf{at least}} two tie questions affirmatively for 3/4 decision; \textcolor{rpTypeEvidence}{\textbf{evidence}} linked superficially. Do not raise beyond 4 for quotation-heavy essays lacking interpretation.\par ... [3 lines omitted] ...\par - Development \& \textcolor{rpTypeWriting}{\textbf{Organization}}: Weak structure; uneven development or some disjointed paragraphs.\par - Use of \textcolor{rpTypeEvidence}{\textbf{Evidence}}: \textcolor{rpTypeEvidence}{\textbf{Evidence}} inconsistently linked; often used as summary or is minimal.\par ... [1 lines omitted] ...\par - Signals to award 3: One or none of the tie questions clearly met; errors present but do not reach "pervasive" \textcolor{rpTypeRule}{\textbf{threshold}} that would force a 2.\par ... [3 lines omitted] ...\par - Development \& \textcolor{rpTypeWriting}{\textbf{Organization}}: Poor, fragmented, illogical structure.\par - Use of \textcolor{rpTypeEvidence}{\textbf{Evidence}}: Little or no effective use of \textcolor{rpTypeEvidence}{\textbf{evidence}}; quotations are present but unexplained or misused.\par ... [1 lines omitted] ...\par - Signals to award 2: Error frequency meets "pervasive" \textcolor{rpTypeRule}{\textbf{threshold}} (>\textasciitilde{}25\textcolor{rpTypeRule}{\textbf{\%}} of sentences seriously problematic) and/or a significant factual misunderstanding exists (see Factual‑misunderstanding rule) that reduces comprehension by \textcolor{rpTypeRule}{\textbf{at least}} one band.\par ... [2 lines omitted] ...\par - Comprehension \& Analysis: Little to no comprehension; incoherent, nonsensical, or \textcolor{rpTypeEvidence}{\textbf{irrelevant}} content.\par - Development \& \textcolor{rpTypeWriting}{\textbf{Organization}}: No usable \textcolor{rpTypeWriting}{\textbf{organization}}; ideas largely incomprehensible.\par - Use of \textcolor{rpTypeEvidence}{\textbf{Evidence}}: No meaningful \textcolor{rpTypeEvidence}{\textbf{evidence}} usage.\par ... [5 lines omitted] ...\par   - \textcolor{rpTypeRule}{\textbf{If}} the essay has a clear central claim, reasonable paragraphing, and meaning is extractable without repeated effort, favor 4 (Competent) rather than 3-even \textcolor{rpTypeRule}{\textbf{if}} errors are frequent-so long as errors do not meet the "pervasive" \textcolor{rpTypeRule}{\textbf{threshold}}. Rationale must note mechanical issues as secondary drivers.\par   - Reserve 3 \textcolor{rpTypeRule}{\textbf{when}} \textcolor{rpTypeWriting}{\textbf{organization}}/analysis is weak in addition to frequent (but non‑pervasive) mechanical errors.\par ... [2 lines omitted] ...\par   - \textcolor{rpTypeEvidence}{\textbf{Cap}} at 4 unless the writer interprets, connects, and explains quotations to support a central claim. Quantity of accurate facts/quotations alone is insufficient for 5-6.\par ... [2 lines omitted] ...\par   - \textcolor{rpTypeRule}{\textbf{If}} a single significant factual contradiction to the passage or core knowledge appears, reduce the overall score by \textcolor{rpTypeRule}{\textbf{at least}} one band from what it would otherwise be. Cite the factual error in the justification.\par ... [3 lines omitted] ...\par   - \textcolor{rpTypeRule}{\textbf{If}} topical content is present but sentence‑level errors force repeated inference of basic claims (reader must re‑read repeatedly), assign 2 (Minimal). Only assign 3 \textcolor{rpTypeRule}{\textbf{when}} meaning is generally extractable without repeated inference.\par ... [1 lines omitted] ...\par Reviewer \textcolor{rpTypeRule}{\textbf{workflow}} \textcolor{rpTypeRule}{\textbf{checklist}} (to enforce consistent application)\par 1. Apply Mechanical‑error rule \textcolor{rpTypeRule}{\textbf{first}}: does the essay meet "pervasive" error \textcolor{rpTypeRule}{\textbf{threshold}}? \textcolor{rpTypeRule}{\textbf{If}} yes, consider 1-2 per definitions above.\par 2. Check for significant factual misunderstandings: do any errors contradict the passage's core claims? \textcolor{rpTypeRule}{\textbf{If}} yes, lower \textcolor{rpTypeRule}{\textbf{at least}} one band and document it.\par 3. Evaluate central claim and \textcolor{rpTypeWriting}{\textbf{organization}}: apply tie‑questions and operational \textcolor{rpTypeRule}{\textbf{threshold}}s.\par 4. Evaluate \textcolor{rpTypeEvidence}{\textbf{evidence}} integration: are quotations/paraphrases interpreted and linked to claims? \textcolor{rpTypeRule}{\textbf{If}} not, \textcolor{rpTypeEvidence}{\textbf{cap}} at 4.\par 5. Assign score and write a concise justification naming primary drivers (language problems, \textcolor{rpTypeEvidence}{\textbf{evidence}} integration, factual misunderstanding, or development).\par 6. \textcolor{rpTypeRule}{\textbf{If}} \textcolor{rpTypeRule}{\textbf{borderline}}, explicitly state which dimension tipped the decision.\par ... [2 lines omitted] ...\par - For every score, include a one‑sentence annotation listing the primary reason(s) for the score (\textcolor{rpTypeEvidence}{\textbf{e.g.}}, "Capped at 4 because \textcolor{rpTypeEvidence}{\textbf{evidence}} is quoted but not explained; frequent but non‑pervasive mechanical errors"). This is required to make disagreements traceable.\par ... [1 lines omitted] ...\par This revised guidance focuses on clearer operational \textcolor{rpTypeRule}{\textbf{threshold}}s for mechanical errors, stricter enforcement of factual‑misunderstanding penalties, and firmer rules for \textcolor{rpTypeEvidence}{\textbf{evidence}} integration and caps on quotation‑heavy essays. Applying the \textcolor{rpTypeRule}{\textbf{workflow}} \textcolor{rpTypeRule}{\textbf{checklist}} should reduce the common over- and under‑scoring patterns observed in past ratings.
\end{tcolorbox}
\caption{Pattern-focused view of the optimized rubric (ASAP2, openai\_gpt-5-mini, base\_simplest\_True\_train100\_iteration5\_top3\_bs4-8-12\_mc4). Colored bold spans indicate regex-matched rubric cues. Color types: \textcolor{rpTypeRule}{\textbf{Rule Structure}} (Explicit decision logic for scoring: conditional branches, boundary tie-breakers, stepwise workflows, and numeric thresholds.); \textcolor{rpTypeEvidence}{\textbf{Evidence Handling}} (How evidence is validated and counted: specific-example requirements, repetition/non-double-count rules, and cap rules for weak evidence.); \textcolor{rpTypeWriting}{\textbf{Writing Quality}} (Language-quality criteria affecting score bands: organization/coherence/transition quality and grammar/mechanics severity.). Matched pattern categories: Conditional Gating (n=20); Boundary / Tie-Break Guidance (n=17); Stepwise Rating Workflow (n=6); Specific Evidence Requirement (n=34); Off-Topic / Summary Cap (n=4); Organization / Coherence Signal (n=15); Grammar / Mechanics Signal (n=2); Repetition Non-Count Rule (n=1); Quantitative Threshold (n=11).}
\label{fig:rubric_pattern_ASAP2_openai_gpt_5_mini_base_simplest_True_train100_iteration5_top3_bs4_8_12_mc4}
\end{figure*}
