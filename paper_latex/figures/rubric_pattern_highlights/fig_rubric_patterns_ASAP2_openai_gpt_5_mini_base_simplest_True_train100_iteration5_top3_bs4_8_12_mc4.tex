\colorlet{rpTypeRule}{red!80!black}
\colorlet{rpTypeEvidence}{blue!80!black}
\colorlet{rpTypeWriting}{teal!80!black}
\begin{figure*}[t]
\centering
\begin{tcolorbox}[colback=white,colframe=black!25,title=Pattern Legend,fonttitle=\bfseries\small,fontupper=\scriptsize,boxsep=1pt,left=2pt,right=2pt,top=2pt,bottom=2pt]
\textcolor{rpTypeRule}{\textbf{Rule Structure}} (if/threshold/stepwise guidance) \quad \textcolor{rpTypeEvidence}{\textbf{Evidence Handling}} (examples, repetition, and caps) \quad \textcolor{rpTypeWriting}{\textbf{Writing Quality}} (organization and grammar/mechanics)
\end{tcolorbox}
\vspace{2mm}
\begin{minipage}[t]{0.485\textwidth}
\begin{tcolorbox}[colback=white,colframe=black!25,title=Initial Rubric,fonttitle=\bfseries\small,fontupper=\scriptsize,breakable]
\ttfamily
Based on the response's content, rate the response on a scale of 1 to 6.
\end{tcolorbox}
\end{minipage}
\hfill
\begin{minipage}[t]{0.485\textwidth}
\begin{tcolorbox}[colback=white,colframe=black!25,title=Optimized Rubric,fonttitle=\bfseries\small,fontupper=\scriptsize,breakable]
\ttfamily
Overview\par - Assign a single holistic score (1-6) reflecting overall performance across five dimensions: Comprehension \& Analysis, Development \& \textcolor{rpTypeWriting}{Organization}, Use of \textcolor{rpTypeEvidence}{Evidence}, Clarity \& Control of Language, and Overall Communicative Effectiveness.\par - For every score, provide a brief written justification naming the primary driver(s) (\textcolor{rpTypeEvidence}{e.g.}, "lowered to 2 due to pervasive sentence‑level errors" or "capped at 4 because \textcolor{rpTypeEvidence}{evidence} is quoted but not explained").\par \par Primary discriminators (updated)\par - To separate 1-2 from 3-6: place heavier weight on Clarity \& Control of Language (apply the strengthened Mechanical‑error rule \textcolor{rpTypeRule}{first}) and on Comprehension \& Analysis (including Factual‑misunderstanding rule).\par - To separate 3-4 from 5-6: place heavier weight on Use of \textcolor{rpTypeEvidence}{Evidence} and Development \& \textcolor{rpTypeWriting}{Organization} (especially \textcolor{rpTypeEvidence}{evidence} integration and explicit explanation).\par - To separate adjacent mid-scores (3 vs 4, \textcolor{rpTypeRule}{4 vs 5}): apply the three tie‑questions (see Conservatism in midpoints) but with concrete \textcolor{rpTypeRule}{threshold}s (see "Operational \textcolor{rpTypeRule}{threshold}s" below).\par \par Operational \textcolor{rpTypeRule}{threshold}s and clarified decision rules\par - Mechanical‑error rule (clarified):\par   - "Pervasive" errors = \textcolor{rpTypeWriting}{spelling}/word-choice/\textcolor{rpTypeWriting}{punctuation} or sentence‑level breakdowns in more than \textasciitilde{}25\textcolor{rpTypeRule}{\%} of sentences, OR repeated sentence fragments/run-ons that force reader to infer basic claims repeatedly.\par   - Assign 1 \textcolor{rpTypeRule}{when} meaning is largely or frequently incomprehensible even to a careful reader (multiple sentences/paragraphs where core claims cannot be reliably extracted).\par   - Assign 2 \textcolor{rpTypeRule}{when} pervasive errors often impede comprehension and the reader must repeatedly infer basic claims (meaning extractable only with repeated effort).\par   - Assign 3 \textcolor{rpTypeRule}{when} mechanical errors are frequent but the reader can generally extract the intended meaning without repeated inference (errors are distracting but not obstructive).\par   - Assign 4-6 only \textcolor{rpTypeRule}{when} meaning is consistently extractable and mechanical issues do not repeatedly force inference. (4 may have noticeable errors; 5-6 should have minor/no errors.)\par   - Raters must count error frequency roughly: \textcolor{rpTypeRule}{if} \textasciitilde{}1 in \textcolor{rpTypeRule}{4 sentences} or more contains a serious mechanical problem that impacts understanding, consider 2 rather than 3.\par \par ... [1 lines omitted] ...\par   - Single minor factual error (one incorrect detail that does not affect central claim) -> deduct or note as a minor issue, but do not automatically lower an entire band.\par   - One significant factual inaccuracy that contradicts a core claim of the source (\textcolor{rpTypeEvidence}{e.g.}, saying "no spacecraft have ever landed on Venus" \textcolor{rpTypeRule}{when} the passage or known facts indicate otherwise) -> lower Comprehension \& Analysis \textcolor{rpTypeRule}{at least} one score band (\textcolor{rpTypeEvidence}{e.g.}, 4->3, 3->2).\par   - Multiple or repeated significant factual errors or a conceptual misunderstanding of the passage -> reduce overall score by one or more bands depending on severity.\par ... [1 lines omitted] ...\par \par - \textcolor{rpTypeEvidence}{Evidence}‑integration rule (strengthened):\par   - To award 5 or 6, the essay must (a) include relevant \textcolor{rpTypeEvidence}{evidence} for each major claim and (b) explicitly interpret or explain how each key piece of \textcolor{rpTypeEvidence}{evidence} supports the claim (linking, not just quoting).\par   - Essays that predominantly summarize the passage or string quotations with little explanation should \textcolor{rpTypeEvidence}{cap} at 4.\par   - Essays that use quotations but fail to interpret or link them to a coherent claim should not exceed 3.\par \par - Conservatism in midpoints (3 vs 4 and \textcolor{rpTypeRule}{4 vs 5}) - concrete tie questions and \textcolor{rpTypeRule}{threshold}s:\par   For each pair (3 vs 4 and \textcolor{rpTypeRule}{4 vs 5}), apply these three questions and score "yes" \textcolor{rpTypeRule}{when} the criterion is met clearly:\par   (a) Is there a clear central claim or thesis that can be restated in one sentence? (Yes = clear, focused thesis; No = implicit or absent.)\par   (b) Is \textcolor{rpTypeEvidence}{evidence} explicitly linked to that claim (i.e., each major supporting point has \textcolor{rpTypeEvidence}{evidence} followed by explanation of how it supports the thesis)? (Yes = explicit linkage for most major claims; No = mostly paraphrase/quotation without explanation.)\par   (c) Does \textcolor{rpTypeWriting}{organization} show purposeful progression with clear paragraphing and \textcolor{rpTypeWriting}{transition}s (not mere \textcolor{rpTypeEvidence}{repetition} of passage sequence)? (Yes = purposeful development; No = repetitive summary or disjointed paragraphs.)\par   - For 3 vs 4: \textcolor{rpTypeRule}{if} \textcolor{rpTypeRule}{at least} two of three are "Yes", favor 4; otherwise favor 3. But \textcolor{rpTypeRule}{if} mechanical errors meet the "pervasive" \textcolor{rpTypeRule}{threshold} for 2, assign 2 instead.\par   - For \textcolor{rpTypeRule}{4 vs 5}: require all three "Yes" to favor 5. \textcolor{rpTypeRule}{If} only two are "Yes", remain at 4 unless \textcolor{rpTypeEvidence}{evidence} integration is strongly analytic and interpretive (in which case consider 5).\par \par ... [3 lines omitted] ...\par - Comprehension \& Analysis: Thorough, accurate, and nuanced reading; offers original insight (implications, limitations, or counterarguments).\par - Development \& \textcolor{rpTypeWriting}{Organization}: Logical, purposeful progression; distinct paragraphs and smooth \textcolor{rpTypeWriting}{transition}s.\par - Use of \textcolor{rpTypeEvidence}{Evidence}: \textcolor{rpTypeEvidence}{Evidence} is consistently relevant, fully integrated, and each key piece is explicitly explained in support of claims.\par - Clarity \& Control: Minor or no mechanical errors; language is precise and fluent.\par - Signals to award 6: All tie questions met; \textcolor{rpTypeEvidence}{evidence} exemplifies interpretation rather than summary; only isolated minor errors \textcolor{rpTypeRule}{at most}.\par \par ... [1 lines omitted] ...\par - Comprehension \& Analysis: Accurate and sophisticated; addresses task with depth but may lack the nuance of a 6.\par - Development \& \textcolor{rpTypeWriting}{Organization}: Coherent and substantive development; minor lapses possible.\par - Use of \textcolor{rpTypeEvidence}{Evidence}: \textcolor{rpTypeEvidence}{Evidence} usually integrated and explained; most major claims linked to textual support.\par - Clarity \& Control: Mechanical errors may be present but do not impede comprehension; error frequency well below "pervasive" (\textcolor{rpTypeEvidence}{e.g.}, <25\textcolor{rpTypeRule}{\%} of sentences affected).\par - Signals to award 5: All three tie questions largely met; explanation of \textcolor{rpTypeEvidence}{evidence} is analytic and not just paraphrase.\par \par ... [1 lines omitted] ...\par - Comprehension \& Analysis: Understands main ideas; analysis is limited (more explanation than insight) and often summarizes.\par - Development \& \textcolor{rpTypeWriting}{Organization}: \textcolor{rpTypeWriting}{Organization} apparent but may be repetitive or underdeveloped.\par - Use of \textcolor{rpTypeEvidence}{Evidence}: Uses relevant \textcolor{rpTypeEvidence}{evidence}, but integration/explanation is superficial; heavy reliance on summary or quotes without interpretation.\par - Clarity \& Control: Noticeable errors and awkward phrasing but meaning is generally extractable without repeated inference (errors below "pervasive" \textcolor{rpTypeRule}{threshold}).\par - Signals to award 4: Typically answers \textcolor{rpTypeRule}{at least} two tie questions affirmatively for 3/4 decision; \textcolor{rpTypeEvidence}{evidence} linked superficially. Do not raise beyond 4 for quotation-heavy essays lacking interpretation.\par \par ... [1 lines omitted] ...\par - Comprehension \& Analysis: Partial or inconsistent comprehension; analysis simplistic, repetitive, or mainly restates text.\par - Development \& \textcolor{rpTypeWriting}{Organization}: Weak structure; uneven development or some disjointed paragraphs.\par - Use of \textcolor{rpTypeEvidence}{Evidence}: \textcolor{rpTypeEvidence}{Evidence} inconsistently linked; often used as summary or is minimal.\par - Clarity \& Control: Frequent mechanical errors and sentence problems that sometimes obscure meaning, but overall meaning is generally extractable (i.e., reader does not need to repeatedly infer core claims).\par - Signals to award 3: One or none of the tie questions clearly met; errors present but do not reach "pervasive" \textcolor{rpTypeRule}{threshold} that would force a 2.\par \par ... [1 lines omitted] ...\par - Comprehension \& Analysis: Minimal or confused comprehension; key ideas missing or incorrect; limited support and development.\par - Development \& \textcolor{rpTypeWriting}{Organization}: Poor, fragmented, illogical structure.\par - Use of \textcolor{rpTypeEvidence}{Evidence}: Little or no effective use of \textcolor{rpTypeEvidence}{evidence}; quotations are present but unexplained or misused.\par - Clarity \& Control: Pervasive mechanical errors and sentence‑level breakdowns that often impede comprehension; reader must repeatedly infer core claims.\par - Signals to award 2: Error frequency meets "pervasive" \textcolor{rpTypeRule}{threshold} (>\textasciitilde{}25\textcolor{rpTypeRule}{\%} of sentences seriously problematic) and/or a significant factual misunderstanding exists (see Factual‑misunderstanding rule) that reduces comprehension by \textcolor{rpTypeRule}{at least} one band.\par \par 1 - Ineffective / Unintelligible\par - Comprehension \& Analysis: Little to no comprehension; incoherent, nonsensical, or \textcolor{rpTypeEvidence}{irrelevant} content.\par - Development \& \textcolor{rpTypeWriting}{Organization}: No usable \textcolor{rpTypeWriting}{organization}; ideas largely incomprehensible.\par - Use of \textcolor{rpTypeEvidence}{Evidence}: No meaningful \textcolor{rpTypeEvidence}{evidence} usage.\par - Clarity \& Control: Errors so pervasive meaning is frequently or entirely obscured.\par ... [3 lines omitted] ...\par - On essays with many surface-level errors but clear summary and structure (common over/under‑scored cases):\par   - \textcolor{rpTypeRule}{If} the essay has a clear central claim, reasonable paragraphing, and meaning is extractable without repeated effort, favor 4 (Competent) rather than 3-even \textcolor{rpTypeRule}{if} errors are frequent-so long as errors do not meet the "pervasive" \textcolor{rpTypeRule}{threshold}. Rationale must note mechanical issues as secondary drivers.\par   - Reserve 3 \textcolor{rpTypeRule}{when} \textcolor{rpTypeWriting}{organization}/analysis is weak in addition to frequent (but non‑pervasive) mechanical errors.\par \par - On essays that repeat the passage and string quotations:\par   - \textcolor{rpTypeEvidence}{Cap} at 4 unless the writer interprets, connects, and explains quotations to support a central claim. Quantity of accurate facts/quotations alone is insufficient for 5-6.\par \par - On essays with factual inaccuracies:\par   - \textcolor{rpTypeRule}{If} a single significant factual contradiction to the passage or core knowledge appears, reduce the overall score by \textcolor{rpTypeRule}{at least} one band from what it would otherwise be. Cite the factual error in the justification.\par   - Multiple contradictions or repeated misrepresentations warrant further downgrading.\par ... [1 lines omitted] ...\par - On essays with pervasive mechanical errors vs heavy topical content:\par   - \textcolor{rpTypeRule}{If} topical content is present but sentence‑level errors force repeated inference of basic claims (reader must re‑read repeatedly), assign 2 (Minimal). Only assign 3 \textcolor{rpTypeRule}{when} meaning is generally extractable without repeated inference.\par \par Reviewer \textcolor{rpTypeRule}{workflow} \textcolor{rpTypeRule}{checklist} (to enforce consistent application)\par 1. Apply Mechanical‑error rule \textcolor{rpTypeRule}{first}: does the essay meet "pervasive" error \textcolor{rpTypeRule}{threshold}? \textcolor{rpTypeRule}{If} yes, consider 1-2 per definitions above.\par 2. Check for significant factual misunderstandings: do any errors contradict the passage's core claims? \textcolor{rpTypeRule}{If} yes, lower \textcolor{rpTypeRule}{at least} one band and document it.\par 3. Evaluate central claim and \textcolor{rpTypeWriting}{organization}: apply tie‑questions and operational \textcolor{rpTypeRule}{threshold}s.\par 4. Evaluate \textcolor{rpTypeEvidence}{evidence} integration: are quotations/paraphrases interpreted and linked to claims? \textcolor{rpTypeRule}{If} not, \textcolor{rpTypeEvidence}{cap} at 4.\par 5. Assign score and write a concise justification naming primary drivers (language problems, \textcolor{rpTypeEvidence}{evidence} integration, factual misunderstanding, or development).\par 6. \textcolor{rpTypeRule}{If} \textcolor{rpTypeRule}{borderline}, explicitly state which dimension tipped the decision.\par \par Documentation requirement\par - For every score, include a one‑sentence annotation listing the primary reason(s) for the score (\textcolor{rpTypeEvidence}{e.g.}, "Capped at 4 because \textcolor{rpTypeEvidence}{evidence} is quoted but not explained; frequent but non‑pervasive mechanical errors"). This is required to make disagreements traceable.\par \par This revised guidance focuses on clearer operational \textcolor{rpTypeRule}{threshold}s for mechanical errors, stricter enforcement of factual‑misunderstanding penalties, and firmer rules for \textcolor{rpTypeEvidence}{evidence} integration and caps on quotation‑heavy essays. Applying the \textcolor{rpTypeRule}{workflow} \textcolor{rpTypeRule}{checklist} should reduce the common over- and under‑scoring patterns observed in past ratings.
\end{tcolorbox}
\end{minipage}
\caption{Pattern-highlighted rubric comparison (ASAP2, openai\_gpt-5-mini, base\_simplest\_True\_train100\_iteration5\_top3\_bs4-8-12\_mc4). Matched spans are color-coded by regex pattern. Color types: \textcolor{rpTypeRule}{\textbf{Rule Structure}} (if/threshold/stepwise guidance); \textcolor{rpTypeEvidence}{\textbf{Evidence Handling}} (examples, repetition, and caps); \textcolor{rpTypeWriting}{\textbf{Writing Quality}} (organization and grammar/mechanics).}
\label{fig:rubric_pattern_ASAP2_openai_gpt_5_mini_base_simplest_True_train100_iteration5_top3_bs4_8_12_mc4}
\end{figure*}
