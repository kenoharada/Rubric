\colorlet{rpTypeRule}{red!80!black}
\colorlet{rpTypeEvidence}{blue!80!black}
\colorlet{rpTypeWriting}{teal!80!black}
\begin{figure*}[t]
\centering
\begin{tcolorbox}[colback=white,colframe=black!25,title=Pattern Type Guide,fonttitle=\bfseries\small,fontupper=\scriptsize,boxsep=1pt,left=2pt,right=2pt,top=2pt,bottom=2pt]
\textcolor{rpTypeRule}{\textbf{Rule Structure}}: Explicit decision logic for scoring: conditional branches, boundary tie-breakers, stepwise workflows, and numeric thresholds.\par \textcolor{rpTypeEvidence}{\textbf{Evidence Handling}}: How evidence is validated and counted: specific-example requirements, repetition/non-double-count rules, and cap rules for weak evidence.\par \textcolor{rpTypeWriting}{\textbf{Writing Quality}}: Language-quality criteria affecting score bands: organization/coherence/transition quality and grammar/mechanics severity.
\end{tcolorbox}
\vspace{1mm}
\begin{tcolorbox}[colback=white,colframe=black!25,title=Detailed Pattern Notes,fonttitle=\bfseries\small,fontupper=\scriptsize,boxsep=1pt,left=2pt,right=2pt,top=2pt,bottom=2pt]
\textcolor{rpTypeRule}{\textbf{Rule Structure}}:\par \quad \textcolor{rpTypeRule}{\textbf{Conditional Gating}} [n=3] Captures explicit condition-based rules that switch decisions only when a stated condition is met. Typical cues: if, when.\par \textcolor{rpTypeEvidence}{\textbf{Evidence Handling}}:\par \quad \textcolor{rpTypeEvidence}{\textbf{Specific Evidence Requirement}} [n=5] Highlights demands for concrete examples and explicit evidence links instead of generic assertions. Typical cues: for example, e.g., specific example, illustration, anecdote, evidence.\par \quad \textcolor{rpTypeEvidence}{\textbf{Off-Topic / Summary Cap}} [n=2] Identifies cap rules that restrict scores when responses are off-topic, irrelevant, or dominated by summary-only content. Typical cues: off-topic, irrelevant, digression, summary-only, cap.\par \textcolor{rpTypeWriting}{\textbf{Writing Quality}}:\par \quad \textcolor{rpTypeWriting}{\textbf{Organization / Coherence Signal}} [n=3] Detects explicit references to discourse structure and logical flow as scoring criteria. Typical cues: organization, coherence, logical flow, transition.\par \quad \textcolor{rpTypeWriting}{\textbf{Grammar / Mechanics Signal}} [n=5] Detects references to language-form quality, especially grammar, spelling, punctuation, and mechanics. Typical cues: grammar, mechanics, spelling, punctuation.
\end{tcolorbox}
\vspace{1mm}
\begin{tcolorbox}[colback=white,colframe=black!25,title=Optimized Rubric (Pattern-Highlighted),fonttitle=\bfseries\small,fontupper=\scriptsize]
\ttfamily
- addresses the topic and task well, providing specific, relevant, and sophisticated details or examples; an unfinished conclusion or mechanical errors (\textcolor{rpTypeWriting}{\textbf{spelling}}, typos, word form) do not automatically disqualify an essay \textcolor{rpTypeRule}{\textbf{if}} the preceding development is thorough and the logic is compelling\par ... [1 lines omitted] ...\par - displays unity and \textcolor{rpTypeWriting}{\textbf{coherence}}; the argument is sustained throughout the response, and any minor errors do not interfere with the reader's ability to follow sophisticated reasoning\par - displays facility in the use of language, demonstrating syntactic variety and range of vocabulary; even \textcolor{rpTypeRule}{\textbf{if}} there is a noticeable "accumulation" of surface-level errors (\textcolor{rpTypeEvidence}{\textbf{e.g.}}, "patern," "nowdays," "thining," "Eddison"), the essay remains at this level \textcolor{rpTypeRule}{\textbf{if}} the writer maintains a consistent rhythm and control of complex sentence structures\par ... [4 lines omitted] ...\par - displays unity and \textcolor{rpTypeWriting}{\textbf{coherence}}, but the connection of ideas is occasionally obscured, or the essay moves mechanically from one point to another without deep exploration\par - demonstrates inconsistent facility in sentence formation; while the meaning is generally clear, frequent errors in \textcolor{rpTypeWriting}{\textbf{grammar}}, \textcolor{rpTypeWriting}{\textbf{spelling}}, or word choice (\textcolor{rpTypeEvidence}{\textbf{e.g.}}, "more risk more yield," "out of different reasons," "there were made several surveys") indicate a lack of sustained control over the language or a reliance on overly simple sentence structures\par - may stray into \textcolor{rpTypeEvidence}{\textbf{irrelevant}} \textcolor{rpTypeEvidence}{\textbf{digression}}s, personal \textcolor{rpTypeEvidence}{\textbf{anecdote}}s that provide only tangential support, or meta-commentary about the test-taking process\par ... [4 lines omitted] ...\par - inadequate \textcolor{rpTypeWriting}{\textbf{organization}} or connection of ideas; the essay may struggle to move beyond a few basic thoughts or may end abruptly mid-sentence (\textcolor{rpTypeEvidence}{\textbf{e.g.}}, "maybe we would be who")\par - a pervasive accumulation of errors in sentence structure, usage, \textcolor{rpTypeWriting}{\textbf{spelling}}, and \textcolor{rpTypeWriting}{\textbf{grammar}} (\textcolor{rpTypeEvidence}{\textbf{e.g.}}, "it can be happen," "how I can happy," "rock question," "anwear") that seriously interrupts the flow of reading and suggests a lack of basic linguistic competence
\end{tcolorbox}
\caption{Pattern-focused view of the optimized rubric (ets3, google\_gemini-3-flash-preview, base\_expert\_True\_train100\_iteration5\_top3\_bs4-8-12\_mc4). Colored bold spans indicate regex-matched rubric cues. Color types: \textcolor{rpTypeRule}{\textbf{Rule Structure}} (Explicit decision logic for scoring: conditional branches, boundary tie-breakers, stepwise workflows, and numeric thresholds.); \textcolor{rpTypeEvidence}{\textbf{Evidence Handling}} (How evidence is validated and counted: specific-example requirements, repetition/non-double-count rules, and cap rules for weak evidence.); \textcolor{rpTypeWriting}{\textbf{Writing Quality}} (Language-quality criteria affecting score bands: organization/coherence/transition quality and grammar/mechanics severity.). Matched pattern categories: Conditional Gating (n=3); Specific Evidence Requirement (n=5); Off-Topic / Summary Cap (n=2); Organization / Coherence Signal (n=3); Grammar / Mechanics Signal (n=5).}
\label{fig:rubric_pattern_ets3_google_gemini_3_flash_preview_base_expert_True_train100_iteration5_top3_bs4_8_12_mc4}
\end{figure*}
