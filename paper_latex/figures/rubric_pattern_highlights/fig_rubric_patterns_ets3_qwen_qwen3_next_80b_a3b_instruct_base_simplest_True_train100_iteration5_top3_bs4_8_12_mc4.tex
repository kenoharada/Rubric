\colorlet{rpTypeRule}{red!80!black}
\colorlet{rpTypeEvidence}{blue!80!black}
\colorlet{rpTypeWriting}{teal!80!black}
\begin{figure*}[t]
\centering
\begin{tcolorbox}[colback=white,colframe=black!25,title=Pattern Type Guide,fonttitle=\bfseries\small,fontupper=\scriptsize,boxsep=1pt,left=2pt,right=2pt,top=2pt,bottom=2pt]
\textcolor{rpTypeRule}{\textbf{Rule Structure}}: Explicit decision logic for scoring: conditional branches, boundary tie-breakers, stepwise workflows, and numeric thresholds.\par \textcolor{rpTypeEvidence}{\textbf{Evidence Handling}}: How evidence is validated and counted: specific-example requirements, repetition/non-double-count rules, and cap rules for weak evidence.\par \textcolor{rpTypeWriting}{\textbf{Writing Quality}}: Language-quality criteria affecting score bands: organization/coherence/transition quality and grammar/mechanics severity.
\end{tcolorbox}
\vspace{1mm}
\begin{tcolorbox}[colback=white,colframe=black!25,title=Detailed Pattern Notes,fonttitle=\bfseries\small,fontupper=\scriptsize,boxsep=1pt,left=2pt,right=2pt,top=2pt,bottom=2pt]
\textcolor{rpTypeRule}{\textbf{Rule Structure}}:\par \quad \textcolor{rpTypeRule}{\textbf{Conditional Gating}} [n=16] Captures explicit condition-based rules that switch decisions only when a stated condition is met. Typical cues: if, when.\par \textcolor{rpTypeEvidence}{\textbf{Evidence Handling}}:\par \quad \textcolor{rpTypeEvidence}{\textbf{Specific Evidence Requirement}} [n=7] Highlights demands for concrete examples and explicit evidence links instead of generic assertions. Typical cues: for example, e.g., specific example, illustration, anecdote, evidence.\par \quad \textcolor{rpTypeEvidence}{\textbf{Off-Topic / Summary Cap}} [n=2] Identifies cap rules that restrict scores when responses are off-topic, irrelevant, or dominated by summary-only content. Typical cues: off-topic, irrelevant, digression, summary-only, cap.\par \quad \textcolor{rpTypeEvidence}{\textbf{Repetition Non-Count Rule}} [n=1] Captures rules that treat repetition/restatement as non-distinct support and prevent double-counting. Typical cues: repetition, restatement, double-count, do not double-count.\par \textcolor{rpTypeWriting}{\textbf{Writing Quality}}:\par \quad \textcolor{rpTypeWriting}{\textbf{Organization / Coherence Signal}} [n=8] Detects explicit references to discourse structure and logical flow as scoring criteria. Typical cues: organization, coherence, logical flow, transition.\par \quad \textcolor{rpTypeWriting}{\textbf{Grammar / Mechanics Signal}} [n=10] Detects references to language-form quality, especially grammar, spelling, punctuation, and mechanics. Typical cues: grammar, mechanics, spelling, punctuation.
\end{tcolorbox}
\vspace{1mm}
\begin{tcolorbox}[colback=white,colframe=black!25,title=Optimized Rubric (Pattern-Highlighted),fonttitle=\bfseries\small,fontupper=\scriptsize]
\ttfamily
- addresses the topic and task thoroughly, with well-developed, specific, and relevant explanations, exemplifications, and details that directly support the position; examples are clearly explained, logically connected to the argument, and demonstrate insight rather than mere \textcolor{rpTypeEvidence}{\textbf{illustration}} - even \textcolor{rpTypeRule}{\textbf{if}} an example is unconventional or imperfectly factual, it is treated with analytical depth and purposefully tied to the claim; minor factual inaccuracies (\textcolor{rpTypeEvidence}{\textbf{e.g.}}, misnamed individuals, exaggerated statistics) are acceptable only \textcolor{rpTypeRule}{\textbf{if}} they are not central to the argument and do not undermine the credibility of the analysis\par - is clearly and logically organized, with smooth progression and strong \textcolor{rpTypeWriting}{\textbf{coherence}} between ideas; \textcolor{rpTypeWriting}{\textbf{transition}}s are natural, structure is purposeful, and there is no significant redundancy, \textcolor{rpTypeEvidence}{\textbf{digression}}, or unclear connections - minor \textcolor{rpTypeWriting}{\textbf{organization}}al imperfections (\textcolor{rpTypeEvidence}{\textbf{e.g.}}, repetitive phrasing or slightly awkward \textcolor{rpTypeWriting}{\textbf{transition}}s) are acceptable \textcolor{rpTypeRule}{\textbf{if}} the overall argument flows logically and the paragraphing supports the development of ideas\par - displays consistent and effective facility in the use of language, demonstrating varied and sophisticated sentence structures and a broad, precise vocabulary appropriate to the academic context; word choice enhances clarity and impact - occasional non-standard word forms, mis\textcolor{rpTypeWriting}{\textbf{spelling}}s, or idiomatic imprecisions are acceptable only \textcolor{rpTypeRule}{\textbf{if}} they are rare, isolated, and do not impede comprehension or undermine the writer's control; errors must not be systematic or frequent enough to require interpretive effort\par - contains only rare, minor errors in \textcolor{rpTypeWriting}{\textbf{grammar}}, word form, or idiomatic usage that are inconsequential to understanding and do not distract from the message; \textcolor{rpTypeWriting}{\textbf{spelling}} and \textcolor{rpTypeWriting}{\textbf{punctuation}} are consistently accurate - errors must not be predictable, recurring, or pervasive enough to suggest a lack of language proficiency; \textcolor{rpTypeRule}{\textbf{if}} errors appear frequently (\textcolor{rpTypeEvidence}{\textbf{e.g.}}, consistent mis\textcolor{rpTypeWriting}{\textbf{spelling}} of common words like "nowdays," "donot," "belive"), the essay cannot qualify for Score 3, regardless of content strength\par ... [3 lines omitted] ...\par - addresses the topic and task with sufficient development, but explanations, examples, or details may be somewhat general, inconsistently specific, or only partially developed; the argument is present but lacks depth, precision, or analytical insight in places; examples may be relevant but are not fully explained or are loosely tied to the claim - even \textcolor{rpTypeRule}{\textbf{if}} examples are flawed, fictional, or oversimplified, they are recognized as intended \textcolor{rpTypeEvidence}{\textbf{illustration}}s and are not dismissed as invalid \textcolor{rpTypeRule}{\textbf{if}} they serve a clear rhetorical purpose; factual inaccuracies or imprecise examples are acceptable \textcolor{rpTypeRule}{\textbf{if}} they do not dominate the argument or mislead the reader's understanding of the position\par - shows adequate \textcolor{rpTypeWriting}{\textbf{organization}} and logical progression, with clear overall structure, but may contain minor lapses in \textcolor{rpTypeWriting}{\textbf{coherence}}, \textcolor{rpTypeWriting}{\textbf{transition}}al awkwardness, or repetitive points that do not undermine the central argument; ideas are connected but not always with sophistication - redundancy or phrasing \textcolor{rpTypeEvidence}{\textbf{repetition}} is acceptable \textcolor{rpTypeRule}{\textbf{if}} the core logic remains intact and paragraphing provides discernible structure\par - demonstrates partial control of language: sentence structures are generally varied but may include occasional awkwardness, simplicity, or overuse of basic constructions; vocabulary is adequate and mostly appropriate, though sometimes repetitive, imprecise, or colloquial - frequent \textcolor{rpTypeWriting}{\textbf{spelling}}, \textcolor{rpTypeWriting}{\textbf{grammar}}, or word form errors are acceptable \textcolor{rpTypeRule}{\textbf{if}} they are predictable (\textcolor{rpTypeEvidence}{\textbf{e.g.}}, "nowdays," "donot," "belive," "Eddison," "popolation"), systematic but not severe, and do not consistently obscure meaning or require substantial reinterpretation; the writer's intent and position remain clearly discernible despite these errors\par - contains noticeable errors in \textcolor{rpTypeWriting}{\textbf{spelling}}, \textcolor{rpTypeWriting}{\textbf{punctuation}}, or syntax that affect tone or clarity but do not prevent the reader from understanding the writer's intent and position; errors must be frequent enough to be systematic but not so severe as to make comprehension difficult without contextual inference - \textcolor{rpTypeRule}{\textbf{if}} errors are pervasive (\textcolor{rpTypeEvidence}{\textbf{e.g.}}, recurring in nearly every sentence), the essay should not be rated higher than Score 2, even \textcolor{rpTypeRule}{\textbf{if}} the content is well-developed\par ... [3 lines omitted] ...\par - shows minimal or inadequate development of ideas in response to the task; explanations, examples, or details are missing, \textcolor{rpTypeEvidence}{\textbf{irrelevant}}, overly general, fabricated without purpose, or fail to meaningfully support the position; the response may misinterpret the prompt or avoid addressing it directly - fabricated examples are only acceptable \textcolor{rpTypeRule}{\textbf{if}} they are used deliberately and analyzed; otherwise, their presence without analysis or relevance signals Score 1\par - lacks clear \textcolor{rpTypeWriting}{\textbf{organization}} or logical progression; ideas are disjointed, poorly connected, or presented in a confusing sequence with no discernible structure; paragraphing is absent or arbitrary - the absence of paragraphing alone is not sufficient for Score 1 \textcolor{rpTypeRule}{\textbf{if}} ideas are otherwise logically grouped\par - contains frequent and severe errors in sentence structure, word form, \textcolor{rpTypeWriting}{\textbf{spelling}}, and usage that consistently obscure meaning and hinder communication; the language is often unintelligible or requires substantial interpretation to discern intent - errors must be pervasive and systemic enough to make comprehension difficult even with contextual inference\par - demonstrates a very limited or inappropriate use of vocabulary and syntactic structures, making the response difficult to follow even \textcolor{rpTypeRule}{\textbf{when}} content is partially understandable - vocabulary must be consistently primitive or misused, not merely imprecise\par - may include nonsensical examples, factual distortions, or persistent grammatical breakdowns that undermine credibility and prevent the reader from engaging with the argument - isolated factual inaccuracies are not grounds for Score 1 \textcolor{rpTypeRule}{\textbf{if}} the argument otherwise holds together
\end{tcolorbox}
\caption{Pattern-focused view of the optimized rubric (ets3, qwen\_qwen3-next-80b-a3b-instruct, base\_simplest\_True\_train100\_iteration5\_top3\_bs4-8-12\_mc4). Colored bold spans indicate regex-matched rubric cues. Color types: \textcolor{rpTypeRule}{\textbf{Rule Structure}} (Explicit decision logic for scoring: conditional branches, boundary tie-breakers, stepwise workflows, and numeric thresholds.); \textcolor{rpTypeEvidence}{\textbf{Evidence Handling}} (How evidence is validated and counted: specific-example requirements, repetition/non-double-count rules, and cap rules for weak evidence.); \textcolor{rpTypeWriting}{\textbf{Writing Quality}} (Language-quality criteria affecting score bands: organization/coherence/transition quality and grammar/mechanics severity.). Matched pattern categories: Conditional Gating (n=16); Specific Evidence Requirement (n=7); Off-Topic / Summary Cap (n=2); Organization / Coherence Signal (n=8); Grammar / Mechanics Signal (n=10); Repetition Non-Count Rule (n=1).}
\label{fig:rubric_pattern_ets3_qwen_qwen3_next_80b_a3b_instruct_base_simplest_True_train100_iteration5_top3_bs4_8_12_mc4}
\end{figure*}
