\colorlet{rpTypeRule}{red!80!black}
\colorlet{rpTypeEvidence}{blue!80!black}
\colorlet{rpTypeWriting}{teal!80!black}
\begin{figure*}[t]
\centering
\begin{tcolorbox}[colback=white,colframe=black!25,title=Pattern Type Guide,fonttitle=\bfseries\small,fontupper=\scriptsize,boxsep=1pt,left=2pt,right=2pt,top=2pt,bottom=2pt]
\textcolor{rpTypeRule}{\textbf{Rule Structure}}: Explicit decision logic for scoring: conditional branches, boundary tie-breakers, stepwise workflows, and numeric thresholds.\par \textcolor{rpTypeEvidence}{\textbf{Evidence Handling}}: How evidence is validated and counted: specific-example requirements, repetition/non-double-count rules, and cap rules for weak evidence.\par \textcolor{rpTypeWriting}{\textbf{Writing Quality}}: Language-quality criteria affecting score bands: organization/coherence/transition quality and grammar/mechanics severity.
\end{tcolorbox}
\vspace{1mm}
\begin{tcolorbox}[colback=white,colframe=black!25,title=Detailed Pattern Notes,fonttitle=\bfseries\small,fontupper=\scriptsize,boxsep=1pt,left=2pt,right=2pt,top=2pt,bottom=2pt]
\textcolor{rpTypeRule}{\textbf{Rule Structure}}:\par \quad \textcolor{rpTypeRule}{\textbf{Conditional Gating}} [n=2] Captures explicit condition-based rules that switch decisions only when a stated condition is met. Typical cues: if, when.\par \quad \textcolor{rpTypeRule}{\textbf{Stepwise Rating Workflow}} [n=1] Detects ordered procedures and checklists that standardize how raters walk through scoring decisions. Typical cues: step, checklist, workflow, procedure, first/second/third.\par \quad \textcolor{rpTypeRule}{\textbf{Quantitative Threshold}} [n=1] Marks numeric cutoffs used for consistent decisions (minimum/maximum counts, percentages, explicit count thresholds). Typical cues: at least, at most, <=, >=, \%, N reasons/examples/sentences/words.\par \textcolor{rpTypeEvidence}{\textbf{Evidence Handling}}:\par \quad \textcolor{rpTypeEvidence}{\textbf{Specific Evidence Requirement}} [n=9] Highlights demands for concrete examples and explicit evidence links instead of generic assertions. Typical cues: for example, e.g., specific example, illustration, anecdote, evidence.\par \quad \textcolor{rpTypeEvidence}{\textbf{Off-Topic / Summary Cap}} [n=2] Identifies cap rules that restrict scores when responses are off-topic, irrelevant, or dominated by summary-only content. Typical cues: off-topic, irrelevant, digression, summary-only, cap.\par \quad \textcolor{rpTypeEvidence}{\textbf{Repetition Non-Count Rule}} [n=1] Captures rules that treat repetition/restatement as non-distinct support and prevent double-counting. Typical cues: repetition, restatement, double-count, do not double-count.\par \textcolor{rpTypeWriting}{\textbf{Writing Quality}}:\par \quad \textcolor{rpTypeWriting}{\textbf{Organization / Coherence Signal}} [n=4] Detects explicit references to discourse structure and logical flow as scoring criteria. Typical cues: organization, coherence, logical flow, transition.\par \quad \textcolor{rpTypeWriting}{\textbf{Grammar / Mechanics Signal}} [n=5] Detects references to language-form quality, especially grammar, spelling, punctuation, and mechanics. Typical cues: grammar, mechanics, spelling, punctuation.
\end{tcolorbox}
\vspace{1mm}
\begin{tcolorbox}[colback=white,colframe=black!25,title=Optimized Rubric (Pattern-Highlighted),fonttitle=\bfseries\small,fontupper=\scriptsize]
\ttfamily
- addresses the topic and task effectively, providing specific reasons and well-developed, concrete examples (\textcolor{rpTypeEvidence}{\textbf{e.g.}}, citing specific historical figures, detailed personal \textcolor{rpTypeEvidence}{\textbf{anecdote}}s, or specific societal trends) that directly support the thesis; the development shows an ability to handle complex ideas and nuanced perspectives\par - is well-organized and displays a clear, logical progression of ideas; while the structure may rely on standard \textcolor{rpTypeWriting}{\textbf{transition}} words (\textcolor{rpTypeEvidence}{\textbf{e.g.}}, "\textcolor{rpTypeRule}{\textbf{First}}," "Secondly"), they are used effectively to guide a coherent argument rather than appearing purely mechanical\par - displays a high facility in the use of language and a range of syntactic variety; although it may contain frequent minor errors in \textcolor{rpTypeWriting}{\textbf{grammar}}, \textcolor{rpTypeWriting}{\textbf{spelling}}, or \textcolor{rpTypeWriting}{\textbf{mechanics}} (\textcolor{rpTypeEvidence}{\textbf{e.g.}}, "doesnot," "yound," "popluare," "well-arouned"), these errors do not obscure meaning or significantly interfere with the strength and flow of the argument\par ... [4 lines omitted] ...\par - addresses the topic and task but development is limited or uneven; the essay may rely on generalities, \textcolor{rpTypeEvidence}{\textbf{repetition}} of ideas, or explanations that lack grounded, specific detail (\textcolor{rpTypeEvidence}{\textbf{e.g.}}, repeating that a situation is "boring" or "hard" without further elaboration)\par - displays unity and \textcolor{rpTypeWriting}{\textbf{coherence}}, but the \textcolor{rpTypeWriting}{\textbf{organization}} may feel overly mechanical or the connection of ideas may be occasionally obscured by linguistic limitations; the essay may resemble a list of points rather than a progression of an argument\par - demonstrates grammatical, \textcolor{rpTypeWriting}{\textbf{spelling}}, or word-choice errors that are persistent enough to distract the reader or suggest a lack of range (\textcolor{rpTypeEvidence}{\textbf{e.g.}}, "earn money hardly," "is important trying," "specialising on"); while the general meaning is discernible, the phrasing often feels unnatural or lacks the complexity/nuance of a higher-level response\par ... [3 lines omitted] ...\par An essay at this level reveals a significant lack of competence \textcolor{rpTypeEvidence}{\textbf{evidence}}d by one or more of the following:\par - limited or very poor development in response to the topic; the essay may be significantly short (under \textcolor{rpTypeRule}{\textbf{200 words}}), fail to address key parts of the prompt, or rely almost entirely on vague, hypothetical "\textcolor{rpTypeRule}{\textbf{if}}/then" scenarios and generalities ("\textcolor{rpTypeRule}{\textbf{if}} we help then world is good") without any concrete \textcolor{rpTypeEvidence}{\textbf{evidence}}\par - inadequate \textcolor{rpTypeWriting}{\textbf{organization}} or connection of ideas, where the reader must frequently pause to reconstruct the author's logic or piece together the relationship between sentences\par - inappropriate, insufficient, or \textcolor{rpTypeEvidence}{\textbf{irrelevant}} examples that fail to support the generalizations made, often resulting in a response that feels "\textcolor{rpTypeEvidence}{\textbf{off-topic}}," purely philosophical, or superficial\par - a pervasive accumulation of serious errors in sentence structure, usage, and \textcolor{rpTypeWriting}{\textbf{spelling}} (\textcolor{rpTypeEvidence}{\textbf{e.g.}}, "cheeting," "funny and rock question," "past the goal") that frequently obscures meaning or results in a lack of clarity throughout the majority of the essay
\end{tcolorbox}
\caption{Pattern-focused view of the optimized rubric (ets3, google\_gemini-3-flash-preview, base\_simplest\_True\_train100\_iteration5\_top3\_bs4-8-12\_mc4). Colored bold spans indicate regex-matched rubric cues. Color types: \textcolor{rpTypeRule}{\textbf{Rule Structure}} (Explicit decision logic for scoring: conditional branches, boundary tie-breakers, stepwise workflows, and numeric thresholds.); \textcolor{rpTypeEvidence}{\textbf{Evidence Handling}} (How evidence is validated and counted: specific-example requirements, repetition/non-double-count rules, and cap rules for weak evidence.); \textcolor{rpTypeWriting}{\textbf{Writing Quality}} (Language-quality criteria affecting score bands: organization/coherence/transition quality and grammar/mechanics severity.). Matched pattern categories: Conditional Gating (n=2); Stepwise Rating Workflow (n=1); Specific Evidence Requirement (n=9); Off-Topic / Summary Cap (n=2); Organization / Coherence Signal (n=4); Grammar / Mechanics Signal (n=5); Repetition Non-Count Rule (n=1); Quantitative Threshold (n=1).}
\label{fig:rubric_pattern_ets3_google_gemini_3_flash_preview_base_simplest_True_train100_iteration5_top3_bs4_8_12_mc4}
\end{figure*}
