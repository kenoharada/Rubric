\colorlet{rpTypeRule}{red!80!black}
\colorlet{rpTypeEvidence}{blue!80!black}
\colorlet{rpTypeWriting}{teal!80!black}
\begin{figure*}[t]
\centering
\begin{tcolorbox}[colback=white,colframe=black!25,title=Pattern Legend,fonttitle=\bfseries\small,fontupper=\scriptsize,boxsep=1pt,left=2pt,right=2pt,top=2pt,bottom=2pt]
\textcolor{rpTypeRule}{\textbf{Rule Structure}} (if/threshold/stepwise guidance) \quad \textcolor{rpTypeEvidence}{\textbf{Evidence Handling}} (examples, repetition, and caps) \quad \textcolor{rpTypeWriting}{\textbf{Writing Quality}} (organization and grammar/mechanics)
\end{tcolorbox}
\vspace{2mm}
\begin{minipage}[t]{0.485\textwidth}
\begin{tcolorbox}[colback=white,colframe=black!25,title=Initial Rubric,fonttitle=\bfseries\small,fontupper=\scriptsize,breakable]
\ttfamily
- is generally well organized and well developed, using appropriate and sufficient explanations, exemplifications, and/or details\par - displays unity, progression, and \textcolor{rpTypeWriting}{coherence}, though it may contain occasional redundancy, \textcolor{rpTypeEvidence}{digression}, or unclear connections\par - displays facility in the use of language, demonstrating syntactic variety and range of vocabulary, though it will probably have occasional noticeable minor errors in structure, word form, or use of idiomatic language that do not interfere with meaning\par ... [3 lines omitted] ...\par - addresses the topic and task using somewhat developed explanations, exemplifications, and/or details\par - displays unity, progression, and \textcolor{rpTypeWriting}{coherence}, though connection of ideas may be occasionally obscured\par - may demonstrate inconsistent facility in sentence formation and word choice that may result in lack of clarity and occasionally obscure meaning\par ... [4 lines omitted] ...\par - limited development in response to the topic and task\par - inadequate \textcolor{rpTypeWriting}{organization} or connection of ideas\par - inappropriate or insufficient exemplifications, explanations, or details to support or illustrate generalizations in response to the task
\end{tcolorbox}
\end{minipage}
\hfill
\begin{minipage}[t]{0.485\textwidth}
\begin{tcolorbox}[colback=white,colframe=black!25,title=Optimized Rubric,fonttitle=\bfseries\small,fontupper=\scriptsize,breakable]
\ttfamily
An essay at this level largely accomplishes all of the following:\par - addresses the topic and task effectively, providing specific reasons and well-developed, concrete examples (\textcolor{rpTypeEvidence}{e.g.}, citing specific historical figures, detailed personal \textcolor{rpTypeEvidence}{anecdote}s, or specific societal trends) that directly support the thesis; the development shows an ability to handle complex ideas and nuanced perspectives\par - is well-organized and displays a clear, logical progression of ideas; while the structure may rely on standard \textcolor{rpTypeWriting}{transition} words (\textcolor{rpTypeEvidence}{e.g.}, "\textcolor{rpTypeRule}{First}," "Secondly"), they are used effectively to guide a coherent argument rather than appearing purely mechanical\par - displays a high facility in the use of language and a range of syntactic variety; although it may contain frequent minor errors in \textcolor{rpTypeWriting}{grammar}, \textcolor{rpTypeWriting}{spelling}, or \textcolor{rpTypeWriting}{mechanics} (\textcolor{rpTypeEvidence}{e.g.}, "doesnot," "yound," "popluare," "well-arouned"), these errors do not obscure meaning or significantly interfere with the strength and flow of the argument\par - demonstrates a level of detail and depth of thought that goes beyond simple observations; the writing feels substantial and the author's control over the argument remains strong despite linguistic imperfections\par ... [2 lines omitted] ...\par An essay at this level is marked by one or more of the following:\par - addresses the topic and task but development is limited or uneven; the essay may rely on generalities, \textcolor{rpTypeEvidence}{repetition} of ideas, or explanations that lack grounded, specific detail (\textcolor{rpTypeEvidence}{e.g.}, repeating that a situation is "boring" or "hard" without further elaboration)\par - displays unity and \textcolor{rpTypeWriting}{coherence}, but the \textcolor{rpTypeWriting}{organization} may feel overly mechanical or the connection of ideas may be occasionally obscured by linguistic limitations; the essay may resemble a list of points rather than a progression of an argument\par - demonstrates grammatical, \textcolor{rpTypeWriting}{spelling}, or word-choice errors that are persistent enough to distract the reader or suggest a lack of range (\textcolor{rpTypeEvidence}{e.g.}, "earn money hardly," "is important trying," "specialising on"); while the general meaning is discernible, the phrasing often feels unnatural or lacks the complexity/nuance of a higher-level response\par - the essay may end abruptly or contain sections where the author's intent is clear but the execution is significantly hindered by a limited vocabulary or repetitive sentence structures\par ... [1 lines omitted] ...\par \#\# Score 1\par An essay at this level reveals a significant lack of competence \textcolor{rpTypeEvidence}{evidence}d by one or more of the following:\par - limited or very poor development in response to the topic; the essay may be significantly short (under \textcolor{rpTypeRule}{200 words}), fail to address key parts of the prompt, or rely almost entirely on vague, hypothetical "\textcolor{rpTypeRule}{if}/then" scenarios and generalities ("\textcolor{rpTypeRule}{if} we help then world is good") without any concrete \textcolor{rpTypeEvidence}{evidence}\par - inadequate \textcolor{rpTypeWriting}{organization} or connection of ideas, where the reader must frequently pause to reconstruct the author's logic or piece together the relationship between sentences\par - inappropriate, insufficient, or \textcolor{rpTypeEvidence}{irrelevant} examples that fail to support the generalizations made, often resulting in a response that feels "\textcolor{rpTypeEvidence}{off-topic}," purely philosophical, or superficial\par - a pervasive accumulation of serious errors in sentence structure, usage, and \textcolor{rpTypeWriting}{spelling} (\textcolor{rpTypeEvidence}{e.g.}, "cheeting," "funny and rock question," "past the goal") that frequently obscures meaning or results in a lack of clarity throughout the majority of the essay
\end{tcolorbox}
\end{minipage}
\caption{Pattern-highlighted rubric comparison (ets3, google\_gemini-3-flash-preview, base\_simplest\_True\_train100\_iteration5\_top3\_bs4-8-12\_mc4). Matched spans are color-coded by regex pattern. Color types: \textcolor{rpTypeRule}{\textbf{Rule Structure}} (if/threshold/stepwise guidance); \textcolor{rpTypeEvidence}{\textbf{Evidence Handling}} (examples, repetition, and caps); \textcolor{rpTypeWriting}{\textbf{Writing Quality}} (organization and grammar/mechanics).}
\label{fig:rubric_pattern_ets3_google_gemini_3_flash_preview_base_simplest_True_train100_iteration5_top3_bs4_8_12_mc4}
\end{figure*}
