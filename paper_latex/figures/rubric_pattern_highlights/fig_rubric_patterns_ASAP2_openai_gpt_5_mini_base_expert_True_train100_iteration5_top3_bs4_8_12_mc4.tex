\colorlet{rpTypeRule}{red!80!black}
\colorlet{rpTypeEvidence}{blue!80!black}
\colorlet{rpTypeWriting}{teal!80!black}
\begin{figure*}[t]
\centering
\begin{tcolorbox}[colback=white,colframe=black!25,title=Pattern Legend,fonttitle=\bfseries\small,fontupper=\scriptsize,boxsep=1pt,left=2pt,right=2pt,top=2pt,bottom=2pt]
\textcolor{rpTypeRule}{\textbf{Rule Structure}} (if/threshold/stepwise guidance) \quad \textcolor{rpTypeEvidence}{\textbf{Evidence Handling}} (examples, repetition, and caps) \quad \textcolor{rpTypeWriting}{\textbf{Writing Quality}} (organization and grammar/mechanics)
\end{tcolorbox}
\vspace{2mm}
\begin{minipage}[t]{0.485\textwidth}
\begin{tcolorbox}[colback=white,colframe=black!25,title=Initial Rubric,fonttitle=\bfseries\small,fontupper=\scriptsize,breakable]
\ttfamily
After reading each essay and completing the analytical rating form, assign a holistic score based on the rubric below. For the following evaluations you will need to use a grading scale between 1 (minimum) and 6 (maximum). As with the analytical rating form, the distance between each grade (\textcolor{rpTypeEvidence}{e.g.}, 1-2, 3-4, 4-5) should be considered equal.\par \par SCORE OF 6: An essay in this category demonstrates clear and consistent mastery, although it may have a few minor errors. A typical essay effectively and insightfully develops a point of view on the issue and demonstrates outstanding critical thinking; the essay uses clearly appropriate examples, reasons, and other \textcolor{rpTypeEvidence}{evidence} taken from the source text(s) to support its position; the essay is well organized and clearly focused, demonstrating clear \textcolor{rpTypeWriting}{coherence} and smooth progression of ideas; the essay exhibits skillful use of language, using a varied, accurate, and apt vocabulary and demonstrates meaningful variety in sentence structure; the essay is free of most errors in \textcolor{rpTypeWriting}{grammar}, usage, and \textcolor{rpTypeWriting}{mechanics}.\par \par SCORE OF 5: An essay in this category demonstrates reasonably consistent mastery, although it will have occasional errors or lapses in quality. A typical essay effectively develops a point of view on the issue and demonstrates strong critical thinking; the essay generally using appropriate examples, reasons, and other \textcolor{rpTypeEvidence}{evidence} taken from the source text(s) to support its position; the essay is well organized and focused, demonstrating \textcolor{rpTypeWriting}{coherence} and progression of ideas; the essay exhibits facility in the use of language, using appropriate vocabulary demonstrates variety in sentence structure; the essay is generally free of most errors in \textcolor{rpTypeWriting}{grammar}, usage, and \textcolor{rpTypeWriting}{mechanics}.\par \par SCORE OF 4: An essay in this category demonstrates adequate mastery, although it will have lapses in quality. A typical essay develops a point of view on the issue and demonstrates competent critical thinking; the essay using adequate examples, reasons, and other \textcolor{rpTypeEvidence}{evidence} taken from the source text(s) to support its position; the essay is generally organized and focused, demonstrating some \textcolor{rpTypeWriting}{coherence} and progression of ideas exhibits adequate; the essay may demonstrate inconsistent facility in the use of language, using generally appropriate vocabulary demonstrates some variety in sentence structure; the essay may have some errors in \textcolor{rpTypeWriting}{grammar}, usage, and \textcolor{rpTypeWriting}{mechanics}.\par \par SCORE OF 3: An essay in this category demonstrates developing mastery, and is marked by ONE OR MORE of the following weaknesses: develops a point of view on the issue, demonstrating some critical thinking, but may do so inconsistently or use inadequate examples, reasons, or other \textcolor{rpTypeEvidence}{evidence} taken from the source texts to support its position; the essay is limited in its \textcolor{rpTypeWriting}{organization} or focus, or may demonstrate some lapses in \textcolor{rpTypeWriting}{coherence} or progression of ideas displays; the essay may demonstrate facility in the use of language, but sometimes uses weak vocabulary or inappropriate word choice and/or lacks variety or demonstrates problems in sentence structure; the essay may contain an accumulation of errors in \textcolor{rpTypeWriting}{grammar}, usage, and \textcolor{rpTypeWriting}{mechanics}.\par \par SCORE OF 2: An essay in this category demonstrates little mastery, and is flawed by ONE OR MORE of the following weaknesses: develops a point of view on the issue that is vague or seriously limited, and demonstrates weak critical thinking; the essay providesinappropriate or insufficient examples, reasons, or other \textcolor{rpTypeEvidence}{evidence} taken from the source text to support its position; the essay is poorly organized and/or focused, or demonstrates serious problems with \textcolor{rpTypeWriting}{coherence} or progression of ideas; the essay displays very little facility in the use of language, using very limited vocabulary or incorrect word choice and/or demonstrates frequent problems in sentence structure; the essay contains errors in \textcolor{rpTypeWriting}{grammar}, usage, and \textcolor{rpTypeWriting}{mechanics} so serious that meaning is somewhat obscured.\par \par SCORE OF 1: An essay in this category demonstrates very little or no mastery, and is severely flawed by ONE OR MORE of the following weaknesses: develops no viable point of view on the issue, or provides little or no \textcolor{rpTypeEvidence}{evidence} to support its position; the essay is disorganized or unfocused, resulting in a disjointed orincoherent essay; the essay displays fundamental errors in vocabulary and/or demonstrates severe flaws in sentence structure; the essay contains pervasive errors in \textcolor{rpTypeWriting}{grammar}, usage, or \textcolor{rpTypeWriting}{mechanics} that persistently interfere with meaning.
\end{tcolorbox}
\end{minipage}
\hfill
\begin{minipage}[t]{0.485\textwidth}
\begin{tcolorbox}[colback=white,colframe=black!25,title=Optimized Rubric,fonttitle=\bfseries\small,fontupper=\scriptsize,breakable]
\ttfamily
  - Over-penalizing essays with adequate development but surface errors (assistant gave too-low scores in some cases).\par   - Under-penalizing \textcolor{rpTypeEvidence}{summary-only} responses and pervasive mechanical errors (assistant sometimes gave too-high scores).\par   - Inconsistent application of the \textcolor{rpTypeEvidence}{Summary-only} \textcolor{rpTypeEvidence}{cap}, \textcolor{rpTypeEvidence}{Evidence}-count rule, and Mechanical-distortion penalty.\par - Make decision rules more prescriptive and algorithmic so raters apply caps, penalties, and aggregations consistently.\par ... [1 lines omitted] ...\par High-level principles (unchanged priorities, clarified weighting)\par 1. Prioritize Development (B) and \textcolor{rpTypeWriting}{Mechanics} (E) \textcolor{rpTypeRule}{first}: B determines maximum possible band (3 vs \textcolor{rpTypeRule}{4 vs 5}/6), then apply E penalties. A and C then refine placement; D refines ties within a band.\par 2. Use explicit, ordered scoring steps below rather than global judgment \textcolor{rpTypeRule}{first}.\par 3. Always record which caps/penalties were applied \textcolor{rpTypeRule}{when} scoring.\par \par Definitions and strict interpretations (to reduce ambiguity)\par - "\textcolor{rpTypeEvidence}{Summary-only}" (B = Weak/Insufficient): The response predominantly restates source points or lists details with no explanatory linkage to the claim. Indicators: repeated paraphrase phrases, no cause/effect/implication/interpretation, no explanation of why \textcolor{rpTypeEvidence}{evidence} supports the claim.\par - "Shallow Adequate" (B = Adequate but paraphrase-heavy): \textcolor{rpTypeRule}{At least} one attempt to link \textcolor{rpTypeEvidence}{evidence} to claim is present but explanation is superficial, repetitive, or circular.\par - "Strong development" (B = Strong): Minimum of two distinct, specific, text-based examples/reasons. Each example must be explicitly explained (not only quoted) and tied to the claim through interpretation (cause/effect, implication) or analysis. Single detailed example = potentially Strong only \textcolor{rpTypeRule}{if} explanation is sustained and clearly tied to claim; otherwise counts as Adequate.\par - \textcolor{rpTypeWriting}{Mechanics} severity (E):\par   - Minor: <= 3 isolated errors, meaning always clear.\par ... [3 lines omitted] ...\par Stepwise scoring algorithm (must be followed in order)\par \textcolor{rpTypeRule}{Step} 1 - Assess Development (B) and impose \textcolor{rpTypeEvidence}{Summary-only} caps immediately:\par   - \textcolor{rpTypeRule}{If} B = Weak/Insufficient due to \textcolor{rpTypeEvidence}{summary-only} (paraphrase/\textcolor{rpTypeEvidence}{repetition} with no explanation), set MAX\_POSSIBLE = 3 and continue to \textcolor{rpTypeRule}{Step} 2.\par   - Else \textcolor{rpTypeRule}{if} B = Adequate but paraphrase-heavy with only one shallow link to claim, set MAX\_POSSIBLE = 4 and continue to \textcolor{rpTypeRule}{Step} 2.\par   - Else (B = Strong or Adequate-with-convincing explanation), set MAX\_POSSIBLE = 6 and continue.\par \par \textcolor{rpTypeRule}{Step} 2 - Assess \textcolor{rpTypeWriting}{Mechanics} (E) and apply Mechanical-distortion penalty:\par   - \textcolor{rpTypeRule}{If} E = Severe and errors frequently obscure meaning: reduce MAX\_POSSIBLE by 1.\par   - \textcolor{rpTypeRule}{If} E = Severe and meaning is largely unintelligible: reduce MAX\_POSSIBLE by 2.\par   - \textcolor{rpTypeRule}{If} E = Noticeable: do not automatically reduce MAX\_POSSIBLE; note for \textcolor{rpTypeRule}{tie-break}ing and possible downward adjustment in \textcolor{rpTypeRule}{Step} 4.\par   - \textcolor{rpTypeRule}{If} E = Minor: no reduction.\par \par \textcolor{rpTypeRule}{Step} 3 - Assess Central Claim (A), \textcolor{rpTypeWriting}{Organization} (C), Language (D)\par   - Convert qualitative labels to points for aggregation purposes:\par ... [9 lines omitted] ...\par \par \textcolor{rpTypeRule}{Step} 4 - Enforce MAX\_POSSIBLE and apply \textcolor{rpTypeRule}{tie-break}ers / final adjustments\par   - Final\_score = min(candidate, MAX\_POSSIBLE).\par   - \textcolor{rpTypeRule}{Tie-break}ers (used only \textcolor{rpTypeRule}{if} candidate <= MAX\_POSSIBLE and multiple close alternatives):\par     1. Favor essays with more and higher-quality explained \textcolor{rpTypeEvidence}{evidence} (B) - a Strong B pushes toward higher score within the band.\par     2. \textcolor{rpTypeRule}{If} E = Noticeable and candidate is 5, downgrade to 4 unless B = Strong with multiple explained examples and C = Strong.\par     3. \textcolor{rpTypeRule}{If} E = Severe but MAX\_POSSIBLE reduction already applied in \textcolor{rpTypeRule}{Step} 2, allow no further upward adjustment.\par     4. \textcolor{rpTypeRule}{If} A and C are Strong but B = Adequate (with one convincing explained piece), and E = Minor, candidate can be 5 (only \textcolor{rpTypeRule}{if} MAX\_POSSIBLE >= 5).\par   - Additional automatic adjustments:\par     - \textcolor{rpTypeRule}{If} candidate = 5 but B = Adequate with only one shallow link, force Final\_score <= 4.\par     - \textcolor{rpTypeRule}{If} candidate = 4 but B = Adequate mainly paraphrase and (D = Weak or E = Noticeable/Severe), consider Final\_score = 3.\par \par Minimum-development caps and mechanical-penalty rules (refined \& stricter)\par 1. \textcolor{rpTypeEvidence}{Summary-only} \textcolor{rpTypeEvidence}{cap} (refined):\par    - B = Weak due to \textcolor{rpTypeEvidence}{summary-only} -> MAX\_POSSIBLE = 3 always (no exception).\par    - B = Adequate but paraphrase-heavy with only one shallow link -> MAX\_POSSIBLE = 4.\par ... [1 lines omitted] ...\par \par 2. \textcolor{rpTypeEvidence}{Evidence}-count requirement for 5-6 (clarified):\par    - To place candidate >= 5, require:\par      - A = Strong; AND\par      - Either B = Strong (>=2 distinct explained pieces) OR (B = Adequate with \textcolor{rpTypeRule}{at least} one sustained, convincing, well-explained example that clearly ties to claim and C = Strong). Single brief example without sustained explanation cannot support 5.\par    - For 6 specifically require: A = Strong; B = Strong; C = Strong; D = Strong; E = Minor.\par \par 3. Mechanical-distortion penalty (refined \textcolor{rpTypeRule}{threshold}s):\par    - E = Severe -> reduce MAX\_POSSIBLE by 1 point as default; \textcolor{rpTypeRule}{if} meaning largely unintelligible reduce by 2 points.\par    - E = Noticeable -> no automatic reduction from MAX\_POSSIBLE, but treat Noticeable as negative \textcolor{rpTypeRule}{tie-break}er preventing upward movement: do not promote an essay above 4 \textcolor{rpTypeRule}{if} E = Noticeable unless B = Strong and multiple content strengths exist.\par \par 4. Development vs. \textcolor{rpTypeWriting}{Mechanics} trade-off clarified numerically:\par    - \textcolor{rpTypeRule}{If} B = Strong and C = Strong, allow up to +1 above what E = Noticeable would otherwise permit (i.e., \textcolor{rpTypeRule}{if} candidate = 5 but E = Noticeable, allow 5 only \textcolor{rpTypeRule}{if} both B and C are Strong).\par    - Conversely, \textcolor{rpTypeRule}{if} B = Adequate or Weak, strong D/E cannot move essay upward past MAX\_POSSIBLE.\par \par ... [1 lines omitted] ...\par - Convert dimension labels to numeric points (Strong=2, Adequate=1, Weak=0) for A, B, C, D. (E handled separately).\par - Use \textcolor{rpTypeRule}{Step} 3 mapping to candidate score, then enforce MAX\_POSSIBLE from \textcolor{rpTypeRule}{Step} 1 and penalties from \textcolor{rpTypeRule}{Step} 2.\par - Common aggregate patterns and mandatory outcomes:\par   - 5 dimensions Strong -> 6 (\textcolor{rpTypeRule}{if} E = Minor). \textcolor{rpTypeRule}{If} E = Noticeable, still can be 5; \textcolor{rpTypeRule}{if} E = Severe, apply penalty and re-evaluate.\par   - 4 Strong + 1 Adequate -> 5 \textcolor{rpTypeRule}{if} Adequate is not B; \textcolor{rpTypeRule}{if} the Adequate is B and it is paraphrase-heavy, \textcolor{rpTypeEvidence}{cap} at 4.\par   - 3 Strong, 1 Adequate, 1 Weak -> normally 4. Upgrade to 5 only \textcolor{rpTypeRule}{if} B = Strong or B = Adequate WITH one strong explained example AND E = Minor.\par   - 2 Strong, rest Adequate -> normally 3. Upgrade to 4 only \textcolor{rpTypeRule}{if} E = Minor and B shows \textcolor{rpTypeRule}{at least} one clear, well-explained example (not mere paraphrase).\par   - 1 Strong or 0 Strong -> 1 or 2 depending on E severity and whether any coherent claim/\textcolor{rpTypeEvidence}{evidence} exists.\par \par Common scoring pitfalls and explicit corrective guidance (targeted to calibration errors)\par - Pitfall A: Scoring a paraphrase-heavy essay as 4 \textcolor{rpTypeRule}{when} development is shallow and \textcolor{rpTypeWriting}{mechanics} are Noticeable.\par   - Fix: \textcolor{rpTypeRule}{If} B = Adequate but primarily paraphrase and E = Noticeable, default to 3 unless there is \textcolor{rpTypeRule}{at least} one convincingly explained example (sustained explanation) -> then 4.\par - Pitfall B: Scoring a fragmented essay with Severe \textcolor{rpTypeWriting}{mechanics} too highly (assistant sometimes was too high).\par   - Fix: \textcolor{rpTypeRule}{If} E = Severe, reduce MAX\_POSSIBLE by \textcolor{rpTypeRule}{at least} 1; \textcolor{rpTypeRule}{if} E = Severe + B = Weak or A absent, final should be 1 or 2.\par - Pitfall C: Failing to \textcolor{rpTypeEvidence}{cap} at 3 for clear \textcolor{rpTypeEvidence}{summary-only} essays (assistant sometimes gave 4).\par   - Fix: Apply \textcolor{rpTypeEvidence}{Summary-only} \textcolor{rpTypeEvidence}{cap} in \textcolor{rpTypeRule}{Step} 1 strictly: B = Weak (\textcolor{rpTypeEvidence}{summary-only}) -> MAX\_POSSIBLE = 3 regardless of other apparent strengths.\par - Pitfall D: Allowing single brief fact to justify 5 or 6.\par   - Fix: Enforce \textcolor{rpTypeEvidence}{Evidence}-count: \textcolor{rpTypeRule}{at least} two distinct explained evidentiary elements (or one sustained, convincing analysis + strong \textcolor{rpTypeWriting}{organization}) for 5; two for 6 with other dimensions Strong.\par - Pitfall E: Letting good \textcolor{rpTypeWriting}{mechanics} rescue very shallow content.\par   - Fix: B and A must be \textcolor{rpTypeRule}{at least} Adequate for score >= 4. Strong \textcolor{rpTypeWriting}{mechanics} and language without development should not push above MAX\_POSSIBLE from \textcolor{rpTypeRule}{Step} 1.\par \par ... [1 lines omitted] ...\par - Every rating must include a one-line summary listing labels A-E used and any caps/penalties applied (examples required):\par   - Example: "A=Adequate; B=Adequate (paraphrase-heavy); C=Adequate; D=Weak; E=Noticeable; Capped at 3 due to B=\textcolor{rpTypeEvidence}{summary-only}" (adapt as appropriate).\par - \textcolor{rpTypeRule}{If} any automatic \textcolor{rpTypeEvidence}{cap} or penalty was applied (\textcolor{rpTypeEvidence}{Summary-only} \textcolor{rpTypeEvidence}{cap}, Mechanical-distortion penalty), state it explicitly.\par \par Quick decision flow (for rapid consistency)\par 1. Is development \textcolor{rpTypeEvidence}{summary-only}? Yes -> \textcolor{rpTypeEvidence}{cap} = 3. No -> continue.\par 2. Is B Adequate but paraphrase-heavy with single shallow link? Yes -> \textcolor{rpTypeEvidence}{cap} = 4. No -> no \textcolor{rpTypeEvidence}{cap}.\par 3. Is E Severe? Yes -> reduce \textcolor{rpTypeEvidence}{cap} by 1 (or 2 \textcolor{rpTypeRule}{if} largely unintelligible).\par 4. Aggregate A/B/C/D numeric points to get preliminary candidate score.\par 5. Final\_score = min(candidate, \textcolor{rpTypeEvidence}{cap}); apply \textcolor{rpTypeRule}{tie-break}ers per \textcolor{rpTypeRule}{Step} 4.\par \par Calibration examples (interpretable rules derived from examples)\par - Essays that summarize source with no analysis -> final = 3 (apply \textcolor{rpTypeEvidence}{summary-only} \textcolor{rpTypeEvidence}{cap}).\par - Essays with multiple paraphrased details and noticeable \textcolor{rpTypeWriting}{mechanics} but \textcolor{rpTypeRule}{at least} one clear linkage -> usually 4.\par - Essays with many facts but Severe mechanical distortion (obscuring meaning) and \textcolor{rpTypeEvidence}{summary-only} development -> 1-2 depending on unintelligibility.\par - Essays with clear claim, multiple explained references, good \textcolor{rpTypeWriting}{organization}, and Minor \textcolor{rpTypeWriting}{mechanics} -> 5 or 6 depending on number/quality of explained \textcolor{rpTypeEvidence}{evidence} and strength of language.\par \par Final notes to raters (short and mandatory)\par - Always follow the Stepwise algorithm; do not skip the \textcolor{rpTypeEvidence}{Summary-only} \textcolor{rpTypeEvidence}{cap} in \textcolor{rpTypeRule}{Step} 1.\par - B and E decide bands \textcolor{rpTypeRule}{first}; A/C/D refine position within band.\par - Record labels for A-E and any caps/penalties with every score.\par - Use the \textcolor{rpTypeEvidence}{Evidence}-count requirement strictly for 5-6 placements.\par 
\end{tcolorbox}
\end{minipage}
\caption{Pattern-highlighted rubric comparison (ASAP2, openai\_gpt-5-mini, base\_expert\_True\_train100\_iteration5\_top3\_bs4-8-12\_mc4). Matched spans are color-coded by regex pattern. Color types: \textcolor{rpTypeRule}{\textbf{Rule Structure}} (if/threshold/stepwise guidance); \textcolor{rpTypeEvidence}{\textbf{Evidence Handling}} (examples, repetition, and caps); \textcolor{rpTypeWriting}{\textbf{Writing Quality}} (organization and grammar/mechanics).}
\label{fig:rubric_pattern_ASAP2_openai_gpt_5_mini_base_expert_True_train100_iteration5_top3_bs4_8_12_mc4}
\end{figure*}
