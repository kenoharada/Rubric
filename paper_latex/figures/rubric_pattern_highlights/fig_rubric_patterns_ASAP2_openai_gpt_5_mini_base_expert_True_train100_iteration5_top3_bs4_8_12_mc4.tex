\colorlet{rpTypeRule}{red!80!black}
\colorlet{rpTypeEvidence}{blue!80!black}
\colorlet{rpTypeWriting}{teal!80!black}
\begin{figure*}[t]
\centering
\begin{tcolorbox}[colback=white,colframe=black!25,title=Pattern Type Guide,fonttitle=\bfseries\small,fontupper=\scriptsize,boxsep=1pt,left=2pt,right=2pt,top=2pt,bottom=2pt]
\textcolor{rpTypeRule}{\textbf{Rule Structure}}: Explicit decision logic for scoring: conditional branches, boundary tie-breakers, stepwise workflows, and numeric thresholds.\par \textcolor{rpTypeEvidence}{\textbf{Evidence Handling}}: How evidence is validated and counted: specific-example requirements, repetition/non-double-count rules, and cap rules for weak evidence.\par \textcolor{rpTypeWriting}{\textbf{Writing Quality}}: Language-quality criteria affecting score bands: organization/coherence/transition quality and grammar/mechanics severity.
\end{tcolorbox}
\vspace{1mm}
\begin{tcolorbox}[colback=white,colframe=black!25,title=Detailed Pattern Notes,fonttitle=\bfseries\small,fontupper=\scriptsize,boxsep=1pt,left=2pt,right=2pt,top=2pt,bottom=2pt]
\textcolor{rpTypeRule}{\textbf{Rule Structure}}:\par \quad \textcolor{rpTypeRule}{\textbf{Conditional Gating}} [n=34] Captures explicit condition-based rules that switch decisions only when a stated condition is met. Typical cues: if, when.\par \quad \textcolor{rpTypeRule}{\textbf{Boundary / Tie-Break Guidance}} [n=7] Marks criteria used to resolve borderline cases between adjacent score bands (e.g., 4 vs 5). Typical cues: tie-break, borderline, boundary, threshold, 4 vs 5.\par \quad \textcolor{rpTypeRule}{\textbf{Stepwise Rating Workflow}} [n=18] Detects ordered procedures and checklists that standardize how raters walk through scoring decisions. Typical cues: step, checklist, workflow, procedure, first/second/third.\par \quad \textcolor{rpTypeRule}{\textbf{Quantitative Threshold}} [n=8] Marks numeric cutoffs used for consistent decisions (minimum/maximum counts, percentages, explicit count thresholds). Typical cues: at least, at most, <=, >=, \%, N reasons/examples/sentences/words.\par \textcolor{rpTypeEvidence}{\textbf{Evidence Handling}}:\par \quad \textcolor{rpTypeEvidence}{\textbf{Specific Evidence Requirement}} [n=9] Highlights demands for concrete examples and explicit evidence links instead of generic assertions. Typical cues: for example, e.g., specific example, illustration, anecdote, evidence.\par \quad \textcolor{rpTypeEvidence}{\textbf{Off-Topic / Summary Cap}} [n=30] Identifies cap rules that restrict scores when responses are off-topic, irrelevant, or dominated by summary-only content. Typical cues: off-topic, irrelevant, digression, summary-only, cap.\par \quad \textcolor{rpTypeEvidence}{\textbf{Repetition Non-Count Rule}} [n=1] Captures rules that treat repetition/restatement as non-distinct support and prevent double-counting. Typical cues: repetition, restatement, double-count, do not double-count.\par \textcolor{rpTypeWriting}{\textbf{Writing Quality}}:\par \quad \textcolor{rpTypeWriting}{\textbf{Organization / Coherence Signal}} [n=3] Detects explicit references to discourse structure and logical flow as scoring criteria. Typical cues: organization, coherence, logical flow, transition.\par \quad \textcolor{rpTypeWriting}{\textbf{Grammar / Mechanics Signal}} [n=10] Detects references to language-form quality, especially grammar, spelling, punctuation, and mechanics. Typical cues: grammar, mechanics, spelling, punctuation.
\end{tcolorbox}
\vspace{1mm}
\begin{tcolorbox}[colback=white,colframe=black!25,title=Optimized Rubric (Pattern-Highlighted),fonttitle=\bfseries\small,fontupper=\scriptsize]
\ttfamily
  - Under-penalizing \textcolor{rpTypeEvidence}{\textbf{summary-only}} responses and pervasive mechanical errors (assistant sometimes gave too-high scores).\par   - Inconsistent application of the \textcolor{rpTypeEvidence}{\textbf{Summary-only}} \textcolor{rpTypeEvidence}{\textbf{cap}}, \textcolor{rpTypeEvidence}{\textbf{Evidence}}-count rule, and Mechanical-distortion penalty.\par ... [3 lines omitted] ...\par 1. Prioritize Development (B) and \textcolor{rpTypeWriting}{\textbf{Mechanics}} (E) \textcolor{rpTypeRule}{\textbf{first}}: B determines maximum possible band (3 vs \textcolor{rpTypeRule}{\textbf{4 vs 5}}/6), then apply E penalties. A and C then refine placement; D refines ties within a band.\par 2. Use explicit, ordered scoring steps below rather than global judgment \textcolor{rpTypeRule}{\textbf{first}}.\par 3. Always record which caps/penalties were applied \textcolor{rpTypeRule}{\textbf{when}} scoring.\par ... [2 lines omitted] ...\par - "\textcolor{rpTypeEvidence}{\textbf{Summary-only}}" (B = Weak/Insufficient): The response predominantly restates source points or lists details with no explanatory linkage to the claim. Indicators: repeated paraphrase phrases, no cause/effect/implication/interpretation, no explanation of why \textcolor{rpTypeEvidence}{\textbf{evidence}} supports the claim.\par - "Shallow Adequate" (B = Adequate but paraphrase-heavy): \textcolor{rpTypeRule}{\textbf{At least}} one attempt to link \textcolor{rpTypeEvidence}{\textbf{evidence}} to claim is present but explanation is superficial, repetitive, or circular.\par - "Strong development" (B = Strong): Minimum of two distinct, specific, text-based examples/reasons. Each example must be explicitly explained (not only quoted) and tied to the claim through interpretation (cause/effect, implication) or analysis. Single detailed example = potentially Strong only \textcolor{rpTypeRule}{\textbf{if}} explanation is sustained and clearly tied to claim; otherwise counts as Adequate.\par - \textcolor{rpTypeWriting}{\textbf{Mechanics}} severity (E):\par ... [5 lines omitted] ...\par \textcolor{rpTypeRule}{\textbf{Step}} 1 - Assess Development (B) and impose \textcolor{rpTypeEvidence}{\textbf{Summary-only}} caps immediately:\par   - \textcolor{rpTypeRule}{\textbf{If}} B = Weak/Insufficient due to \textcolor{rpTypeEvidence}{\textbf{summary-only}} (paraphrase/\textcolor{rpTypeEvidence}{\textbf{repetition}} with no explanation), set MAX\_POSSIBLE = 3 and continue to \textcolor{rpTypeRule}{\textbf{Step}} 2.\par   - Else \textcolor{rpTypeRule}{\textbf{if}} B = Adequate but paraphrase-heavy with only one shallow link to claim, set MAX\_POSSIBLE = 4 and continue to \textcolor{rpTypeRule}{\textbf{Step}} 2.\par ... [2 lines omitted] ...\par \textcolor{rpTypeRule}{\textbf{Step}} 2 - Assess \textcolor{rpTypeWriting}{\textbf{Mechanics}} (E) and apply Mechanical-distortion penalty:\par   - \textcolor{rpTypeRule}{\textbf{If}} E = Severe and errors frequently obscure meaning: reduce MAX\_POSSIBLE by 1.\par   - \textcolor{rpTypeRule}{\textbf{If}} E = Severe and meaning is largely unintelligible: reduce MAX\_POSSIBLE by 2.\par   - \textcolor{rpTypeRule}{\textbf{If}} E = Noticeable: do not automatically reduce MAX\_POSSIBLE; note for \textcolor{rpTypeRule}{\textbf{tie-break}}ing and possible downward adjustment in \textcolor{rpTypeRule}{\textbf{Step}} 4.\par   - \textcolor{rpTypeRule}{\textbf{If}} E = Minor: no reduction.\par ... [1 lines omitted] ...\par \textcolor{rpTypeRule}{\textbf{Step}} 3 - Assess Central Claim (A), \textcolor{rpTypeWriting}{\textbf{Organization}} (C), Language (D)\par ... [11 lines omitted] ...\par \textcolor{rpTypeRule}{\textbf{Step}} 4 - Enforce MAX\_POSSIBLE and apply \textcolor{rpTypeRule}{\textbf{tie-break}}ers / final adjustments\par ... [1 lines omitted] ...\par   - \textcolor{rpTypeRule}{\textbf{Tie-break}}ers (used only \textcolor{rpTypeRule}{\textbf{if}} candidate <= MAX\_POSSIBLE and multiple close alternatives):\par     1. Favor essays with more and higher-quality explained \textcolor{rpTypeEvidence}{\textbf{evidence}} (B) - a Strong B pushes toward higher score within the band.\par     2. \textcolor{rpTypeRule}{\textbf{If}} E = Noticeable and candidate is 5, downgrade to 4 unless B = Strong with multiple explained examples and C = Strong.\par     3. \textcolor{rpTypeRule}{\textbf{If}} E = Severe but MAX\_POSSIBLE reduction already applied in \textcolor{rpTypeRule}{\textbf{Step}} 2, allow no further upward adjustment.\par     4. \textcolor{rpTypeRule}{\textbf{If}} A and C are Strong but B = Adequate (with one convincing explained piece), and E = Minor, candidate can be 5 (only \textcolor{rpTypeRule}{\textbf{if}} MAX\_POSSIBLE >= 5).\par ... [1 lines omitted] ...\par     - \textcolor{rpTypeRule}{\textbf{If}} candidate = 5 but B = Adequate with only one shallow link, force Final\_score <= 4.\par     - \textcolor{rpTypeRule}{\textbf{If}} candidate = 4 but B = Adequate mainly paraphrase and (D = Weak or E = Noticeable/Severe), consider Final\_score = 3.\par ... [2 lines omitted] ...\par 1. \textcolor{rpTypeEvidence}{\textbf{Summary-only}} \textcolor{rpTypeEvidence}{\textbf{cap}} (refined):\par    - B = Weak due to \textcolor{rpTypeEvidence}{\textbf{summary-only}} -> MAX\_POSSIBLE = 3 always (no exception).\par ... [3 lines omitted] ...\par 2. \textcolor{rpTypeEvidence}{\textbf{Evidence}}-count requirement for 5-6 (clarified):\par ... [2 lines omitted] ...\par      - Either B = Strong (>=2 distinct explained pieces) OR (B = Adequate with \textcolor{rpTypeRule}{\textbf{at least}} one sustained, convincing, well-explained example that clearly ties to claim and C = Strong). Single brief example without sustained explanation cannot support 5.\par ... [2 lines omitted] ...\par 3. Mechanical-distortion penalty (refined \textcolor{rpTypeRule}{\textbf{threshold}}s):\par    - E = Severe -> reduce MAX\_POSSIBLE by 1 point as default; \textcolor{rpTypeRule}{\textbf{if}} meaning largely unintelligible reduce by 2 points.\par    - E = Noticeable -> no automatic reduction from MAX\_POSSIBLE, but treat Noticeable as negative \textcolor{rpTypeRule}{\textbf{tie-break}}er preventing upward movement: do not promote an essay above 4 \textcolor{rpTypeRule}{\textbf{if}} E = Noticeable unless B = Strong and multiple content strengths exist.\par ... [1 lines omitted] ...\par 4. Development vs. \textcolor{rpTypeWriting}{\textbf{Mechanics}} trade-off clarified numerically:\par    - \textcolor{rpTypeRule}{\textbf{If}} B = Strong and C = Strong, allow up to +1 above what E = Noticeable would otherwise permit (i.e., \textcolor{rpTypeRule}{\textbf{if}} candidate = 5 but E = Noticeable, allow 5 only \textcolor{rpTypeRule}{\textbf{if}} both B and C are Strong).\par    - Conversely, \textcolor{rpTypeRule}{\textbf{if}} B = Adequate or Weak, strong D/E cannot move essay upward past MAX\_POSSIBLE.\par ... [3 lines omitted] ...\par - Use \textcolor{rpTypeRule}{\textbf{Step}} 3 mapping to candidate score, then enforce MAX\_POSSIBLE from \textcolor{rpTypeRule}{\textbf{Step}} 1 and penalties from \textcolor{rpTypeRule}{\textbf{Step}} 2.\par ... [1 lines omitted] ...\par   - 5 dimensions Strong -> 6 (\textcolor{rpTypeRule}{\textbf{if}} E = Minor). \textcolor{rpTypeRule}{\textbf{If}} E = Noticeable, still can be 5; \textcolor{rpTypeRule}{\textbf{if}} E = Severe, apply penalty and re-evaluate.\par   - 4 Strong + 1 Adequate -> 5 \textcolor{rpTypeRule}{\textbf{if}} Adequate is not B; \textcolor{rpTypeRule}{\textbf{if}} the Adequate is B and it is paraphrase-heavy, \textcolor{rpTypeEvidence}{\textbf{cap}} at 4.\par   - 3 Strong, 1 Adequate, 1 Weak -> normally 4. Upgrade to 5 only \textcolor{rpTypeRule}{\textbf{if}} B = Strong or B = Adequate WITH one strong explained example AND E = Minor.\par   - 2 Strong, rest Adequate -> normally 3. Upgrade to 4 only \textcolor{rpTypeRule}{\textbf{if}} E = Minor and B shows \textcolor{rpTypeRule}{\textbf{at least}} one clear, well-explained example (not mere paraphrase).\par   - 1 Strong or 0 Strong -> 1 or 2 depending on E severity and whether any coherent claim/\textcolor{rpTypeEvidence}{\textbf{evidence}} exists.\par ... [2 lines omitted] ...\par - Pitfall A: Scoring a paraphrase-heavy essay as 4 \textcolor{rpTypeRule}{\textbf{when}} development is shallow and \textcolor{rpTypeWriting}{\textbf{mechanics}} are Noticeable.\par   - Fix: \textcolor{rpTypeRule}{\textbf{If}} B = Adequate but primarily paraphrase and E = Noticeable, default to 3 unless there is \textcolor{rpTypeRule}{\textbf{at least}} one convincingly explained example (sustained explanation) -> then 4.\par - Pitfall B: Scoring a fragmented essay with Severe \textcolor{rpTypeWriting}{\textbf{mechanics}} too highly (assistant sometimes was too high).\par   - Fix: \textcolor{rpTypeRule}{\textbf{If}} E = Severe, reduce MAX\_POSSIBLE by \textcolor{rpTypeRule}{\textbf{at least}} 1; \textcolor{rpTypeRule}{\textbf{if}} E = Severe + B = Weak or A absent, final should be 1 or 2.\par - Pitfall C: Failing to \textcolor{rpTypeEvidence}{\textbf{cap}} at 3 for clear \textcolor{rpTypeEvidence}{\textbf{summary-only}} essays (assistant sometimes gave 4).\par   - Fix: Apply \textcolor{rpTypeEvidence}{\textbf{Summary-only}} \textcolor{rpTypeEvidence}{\textbf{cap}} in \textcolor{rpTypeRule}{\textbf{Step}} 1 strictly: B = Weak (\textcolor{rpTypeEvidence}{\textbf{summary-only}}) -> MAX\_POSSIBLE = 3 regardless of other apparent strengths.\par ... [1 lines omitted] ...\par   - Fix: Enforce \textcolor{rpTypeEvidence}{\textbf{Evidence}}-count: \textcolor{rpTypeRule}{\textbf{at least}} two distinct explained evidentiary elements (or one sustained, convincing analysis + strong \textcolor{rpTypeWriting}{\textbf{organization}}) for 5; two for 6 with other dimensions Strong.\par - Pitfall E: Letting good \textcolor{rpTypeWriting}{\textbf{mechanics}} rescue very shallow content.\par   - Fix: B and A must be \textcolor{rpTypeRule}{\textbf{at least}} Adequate for score >= 4. Strong \textcolor{rpTypeWriting}{\textbf{mechanics}} and language without development should not push above MAX\_POSSIBLE from \textcolor{rpTypeRule}{\textbf{Step}} 1.\par ... [3 lines omitted] ...\par   - Example: "A=Adequate; B=Adequate (paraphrase-heavy); C=Adequate; D=Weak; E=Noticeable; Capped at 3 due to B=\textcolor{rpTypeEvidence}{\textbf{summary-only}}" (adapt as appropriate).\par - \textcolor{rpTypeRule}{\textbf{If}} any automatic \textcolor{rpTypeEvidence}{\textbf{cap}} or penalty was applied (\textcolor{rpTypeEvidence}{\textbf{Summary-only}} \textcolor{rpTypeEvidence}{\textbf{cap}}, Mechanical-distortion penalty), state it explicitly.\par ... [2 lines omitted] ...\par 1. Is development \textcolor{rpTypeEvidence}{\textbf{summary-only}}? Yes -> \textcolor{rpTypeEvidence}{\textbf{cap}} = 3. No -> continue.\par 2. Is B Adequate but paraphrase-heavy with single shallow link? Yes -> \textcolor{rpTypeEvidence}{\textbf{cap}} = 4. No -> no \textcolor{rpTypeEvidence}{\textbf{cap}}.\par 3. Is E Severe? Yes -> reduce \textcolor{rpTypeEvidence}{\textbf{cap}} by 1 (or 2 \textcolor{rpTypeRule}{\textbf{if}} largely unintelligible).\par ... [1 lines omitted] ...\par 5. Final\_score = min(candidate, \textcolor{rpTypeEvidence}{\textbf{cap}}); apply \textcolor{rpTypeRule}{\textbf{tie-break}}ers per \textcolor{rpTypeRule}{\textbf{Step}} 4.\par ... [2 lines omitted] ...\par - Essays that summarize source with no analysis -> final = 3 (apply \textcolor{rpTypeEvidence}{\textbf{summary-only}} \textcolor{rpTypeEvidence}{\textbf{cap}}).\par - Essays with multiple paraphrased details and noticeable \textcolor{rpTypeWriting}{\textbf{mechanics}} but \textcolor{rpTypeRule}{\textbf{at least}} one clear linkage -> usually 4.\par - Essays with many facts but Severe mechanical distortion (obscuring meaning) and \textcolor{rpTypeEvidence}{\textbf{summary-only}} development -> 1-2 depending on unintelligibility.\par - Essays with clear claim, multiple explained references, good \textcolor{rpTypeWriting}{\textbf{organization}}, and Minor \textcolor{rpTypeWriting}{\textbf{mechanics}} -> 5 or 6 depending on number/quality of explained \textcolor{rpTypeEvidence}{\textbf{evidence}} and strength of language.\par ... [2 lines omitted] ...\par - Always follow the Stepwise algorithm; do not skip the \textcolor{rpTypeEvidence}{\textbf{Summary-only}} \textcolor{rpTypeEvidence}{\textbf{cap}} in \textcolor{rpTypeRule}{\textbf{Step}} 1.\par - B and E decide bands \textcolor{rpTypeRule}{\textbf{first}}; A/C/D refine position within band.\par ... [1 lines omitted] ...\par - Use the \textcolor{rpTypeEvidence}{\textbf{Evidence}}-count requirement strictly for 5-6 placements.
\end{tcolorbox}
\caption{Pattern-focused view of the optimized rubric (ASAP2, openai\_gpt-5-mini, base\_expert\_True\_train100\_iteration5\_top3\_bs4-8-12\_mc4). Colored bold spans indicate regex-matched rubric cues. Color types: \textcolor{rpTypeRule}{\textbf{Rule Structure}} (Explicit decision logic for scoring: conditional branches, boundary tie-breakers, stepwise workflows, and numeric thresholds.); \textcolor{rpTypeEvidence}{\textbf{Evidence Handling}} (How evidence is validated and counted: specific-example requirements, repetition/non-double-count rules, and cap rules for weak evidence.); \textcolor{rpTypeWriting}{\textbf{Writing Quality}} (Language-quality criteria affecting score bands: organization/coherence/transition quality and grammar/mechanics severity.). Matched pattern categories: Conditional Gating (n=34); Boundary / Tie-Break Guidance (n=7); Stepwise Rating Workflow (n=18); Specific Evidence Requirement (n=9); Off-Topic / Summary Cap (n=30); Organization / Coherence Signal (n=3); Grammar / Mechanics Signal (n=10); Repetition Non-Count Rule (n=1); Quantitative Threshold (n=8).}
\label{fig:rubric_pattern_ASAP2_openai_gpt_5_mini_base_expert_True_train100_iteration5_top3_bs4_8_12_mc4}
\end{figure*}
