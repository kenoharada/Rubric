\colorlet{rpTypeRule}{red!80!black}
\colorlet{rpTypeEvidence}{blue!80!black}
\colorlet{rpTypeWriting}{teal!80!black}
\begin{figure*}[t]
\centering
\begin{tcolorbox}[colback=white,colframe=black!25,title=Pattern Type Guide,fonttitle=\bfseries\small,fontupper=\scriptsize,boxsep=1pt,left=2pt,right=2pt,top=2pt,bottom=2pt]
\textcolor{rpTypeRule}{\textbf{Rule Structure}}: Explicit decision logic for scoring: conditional branches, boundary tie-breakers, stepwise workflows, and numeric thresholds.\par \textcolor{rpTypeEvidence}{\textbf{Evidence Handling}}: How evidence is validated and counted: specific-example requirements, repetition/non-double-count rules, and cap rules for weak evidence.\par \textcolor{rpTypeWriting}{\textbf{Writing Quality}}: Language-quality criteria affecting score bands: organization/coherence/transition quality and grammar/mechanics severity.
\end{tcolorbox}
\vspace{1mm}
\begin{tcolorbox}[colback=white,colframe=black!25,title=Detailed Pattern Notes,fonttitle=\bfseries\small,fontupper=\scriptsize,boxsep=1pt,left=2pt,right=2pt,top=2pt,bottom=2pt]
\textcolor{rpTypeRule}{\textbf{Rule Structure}}:\par \quad \textcolor{rpTypeRule}{\textbf{Conditional Gating}} [n=21] Captures explicit condition-based rules that switch decisions only when a stated condition is met. Typical cues: if, when.\par \textcolor{rpTypeEvidence}{\textbf{Evidence Handling}}:\par \quad \textcolor{rpTypeEvidence}{\textbf{Specific Evidence Requirement}} [n=23] Highlights demands for concrete examples and explicit evidence links instead of generic assertions. Typical cues: for example, e.g., specific example, illustration, anecdote, evidence.\par \quad \textcolor{rpTypeEvidence}{\textbf{Off-Topic / Summary Cap}} [n=1] Identifies cap rules that restrict scores when responses are off-topic, irrelevant, or dominated by summary-only content. Typical cues: off-topic, irrelevant, digression, summary-only, cap.\par \textcolor{rpTypeWriting}{\textbf{Writing Quality}}:\par \quad \textcolor{rpTypeWriting}{\textbf{Organization / Coherence Signal}} [n=4] Detects explicit references to discourse structure and logical flow as scoring criteria. Typical cues: organization, coherence, logical flow, transition.\par \quad \textcolor{rpTypeWriting}{\textbf{Grammar / Mechanics Signal}} [n=3] Detects references to language-form quality, especially grammar, spelling, punctuation, and mechanics. Typical cues: grammar, mechanics, spelling, punctuation.
\end{tcolorbox}
\vspace{1mm}
\begin{tcolorbox}[colback=white,colframe=black!25,title=Optimized Rubric (Pattern-Highlighted),fonttitle=\bfseries\small,fontupper=\scriptsize]
\ttfamily
A score of 6 requires a well-organized, compelling argument with clear, credible, and \textcolor{rpTypeEvidence}{\textbf{specific example}}s; effective persuasion grounded in logical reasoning; and minimal language errors that do not distract from the message. Placeholders (\textcolor{rpTypeEvidence}{\textbf{e.g.}}, @NUM1, @CAPS1) are acceptable only \textcolor{rpTypeRule}{\textbf{if}} they are sparse, contextually clear, and do not replace substantive \textcolor{rpTypeEvidence}{\textbf{evidence}} or personal insight. Fabricated statistics, fictional names, or unsupported claims presented as fact-even \textcolor{rpTypeRule}{\textbf{if}} embedded in placeholders-undermine credibility and disqualify a score of 6.  \par ... [1 lines omitted] ...\par A score of 5 requires a strong, coherent argument with relevant, developed examples and effective persuasion. Language errors may be occasional and non-obstructive. Placeholders are acceptable \textcolor{rpTypeRule}{\textbf{if}} they are clearly intended as stand-ins for real-world references (\textcolor{rpTypeEvidence}{\textbf{e.g.}}, @LOCATION1 for a city name, @\textcolor{rpTypeWriting}{\textbf{ORGANIZATION}}1 for a known institution) and do not replace meaningful analysis, personal experience, or verifiable reasoning. A response may achieve a score of 5 even \textcolor{rpTypeRule}{\textbf{if}} it contains multiple placeholders, provided: (1) the core claims are reasonable and grounded in plausible real-world phenomena; (2) the placeholders serve as convenient abbreviations or anonymizations rather than fabrications (\textcolor{rpTypeEvidence}{\textbf{e.g.}}, @PERSON2 for "a friend" or "a teacher"); (3) the argument's logic, structure, and persuasive intent remain intact and credible; and (4) the reader can reasonably infer the intended meaning without being misled into believing the placeholders represent falsified data. Fabricated statistics, fictional \textcolor{rpTypeEvidence}{\textbf{anecdote}}s, or implausible events presented as fact-even \textcolor{rpTypeRule}{\textbf{if}} labeled with placeholders-do not automatically disqualify a score of 5 \textcolor{rpTypeRule}{\textbf{if}} they are clearly used as illustrative proxies for real trends (\textcolor{rpTypeEvidence}{\textbf{e.g.}}, "@PERCENT1 of teens" meaning "many teens") and the overall reasoning remains internally consistent and persuasive. The key is whether the response functions as a credible, thoughtful letter to the editor, not whether every detail is verifiable.  \par ... [1 lines omitted] ...\par A score of 4 requires a clear position with logical reasoning and sufficient supporting details, even \textcolor{rpTypeRule}{\textbf{if}} the language contains frequent but understandable errors (\textcolor{rpTypeEvidence}{\textbf{e.g.}}, mis\textcolor{rpTypeWriting}{\textbf{spelling}}s, awkward phrasing, minor \textcolor{rpTypeWriting}{\textbf{grammar}} issues). The argument remains comprehensible and persuasive despite imperfections. Placeholders are acceptable \textcolor{rpTypeRule}{\textbf{if}} they are limited in number, contextually clear, and do not substitute for core claims or \textcolor{rpTypeEvidence}{\textbf{evidence}}. Responses that rely on fabricated or implausible \textcolor{rpTypeEvidence}{\textbf{anecdote}}s (\textcolor{rpTypeEvidence}{\textbf{e.g.}}, "@PERSON3 overdosed because of cyberbullying") or unverifiable statistics (\textcolor{rpTypeEvidence}{\textbf{e.g.}}, "@PERCENT1 of teenagers") are scored no higher than 4 \textcolor{rpTypeRule}{\textbf{if}} the placeholders dominate the \textcolor{rpTypeEvidence}{\textbf{evidence}} base or \textcolor{rpTypeRule}{\textbf{if}} the claims are presented as factual rather than illustrative. However, \textcolor{rpTypeRule}{\textbf{if}} the core reasoning is sound, the structure is coherent, and the placeholders are used minimally to represent generic real-world phenomena (\textcolor{rpTypeEvidence}{\textbf{e.g.}}, "@CAPS1" for "the internet"), a score of 4 is appropriate.  \par ... [1 lines omitted] ...\par A score of 3 indicates a partially developed argument with some relevance to the prompt but significant language errors or dis\textcolor{rpTypeWriting}{\textbf{organization}} that hinder clarity. Ideas may be fragmented or inconsistently supported. Placeholders are used excessively or in ways that obscure meaning, but the core intent to persuade is discernible.  \par ... [1 lines omitted] ...\par A score of 2 indicates a weak or confused argument with severe language problems, incoherent structure, or nonsensical phrasing that makes the intent difficult to discern. Some relevance may be present, but the response fails to function as a persuasive letter. Excessive placeholder use that replaces meaningful content (\textcolor{rpTypeEvidence}{\textbf{e.g.}}, @CAPS1 used as a generic substitute for "computer" in nearly every sentence) or reliance on absurd, unexplained claims (\textcolor{rpTypeEvidence}{\textbf{e.g.}}, "@CAPS1 destroys the ozone layer") qualifies for a score of 2 \textcolor{rpTypeRule}{\textbf{if}} the tone is unprofessional or the logic is too broken to follow. Additionally, responses that present only superficial, repetitive, or trivial claims without meaningful development-such as listing benefits or harms in a simplistic, unsupported manner (\textcolor{rpTypeEvidence}{\textbf{e.g.}}, "computers are good because you can play games and do homework")-and fail to engage with the complexity of the issue, even \textcolor{rpTypeRule}{\textbf{if}} language errors are present, must be scored as 2. Persuasive intent requires substantive reasoning, not just enumeration of obvious or shallow points.  \par ... [1 lines omitted] ...\par A score of 1 indicates a response that is largely \textcolor{rpTypeEvidence}{\textbf{irrelevant}}, incoherent, or dominated by meaningless placeholders, fabricated data, or random text with no discernible argument or purpose. Placeholders are used as the primary content, replacing all substantive ideas, or the response is filled with nonsensical claims, fictional tragedies, or random symbols that convey no persuasive intent.  \par ... [1 lines omitted] ...\par Note: Do not penalize heavily for \textcolor{rpTypeWriting}{\textbf{spelling}} or grammatical errors \textcolor{rpTypeRule}{\textbf{if}} the core argument is clear, logically structured, and persuasively developed. Conversely, do not award high scores for structure or ideas \textcolor{rpTypeRule}{\textbf{if}} the argument relies on fabricated \textcolor{rpTypeEvidence}{\textbf{evidence}}, implausible \textcolor{rpTypeEvidence}{\textbf{anecdote}}s, or excessive placeholders that replace authentic content and undermine credibility-even \textcolor{rpTypeRule}{\textbf{if}} the language is otherwise fluent. A persuasive letter must be credible as well as coherent. However, credibility is assessed holistically: a response may still be persuasive and worthy of a score of 5 \textcolor{rpTypeRule}{\textbf{if}} its core claims are reasonable, its structure is sound, and its use of placeholders is clearly meant to represent real-world references rather than to fabricate false authority or data. Crucially, responses that merely list surface-level observations without analysis, connection to broader implications, or personal insight-regardless of grammatical correctness-lack the depth required for a score above 2. A score of 4 or higher requires \textcolor{rpTypeEvidence}{\textbf{evidence}} of critical thinking, not just assertion.  \par ... [3 lines omitted] ...\par - A response with multiple placeholders (\textcolor{rpTypeEvidence}{\textbf{e.g.}}, @\textcolor{rpTypeWriting}{\textbf{ORGANIZATION}}1, @PERSON2, @PERCENT1) may still earn a 5 \textcolor{rpTypeRule}{\textbf{if}} the underlying argument is logically structured, the examples are plausible, and the placeholders clearly represent common, real-world entities or trends (\textcolor{rpTypeEvidence}{\textbf{e.g.}}, "@PERCENT1" = "a large percentage," "@\textcolor{rpTypeWriting}{\textbf{ORGANIZATION}}1" = "a major tech company," "@PERSON2" = "a teacher or expert").  \par - A score of 5 is awarded \textcolor{rpTypeRule}{\textbf{when}} the reader can reasonably infer the intended real-world reference and the argument's persuasive power is not dependent on the placeholder's specificity.  \par - A score of 4 is assigned \textcolor{rpTypeRule}{\textbf{when}} placeholders replace substantive \textcolor{rpTypeEvidence}{\textbf{evidence}} in a way that weakens credibility, but the argument's structure and intent remain clear.  \par - A score of 2 or 1 is reserved only \textcolor{rpTypeRule}{\textbf{when}} placeholders are used so pervasively or absurdly that the response becomes unintelligible, nonsensical, or devoid of meaningful reasoning.
\end{tcolorbox}
\caption{Pattern-focused view of the optimized rubric (asap\_1, qwen\_qwen3-next-80b-a3b-instruct, base\_simplest\_True\_train100\_iteration5\_top3\_bs4-8-12\_mc4). Colored bold spans indicate regex-matched rubric cues. Color types: \textcolor{rpTypeRule}{\textbf{Rule Structure}} (Explicit decision logic for scoring: conditional branches, boundary tie-breakers, stepwise workflows, and numeric thresholds.); \textcolor{rpTypeEvidence}{\textbf{Evidence Handling}} (How evidence is validated and counted: specific-example requirements, repetition/non-double-count rules, and cap rules for weak evidence.); \textcolor{rpTypeWriting}{\textbf{Writing Quality}} (Language-quality criteria affecting score bands: organization/coherence/transition quality and grammar/mechanics severity.). Matched pattern categories: Conditional Gating (n=21); Specific Evidence Requirement (n=23); Off-Topic / Summary Cap (n=1); Organization / Coherence Signal (n=4); Grammar / Mechanics Signal (n=3).}
\label{fig:rubric_pattern_asap_1_qwen_qwen3_next_80b_a3b_instruct_base_simplest_True_train100_iteration5_top3_bs4_8_12_mc4}
\end{figure*}
