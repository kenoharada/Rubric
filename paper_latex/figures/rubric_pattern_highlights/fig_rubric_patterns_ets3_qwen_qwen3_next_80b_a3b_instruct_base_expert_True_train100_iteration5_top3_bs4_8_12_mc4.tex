\colorlet{rpTypeRule}{red!80!black}
\colorlet{rpTypeEvidence}{blue!80!black}
\colorlet{rpTypeWriting}{teal!80!black}
\begin{figure*}[t]
\centering
\begin{tcolorbox}[colback=white,colframe=black!25,title=Pattern Type Guide,fonttitle=\bfseries\small,fontupper=\scriptsize,boxsep=1pt,left=2pt,right=2pt,top=2pt,bottom=2pt]
\textcolor{rpTypeRule}{\textbf{Rule Structure}}: Explicit decision logic for scoring: conditional branches, boundary tie-breakers, stepwise workflows, and numeric thresholds.\par \textcolor{rpTypeEvidence}{\textbf{Evidence Handling}}: How evidence is validated and counted: specific-example requirements, repetition/non-double-count rules, and cap rules for weak evidence.\par \textcolor{rpTypeWriting}{\textbf{Writing Quality}}: Language-quality criteria affecting score bands: organization/coherence/transition quality and grammar/mechanics severity.
\end{tcolorbox}
\vspace{1mm}
\begin{tcolorbox}[colback=white,colframe=black!25,title=Detailed Pattern Notes,fonttitle=\bfseries\small,fontupper=\scriptsize,boxsep=1pt,left=2pt,right=2pt,top=2pt,bottom=2pt]
\textcolor{rpTypeRule}{\textbf{Rule Structure}}:\par \quad \textcolor{rpTypeRule}{\textbf{Conditional Gating}} [n=6] Captures explicit condition-based rules that switch decisions only when a stated condition is met. Typical cues: if, when.\par \textcolor{rpTypeEvidence}{\textbf{Evidence Handling}}:\par \quad \textcolor{rpTypeEvidence}{\textbf{Specific Evidence Requirement}} [n=3] Highlights demands for concrete examples and explicit evidence links instead of generic assertions. Typical cues: for example, e.g., specific example, illustration, anecdote, evidence.\par \quad \textcolor{rpTypeEvidence}{\textbf{Off-Topic / Summary Cap}} [n=5] Identifies cap rules that restrict scores when responses are off-topic, irrelevant, or dominated by summary-only content. Typical cues: off-topic, irrelevant, digression, summary-only, cap.\par \textcolor{rpTypeWriting}{\textbf{Writing Quality}}:\par \quad \textcolor{rpTypeWriting}{\textbf{Organization / Coherence Signal}} [n=5] Detects explicit references to discourse structure and logical flow as scoring criteria. Typical cues: organization, coherence, logical flow, transition.\par \quad \textcolor{rpTypeWriting}{\textbf{Grammar / Mechanics Signal}} [n=3] Detects references to language-form quality, especially grammar, spelling, punctuation, and mechanics. Typical cues: grammar, mechanics, spelling, punctuation.
\end{tcolorbox}
\vspace{1mm}
\begin{tcolorbox}[colback=white,colframe=black!25,title=Optimized Rubric (Pattern-Highlighted),fonttitle=\bfseries\small,fontupper=\scriptsize]
\ttfamily
- addresses the topic and task with clear, thorough, and well-developed explanations, exemplifications, and details that directly and meaningfully support the position; examples are specific, concrete, fully elaborated, and explicitly connected to the argument - not merely named, vaguely referenced, or superficially mentioned; even \textcolor{rpTypeRule}{\textbf{if}} minor factual inaccuracies exist (\textcolor{rpTypeEvidence}{\textbf{e.g.}}, misspelled names like "Eddison" instead of "Edison"), they do not undermine the logical \textcolor{rpTypeWriting}{\textbf{coherence}} or depth of the analysis\par - is logically organized with strong unity, clear progression, and coherent \textcolor{rpTypeWriting}{\textbf{transition}}s between ideas; no significant \textcolor{rpTypeEvidence}{\textbf{digression}}s, redundancy, or unclear connections exist; structure supports argumentative flow, even \textcolor{rpTypeRule}{\textbf{if}} the conclusion is slightly truncated or implied rather than formally stated\par - displays consistent and effective facility in language use, including syntactic variety and a broad range of precise vocabulary; grammatical, word form, or idiomatic errors are extremely rare, minor, and do not impede clarity, fluency, or meaning in any way - even \textcolor{rpTypeRule}{\textbf{when}} present, they are isolated and do not affect comprehension or flow; \textcolor{rpTypeWriting}{\textbf{spelling}} errors or non-standard word forms (\textcolor{rpTypeEvidence}{\textbf{e.g.}}, "popolation," "essectial") are acceptable \textcolor{rpTypeRule}{\textbf{if}} they are isolated and do not obscure meaning or disrupt the reader's ability to follow the argument\par ... [3 lines omitted] ...\par - addresses the topic and task with partially developed explanations, examples, or details that may be vague, incomplete, repetitive, or only indirectly relevant, but the central position remains discernible; examples may be mentioned but not fully explained, or their connection to the argument is unclear or underdeveloped - however, the writer demonstrates intent to support the claim and the core logic is present, even \textcolor{rpTypeRule}{\textbf{if}} execution is flawed\par - demonstrates adequate but inconsistent \textcolor{rpTypeWriting}{\textbf{organization}}; ideas may be connected in a general way, but \textcolor{rpTypeWriting}{\textbf{transition}}s may be awkward, missing, repetitive, or overly simplistic; there may be some \textcolor{rpTypeEvidence}{\textbf{digression}} or redundancy that disrupts flow, though the overall structure is recognizable and the progression of ideas is discernible\par - shows limited or inconsistent facility in language use: vocabulary may be repetitive, imprecise, or occasionally inappropriate; syntactic structures may be simplistic, faulty, or awkward; and errors in \textcolor{rpTypeWriting}{\textbf{grammar}}, word form, or usage are frequent enough to reduce fluency or occasionally obscure meaning, but the main ideas remain comprehensible despite these flaws - errors are systemic but not so severe as to prevent understanding of the core argument; mis\textcolor{rpTypeWriting}{\textbf{spelling}}s, non-standard phrasing, or awkward constructions (\textcolor{rpTypeEvidence}{\textbf{e.g.}}, "popolation," "it going with the light of the Sun") are tolerated \textcolor{rpTypeRule}{\textbf{if}} the intended meaning can be reasonably inferred without requiring significant reconstruction\par ... [3 lines omitted] ...\par - fails to adequately address the topic or task; response is \textcolor{rpTypeEvidence}{\textbf{off-topic}}, superficial, lacks a clear position, or is largely \textcolor{rpTypeEvidence}{\textbf{irrelevant}}; the writer may appear to misunderstand the prompt or provide no coherent stance\par - demonstrates poor or incoherent \textcolor{rpTypeWriting}{\textbf{organization}}; ideas are disconnected, illogical, or lack any meaningful progression; paragraphs may be absent, randomly structured, or fail to serve a functional purpose\par - provides inadequate, \textcolor{rpTypeEvidence}{\textbf{irrelevant}}, or absent exemplifications, explanations, or details to support claims; any examples given are unelaborated, nonsensical, unrelated, or cut off mid-thought with no discernible intent to develop them
\end{tcolorbox}
\caption{Pattern-focused view of the optimized rubric (ets3, qwen\_qwen3-next-80b-a3b-instruct, base\_expert\_True\_train100\_iteration5\_top3\_bs4-8-12\_mc4). Colored bold spans indicate regex-matched rubric cues. Color types: \textcolor{rpTypeRule}{\textbf{Rule Structure}} (Explicit decision logic for scoring: conditional branches, boundary tie-breakers, stepwise workflows, and numeric thresholds.); \textcolor{rpTypeEvidence}{\textbf{Evidence Handling}} (How evidence is validated and counted: specific-example requirements, repetition/non-double-count rules, and cap rules for weak evidence.); \textcolor{rpTypeWriting}{\textbf{Writing Quality}} (Language-quality criteria affecting score bands: organization/coherence/transition quality and grammar/mechanics severity.). Matched pattern categories: Conditional Gating (n=6); Specific Evidence Requirement (n=3); Off-Topic / Summary Cap (n=5); Organization / Coherence Signal (n=5); Grammar / Mechanics Signal (n=3).}
\label{fig:rubric_pattern_ets3_qwen_qwen3_next_80b_a3b_instruct_base_expert_True_train100_iteration5_top3_bs4_8_12_mc4}
\end{figure*}
