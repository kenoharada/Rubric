\colorlet{rpTypeRule}{red!80!black}
\colorlet{rpTypeEvidence}{blue!80!black}
\colorlet{rpTypeWriting}{teal!80!black}
\begin{figure*}[t]
\centering
\begin{tcolorbox}[colback=white,colframe=black!25,title=Pattern Legend,fonttitle=\bfseries\small,fontupper=\scriptsize,boxsep=1pt,left=2pt,right=2pt,top=2pt,bottom=2pt]
\textcolor{rpTypeRule}{\textbf{Rule Structure}} (if/threshold/stepwise guidance) \quad \textcolor{rpTypeEvidence}{\textbf{Evidence Handling}} (examples, repetition, and caps) \quad \textcolor{rpTypeWriting}{\textbf{Writing Quality}} (organization and grammar/mechanics)
\end{tcolorbox}
\vspace{2mm}
\begin{minipage}[t]{0.485\textwidth}
\begin{tcolorbox}[colback=white,colframe=black!25,title=Initial Rubric,fonttitle=\bfseries\small,fontupper=\scriptsize,breakable]
\ttfamily
After reading each essay and completing the analytical rating form, assign a holistic score based on the rubric below. For the following evaluations you will need to use a grading scale between 1 (minimum) and 6 (maximum). As with the analytical rating form, the distance between each grade (\textcolor{rpTypeEvidence}{e.g.}, 1-2, 3-4, 4-5) should be considered equal.\par \par SCORE OF 6: An essay in this category demonstrates clear and consistent mastery, although it may have a few minor errors. A typical essay effectively and insightfully develops a point of view on the issue and demonstrates outstanding critical thinking; the essay uses clearly appropriate examples, reasons, and other \textcolor{rpTypeEvidence}{evidence} taken from the source text(s) to support its position; the essay is well organized and clearly focused, demonstrating clear \textcolor{rpTypeWriting}{coherence} and smooth progression of ideas; the essay exhibits skillful use of language, using a varied, accurate, and apt vocabulary and demonstrates meaningful variety in sentence structure; the essay is free of most errors in \textcolor{rpTypeWriting}{grammar}, usage, and \textcolor{rpTypeWriting}{mechanics}.\par \par SCORE OF 5: An essay in this category demonstrates reasonably consistent mastery, although it will have occasional errors or lapses in quality. A typical essay effectively develops a point of view on the issue and demonstrates strong critical thinking; the essay generally using appropriate examples, reasons, and other \textcolor{rpTypeEvidence}{evidence} taken from the source text(s) to support its position; the essay is well organized and focused, demonstrating \textcolor{rpTypeWriting}{coherence} and progression of ideas; the essay exhibits facility in the use of language, using appropriate vocabulary demonstrates variety in sentence structure; the essay is generally free of most errors in \textcolor{rpTypeWriting}{grammar}, usage, and \textcolor{rpTypeWriting}{mechanics}.\par \par SCORE OF 4: An essay in this category demonstrates adequate mastery, although it will have lapses in quality. A typical essay develops a point of view on the issue and demonstrates competent critical thinking; the essay using adequate examples, reasons, and other \textcolor{rpTypeEvidence}{evidence} taken from the source text(s) to support its position; the essay is generally organized and focused, demonstrating some \textcolor{rpTypeWriting}{coherence} and progression of ideas exhibits adequate; the essay may demonstrate inconsistent facility in the use of language, using generally appropriate vocabulary demonstrates some variety in sentence structure; the essay may have some errors in \textcolor{rpTypeWriting}{grammar}, usage, and \textcolor{rpTypeWriting}{mechanics}.\par \par SCORE OF 3: An essay in this category demonstrates developing mastery, and is marked by ONE OR MORE of the following weaknesses: develops a point of view on the issue, demonstrating some critical thinking, but may do so inconsistently or use inadequate examples, reasons, or other \textcolor{rpTypeEvidence}{evidence} taken from the source texts to support its position; the essay is limited in its \textcolor{rpTypeWriting}{organization} or focus, or may demonstrate some lapses in \textcolor{rpTypeWriting}{coherence} or progression of ideas displays; the essay may demonstrate facility in the use of language, but sometimes uses weak vocabulary or inappropriate word choice and/or lacks variety or demonstrates problems in sentence structure; the essay may contain an accumulation of errors in \textcolor{rpTypeWriting}{grammar}, usage, and \textcolor{rpTypeWriting}{mechanics}.\par \par SCORE OF 2: An essay in this category demonstrates little mastery, and is flawed by ONE OR MORE of the following weaknesses: develops a point of view on the issue that is vague or seriously limited, and demonstrates weak critical thinking; the essay providesinappropriate or insufficient examples, reasons, or other \textcolor{rpTypeEvidence}{evidence} taken from the source text to support its position; the essay is poorly organized and/or focused, or demonstrates serious problems with \textcolor{rpTypeWriting}{coherence} or progression of ideas; the essay displays very little facility in the use of language, using very limited vocabulary or incorrect word choice and/or demonstrates frequent problems in sentence structure; the essay contains errors in \textcolor{rpTypeWriting}{grammar}, usage, and \textcolor{rpTypeWriting}{mechanics} so serious that meaning is somewhat obscured.\par \par SCORE OF 1: An essay in this category demonstrates very little or no mastery, and is severely flawed by ONE OR MORE of the following weaknesses: develops no viable point of view on the issue, or provides little or no \textcolor{rpTypeEvidence}{evidence} to support its position; the essay is disorganized or unfocused, resulting in a disjointed orincoherent essay; the essay displays fundamental errors in vocabulary and/or demonstrates severe flaws in sentence structure; the essay contains pervasive errors in \textcolor{rpTypeWriting}{grammar}, usage, or \textcolor{rpTypeWriting}{mechanics} that persistently interfere with meaning.
\end{tcolorbox}
\end{minipage}
\hfill
\begin{minipage}[t]{0.485\textwidth}
\begin{tcolorbox}[colback=white,colframe=black!25,title=Optimized Rubric,fonttitle=\bfseries\small,fontupper=\scriptsize,breakable]
\ttfamily
After reading each essay and completing the analytical rating form, assign a holistic score based on the rubric below. For the following evaluations you will need to use a grading scale between 1 (minimum) and 6 (maximum). The distance between each grade should be considered equal. \textcolor{rpTypeRule}{When} scoring, prioritize the quality of critical thinking and the student's ability to use \textcolor{rpTypeEvidence}{evidence} over surface-level mechanical errors, unless those errors obscure meaning.\par \par SCORE OF 6: Outstanding Mastery. An essay in this category demonstrates clear and consistent mastery. It effectively and insightfully develops a point of view, showing sophisticated critical thinking (\textcolor{rpTypeEvidence}{e.g.}, analyzing tone, diction, or complex contradictions). It uses clearly appropriate examples and \textcolor{rpTypeEvidence}{evidence} from the source text to support its position; it is well-organized, focused, and demonstrates clear \textcolor{rpTypeWriting}{coherence} and smooth progression of ideas. It exhibits skillful use of language, varied vocabulary, and meaningful variety in sentence structure.\par \par SCORE OF 5: Strong Mastery. An essay in this category demonstrates reasonably consistent mastery, though it will have occasional errors or lapses in quality. The essay effectively develops a point of view and demonstrates strong critical thinking; it generally uses appropriate \textcolor{rpTypeEvidence}{evidence} from the source text; it is well-organized and focused, demonstrating \textcolor{rpTypeWriting}{coherence} and progression; it exhibits facility in language, using appropriate vocabulary and varied sentence structure.\par \par SCORE OF 4: Adequate Mastery. An essay in this category demonstrates adequate mastery. The essay develops a clear point of view and demonstrates competent critical thinking by connecting the text to a broader argument, synthesizing several parts of the text to support a theme, or by analyzing the author's methods (\textcolor{rpTypeEvidence}{e.g.}, how the author uses facts or examples to persuade). \textcolor{rpTypeRule}{If} the essay moves beyond a chronological retelling to a thematic \textcolor{rpTypeWriting}{organization}-even \textcolor{rpTypeRule}{if} it contains frequent errors in \textcolor{rpTypeWriting}{grammar} and \textcolor{rpTypeWriting}{mechanics}-it should receive a 4. The student must demonstrate an understanding of the text's construction rather than just its content.\par \par SCORE OF 3: Developing Mastery. An essay in this category demonstrates developing mastery and is marked by one or more of the following: it develops a point of view but does so inconsistently; it shows some attempt at analysis but is limited in depth or relies more on paraphrasing than critical evaluation. A 3 may identify the author's argument and provide \textcolor{rpTypeEvidence}{evidence} but fails to explain *how* or *why* the \textcolor{rpTypeEvidence}{evidence} supports the argument in a meaningful way. It may follow the text's chronology too closely (\textcolor{rpTypeEvidence}{e.g.}, "In paragraph 1... in paragraph 2...") but must offer \textcolor{rpTypeRule}{at least} some original interpretation or evaluation of the ideas to remain in this category.\par \par SCORE OF 2: Little Mastery. An essay in this category demonstrates little mastery and is flawed by one or more of the following: it develops a vague, simplistic, or seriously limited point of view; it relies almost exclusively on listing facts or providing a sequential summary of the text. Common traits of a 2 include substituting conversational fillers, repetitive rhetorical questions (\textcolor{rpTypeEvidence}{e.g.}, "Don't you want to find out?"), or a simple "I agree" for an actual argument. Even \textcolor{rpTypeRule}{if} the writing is relatively clear and the structure is logical, an essay that is essentially a summary with a brief personal opinion tacked on belongs in this category.\par 
\end{tcolorbox}
\end{minipage}
\caption{Pattern-highlighted rubric comparison (ASAP2, google\_gemini-3-flash-preview, base\_expert\_True\_train100\_iteration5\_top3\_bs4-8-12\_mc4). Matched spans are color-coded by regex pattern. Color types: \textcolor{rpTypeRule}{\textbf{Rule Structure}} (if/threshold/stepwise guidance); \textcolor{rpTypeEvidence}{\textbf{Evidence Handling}} (examples, repetition, and caps); \textcolor{rpTypeWriting}{\textbf{Writing Quality}} (organization and grammar/mechanics).}
\label{fig:rubric_pattern_ASAP2_google_gemini_3_flash_preview_base_expert_True_train100_iteration5_top3_bs4_8_12_mc4}
\end{figure*}
