\colorlet{rpTypeRule}{red!80!black}
\colorlet{rpTypeEvidence}{blue!80!black}
\colorlet{rpTypeWriting}{teal!80!black}
\begin{figure*}[t]
\centering
\begin{tcolorbox}[colback=white,colframe=black!25,title=Pattern Type Guide,fonttitle=\bfseries\small,fontupper=\scriptsize,boxsep=1pt,left=2pt,right=2pt,top=2pt,bottom=2pt]
\textcolor{rpTypeRule}{\textbf{Rule Structure}}: Explicit decision logic for scoring: conditional branches, boundary tie-breakers, stepwise workflows, and numeric thresholds.\par \textcolor{rpTypeEvidence}{\textbf{Evidence Handling}}: How evidence is validated and counted: specific-example requirements, repetition/non-double-count rules, and cap rules for weak evidence.\par \textcolor{rpTypeWriting}{\textbf{Writing Quality}}: Language-quality criteria affecting score bands: organization/coherence/transition quality and grammar/mechanics severity.
\end{tcolorbox}
\vspace{1mm}
\begin{tcolorbox}[colback=white,colframe=black!25,title=Detailed Pattern Notes,fonttitle=\bfseries\small,fontupper=\scriptsize,boxsep=1pt,left=2pt,right=2pt,top=2pt,bottom=2pt]
\textcolor{rpTypeRule}{\textbf{Rule Structure}}:\par \quad \textcolor{rpTypeRule}{\textbf{Conditional Gating}} [n=4] Captures explicit condition-based rules that switch decisions only when a stated condition is met. Typical cues: if, when.\par \quad \textcolor{rpTypeRule}{\textbf{Quantitative Threshold}} [n=1] Marks numeric cutoffs used for consistent decisions (minimum/maximum counts, percentages, explicit count thresholds). Typical cues: at least, at most, <=, >=, \%, N reasons/examples/sentences/words.\par \textcolor{rpTypeEvidence}{\textbf{Evidence Handling}}:\par \quad \textcolor{rpTypeEvidence}{\textbf{Specific Evidence Requirement}} [n=9] Highlights demands for concrete examples and explicit evidence links instead of generic assertions. Typical cues: for example, e.g., specific example, illustration, anecdote, evidence.\par \textcolor{rpTypeWriting}{\textbf{Writing Quality}}:\par \quad \textcolor{rpTypeWriting}{\textbf{Organization / Coherence Signal}} [n=3] Detects explicit references to discourse structure and logical flow as scoring criteria. Typical cues: organization, coherence, logical flow, transition.\par \quad \textcolor{rpTypeWriting}{\textbf{Grammar / Mechanics Signal}} [n=2] Detects references to language-form quality, especially grammar, spelling, punctuation, and mechanics. Typical cues: grammar, mechanics, spelling, punctuation.
\end{tcolorbox}
\vspace{1mm}
\begin{tcolorbox}[colback=white,colframe=black!25,title=Optimized Rubric (Pattern-Highlighted),fonttitle=\bfseries\small,fontupper=\scriptsize]
\ttfamily
After reading each essay and completing the analytical rating form, assign a holistic score based on the rubric below. For the following evaluations you will need to use a grading scale between 1 (minimum) and 6 (maximum). The distance between each grade should be considered equal. \textcolor{rpTypeRule}{\textbf{When}} scoring, prioritize the quality of critical thinking and the student's ability to use \textcolor{rpTypeEvidence}{\textbf{evidence}} over surface-level mechanical errors, unless those errors obscure meaning.\par ... [1 lines omitted] ...\par SCORE OF 6: Outstanding Mastery. An essay in this category demonstrates clear and consistent mastery. It effectively and insightfully develops a point of view, showing sophisticated critical thinking (\textcolor{rpTypeEvidence}{\textbf{e.g.}}, analyzing tone, diction, or complex contradictions). It uses clearly appropriate examples and \textcolor{rpTypeEvidence}{\textbf{evidence}} from the source text to support its position; it is well-organized, focused, and demonstrates clear \textcolor{rpTypeWriting}{\textbf{coherence}} and smooth progression of ideas. It exhibits skillful use of language, varied vocabulary, and meaningful variety in sentence structure.\par ... [1 lines omitted] ...\par SCORE OF 5: Strong Mastery. An essay in this category demonstrates reasonably consistent mastery, though it will have occasional errors or lapses in quality. The essay effectively develops a point of view and demonstrates strong critical thinking; it generally uses appropriate \textcolor{rpTypeEvidence}{\textbf{evidence}} from the source text; it is well-organized and focused, demonstrating \textcolor{rpTypeWriting}{\textbf{coherence}} and progression; it exhibits facility in language, using appropriate vocabulary and varied sentence structure.\par ... [1 lines omitted] ...\par SCORE OF 4: Adequate Mastery. An essay in this category demonstrates adequate mastery. The essay develops a clear point of view and demonstrates competent critical thinking by connecting the text to a broader argument, synthesizing several parts of the text to support a theme, or by analyzing the author's methods (\textcolor{rpTypeEvidence}{\textbf{e.g.}}, how the author uses facts or examples to persuade). \textcolor{rpTypeRule}{\textbf{If}} the essay moves beyond a chronological retelling to a thematic \textcolor{rpTypeWriting}{\textbf{organization}}-even \textcolor{rpTypeRule}{\textbf{if}} it contains frequent errors in \textcolor{rpTypeWriting}{\textbf{grammar}} and \textcolor{rpTypeWriting}{\textbf{mechanics}}-it should receive a 4. The student must demonstrate an understanding of the text's construction rather than just its content.\par ... [1 lines omitted] ...\par SCORE OF 3: Developing Mastery. An essay in this category demonstrates developing mastery and is marked by one or more of the following: it develops a point of view but does so inconsistently; it shows some attempt at analysis but is limited in depth or relies more on paraphrasing than critical evaluation. A 3 may identify the author's argument and provide \textcolor{rpTypeEvidence}{\textbf{evidence}} but fails to explain *how* or *why* the \textcolor{rpTypeEvidence}{\textbf{evidence}} supports the argument in a meaningful way. It may follow the text's chronology too closely (\textcolor{rpTypeEvidence}{\textbf{e.g.}}, "In paragraph 1... in paragraph 2...") but must offer \textcolor{rpTypeRule}{\textbf{at least}} some original interpretation or evaluation of the ideas to remain in this category.\par ... [1 lines omitted] ...\par SCORE OF 2: Little Mastery. An essay in this category demonstrates little mastery and is flawed by one or more of the following: it develops a vague, simplistic, or seriously limited point of view; it relies almost exclusively on listing facts or providing a sequential summary of the text. Common traits of a 2 include substituting conversational fillers, repetitive rhetorical questions (\textcolor{rpTypeEvidence}{\textbf{e.g.}}, "Don't you want to find out?"), or a simple "I agree" for an actual argument. Even \textcolor{rpTypeRule}{\textbf{if}} the writing is relatively clear and the structure is logical, an essay that is essentially a summary with a brief personal opinion tacked on belongs in this category.
\end{tcolorbox}
\caption{Pattern-focused view of the optimized rubric (ASAP2, google\_gemini-3-flash-preview, base\_expert\_True\_train100\_iteration5\_top3\_bs4-8-12\_mc4). Colored bold spans indicate regex-matched rubric cues. Color types: \textcolor{rpTypeRule}{\textbf{Rule Structure}} (Explicit decision logic for scoring: conditional branches, boundary tie-breakers, stepwise workflows, and numeric thresholds.); \textcolor{rpTypeEvidence}{\textbf{Evidence Handling}} (How evidence is validated and counted: specific-example requirements, repetition/non-double-count rules, and cap rules for weak evidence.); \textcolor{rpTypeWriting}{\textbf{Writing Quality}} (Language-quality criteria affecting score bands: organization/coherence/transition quality and grammar/mechanics severity.). Matched pattern categories: Conditional Gating (n=4); Specific Evidence Requirement (n=9); Organization / Coherence Signal (n=3); Grammar / Mechanics Signal (n=2); Quantitative Threshold (n=1).}
\label{fig:rubric_pattern_ASAP2_google_gemini_3_flash_preview_base_expert_True_train100_iteration5_top3_bs4_8_12_mc4}
\end{figure*}
