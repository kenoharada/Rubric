\colorlet{rpTypeRule}{red!80!black}
\colorlet{rpTypeEvidence}{blue!80!black}
\colorlet{rpTypeWriting}{teal!80!black}
\begin{figure*}[t]
\centering
\begin{tcolorbox}[colback=white,colframe=black!25,title=Pattern Legend,fonttitle=\bfseries\small,fontupper=\scriptsize,boxsep=1pt,left=2pt,right=2pt,top=2pt,bottom=2pt]
\textcolor{rpTypeRule}{\textbf{Rule Structure}} (if/threshold/stepwise guidance) \quad \textcolor{rpTypeEvidence}{\textbf{Evidence Handling}} (examples, repetition, and caps) \quad \textcolor{rpTypeWriting}{\textbf{Writing Quality}} (organization and grammar/mechanics)
\end{tcolorbox}
\vspace{2mm}
\begin{minipage}[t]{0.485\textwidth}
\begin{tcolorbox}[colback=white,colframe=black!25,title=Initial Rubric,fonttitle=\bfseries\small,fontupper=\scriptsize,breakable]
\ttfamily
- Contains only general reasons with unelaborated and/or list-like details.\par - Shows little or no \textcolor{rpTypeEvidence}{evidence} of \textcolor{rpTypeWriting}{organization}.\par - May be awkward and confused or simplistic.\par ... [3 lines omitted] ...\par - Has reasons with minimal elaboration and more general than specific details.\par - Shows some \textcolor{rpTypeWriting}{organization}.\par - May be awkward in parts with few \textcolor{rpTypeWriting}{transition}s.\par - Shows some awareness of audience.\par ... [2 lines omitted] ...\par - Has adequately elaborated reasons with a mix of general and specific details.\par - Shows satisfactory \textcolor{rpTypeWriting}{organization}.\par - May be somewhat fluent with some \textcolor{rpTypeWriting}{transition}al language.\par - Shows adequate awareness of audience.\par ... [2 lines omitted] ...\par - Has moderately well elaborated reasons with mostly specific details.\par - Exhibits generally strong \textcolor{rpTypeWriting}{organization}.\par - May be moderately fluent with \textcolor{rpTypeWriting}{transition}al language throughout.\par - May show a consistent awareness of audience.\par ... [2 lines omitted] ...\par - Has fully elaborated reasons with specific details.\par - Exhibits strong \textcolor{rpTypeWriting}{organization}.\par - Is fluent and uses sophisticated \textcolor{rpTypeWriting}{transition}al language.\par - May show a heightened awareness of audience.\par ... [1 lines omitted] ...\par Note: \par I have made an effort to remove personally identifying information from the essays using the Named Entity Recognizer (NER). The relevant entities are identified in the text and then replaced with a string such as "PERSON", "\textcolor{rpTypeWriting}{ORGANIZATION}", "LOCATION", "DATE", "TIME", "MONEY", "PERCENT", "CAPS" (any capitalized word) and "NUM" (any digits). Please do not penalize the essay because of the anonymizations.
\end{tcolorbox}
\end{minipage}
\hfill
\begin{minipage}[t]{0.485\textwidth}
\begin{tcolorbox}[colback=white,colframe=black!25,title=Optimized Rubric,fonttitle=\bfseries\small,fontupper=\scriptsize,breakable]
\ttfamily
General note about anonymization:\par - Do NOT deduct points for named-entity placeholders (PERSON, LOCATION, NUM, PERCENT, etc.). Treat them as neutral substitutions for real details; evaluate the presence, clarity, and specificity of ideas rather than the literal labels. Count placeholders as specific details \textcolor{rpTypeRule}{when} the writer clearly intends a concrete fact, example, or \textcolor{rpTypeEvidence}{anecdote}.\par \par ... [2 lines omitted] ...\par 2. Number and specificity of supporting reasons/examples (count distinct reasons and whether each has elaboration).\par 3. \textcolor{rpTypeWriting}{Organization}/\textcolor{rpTypeWriting}{coherence} (intro-body-conclusion, paragraphing, \textcolor{rpTypeWriting}{transition}s).\par 4. Development depth and insight (analysis, counterargument).\par ... [5 lines omitted] ...\par   b. Minimal elaboration (general explanation, vague example) -> "min-elab".\par   c. Specific elaboration/example or clear personal \textcolor{rpTypeEvidence}{anecdote}/data -> "specific".\par - Treat personal \textcolor{rpTypeEvidence}{anecdote}s and clearly-intended placeholders-as-\textcolor{rpTypeEvidence}{evidence} as valid "specific".\par - \textcolor{rpTypeEvidence}{Do not double-count} repeated \textcolor{rpTypeEvidence}{restatement}s of the same reason. Different examples that support the same reason count as strengthening that one reason (do not convert them into separate reasons unless they support a genuinely different claim).\par - \textcolor{rpTypeRule}{When} in doubt about whether two supports are distinct reasons or sub-points of the same reason, prefer to count them as the same reason unless they address different effects, audiences, or mechanisms.\par \par ... [1 lines omitted] ...\par - Position: May state a position or may be off-task; little or no purposeful response to the prompt.\par - Development: Few or no reasons; \textcolor{rpTypeRule}{if} present they are list-only or \textcolor{rpTypeEvidence}{irrelevant}. No meaningful examples or elaboration.\par - \textcolor{rpTypeWriting}{Organization} \& \textcolor{rpTypeWriting}{coherence}: Fragmented, chaotic, or extremely hard to follow; may be one or two disjointed sentences.\par - Language: \textcolor{rpTypeWriting}{Grammar} and usage may prevent comprehension.\par - Use \textcolor{rpTypeRule}{when} the essay essentially fails to form an argument or provide any supporting content.\par \par ... [2 lines omitted] ...\par - Development: Mostly list-only reasons or 1-2 general reasons with no or minimal elaboration (classified as "list-only" or mostly "min-elab"). Distinctness of reasons is low.\par - \textcolor{rpTypeWriting}{Organization} \& \textcolor{rpTypeWriting}{coherence}: Little or no logical \textcolor{rpTypeWriting}{organization}; \textcolor{rpTypeWriting}{transition}s absent and sequencing is weak.\par - Language: Frequent errors that sometimes impede comprehension.\par - Use \textcolor{rpTypeRule}{when} the response is more than a sentence or two but lacks development, explanation, and clear structure.\par \par Score Point 3 - "Minimally developed / some \textcolor{rpTypeWriting}{organization}"\par - Position: Takes a position that is generally clear.\par - Development: Provides basic reasons with minimal elaboration. Typical patterns:\par   - 1-2 distinct reasons where \textcolor{rpTypeRule}{at least} one has min-elab; or\par   - 2+ reasons but most are list-only or repetitive \textcolor{rpTypeEvidence}{restatement}s.\par   - May include one brief \textcolor{rpTypeEvidence}{specific example} or \textcolor{rpTypeEvidence}{anecdote}, but it is not well developed or persuasive.\par - \textcolor{rpTypeWriting}{Organization} \& \textcolor{rpTypeWriting}{coherence}: Some sense of \textcolor{rpTypeWriting}{organization} (intro, body, conclusion or paragraphing) though progression may be weak; limited \textcolor{rpTypeWriting}{transition}s.\par - Language: Errors are frequent but overall meaning is still understandable.\par - Use \textcolor{rpTypeRule}{when} the essay demonstrates a clear stance and rudimentary structure with limited support and few specific, distinct examples.\par \par ... [1 lines omitted] ...\par - Position: Clearly stated position throughout.\par - Development: Offers adequately elaborated reasons with a mix of general and specific details/examples. Typical \textcolor{rpTypeRule}{threshold}:\par   - \textcolor{rpTypeRule}{At least} 2 distinct reasons each with \textcolor{rpTypeRule}{at least} min-elab AND \textcolor{rpTypeRule}{at least} one clear \textcolor{rpTypeEvidence}{specific example} supporting any one of the reasons; OR\par   - 2-3 distinct reasons with mostly min-elab development and \textcolor{rpTypeRule}{at least} one specific that meaningfully strengthens the argument.\par - \textcolor{rpTypeWriting}{Organization} \& \textcolor{rpTypeWriting}{coherence}: Satisfactory \textcolor{rpTypeWriting}{organization} with clear paragraphing and some \textcolor{rpTypeWriting}{transition}s; readers can follow the argument.\par - Language: Noticeable errors may be present but do not substantially obscure meaning.\par - \textcolor{rpTypeRule}{Tie-break}er to promote consistency: \textcolor{rpTypeRule}{If} an essay has 2+ distinct reasons each with min-elab and \textcolor{rpTypeRule}{at least} one clearly functioning specific (including placeholders or brief \textcolor{rpTypeEvidence}{anecdote}s), prefer 4 over 3-even \textcolor{rpTypeRule}{when} language is weak.\par \par ... [2 lines omitted] ...\par - Development: Stronger elaboration than 4. Typical patterns (choose the rule that fits):\par   - 3+ distinct reasons with \textcolor{rpTypeRule}{at least} two being "specific" examples/\textcolor{rpTypeEvidence}{anecdote}s; OR\par   - 2 distinct reasons each with robust, specific elaboration and varied supporting details; OR\par   - 2+ reasons plus multiple concrete personal \textcolor{rpTypeEvidence}{anecdote}s or data points that combine to make the argument persuasive.\par - \textcolor{rpTypeWriting}{Organization} \& \textcolor{rpTypeWriting}{coherence}: Generally strong \textcolor{rpTypeWriting}{organization} and logical progression; effective paragraphing and \textcolor{rpTypeWriting}{transition}al language throughout.\par - Language: Generally fluent; errors present but not distracting.\par - Important constraint to reduce over-scoring: Do NOT award a 5 \textcolor{rpTypeRule}{if} the "specific" supports are repetitive \textcolor{rpTypeEvidence}{restatement}s or the same example reused to pad counts. Multiple \textcolor{rpTypeEvidence}{specific example}s must be distinct in content or context (different data points, different \textcolor{rpTypeEvidence}{anecdote}s, different illustrative scenarios).\par \par Score Point 6 - "Well-developed / thoughtful \& sophisticated"\par - Position: Takes a thoughtful, nuanced, and compelling position that goes beyond the obvious OR demonstrates exceptional development through multiple distinct specifics and strong \textcolor{rpTypeWriting}{organization}.\par - Development: One of the following should be true:\par   - The essay presents a nuanced/complex stance (acknowledges trade-offs, limits, or a qualified position) AND offers multiple specific, relevant examples and explanation; OR\par   - The essay contains 3+ distinct reasons each supported by clear, separate "specific" examples (not merely repeated \textcolor{rpTypeEvidence}{restatement}s), combined with cohesive \textcolor{rpTypeWriting}{organization} and some synthesis (linking reasons, explaining implications) even \textcolor{rpTypeRule}{if} explicit counterargument is brief or implicit.\par - \textcolor{rpTypeWriting}{Organization} \& \textcolor{rpTypeWriting}{coherence}: Strong, logical \textcolor{rpTypeWriting}{organization} with clear, effective \textcolor{rpTypeWriting}{transition}s and paragraphing; the argument builds cohesively.\par - Language: Fluent and controlled with varied sentence structure and vocabulary; minor errors may occur but do not interfere with meaning.\par - Audience awareness: Heightened-persuasive techniques tailored to the intended audience.\par - Use 6 \textcolor{rpTypeRule}{when} the essay demonstrates either clear analytic depth (counterargument, trade-offs, synthesis) OR very strong breadth and specificity of development (three distinct, well-supported reasons) plus clear \textcolor{rpTypeWriting}{organization}.\par \par ... [2 lines omitted] ...\par    - Use the three-tier label (list-only / min-elab / specific) per reason.\par    - 2+ adequately elaborated reasons (min-elab) with \textcolor{rpTypeRule}{at least} one specific -> lean 4.\par    - 3+ distinct reasons with 2+ \textcolor{rpTypeEvidence}{specific example}s/\textcolor{rpTypeEvidence}{anecdote}s -> lean 5; \textcolor{rpTypeRule}{if} those 3+ specifics are present and \textcolor{rpTypeWriting}{organization} is strong, consider 6 (see 6's alternate path).\par    - 2 strongly specific, well-connected reasons with varied \textcolor{rpTypeEvidence}{evidence} or a clear rebuttal -> can justify 5.\par    - Counterargument or analytical depth -> consider 6.\par 2. Favor development over surface fluency:\par    - Frequent mechanical errors should lower the fluency descriptor but should not automatically move a piece from 4/5/6 down to 2/3 \textcolor{rpTypeRule}{if} the essay contains clear \textcolor{rpTypeWriting}{organization} and multiple \textcolor{rpTypeEvidence}{specific example}s.\par    - However, severe breakdowns in \textcolor{rpTypeWriting}{grammar} that impede comprehension of key supports should lower the score.\par 3. Treat personal \textcolor{rpTypeEvidence}{anecdote}s and placeholder-based "studies" as valid \textcolor{rpTypeEvidence}{evidence}:\par    - \textcolor{rpTypeRule}{If} the writer provides a personal story or clearly intended study/example (even with placeholders), count it as a "specific" example for development-unless the placeholder is so vague that it does not function as \textcolor{rpTypeEvidence}{evidence}.\par 4. Distinguish \textcolor{rpTypeEvidence}{repetition} vs. distinct \textcolor{rpTypeEvidence}{evidence}:\par    - \textcolor{rpTypeEvidence}{Repetition} or rephrasing of the same example should not be counted as multiple specifics.\par    - Different contexts or different concrete examples (even \textcolor{rpTypeRule}{if} they support the same general reason) strengthen that reason but \textcolor{rpTypeRule}{do not count} as additional distinct reasons.\par    - To move from 4->5 using the "3+ reasons" route, ensure the \textcolor{rpTypeRule}{third} reason is independent (addresses a different effect or mechanism) and has a \textcolor{rpTypeEvidence}{specific example}.\par 5. Use \textcolor{rpTypeWriting}{organization} to resolve close calls:\par    - Clear intro/body/conclusion and logical paragraphing can raise a \textcolor{rpTypeRule}{borderline} 3 to a 4 even \textcolor{rpTypeRule}{when} details are modest.\par    - Conversely, poor \textcolor{rpTypeWriting}{organization} can keep a richly supported response from reaching 6 \textcolor{rpTypeRule}{if} the argument fails to cohere.\par 6. Holistic \textcolor{rpTypeRule}{tie-break}ers (final arbitration):\par    - \textcolor{rpTypeRule}{If} features point to different scores, prioritize in order: (a) specificity \& number of supporting details (b) clarity of position (c) \textcolor{rpTypeWriting}{organization}/cohesion.\par    - \textcolor{rpTypeRule}{When} in doubt between 4 and 5, count \textcolor{rpTypeEvidence}{specific example}s carefully-require distinctiveness and substantive support. \textcolor{rpTypeRule}{If} you count 2 full- strength specifics (distinct content) and either a \textcolor{rpTypeRule}{third} reason or \textcolor{rpTypeRule}{second} reason with strong specifics, prefer 5.\par    - \textcolor{rpTypeRule}{When} in doubt between 5 and 6, require either explicit nuance/counterargument OR 3+ distinct reasons each with clear specifics AND strong \textcolor{rpTypeWriting}{organization} for 6.\par 7. Examples of \textcolor{rpTypeRule}{boundary} judgments (updated heuristics):\par    - Several distinct reasons but only list-like, no examples -> Score 2.\par    - Clear stance, 1-\textcolor{rpTypeRule}{2 reasons} with brief examples or \textcolor{rpTypeEvidence}{anecdote}s; some \textcolor{rpTypeWriting}{organization} -> Score 3.\par    - Clear stance, 2+ distinct reasons each with \textcolor{rpTypeRule}{at least} some elaboration and \textcolor{rpTypeRule}{at least} one \textcolor{rpTypeEvidence}{specific example} -> Score 4.\par    - Clear stance, either (a) 3+ reasons with mostly specific, relevant examples (distinct and non-repetitive) OR (b) \textcolor{rpTypeRule}{2 reasons} each with substantial specific elaboration and persuasive flow -> Score 5.\par    - Nuanced stance, thorough analysis, counterargument, or 3+ distinct, well-supported reasons with cohesive synthesis -> Score 6.\par \par Practical scoring \textcolor{rpTypeRule}{checklist} (use \textcolor{rpTypeRule}{when} assigning a score):\par - Is the position clear and consistent? (Yes -> continue; No -> lean 1-2)\par ... [2 lines omitted] ...\par - Count specifics: how many distinct "specific" supports? (Are they different in content/context or repetitive?)\par - Evaluate \textcolor{rpTypeWriting}{organization}: clear paragraphs and \textcolor{rpTypeWriting}{transition}s? (Yes raises \textcolor{rpTypeRule}{borderline} 3->4)\par - Is there counterargument, synthesis, or analytic depth? (Yes -> consider 6)\par - Check language: are errors limiting comprehension? (\textcolor{rpTypeRule}{If} comprehension fails, lower to 1-2; otherwise, do not heavily penalize development)\par - Apply \textcolor{rpTypeRule}{tie-break}er rules above (prioritize specificity, then clarity, then \textcolor{rpTypeWriting}{organization}).\par \par Rationale summary for raters:\par - Increase scores \textcolor{rpTypeRule}{when} multiple distinct, \textcolor{rpTypeEvidence}{specific example}s or \textcolor{rpTypeEvidence}{anecdote}s are present even \textcolor{rpTypeRule}{if} the essay is marred by grammatical errors or placeholders.\par - Do not let surface-level \textcolor{rpTypeEvidence}{repetition} or weak phrasing obscure counting of distinct supports-explicitly count and label supports.\par - Reserve the top score (6) for essays that either add analytical depth (counterargument, synthesis) OR supply broad, distinct, and specific development across 3+ independent reasons with cohesive \textcolor{rpTypeWriting}{organization}.\par - Be stricter about distinctiveness of specifics \textcolor{rpTypeRule}{when} moving between 4, 5, and 6-require different content/context for each counted specific support.
\end{tcolorbox}
\end{minipage}
\caption{Pattern-highlighted rubric comparison (asap\_1, openai\_gpt-5-mini, base\_expert\_True\_train100\_iteration5\_top3\_bs4-8-12\_mc4). Matched spans are color-coded by regex pattern. Color types: \textcolor{rpTypeRule}{\textbf{Rule Structure}} (if/threshold/stepwise guidance); \textcolor{rpTypeEvidence}{\textbf{Evidence Handling}} (examples, repetition, and caps); \textcolor{rpTypeWriting}{\textbf{Writing Quality}} (organization and grammar/mechanics).}
\label{fig:rubric_pattern_asap_1_openai_gpt_5_mini_base_expert_True_train100_iteration5_top3_bs4_8_12_mc4}
\end{figure*}
