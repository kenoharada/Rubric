\colorlet{rpTypeRule}{red!80!black}
\colorlet{rpTypeEvidence}{blue!80!black}
\colorlet{rpTypeWriting}{teal!80!black}
\begin{figure*}[t]
\centering
\begin{tcolorbox}[colback=white,colframe=black!25,title=Pattern Type Guide,fonttitle=\bfseries\small,fontupper=\scriptsize,boxsep=1pt,left=2pt,right=2pt,top=2pt,bottom=2pt]
\textcolor{rpTypeRule}{\textbf{Rule Structure}}: Explicit decision logic for scoring: conditional branches, boundary tie-breakers, stepwise workflows, and numeric thresholds.\par \textcolor{rpTypeEvidence}{\textbf{Evidence Handling}}: How evidence is validated and counted: specific-example requirements, repetition/non-double-count rules, and cap rules for weak evidence.\par \textcolor{rpTypeWriting}{\textbf{Writing Quality}}: Language-quality criteria affecting score bands: organization/coherence/transition quality and grammar/mechanics severity.
\end{tcolorbox}
\vspace{1mm}
\begin{tcolorbox}[colback=white,colframe=black!25,title=Detailed Pattern Notes,fonttitle=\bfseries\small,fontupper=\scriptsize,boxsep=1pt,left=2pt,right=2pt,top=2pt,bottom=2pt]
\textcolor{rpTypeRule}{\textbf{Rule Structure}}:\par \quad \textcolor{rpTypeRule}{\textbf{Conditional Gating}} [n=26] Captures explicit condition-based rules that switch decisions only when a stated condition is met. Typical cues: if, when.\par \quad \textcolor{rpTypeRule}{\textbf{Boundary / Tie-Break Guidance}} [n=7] Marks criteria used to resolve borderline cases between adjacent score bands (e.g., 4 vs 5). Typical cues: tie-break, borderline, boundary, threshold, 4 vs 5.\par \quad \textcolor{rpTypeRule}{\textbf{Stepwise Rating Workflow}} [n=4] Detects ordered procedures and checklists that standardize how raters walk through scoring decisions. Typical cues: step, checklist, workflow, procedure, first/second/third.\par \quad \textcolor{rpTypeRule}{\textbf{Anti-Mechanical Counting}} [n=1] Finds rules that prevent naive counting and require qualitative judgment before assigning higher scores. Typical cues: do not count, not mechanically, do not equate.\par \quad \textcolor{rpTypeRule}{\textbf{Quantitative Threshold}} [n=12] Marks numeric cutoffs used for consistent decisions (minimum/maximum counts, percentages, explicit count thresholds). Typical cues: at least, at most, <=, >=, \%, N reasons/examples/sentences/words.\par \textcolor{rpTypeEvidence}{\textbf{Evidence Handling}}:\par \quad \textcolor{rpTypeEvidence}{\textbf{Specific Evidence Requirement}} [n=26] Highlights demands for concrete examples and explicit evidence links instead of generic assertions. Typical cues: for example, e.g., specific example, illustration, anecdote, evidence.\par \quad \textcolor{rpTypeEvidence}{\textbf{Off-Topic / Summary Cap}} [n=1] Identifies cap rules that restrict scores when responses are off-topic, irrelevant, or dominated by summary-only content. Typical cues: off-topic, irrelevant, digression, summary-only, cap.\par \quad \textcolor{rpTypeEvidence}{\textbf{Repetition Non-Count Rule}} [n=8] Captures rules that treat repetition/restatement as non-distinct support and prevent double-counting. Typical cues: repetition, restatement, double-count, do not double-count.\par \textcolor{rpTypeWriting}{\textbf{Writing Quality}}:\par \quad \textcolor{rpTypeWriting}{\textbf{Organization / Coherence Signal}} [n=40] Detects explicit references to discourse structure and logical flow as scoring criteria. Typical cues: organization, coherence, logical flow, transition.\par \quad \textcolor{rpTypeWriting}{\textbf{Grammar / Mechanics Signal}} [n=2] Detects references to language-form quality, especially grammar, spelling, punctuation, and mechanics. Typical cues: grammar, mechanics, spelling, punctuation.
\end{tcolorbox}
\vspace{1mm}
\begin{tcolorbox}[colback=white,colframe=black!25,title=Optimized Rubric (Pattern-Highlighted),fonttitle=\bfseries\small,fontupper=\scriptsize]
\ttfamily
- Do NOT deduct points for named-entity placeholders (PERSON, LOCATION, NUM, PERCENT, etc.). Treat them as neutral substitutions for real details; evaluate the presence, clarity, and specificity of ideas rather than the literal labels. Count placeholders as specific details \textcolor{rpTypeRule}{\textbf{when}} the writer clearly intends a concrete fact, example, or \textcolor{rpTypeEvidence}{\textbf{anecdote}}.\par ... [4 lines omitted] ...\par 3. \textcolor{rpTypeWriting}{\textbf{Organization}}/\textcolor{rpTypeWriting}{\textbf{coherence}} (intro-body-conclusion, paragraphing, \textcolor{rpTypeWriting}{\textbf{transition}}s).\par ... [7 lines omitted] ...\par   c. Specific elaboration/example or clear personal \textcolor{rpTypeEvidence}{\textbf{anecdote}}/data -> "specific".\par - Treat personal \textcolor{rpTypeEvidence}{\textbf{anecdote}}s and clearly-intended placeholders-as-\textcolor{rpTypeEvidence}{\textbf{evidence}} as valid "specific".\par - \textcolor{rpTypeEvidence}{\textbf{Do not double-count}} repeated \textcolor{rpTypeEvidence}{\textbf{restatement}}s of the same reason. Different examples that support the same reason count as strengthening that one reason (do not convert them into separate reasons unless they support a genuinely different claim).\par - \textcolor{rpTypeRule}{\textbf{When}} in doubt about whether two supports are distinct reasons or sub-points of the same reason, prefer to count them as the same reason unless they address different effects, audiences, or mechanisms.\par ... [3 lines omitted] ...\par - Development: Few or no reasons; \textcolor{rpTypeRule}{\textbf{if}} present they are list-only or \textcolor{rpTypeEvidence}{\textbf{irrelevant}}. No meaningful examples or elaboration.\par - \textcolor{rpTypeWriting}{\textbf{Organization}} \& \textcolor{rpTypeWriting}{\textbf{coherence}}: Fragmented, chaotic, or extremely hard to follow; may be one or two disjointed sentences.\par - Language: \textcolor{rpTypeWriting}{\textbf{Grammar}} and usage may prevent comprehension.\par - Use \textcolor{rpTypeRule}{\textbf{when}} the essay essentially fails to form an argument or provide any supporting content.\par ... [4 lines omitted] ...\par - \textcolor{rpTypeWriting}{\textbf{Organization}} \& \textcolor{rpTypeWriting}{\textbf{coherence}}: Little or no logical \textcolor{rpTypeWriting}{\textbf{organization}}; \textcolor{rpTypeWriting}{\textbf{transition}}s absent and sequencing is weak.\par ... [1 lines omitted] ...\par - Use \textcolor{rpTypeRule}{\textbf{when}} the response is more than a sentence or two but lacks development, explanation, and clear structure.\par ... [1 lines omitted] ...\par Score Point 3 - "Minimally developed / some \textcolor{rpTypeWriting}{\textbf{organization}}"\par ... [2 lines omitted] ...\par   - 1-2 distinct reasons where \textcolor{rpTypeRule}{\textbf{at least}} one has min-elab; or\par   - 2+ reasons but most are list-only or repetitive \textcolor{rpTypeEvidence}{\textbf{restatement}}s.\par   - May include one brief \textcolor{rpTypeEvidence}{\textbf{specific example}} or \textcolor{rpTypeEvidence}{\textbf{anecdote}}, but it is not well developed or persuasive.\par - \textcolor{rpTypeWriting}{\textbf{Organization}} \& \textcolor{rpTypeWriting}{\textbf{coherence}}: Some sense of \textcolor{rpTypeWriting}{\textbf{organization}} (intro, body, conclusion or paragraphing) though progression may be weak; limited \textcolor{rpTypeWriting}{\textbf{transition}}s.\par ... [1 lines omitted] ...\par - Use \textcolor{rpTypeRule}{\textbf{when}} the essay demonstrates a clear stance and rudimentary structure with limited support and few specific, distinct examples.\par ... [3 lines omitted] ...\par - Development: Offers adequately elaborated reasons with a mix of general and specific details/examples. Typical \textcolor{rpTypeRule}{\textbf{threshold}}:\par   - \textcolor{rpTypeRule}{\textbf{At least}} 2 distinct reasons each with \textcolor{rpTypeRule}{\textbf{at least}} min-elab AND \textcolor{rpTypeRule}{\textbf{at least}} one clear \textcolor{rpTypeEvidence}{\textbf{specific example}} supporting any one of the reasons; OR\par   - 2-3 distinct reasons with mostly min-elab development and \textcolor{rpTypeRule}{\textbf{at least}} one specific that meaningfully strengthens the argument.\par - \textcolor{rpTypeWriting}{\textbf{Organization}} \& \textcolor{rpTypeWriting}{\textbf{coherence}}: Satisfactory \textcolor{rpTypeWriting}{\textbf{organization}} with clear paragraphing and some \textcolor{rpTypeWriting}{\textbf{transition}}s; readers can follow the argument.\par ... [1 lines omitted] ...\par - \textcolor{rpTypeRule}{\textbf{Tie-break}}er to promote consistency: \textcolor{rpTypeRule}{\textbf{If}} an essay has 2+ distinct reasons each with min-elab and \textcolor{rpTypeRule}{\textbf{at least}} one clearly functioning specific (including placeholders or brief \textcolor{rpTypeEvidence}{\textbf{anecdote}}s), prefer 4 over 3-even \textcolor{rpTypeRule}{\textbf{when}} language is weak.\par ... [4 lines omitted] ...\par   - 3+ distinct reasons with \textcolor{rpTypeRule}{\textbf{at least}} two being "specific" examples/\textcolor{rpTypeEvidence}{\textbf{anecdote}}s; OR\par ... [1 lines omitted] ...\par   - 2+ reasons plus multiple concrete personal \textcolor{rpTypeEvidence}{\textbf{anecdote}}s or data points that combine to make the argument persuasive.\par - \textcolor{rpTypeWriting}{\textbf{Organization}} \& \textcolor{rpTypeWriting}{\textbf{coherence}}: Generally strong \textcolor{rpTypeWriting}{\textbf{organization}} and logical progression; effective paragraphing and \textcolor{rpTypeWriting}{\textbf{transition}}al language throughout.\par ... [1 lines omitted] ...\par - Important constraint to reduce over-scoring: Do NOT award a 5 \textcolor{rpTypeRule}{\textbf{if}} the "specific" supports are repetitive \textcolor{rpTypeEvidence}{\textbf{restatement}}s or the same example reused to pad counts. Multiple \textcolor{rpTypeEvidence}{\textbf{specific example}}s must be distinct in content or context (different data points, different \textcolor{rpTypeEvidence}{\textbf{anecdote}}s, different illustrative scenarios).\par ... [2 lines omitted] ...\par - Position: Takes a thoughtful, nuanced, and compelling position that goes beyond the obvious OR demonstrates exceptional development through multiple distinct specifics and strong \textcolor{rpTypeWriting}{\textbf{organization}}.\par ... [2 lines omitted] ...\par   - The essay contains 3+ distinct reasons each supported by clear, separate "specific" examples (not merely repeated \textcolor{rpTypeEvidence}{\textbf{restatement}}s), combined with cohesive \textcolor{rpTypeWriting}{\textbf{organization}} and some synthesis (linking reasons, explaining implications) even \textcolor{rpTypeRule}{\textbf{if}} explicit counterargument is brief or implicit.\par - \textcolor{rpTypeWriting}{\textbf{Organization}} \& \textcolor{rpTypeWriting}{\textbf{coherence}}: Strong, logical \textcolor{rpTypeWriting}{\textbf{organization}} with clear, effective \textcolor{rpTypeWriting}{\textbf{transition}}s and paragraphing; the argument builds cohesively.\par ... [2 lines omitted] ...\par - Use 6 \textcolor{rpTypeRule}{\textbf{when}} the essay demonstrates either clear analytic depth (counterargument, trade-offs, synthesis) OR very strong breadth and specificity of development (three distinct, well-supported reasons) plus clear \textcolor{rpTypeWriting}{\textbf{organization}}.\par ... [4 lines omitted] ...\par    - 2+ adequately elaborated reasons (min-elab) with \textcolor{rpTypeRule}{\textbf{at least}} one specific -> lean 4.\par    - 3+ distinct reasons with 2+ \textcolor{rpTypeEvidence}{\textbf{specific example}}s/\textcolor{rpTypeEvidence}{\textbf{anecdote}}s -> lean 5; \textcolor{rpTypeRule}{\textbf{if}} those 3+ specifics are present and \textcolor{rpTypeWriting}{\textbf{organization}} is strong, consider 6 (see 6's alternate path).\par    - 2 strongly specific, well-connected reasons with varied \textcolor{rpTypeEvidence}{\textbf{evidence}} or a clear rebuttal -> can justify 5.\par ... [2 lines omitted] ...\par    - Frequent mechanical errors should lower the fluency descriptor but should not automatically move a piece from 4/5/6 down to 2/3 \textcolor{rpTypeRule}{\textbf{if}} the essay contains clear \textcolor{rpTypeWriting}{\textbf{organization}} and multiple \textcolor{rpTypeEvidence}{\textbf{specific example}}s.\par    - However, severe breakdowns in \textcolor{rpTypeWriting}{\textbf{grammar}} that impede comprehension of key supports should lower the score.\par 3. Treat personal \textcolor{rpTypeEvidence}{\textbf{anecdote}}s and placeholder-based "studies" as valid \textcolor{rpTypeEvidence}{\textbf{evidence}}:\par    - \textcolor{rpTypeRule}{\textbf{If}} the writer provides a personal story or clearly intended study/example (even with placeholders), count it as a "specific" example for development-unless the placeholder is so vague that it does not function as \textcolor{rpTypeEvidence}{\textbf{evidence}}.\par 4. Distinguish \textcolor{rpTypeEvidence}{\textbf{repetition}} vs. distinct \textcolor{rpTypeEvidence}{\textbf{evidence}}:\par    - \textcolor{rpTypeEvidence}{\textbf{Repetition}} or rephrasing of the same example should not be counted as multiple specifics.\par    - Different contexts or different concrete examples (even \textcolor{rpTypeRule}{\textbf{if}} they support the same general reason) strengthen that reason but \textcolor{rpTypeRule}{\textbf{do not count}} as additional distinct reasons.\par    - To move from 4->5 using the "3+ reasons" route, ensure the \textcolor{rpTypeRule}{\textbf{third}} reason is independent (addresses a different effect or mechanism) and has a \textcolor{rpTypeEvidence}{\textbf{specific example}}.\par 5. Use \textcolor{rpTypeWriting}{\textbf{organization}} to resolve close calls:\par    - Clear intro/body/conclusion and logical paragraphing can raise a \textcolor{rpTypeRule}{\textbf{borderline}} 3 to a 4 even \textcolor{rpTypeRule}{\textbf{when}} details are modest.\par    - Conversely, poor \textcolor{rpTypeWriting}{\textbf{organization}} can keep a richly supported response from reaching 6 \textcolor{rpTypeRule}{\textbf{if}} the argument fails to cohere.\par 6. Holistic \textcolor{rpTypeRule}{\textbf{tie-break}}ers (final arbitration):\par    - \textcolor{rpTypeRule}{\textbf{If}} features point to different scores, prioritize in order: (a) specificity \& number of supporting details (b) clarity of position (c) \textcolor{rpTypeWriting}{\textbf{organization}}/cohesion.\par    - \textcolor{rpTypeRule}{\textbf{When}} in doubt between 4 and 5, count \textcolor{rpTypeEvidence}{\textbf{specific example}}s carefully-require distinctiveness and substantive support. \textcolor{rpTypeRule}{\textbf{If}} you count 2 full- strength specifics (distinct content) and either a \textcolor{rpTypeRule}{\textbf{third}} reason or \textcolor{rpTypeRule}{\textbf{second}} reason with strong specifics, prefer 5.\par    - \textcolor{rpTypeRule}{\textbf{When}} in doubt between 5 and 6, require either explicit nuance/counterargument OR 3+ distinct reasons each with clear specifics AND strong \textcolor{rpTypeWriting}{\textbf{organization}} for 6.\par 7. Examples of \textcolor{rpTypeRule}{\textbf{boundary}} judgments (updated heuristics):\par ... [1 lines omitted] ...\par    - Clear stance, 1-\textcolor{rpTypeRule}{\textbf{2 reasons}} with brief examples or \textcolor{rpTypeEvidence}{\textbf{anecdote}}s; some \textcolor{rpTypeWriting}{\textbf{organization}} -> Score 3.\par    - Clear stance, 2+ distinct reasons each with \textcolor{rpTypeRule}{\textbf{at least}} some elaboration and \textcolor{rpTypeRule}{\textbf{at least}} one \textcolor{rpTypeEvidence}{\textbf{specific example}} -> Score 4.\par    - Clear stance, either (a) 3+ reasons with mostly specific, relevant examples (distinct and non-repetitive) OR (b) \textcolor{rpTypeRule}{\textbf{2 reasons}} each with substantial specific elaboration and persuasive flow -> Score 5.\par ... [2 lines omitted] ...\par Practical scoring \textcolor{rpTypeRule}{\textbf{checklist}} (use \textcolor{rpTypeRule}{\textbf{when}} assigning a score):\par ... [4 lines omitted] ...\par - Evaluate \textcolor{rpTypeWriting}{\textbf{organization}}: clear paragraphs and \textcolor{rpTypeWriting}{\textbf{transition}}s? (Yes raises \textcolor{rpTypeRule}{\textbf{borderline}} 3->4)\par ... [1 lines omitted] ...\par - Check language: are errors limiting comprehension? (\textcolor{rpTypeRule}{\textbf{If}} comprehension fails, lower to 1-2; otherwise, do not heavily penalize development)\par - Apply \textcolor{rpTypeRule}{\textbf{tie-break}}er rules above (prioritize specificity, then clarity, then \textcolor{rpTypeWriting}{\textbf{organization}}).\par ... [2 lines omitted] ...\par - Increase scores \textcolor{rpTypeRule}{\textbf{when}} multiple distinct, \textcolor{rpTypeEvidence}{\textbf{specific example}}s or \textcolor{rpTypeEvidence}{\textbf{anecdote}}s are present even \textcolor{rpTypeRule}{\textbf{if}} the essay is marred by grammatical errors or placeholders.\par - Do not let surface-level \textcolor{rpTypeEvidence}{\textbf{repetition}} or weak phrasing obscure counting of distinct supports-explicitly count and label supports.\par - Reserve the top score (6) for essays that either add analytical depth (counterargument, synthesis) OR supply broad, distinct, and specific development across 3+ independent reasons with cohesive \textcolor{rpTypeWriting}{\textbf{organization}}.\par - Be stricter about distinctiveness of specifics \textcolor{rpTypeRule}{\textbf{when}} moving between 4, 5, and 6-require different content/context for each counted specific support.
\end{tcolorbox}
\caption{Pattern-focused view of the optimized rubric (asap\_1, openai\_gpt-5-mini, base\_expert\_True\_train100\_iteration5\_top3\_bs4-8-12\_mc4). Colored bold spans indicate regex-matched rubric cues. Color types: \textcolor{rpTypeRule}{\textbf{Rule Structure}} (Explicit decision logic for scoring: conditional branches, boundary tie-breakers, stepwise workflows, and numeric thresholds.); \textcolor{rpTypeEvidence}{\textbf{Evidence Handling}} (How evidence is validated and counted: specific-example requirements, repetition/non-double-count rules, and cap rules for weak evidence.); \textcolor{rpTypeWriting}{\textbf{Writing Quality}} (Language-quality criteria affecting score bands: organization/coherence/transition quality and grammar/mechanics severity.). Matched pattern categories: Conditional Gating (n=26); Boundary / Tie-Break Guidance (n=7); Stepwise Rating Workflow (n=4); Anti-Mechanical Counting (n=1); Specific Evidence Requirement (n=26); Off-Topic / Summary Cap (n=1); Organization / Coherence Signal (n=40); Grammar / Mechanics Signal (n=2); Repetition Non-Count Rule (n=8); Quantitative Threshold (n=12).}
\label{fig:rubric_pattern_asap_1_openai_gpt_5_mini_base_expert_True_train100_iteration5_top3_bs4_8_12_mc4}
\end{figure*}
