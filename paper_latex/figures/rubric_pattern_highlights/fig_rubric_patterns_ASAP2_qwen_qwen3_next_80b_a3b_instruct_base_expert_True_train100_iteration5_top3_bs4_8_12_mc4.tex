\colorlet{rpTypeRule}{red!80!black}
\colorlet{rpTypeEvidence}{blue!80!black}
\colorlet{rpTypeWriting}{teal!80!black}
\begin{figure*}[t]
\centering
\begin{tcolorbox}[colback=white,colframe=black!25,title=Pattern Legend,fonttitle=\bfseries\small,fontupper=\scriptsize,boxsep=1pt,left=2pt,right=2pt,top=2pt,bottom=2pt]
\textcolor{rpTypeRule}{\textbf{Rule Structure}} (if/threshold/stepwise guidance) \quad \textcolor{rpTypeEvidence}{\textbf{Evidence Handling}} (examples, repetition, and caps) \quad \textcolor{rpTypeWriting}{\textbf{Writing Quality}} (organization and grammar/mechanics)
\end{tcolorbox}
\vspace{2mm}
\begin{minipage}[t]{0.485\textwidth}
\begin{tcolorbox}[colback=white,colframe=black!25,title=Initial Rubric,fonttitle=\bfseries\small,fontupper=\scriptsize,breakable]
\ttfamily
After reading each essay and completing the analytical rating form, assign a holistic score based on the rubric below. For the following evaluations you will need to use a grading scale between 1 (minimum) and 6 (maximum). As with the analytical rating form, the distance between each grade (\textcolor{rpTypeEvidence}{e.g.}, 1-2, 3-4, 4-5) should be considered equal.\par \par SCORE OF 6: An essay in this category demonstrates clear and consistent mastery, although it may have a few minor errors. A typical essay effectively and insightfully develops a point of view on the issue and demonstrates outstanding critical thinking; the essay uses clearly appropriate examples, reasons, and other \textcolor{rpTypeEvidence}{evidence} taken from the source text(s) to support its position; the essay is well organized and clearly focused, demonstrating clear \textcolor{rpTypeWriting}{coherence} and smooth progression of ideas; the essay exhibits skillful use of language, using a varied, accurate, and apt vocabulary and demonstrates meaningful variety in sentence structure; the essay is free of most errors in \textcolor{rpTypeWriting}{grammar}, usage, and \textcolor{rpTypeWriting}{mechanics}.\par \par SCORE OF 5: An essay in this category demonstrates reasonably consistent mastery, although it will have occasional errors or lapses in quality. A typical essay effectively develops a point of view on the issue and demonstrates strong critical thinking; the essay generally using appropriate examples, reasons, and other \textcolor{rpTypeEvidence}{evidence} taken from the source text(s) to support its position; the essay is well organized and focused, demonstrating \textcolor{rpTypeWriting}{coherence} and progression of ideas; the essay exhibits facility in the use of language, using appropriate vocabulary demonstrates variety in sentence structure; the essay is generally free of most errors in \textcolor{rpTypeWriting}{grammar}, usage, and \textcolor{rpTypeWriting}{mechanics}.\par \par SCORE OF 4: An essay in this category demonstrates adequate mastery, although it will have lapses in quality. A typical essay develops a point of view on the issue and demonstrates competent critical thinking; the essay using adequate examples, reasons, and other \textcolor{rpTypeEvidence}{evidence} taken from the source text(s) to support its position; the essay is generally organized and focused, demonstrating some \textcolor{rpTypeWriting}{coherence} and progression of ideas exhibits adequate; the essay may demonstrate inconsistent facility in the use of language, using generally appropriate vocabulary demonstrates some variety in sentence structure; the essay may have some errors in \textcolor{rpTypeWriting}{grammar}, usage, and \textcolor{rpTypeWriting}{mechanics}.\par \par SCORE OF 3: An essay in this category demonstrates developing mastery, and is marked by ONE OR MORE of the following weaknesses: develops a point of view on the issue, demonstrating some critical thinking, but may do so inconsistently or use inadequate examples, reasons, or other \textcolor{rpTypeEvidence}{evidence} taken from the source texts to support its position; the essay is limited in its \textcolor{rpTypeWriting}{organization} or focus, or may demonstrate some lapses in \textcolor{rpTypeWriting}{coherence} or progression of ideas displays; the essay may demonstrate facility in the use of language, but sometimes uses weak vocabulary or inappropriate word choice and/or lacks variety or demonstrates problems in sentence structure; the essay may contain an accumulation of errors in \textcolor{rpTypeWriting}{grammar}, usage, and \textcolor{rpTypeWriting}{mechanics}.\par \par SCORE OF 2: An essay in this category demonstrates little mastery, and is flawed by ONE OR MORE of the following weaknesses: develops a point of view on the issue that is vague or seriously limited, and demonstrates weak critical thinking; the essay providesinappropriate or insufficient examples, reasons, or other \textcolor{rpTypeEvidence}{evidence} taken from the source text to support its position; the essay is poorly organized and/or focused, or demonstrates serious problems with \textcolor{rpTypeWriting}{coherence} or progression of ideas; the essay displays very little facility in the use of language, using very limited vocabulary or incorrect word choice and/or demonstrates frequent problems in sentence structure; the essay contains errors in \textcolor{rpTypeWriting}{grammar}, usage, and \textcolor{rpTypeWriting}{mechanics} so serious that meaning is somewhat obscured.\par \par SCORE OF 1: An essay in this category demonstrates very little or no mastery, and is severely flawed by ONE OR MORE of the following weaknesses: develops no viable point of view on the issue, or provides little or no \textcolor{rpTypeEvidence}{evidence} to support its position; the essay is disorganized or unfocused, resulting in a disjointed orincoherent essay; the essay displays fundamental errors in vocabulary and/or demonstrates severe flaws in sentence structure; the essay contains pervasive errors in \textcolor{rpTypeWriting}{grammar}, usage, or \textcolor{rpTypeWriting}{mechanics} that persistently interfere with meaning.
\end{tcolorbox}
\end{minipage}
\hfill
\begin{minipage}[t]{0.485\textwidth}
\begin{tcolorbox}[colback=white,colframe=black!25,title=Optimized Rubric,fonttitle=\bfseries\small,fontupper=\scriptsize,breakable]
\ttfamily
After reading each essay and completing the analytical rating form, assign a holistic score based on the rubric below. For the following evaluations you will need to use a grading scale between 1 (minimum) and 6 (maximum). As with the analytical rating form, the distance between each grade (\textcolor{rpTypeEvidence}{e.g.}, 1-2, 3-4, 4-5) should be considered equal.\par \par SCORE OF 6: An essay in this category demonstrates clear and consistent mastery, although it may have a few minor errors. A typical essay effectively and insightfully develops a point of view on the issue and demonstrates outstanding critical thinking; the essay uses clearly appropriate examples, reasons, and other \textcolor{rpTypeEvidence}{evidence} taken from the source text(s) to support its position; the essay is well organized and clearly focused, demonstrating clear \textcolor{rpTypeWriting}{coherence} and smooth progression of ideas; the essay exhibits skillful use of language, using a varied, accurate, and apt vocabulary and demonstrates meaningful variety in sentence structure; the essay is free of most errors in \textcolor{rpTypeWriting}{grammar}, usage, and \textcolor{rpTypeWriting}{mechanics}. A score of 6 requires the essay to be a fully developed, original argument that engages deeply with the source material-not merely summarizing, restating, or superficially agreeing with it. The argument must show independent insight, synthesis of ideas, and a distinctive voice that goes beyond paraphrasing the source. The essay must not only cite \textcolor{rpTypeEvidence}{evidence} but interpret its significance, connect multiple ideas across the text, and offer a novel perspective that could not be derived solely from the source. Crucially, a score of 6 demands that the insight be substantive, original, and central to the argument-not merely implied, tangential, or rephrased from the source. Essays that accurately summarize the source, correctly identify its claims, and even correctly critique its reasoning, but fail to generate a new interpretive framework, unexpected connection, or distinctive evaluative stance, do not qualify for a 6.\par \par SCORE OF 5: An essay in this category demonstrates reasonably consistent mastery, although it will have occasional errors or lapses in quality. A typical essay effectively develops a point of view on the issue and demonstrates strong critical thinking; the essay generally uses appropriate examples, reasons, and other \textcolor{rpTypeEvidence}{evidence} taken from the source text(s) to support its position; the essay is well organized and focused, demonstrating \textcolor{rpTypeWriting}{coherence} and progression of ideas; the essay exhibits facility in the use of language, using appropriate vocabulary and demonstrating variety in sentence structure; the essay is generally free of most errors in \textcolor{rpTypeWriting}{grammar}, usage, and \textcolor{rpTypeWriting}{mechanics}. A score of 5 requires the essay to present a clear, substantive position with thoughtful development and \textcolor{rpTypeEvidence}{evidence}-based reasoning that goes beyond simple agreement or summary. The essay must interpret, connect, or extend ideas from the source-\textcolor{rpTypeEvidence}{e.g.}, explaining why a detail matters, comparing multiple claims, evaluating implications, or identifying an unstated assumption-not merely listing facts or repeating the author's claims. Language errors must be infrequent and never obscure meaning. Essays that summarize the source effectively but fail to add interpretive depth, even with good \textcolor{rpTypeWriting}{organization} and language, do not qualify for a 5. Do not award a 5 \textcolor{rpTypeRule}{if} the essay's analysis is dominated by paraphrasing or \textcolor{rpTypeRule}{if} the insight is implied rather than explicitly developed. Importantly, \textcolor{rpTypeRule}{if} the essay's position directly contradicts the source's intent (\textcolor{rpTypeEvidence}{e.g.}, arguing the author's point is invalid \textcolor{rpTypeRule}{when} the author is clearly advocating for it), the essay may still earn a 5 \textcolor{rpTypeRule}{if} it offers a well-supported, coherent, and analytically rich counter-argument grounded in the text.\par \par SCORE OF 4: An essay in this category demonstrates adequate mastery, although it will have lapses in quality. A typical essay develops a point of view on the issue and demonstrates competent critical thinking; the essay uses adequate examples, reasons, and other \textcolor{rpTypeEvidence}{evidence} taken from the source text(s) to support its position; the essay is generally organized and focused, demonstrating some \textcolor{rpTypeWriting}{coherence} and progression of ideas; the essay may demonstrate inconsistent facility in the use of language, using generally appropriate vocabulary and demonstrating some variety in sentence structure; the essay may have some errors in \textcolor{rpTypeWriting}{grammar}, usage, and \textcolor{rpTypeWriting}{mechanics}. A score of 4 requires the essay to assert a discernible position that responds directly to the prompt and to go beyond mere description or summary by offering \textcolor{rpTypeRule}{at least} one clear, developed interpretive insight-not just a single phrase or label. The essay must explain why the \textcolor{rpTypeEvidence}{evidence} matters, how ideas relate, or what the implications are-not just state that the author is "for" or "against" something. \textcolor{rpTypeEvidence}{For example}, saying "the author shows Venus is dangerous" is summary; saying "the author uses Venus's extreme conditions to argue that human ingenuity, not advanced tech, may be the key to exploration" is analysis. Essays that rely heavily on paraphrasing the source without adding analysis, or that repeat the same point without development, should not exceed a score of 3. Language errors may be noticeable but must not prevent understanding of the argument. \textcolor{rpTypeRule}{If} the essay's analysis is superficial (\textcolor{rpTypeEvidence}{e.g.}, "the author doesn't support his claim well enough") without explaining why or how, or \textcolor{rpTypeRule}{if} the evaluation merely repeats source content with no added insight, it should be scored 3 or lower. Do not award a 4 \textcolor{rpTypeRule}{if} the essay's position is vague, its \textcolor{rpTypeEvidence}{evidence} is disconnected, or its insight is buried in summary. Crucially, essays that misinterpret the source (\textcolor{rpTypeEvidence}{e.g.}, claiming the author uses "suspense" \textcolor{rpTypeRule}{when} the text is factual) but still offer a clear, developed interpretation of their own misunderstanding may still receive a 4 \textcolor{rpTypeRule}{if} the analysis is internally consistent and grounded in the text as they read it.\par \par SCORE OF 3: An essay in this category demonstrates developing mastery, and is marked by ONE OR MORE of the following weaknesses: develops a point of view on the issue, demonstrating some critical thinking, but may do so inconsistently or use inadequate examples, reasons, or other \textcolor{rpTypeEvidence}{evidence} taken from the source texts to support its position; the essay is limited in its \textcolor{rpTypeWriting}{organization} or focus, or may demonstrate some lapses in \textcolor{rpTypeWriting}{coherence} or progression of ideas; the essay may demonstrate facility in the use of language, but sometimes uses weak vocabulary or inappropriate word choice and/or lacks variety or demonstrates problems in sentence structure; the essay may contain an accumulation of errors in \textcolor{rpTypeWriting}{grammar}, usage, and \textcolor{rpTypeWriting}{mechanics}. A score of 3 is appropriate \textcolor{rpTypeRule}{when} the essay attempts to take a position but struggles to sustain it, or \textcolor{rpTypeRule}{when} the position is implied rather than clearly stated. Language errors may be frequent but do not consistently obscure meaning. Essays that are primarily descriptive, summarize the source without asserting a clear claim of their own, or that show only surface-level engagement (\textcolor{rpTypeEvidence}{e.g.}, restating source claims without analysis) may still receive a 3 \textcolor{rpTypeRule}{if} they show emerging critical thinking, such as a tentative connection or a single interpretive insight. However, \textcolor{rpTypeRule}{if} the essay is dominated by summary, contains no discernible independent claim, or is so confused in structure that the argument cannot be reconstructed, it must be scored lower. Importantly, essays that merely label the author's argument as "weak" or "not well supported" without explaining the basis for that judgment, or that repeat source phrases verbatim without interpretation, should not receive a 3 unless they contain \textcolor{rpTypeRule}{at least} one clear, albeit underdeveloped, evaluative insight. Do not award a 3 \textcolor{rpTypeRule}{if} the essay contains no discernible position or \textcolor{rpTypeRule}{if} the only "analysis" is restating source claims with minor rewording. Essays that misinterpret the source (\textcolor{rpTypeEvidence}{e.g.}, confusing "suspense" for factual description) but still attempt to build a coherent, \textcolor{rpTypeRule}{if} flawed, argument from that misreading may qualify for a 3 \textcolor{rpTypeRule}{if} the reasoning is internally consistent and shows an attempt to interpret-not just report-the text.\par \par SCORE OF 2: An essay in this category demonstrates little mastery, and is flawed by ONE OR MORE of the following weaknesses: develops a point of view on the issue that is vague or seriously limited, and demonstrates weak critical thinking; the essay provides inappropriate or insufficient examples, reasons, or other \textcolor{rpTypeEvidence}{evidence} taken from the source text to support its position; the essay is poorly organized and/or focused, or demonstrates serious problems with \textcolor{rpTypeWriting}{coherence} or progression of ideas; the essay displays very little facility in the use of language, using very limited vocabulary or incorrect word choice and/or demonstrates frequent problems in sentence structure; the essay contains errors in \textcolor{rpTypeWriting}{grammar}, usage, and \textcolor{rpTypeWriting}{mechanics} so serious that meaning is somewhat obscured. A score of 2 is reserved for essays that either fail to articulate a clear position, rely almost entirely on summary without analysis, or are so linguistically flawed that understanding is significantly impaired-even \textcolor{rpTypeRule}{if} some relevant content is present. Essays that repeat source phrases without interpretation, contain no original claim, and are marred by persistent grammatical errors that require effort to decode should be scored 2. Do not award a 3 \textcolor{rpTypeRule}{if} the essay does not demonstrate even minimal critical thinking or an attempt to move beyond summary. \textcolor{rpTypeRule}{If} the essay's language is so fragmented, misspelled, or syntactically broken that the intended argument must be inferred with difficulty, and the analysis is nonexistent or reduced to isolated phrases (\textcolor{rpTypeEvidence}{e.g.}, "it shows that scientists are studying"), it belongs in this category. \textcolor{rpTypeRule}{If} the essay's only "point" is a vague \textcolor{rpTypeEvidence}{restatement} of the prompt ("Venus is dangerous but interesting") without any explanation, it is a 2. Crucially, essays that misinterpret the source but offer no coherent argument or insight-even \textcolor{rpTypeRule}{if} they use source phrases-should be scored 2 \textcolor{rpTypeRule}{if} the reasoning cannot be reconstructed into a meaningful claim.\par \par SCORE OF 1: An essay in this category demonstrates very little or no mastery, and is severely flawed by ONE OR MORE of the following weaknesses: develops no viable point of view on the issue, or provides little or no \textcolor{rpTypeEvidence}{evidence} to support its position; the essay is disorganized or unfocused, resulting in a disjointed or incoherent essay; the essay displays fundamental errors in vocabulary and/or demonstrates severe flaws in sentence structure; the essay contains pervasive errors in \textcolor{rpTypeWriting}{grammar}, usage, or \textcolor{rpTypeWriting}{mechanics} that persistently interfere with meaning. A score of 1 is reserved for essays that are essentially non-responsive: they may be entirely \textcolor{rpTypeEvidence}{off-topic}, contain no discernible argument, or are so garbled by errors that no coherent meaning can be extracted-even \textcolor{rpTypeRule}{if} isolated phrases reference the source. Essays that are unintelligible, contain no identifiable stance, or consist only of fragmented phrases with no logical connection should receive a 1. \textcolor{rpTypeRule}{If} the essay contains only incoherent \textcolor{rpTypeEvidence}{repetition}s of source phrases, grammatical errors that prevent identification of any claim, or no discernible structure beyond a string of disconnected sentences, it must be scored 1-even \textcolor{rpTypeRule}{if} keywords from the prompt appear.
\end{tcolorbox}
\end{minipage}
\caption{Pattern-highlighted rubric comparison (ASAP2, qwen\_qwen3-next-80b-a3b-instruct, base\_expert\_True\_train100\_iteration5\_top3\_bs4-8-12\_mc4). Matched spans are color-coded by regex pattern. Color types: \textcolor{rpTypeRule}{\textbf{Rule Structure}} (if/threshold/stepwise guidance); \textcolor{rpTypeEvidence}{\textbf{Evidence Handling}} (examples, repetition, and caps); \textcolor{rpTypeWriting}{\textbf{Writing Quality}} (organization and grammar/mechanics).}
\label{fig:rubric_pattern_ASAP2_qwen_qwen3_next_80b_a3b_instruct_base_expert_True_train100_iteration5_top3_bs4_8_12_mc4}
\end{figure*}
