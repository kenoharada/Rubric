\colorlet{rpTypeRule}{red!80!black}
\colorlet{rpTypeEvidence}{blue!80!black}
\colorlet{rpTypeWriting}{teal!80!black}
\begin{figure*}[t]
\centering
\begin{tcolorbox}[colback=white,colframe=black!25,title=Pattern Type Guide,fonttitle=\bfseries\small,fontupper=\scriptsize,boxsep=1pt,left=2pt,right=2pt,top=2pt,bottom=2pt]
\textcolor{rpTypeRule}{\textbf{Rule Structure}}: Explicit decision logic for scoring: conditional branches, boundary tie-breakers, stepwise workflows, and numeric thresholds.\par \textcolor{rpTypeEvidence}{\textbf{Evidence Handling}}: How evidence is validated and counted: specific-example requirements, repetition/non-double-count rules, and cap rules for weak evidence.\par \textcolor{rpTypeWriting}{\textbf{Writing Quality}}: Language-quality criteria affecting score bands: organization/coherence/transition quality and grammar/mechanics severity.
\end{tcolorbox}
\vspace{1mm}
\begin{tcolorbox}[colback=white,colframe=black!25,title=Detailed Pattern Notes,fonttitle=\bfseries\small,fontupper=\scriptsize,boxsep=1pt,left=2pt,right=2pt,top=2pt,bottom=2pt]
\textcolor{rpTypeRule}{\textbf{Rule Structure}}:\par \quad \textcolor{rpTypeRule}{\textbf{Conditional Gating}} [n=1] Captures explicit condition-based rules that switch decisions only when a stated condition is met. Typical cues: if, when.\par \textcolor{rpTypeEvidence}{\textbf{Evidence Handling}}:\par \quad \textcolor{rpTypeEvidence}{\textbf{Specific Evidence Requirement}} [n=8] Highlights demands for concrete examples and explicit evidence links instead of generic assertions. Typical cues: for example, e.g., specific example, illustration, anecdote, evidence.\par \textcolor{rpTypeWriting}{\textbf{Writing Quality}}:\par \quad \textcolor{rpTypeWriting}{\textbf{Organization / Coherence Signal}} [n=3] Detects explicit references to discourse structure and logical flow as scoring criteria. Typical cues: organization, coherence, logical flow, transition.\par \quad \textcolor{rpTypeWriting}{\textbf{Grammar / Mechanics Signal}} [n=3] Detects references to language-form quality, especially grammar, spelling, punctuation, and mechanics. Typical cues: grammar, mechanics, spelling, punctuation.
\end{tcolorbox}
\vspace{1mm}
\begin{tcolorbox}[colback=white,colframe=black!25,title=Optimized Rubric (Pattern-Highlighted),fonttitle=\bfseries\small,fontupper=\scriptsize]
\ttfamily
1 - Minimal/Inadequate: The response provides little to no original input. It may consist of a few isolated facts, a wall of direct quotes without connecting thought, or be so brief and error-ridden (\textcolor{rpTypeWriting}{\textbf{punctuation}}, capitalization, \textcolor{rpTypeWriting}{\textbf{spelling}}) that it remains undeveloped or incoherent. Reasoning is often logically flawed or absent. Even \textcolor{rpTypeRule}{\textbf{if}} it attempts to follow the prompt, the lack of an independent summary or a clear voice places it here.\par ... [1 lines omitted] ...\par 2 - Limited: The response shows a basic understanding but is primarily a chronological, paragraph-by-paragraph retelling or a list-like collection of facts and reactions. It relies heavily on the source's phrasing and \textcolor{rpTypeWriting}{\textbf{organization}}al structure. While there may be a hint of an evaluative stance or a simple conclusion, the "analysis" is almost entirely composed of strings of quotes. It lacks the cohesive structure of a functional summary.\par ... [1 lines omitted] ...\par 3 - Functional: The response demonstrates a clear understanding and provides an independent summary that captures the essence of the text rather than just listing facts. It follows a logical structure (usually the source's chrono\textcolor{rpTypeWriting}{\textbf{logical flow}}). The student's voice is present in connecting ideas, though the reasoning may be simplistic or naive. It uses relevant \textcolor{rpTypeEvidence}{\textbf{evidence}} but does not yet effectively group that \textcolor{rpTypeEvidence}{\textbf{evidence}} by theme or concept (\textcolor{rpTypeEvidence}{\textbf{e.g.}}, it summarizes the text from beginning to end rather than by "pros vs. cons").\par ... [1 lines omitted] ...\par 4 - Developing: The response provides a clear position and a complete, purposeful structure (intro, body, conclusion). The defining characteristic is the shift from a chronological retelling to a thematic grouping of \textcolor{rpTypeEvidence}{\textbf{evidence}} (\textcolor{rpTypeEvidence}{\textbf{e.g.}}, categorizing by "hazards," "geology," or "solutions"). While the student's own \textcolor{rpTypeWriting}{\textbf{organization}}al intent drives the response, it may still rely heavily on direct quotes to fill out the themes, contain repetitive points, or feature simplistic analysis. \par ... [1 lines omitted] ...\par 5 - Advanced: The response provides a strong, purposeful analysis where the student's voice and thematic structure guide the response entirely. It consistently and effectively categorizes different types of \textcolor{rpTypeEvidence}{\textbf{evidence}} to explain how the author supports their claim. At this level, the depth of synthesis and the logical progression of the argument outweigh frequent mechanical or \textcolor{rpTypeWriting}{\textbf{spelling}} errors. The response demonstrates a clear command of how the \textcolor{rpTypeEvidence}{\textbf{evidence}} functions to support the author's broader purpose.\par ... [1 lines omitted] ...\par 6 - Sophisticated: The response takes a critical, analytical stance, evaluating the rhetorical effectiveness of the text (\textcolor{rpTypeEvidence}{\textbf{e.g.}}, analyzing diction, tone, or logical contradictions). It is highly effective and well-organized, demonstrating deep insight into the nuances of the author's argument. It moves beyond simple categorization to synthesize information with high-level proficiency. Minor mechanical errors do not detract from the sophisticated evaluation and synthesis.
\end{tcolorbox}
\caption{Pattern-focused view of the optimized rubric (ASAP2, google\_gemini-3-flash-preview, base\_simplest\_True\_train100\_iteration5\_top3\_bs4-8-12\_mc4). Colored bold spans indicate regex-matched rubric cues. Color types: \textcolor{rpTypeRule}{\textbf{Rule Structure}} (Explicit decision logic for scoring: conditional branches, boundary tie-breakers, stepwise workflows, and numeric thresholds.); \textcolor{rpTypeEvidence}{\textbf{Evidence Handling}} (How evidence is validated and counted: specific-example requirements, repetition/non-double-count rules, and cap rules for weak evidence.); \textcolor{rpTypeWriting}{\textbf{Writing Quality}} (Language-quality criteria affecting score bands: organization/coherence/transition quality and grammar/mechanics severity.). Matched pattern categories: Conditional Gating (n=1); Specific Evidence Requirement (n=8); Organization / Coherence Signal (n=3); Grammar / Mechanics Signal (n=3).}
\label{fig:rubric_pattern_ASAP2_google_gemini_3_flash_preview_base_simplest_True_train100_iteration5_top3_bs4_8_12_mc4}
\end{figure*}
