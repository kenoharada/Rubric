\colorlet{rpTypeRule}{red!80!black}
\colorlet{rpTypeEvidence}{blue!80!black}
\colorlet{rpTypeWriting}{teal!80!black}
\begin{figure*}[t]
\centering
\begin{tcolorbox}[colback=white,colframe=black!25,title=Pattern Legend,fonttitle=\bfseries\small,fontupper=\scriptsize,boxsep=1pt,left=2pt,right=2pt,top=2pt,bottom=2pt]
\textcolor{rpTypeRule}{\textbf{Rule Structure}} (if/threshold/stepwise guidance) \quad \textcolor{rpTypeEvidence}{\textbf{Evidence Handling}} (examples, repetition, and caps) \quad \textcolor{rpTypeWriting}{\textbf{Writing Quality}} (organization and grammar/mechanics)
\end{tcolorbox}
\vspace{2mm}
\begin{minipage}[t]{0.485\textwidth}
\begin{tcolorbox}[colback=white,colframe=black!25,title=Initial Rubric,fonttitle=\bfseries\small,fontupper=\scriptsize,breakable]
\ttfamily
Based on the response's content, rate the response on a scale of 1 to 6.
\end{tcolorbox}
\end{minipage}
\hfill
\begin{minipage}[t]{0.485\textwidth}
\begin{tcolorbox}[colback=white,colframe=black!25,title=Optimized Rubric,fonttitle=\bfseries\small,fontupper=\scriptsize,breakable]
\ttfamily
\par 1 - Minimal/Inadequate: The response provides little to no original input. It may consist of a few isolated facts, a wall of direct quotes without connecting thought, or be so brief and error-ridden (\textcolor{rpTypeWriting}{punctuation}, capitalization, \textcolor{rpTypeWriting}{spelling}) that it remains undeveloped or incoherent. Reasoning is often logically flawed or absent. Even \textcolor{rpTypeRule}{if} it attempts to follow the prompt, the lack of an independent summary or a clear voice places it here.\par \par 2 - Limited: The response shows a basic understanding but is primarily a chronological, paragraph-by-paragraph retelling or a list-like collection of facts and reactions. It relies heavily on the source's phrasing and \textcolor{rpTypeWriting}{organization}al structure. While there may be a hint of an evaluative stance or a simple conclusion, the "analysis" is almost entirely composed of strings of quotes. It lacks the cohesive structure of a functional summary.\par \par 3 - Functional: The response demonstrates a clear understanding and provides an independent summary that captures the essence of the text rather than just listing facts. It follows a logical structure (usually the source's chrono\textcolor{rpTypeWriting}{logical flow}). The student's voice is present in connecting ideas, though the reasoning may be simplistic or naive. It uses relevant \textcolor{rpTypeEvidence}{evidence} but does not yet effectively group that \textcolor{rpTypeEvidence}{evidence} by theme or concept (\textcolor{rpTypeEvidence}{e.g.}, it summarizes the text from beginning to end rather than by "pros vs. cons").\par \par 4 - Developing: The response provides a clear position and a complete, purposeful structure (intro, body, conclusion). The defining characteristic is the shift from a chronological retelling to a thematic grouping of \textcolor{rpTypeEvidence}{evidence} (\textcolor{rpTypeEvidence}{e.g.}, categorizing by "hazards," "geology," or "solutions"). While the student's own \textcolor{rpTypeWriting}{organization}al intent drives the response, it may still rely heavily on direct quotes to fill out the themes, contain repetitive points, or feature simplistic analysis. \par \par 5 - Advanced: The response provides a strong, purposeful analysis where the student's voice and thematic structure guide the response entirely. It consistently and effectively categorizes different types of \textcolor{rpTypeEvidence}{evidence} to explain how the author supports their claim. At this level, the depth of synthesis and the logical progression of the argument outweigh frequent mechanical or \textcolor{rpTypeWriting}{spelling} errors. The response demonstrates a clear command of how the \textcolor{rpTypeEvidence}{evidence} functions to support the author's broader purpose.\par \par 6 - Sophisticated: The response takes a critical, analytical stance, evaluating the rhetorical effectiveness of the text (\textcolor{rpTypeEvidence}{e.g.}, analyzing diction, tone, or logical contradictions). It is highly effective and well-organized, demonstrating deep insight into the nuances of the author's argument. It moves beyond simple categorization to synthesize information with high-level proficiency. Minor mechanical errors do not detract from the sophisticated evaluation and synthesis.
\end{tcolorbox}
\end{minipage}
\caption{Pattern-highlighted rubric comparison (ASAP2, google\_gemini-3-flash-preview, base\_simplest\_True\_train100\_iteration5\_top3\_bs4-8-12\_mc4). Matched spans are color-coded by regex pattern. Color types: \textcolor{rpTypeRule}{\textbf{Rule Structure}} (if/threshold/stepwise guidance); \textcolor{rpTypeEvidence}{\textbf{Evidence Handling}} (examples, repetition, and caps); \textcolor{rpTypeWriting}{\textbf{Writing Quality}} (organization and grammar/mechanics).}
\label{fig:rubric_pattern_ASAP2_google_gemini_3_flash_preview_base_simplest_True_train100_iteration5_top3_bs4_8_12_mc4}
\end{figure*}
