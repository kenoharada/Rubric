\colorlet{rpTypeRule}{red!80!black}
\colorlet{rpTypeEvidence}{blue!80!black}
\colorlet{rpTypeWriting}{teal!80!black}
\begin{figure*}[t]
\centering
\begin{tcolorbox}[colback=white,colframe=black!25,title=Pattern Type Guide,fonttitle=\bfseries\small,fontupper=\scriptsize,boxsep=1pt,left=2pt,right=2pt,top=2pt,bottom=2pt]
\textcolor{rpTypeRule}{\textbf{Rule Structure}}: Explicit decision logic for scoring: conditional branches, boundary tie-breakers, stepwise workflows, and numeric thresholds.\par \textcolor{rpTypeEvidence}{\textbf{Evidence Handling}}: How evidence is validated and counted: specific-example requirements, repetition/non-double-count rules, and cap rules for weak evidence.\par \textcolor{rpTypeWriting}{\textbf{Writing Quality}}: Language-quality criteria affecting score bands: organization/coherence/transition quality and grammar/mechanics severity.
\end{tcolorbox}
\vspace{1mm}
\begin{tcolorbox}[colback=white,colframe=black!25,title=Detailed Pattern Notes,fonttitle=\bfseries\small,fontupper=\scriptsize,boxsep=1pt,left=2pt,right=2pt,top=2pt,bottom=2pt]
\textcolor{rpTypeRule}{\textbf{Rule Structure}}:\par \quad \textcolor{rpTypeRule}{\textbf{Conditional Gating}} [n=5] Captures explicit condition-based rules that switch decisions only when a stated condition is met. Typical cues: if, when.\par \quad \textcolor{rpTypeRule}{\textbf{Stepwise Rating Workflow}} [n=1] Detects ordered procedures and checklists that standardize how raters walk through scoring decisions. Typical cues: step, checklist, workflow, procedure, first/second/third.\par \textcolor{rpTypeEvidence}{\textbf{Evidence Handling}}:\par \quad \textcolor{rpTypeEvidence}{\textbf{Specific Evidence Requirement}} [n=9] Highlights demands for concrete examples and explicit evidence links instead of generic assertions. Typical cues: for example, e.g., specific example, illustration, anecdote, evidence.\par \quad \textcolor{rpTypeEvidence}{\textbf{Off-Topic / Summary Cap}} [n=1] Identifies cap rules that restrict scores when responses are off-topic, irrelevant, or dominated by summary-only content. Typical cues: off-topic, irrelevant, digression, summary-only, cap.\par \textcolor{rpTypeWriting}{\textbf{Writing Quality}}:\par \quad \textcolor{rpTypeWriting}{\textbf{Organization / Coherence Signal}} [n=3] Detects explicit references to discourse structure and logical flow as scoring criteria. Typical cues: organization, coherence, logical flow, transition.\par \quad \textcolor{rpTypeWriting}{\textbf{Grammar / Mechanics Signal}} [n=3] Detects references to language-form quality, especially grammar, spelling, punctuation, and mechanics. Typical cues: grammar, mechanics, spelling, punctuation.
\end{tcolorbox}
\vspace{1mm}
\begin{tcolorbox}[colback=white,colframe=black!25,title=Optimized Rubric (Pattern-Highlighted),fonttitle=\bfseries\small,fontupper=\scriptsize]
\ttfamily
Score the essay on a scale of 1 to 6 based on the following criteria. The primary determinants of the score are the writer's ability to address the prompt with original thought, the depth/thickness of supporting details, and the clarity of the \textcolor{rpTypeWriting}{\textbf{organization}}al structure.\par ... [1 lines omitted] ...\par CRITICAL SCORING PRINCIPLE: Surface-level errors in \textcolor{rpTypeWriting}{\textbf{spelling}}, \textcolor{rpTypeWriting}{\textbf{grammar}}, syntax, and \textcolor{rpTypeWriting}{\textbf{punctuation}}-even \textcolor{rpTypeRule}{\textbf{if}} frequent, severe, or making the text difficult to read-should NOT prevent a high score (4, 5, or 6) \textcolor{rpTypeRule}{\textbf{if}} the logic is discernible and the content is developed. Focus on the "voice" and the richness of the \textcolor{rpTypeEvidence}{\textbf{evidence}} provided rather than mechanical accuracy.\par ... [1 lines omitted] ...\par - 6 (Superior): The response provides a sophisticated, persuasive argument that goes beyond standard responses by offering unique perspectives or creative logic (\textcolor{rpTypeEvidence}{\textbf{e.g.}}, reframing a common disadvantage into a situational advantage or providing vivid, sensory imagery). It is highly organized with a clear, rhythmic progression of ideas. It is distinguished by "authoritative" voice and layered, original supporting details that feel authentic/personal rather than just meeting a length requirement.\par ... [1 lines omitted] ...\par - 5 (Strong): The response takes a clear stance and is well-developed with "thick" body paragraphs. It is distinguished from a 4 by its use of varied \textcolor{rpTypeWriting}{\textbf{transition}}s and nuanced reasoning (\textcolor{rpTypeEvidence}{\textbf{e.g.}}, comparing digital experiences to physical reality or discussing long-term psychological impacts). While it may rely on more common arguments, it supports them with extensive personal \textcolor{rpTypeEvidence}{\textbf{anecdote}}s or specific, varied examples that show a depth of reflection beyond a simple list.\par ... [1 lines omitted] ...\par - 4 (Competent): The response is the baseline for a "successful" essay. It addresses the prompt with a clear opinion and a discernible structure. To earn a 4, the writer MUST provide some original expansion or \textcolor{rpTypeEvidence}{\textbf{specific example}}s (\textcolor{rpTypeEvidence}{\textbf{e.g.}}, a specific personal story, naming a specific website, or a unique hypothetical scenario). Even \textcolor{rpTypeRule}{\textbf{if}} the language is "broken," \textcolor{rpTypeRule}{\textbf{if}} the writer moves beyond simply repeating/listing the prompt's ideas and provides a clear "why" or "how" for their points, it earns a 4.\par ... [1 lines omitted] ...\par - 3 (Developing): The response is limited and feels "thin." While it may have an introduction, body, and conclusion, it relies heavily on repeating the prompt's own language or listing common reasons (\textcolor{rpTypeEvidence}{\textbf{e.g.}}, "you can talk to friends," "it helps with homework") without providing original details or unique \textcolor{rpTypeEvidence}{\textbf{evidence}}. A 3 often feels like a \textcolor{rpTypeRule}{\textbf{checklist}} of the prompt's suggestions. \textcolor{rpTypeRule}{\textbf{If}} the essay is long but repetitive or lacks specific, original \textcolor{rpTypeEvidence}{\textbf{anecdote}}s, it stays at a 3.\par ... [1 lines omitted] ...\par - 2 (Limited): The response shows minimal control of language and \textcolor{rpTypeWriting}{\textbf{organization}}. Ideas are thin, highly fragmented, or consist of only a few repetitive sentences. It fails to build a coherent argument. It may be very short or comprise a list of disjointed thoughts that barely move beyond the prompt's own words.\par ... [1 lines omitted] ...\par - 1 (Inadequate): The response is \textcolor{rpTypeEvidence}{\textbf{off-topic}}, too brief to evaluate, or largely unintelligible due to a total lack of language control that prevents any logic from emerging.
\end{tcolorbox}
\caption{Pattern-focused view of the optimized rubric (asap\_1, google\_gemini-3-flash-preview, base\_simplest\_True\_train100\_iteration5\_top3\_bs4-8-12\_mc4). Colored bold spans indicate regex-matched rubric cues. Color types: \textcolor{rpTypeRule}{\textbf{Rule Structure}} (Explicit decision logic for scoring: conditional branches, boundary tie-breakers, stepwise workflows, and numeric thresholds.); \textcolor{rpTypeEvidence}{\textbf{Evidence Handling}} (How evidence is validated and counted: specific-example requirements, repetition/non-double-count rules, and cap rules for weak evidence.); \textcolor{rpTypeWriting}{\textbf{Writing Quality}} (Language-quality criteria affecting score bands: organization/coherence/transition quality and grammar/mechanics severity.). Matched pattern categories: Conditional Gating (n=5); Stepwise Rating Workflow (n=1); Specific Evidence Requirement (n=9); Off-Topic / Summary Cap (n=1); Organization / Coherence Signal (n=3); Grammar / Mechanics Signal (n=3).}
\label{fig:rubric_pattern_asap_1_google_gemini_3_flash_preview_base_simplest_True_train100_iteration5_top3_bs4_8_12_mc4}
\end{figure*}
