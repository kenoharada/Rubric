\colorlet{rpTypeRule}{red!80!black}
\colorlet{rpTypeEvidence}{blue!80!black}
\colorlet{rpTypeWriting}{teal!80!black}
\begin{figure*}[t]
\centering
\begin{tcolorbox}[colback=white,colframe=black!25,title=Pattern Legend,fonttitle=\bfseries\small,fontupper=\scriptsize,boxsep=1pt,left=2pt,right=2pt,top=2pt,bottom=2pt]
\textcolor{rpTypeRule}{\textbf{Rule Structure}} (if/threshold/stepwise guidance) \quad \textcolor{rpTypeEvidence}{\textbf{Evidence Handling}} (examples, repetition, and caps) \quad \textcolor{rpTypeWriting}{\textbf{Writing Quality}} (organization and grammar/mechanics)
\end{tcolorbox}
\vspace{2mm}
\begin{minipage}[t]{0.485\textwidth}
\begin{tcolorbox}[colback=white,colframe=black!25,title=Initial Rubric,fonttitle=\bfseries\small,fontupper=\scriptsize,breakable]
\ttfamily
Based on the response's content, rate the response on a scale of 1 to 6.
\end{tcolorbox}
\end{minipage}
\hfill
\begin{minipage}[t]{0.485\textwidth}
\begin{tcolorbox}[colback=white,colframe=black!25,title=Optimized Rubric,fonttitle=\bfseries\small,fontupper=\scriptsize,breakable]
\ttfamily
Score the essay on a scale of 1 to 6 based on the following criteria. The primary determinants of the score are the writer's ability to address the prompt with original thought, the depth/thickness of supporting details, and the clarity of the \textcolor{rpTypeWriting}{organization}al structure.\par \par CRITICAL SCORING PRINCIPLE: Surface-level errors in \textcolor{rpTypeWriting}{spelling}, \textcolor{rpTypeWriting}{grammar}, syntax, and \textcolor{rpTypeWriting}{punctuation}-even \textcolor{rpTypeRule}{if} frequent, severe, or making the text difficult to read-should NOT prevent a high score (4, 5, or 6) \textcolor{rpTypeRule}{if} the logic is discernible and the content is developed. Focus on the "voice" and the richness of the \textcolor{rpTypeEvidence}{evidence} provided rather than mechanical accuracy.\par \par - 6 (Superior): The response provides a sophisticated, persuasive argument that goes beyond standard responses by offering unique perspectives or creative logic (\textcolor{rpTypeEvidence}{e.g.}, reframing a common disadvantage into a situational advantage or providing vivid, sensory imagery). It is highly organized with a clear, rhythmic progression of ideas. It is distinguished by "authoritative" voice and layered, original supporting details that feel authentic/personal rather than just meeting a length requirement.\par \par - 5 (Strong): The response takes a clear stance and is well-developed with "thick" body paragraphs. It is distinguished from a 4 by its use of varied \textcolor{rpTypeWriting}{transition}s and nuanced reasoning (\textcolor{rpTypeEvidence}{e.g.}, comparing digital experiences to physical reality or discussing long-term psychological impacts). While it may rely on more common arguments, it supports them with extensive personal \textcolor{rpTypeEvidence}{anecdote}s or specific, varied examples that show a depth of reflection beyond a simple list.\par \par - 4 (Competent): The response is the baseline for a "successful" essay. It addresses the prompt with a clear opinion and a discernible structure. To earn a 4, the writer MUST provide some original expansion or \textcolor{rpTypeEvidence}{specific example}s (\textcolor{rpTypeEvidence}{e.g.}, a specific personal story, naming a specific website, or a unique hypothetical scenario). Even \textcolor{rpTypeRule}{if} the language is "broken," \textcolor{rpTypeRule}{if} the writer moves beyond simply repeating/listing the prompt's ideas and provides a clear "why" or "how" for their points, it earns a 4.\par \par - 3 (Developing): The response is limited and feels "thin." While it may have an introduction, body, and conclusion, it relies heavily on repeating the prompt's own language or listing common reasons (\textcolor{rpTypeEvidence}{e.g.}, "you can talk to friends," "it helps with homework") without providing original details or unique \textcolor{rpTypeEvidence}{evidence}. A 3 often feels like a \textcolor{rpTypeRule}{checklist} of the prompt's suggestions. \textcolor{rpTypeRule}{If} the essay is long but repetitive or lacks specific, original \textcolor{rpTypeEvidence}{anecdote}s, it stays at a 3.\par \par - 2 (Limited): The response shows minimal control of language and \textcolor{rpTypeWriting}{organization}. Ideas are thin, highly fragmented, or consist of only a few repetitive sentences. It fails to build a coherent argument. It may be very short or comprise a list of disjointed thoughts that barely move beyond the prompt's own words.\par \par - 1 (Inadequate): The response is \textcolor{rpTypeEvidence}{off-topic}, too brief to evaluate, or largely unintelligible due to a total lack of language control that prevents any logic from emerging.
\end{tcolorbox}
\end{minipage}
\caption{Pattern-highlighted rubric comparison (asap\_1, google\_gemini-3-flash-preview, base\_simplest\_True\_train100\_iteration5\_top3\_bs4-8-12\_mc4). Matched spans are color-coded by regex pattern. Color types: \textcolor{rpTypeRule}{\textbf{Rule Structure}} (if/threshold/stepwise guidance); \textcolor{rpTypeEvidence}{\textbf{Evidence Handling}} (examples, repetition, and caps); \textcolor{rpTypeWriting}{\textbf{Writing Quality}} (organization and grammar/mechanics).}
\label{fig:rubric_pattern_asap_1_google_gemini_3_flash_preview_base_simplest_True_train100_iteration5_top3_bs4_8_12_mc4}
\end{figure*}
