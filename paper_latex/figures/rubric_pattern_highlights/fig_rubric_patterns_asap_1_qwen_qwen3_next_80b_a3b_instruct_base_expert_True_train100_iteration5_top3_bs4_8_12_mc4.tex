\colorlet{rpTypeRule}{red!80!black}
\colorlet{rpTypeEvidence}{blue!80!black}
\colorlet{rpTypeWriting}{teal!80!black}
\begin{figure*}[t]
\centering
\begin{tcolorbox}[colback=white,colframe=black!25,title=Pattern Type Guide,fonttitle=\bfseries\small,fontupper=\scriptsize,boxsep=1pt,left=2pt,right=2pt,top=2pt,bottom=2pt]
\textcolor{rpTypeRule}{\textbf{Rule Structure}}: Explicit decision logic for scoring: conditional branches, boundary tie-breakers, stepwise workflows, and numeric thresholds.\par \textcolor{rpTypeEvidence}{\textbf{Evidence Handling}}: How evidence is validated and counted: specific-example requirements, repetition/non-double-count rules, and cap rules for weak evidence.\par \textcolor{rpTypeWriting}{\textbf{Writing Quality}}: Language-quality criteria affecting score bands: organization/coherence/transition quality and grammar/mechanics severity.
\end{tcolorbox}
\vspace{1mm}
\begin{tcolorbox}[colback=white,colframe=black!25,title=Detailed Pattern Notes,fonttitle=\bfseries\small,fontupper=\scriptsize,boxsep=1pt,left=2pt,right=2pt,top=2pt,bottom=2pt]
\textcolor{rpTypeRule}{\textbf{Rule Structure}}:\par \quad \textcolor{rpTypeRule}{\textbf{Conditional Gating}} [n=28] Captures explicit condition-based rules that switch decisions only when a stated condition is met. Typical cues: if, when.\par \quad \textcolor{rpTypeRule}{\textbf{Stepwise Rating Workflow}} [n=2] Detects ordered procedures and checklists that standardize how raters walk through scoring decisions. Typical cues: step, checklist, workflow, procedure, first/second/third.\par \quad \textcolor{rpTypeRule}{\textbf{Quantitative Threshold}} [n=12] Marks numeric cutoffs used for consistent decisions (minimum/maximum counts, percentages, explicit count thresholds). Typical cues: at least, at most, <=, >=, \%, N reasons/examples/sentences/words.\par \textcolor{rpTypeEvidence}{\textbf{Evidence Handling}}:\par \quad \textcolor{rpTypeEvidence}{\textbf{Specific Evidence Requirement}} [n=52] Highlights demands for concrete examples and explicit evidence links instead of generic assertions. Typical cues: for example, e.g., specific example, illustration, anecdote, evidence.\par \quad \textcolor{rpTypeEvidence}{\textbf{Off-Topic / Summary Cap}} [n=1] Identifies cap rules that restrict scores when responses are off-topic, irrelevant, or dominated by summary-only content. Typical cues: off-topic, irrelevant, digression, summary-only, cap.\par \textcolor{rpTypeWriting}{\textbf{Writing Quality}}:\par \quad \textcolor{rpTypeWriting}{\textbf{Organization / Coherence Signal}} [n=17] Detects explicit references to discourse structure and logical flow as scoring criteria. Typical cues: organization, coherence, logical flow, transition.\par \quad \textcolor{rpTypeWriting}{\textbf{Grammar / Mechanics Signal}} [n=4] Detects references to language-form quality, especially grammar, spelling, punctuation, and mechanics. Typical cues: grammar, mechanics, spelling, punctuation.
\end{tcolorbox}
\vspace{1mm}
\begin{tcolorbox}[colback=white,colframe=black!25,title=Optimized Rubric (Pattern-Highlighted),fonttitle=\bfseries\small,fontupper=\scriptsize]
\ttfamily
- Anonymized placeholders (@CAPS, @\textcolor{rpTypeWriting}{\textbf{ORGANIZATION}}, etc.) dominate the text and prevent coherent interpretation of ideas, rendering key claims unverifiable or unintelligible.\par - Shows no consistent awareness of audience or purpose; letter format, \textcolor{rpTypeRule}{\textbf{if}} present, is incorrectly applied or \textcolor{rpTypeEvidence}{\textbf{irrelevant}}.\par ... [1 lines omitted] ...\par - Critical Note: \textcolor{rpTypeRule}{\textbf{If}} anonymized placeholders are randomly inserted and render core claims unintelligible (\textcolor{rpTypeEvidence}{\textbf{e.g.}}, "@CAPS9 he helps you..."), this qualifies as Score 1. However, \textcolor{rpTypeRule}{\textbf{if}} the structure and intent are discernible despite grammatical errors, and placeholders are used as contextual substitutes (even \textcolor{rpTypeRule}{\textbf{if}} imperfect), do not assign Score 1.\par ... [2 lines omitted] ...\par - Contains only broad, general claims with no \textcolor{rpTypeEvidence}{\textbf{specific example}}s or personal context; reasons are listed but not explained.\par - Shows minimal \textcolor{rpTypeWriting}{\textbf{organization}}, with little to no paragraphing or \textcolor{rpTypeWriting}{\textbf{logical flow}}; \textcolor{rpTypeWriting}{\textbf{transition}}s are absent or nonsensical.\par ... [1 lines omitted] ...\par - Anonymized placeholders appear but do not dominate the text; some attempt at audience awareness (\textcolor{rpTypeEvidence}{\textbf{e.g.}}, letter format) is present but ineffective.\par - Ideas are superficial and lack development; no \textcolor{rpTypeEvidence}{\textbf{evidence}} of reflection, analysis, or persuasive intent beyond surface-level statements.\par - Critical Note: \textcolor{rpTypeRule}{\textbf{If}} the essay contains \textcolor{rpTypeRule}{\textbf{at least}} one identifiable personal \textcolor{rpTypeEvidence}{\textbf{anecdote}}, observable behavior, or contextual reference (even \textcolor{rpTypeRule}{\textbf{if}} poorly expressed), and the position is clear, it should not be scored as 1. Score 2 is reserved for responses where the argument is present but entirely unsubstantiated by any concrete detail-even anonymized.\par ... [2 lines omitted] ...\par - Presents 1-\textcolor{rpTypeRule}{\textbf{2 reasons}} with some attempt at elaboration, though details remain general or inconsistently developed.\par - Shows basic \textcolor{rpTypeWriting}{\textbf{organization}}: introduction, body, and conclusion are recognizable, though paragraphing may be weak or uneven.\par - Uses occasional \textcolor{rpTypeEvidence}{\textbf{specific example}}s (\textcolor{rpTypeEvidence}{\textbf{e.g.}}, "I talk to my cousin in Colombia") or anonymized placeholders used contextually as \textcolor{rpTypeEvidence}{\textbf{evidence}} (\textcolor{rpTypeEvidence}{\textbf{e.g.}}, "@PERCENT1 of students," "@PERSON1 says...")-even \textcolor{rpTypeRule}{\textbf{if}} embedded in awkward phrasing or grammatical errors.\par - Shows partial awareness of audience (\textcolor{rpTypeEvidence}{\textbf{e.g.}}, uses letter format, addresses "readers"), but tone is inconsistent or immature.\par - \textcolor{rpTypeWriting}{\textbf{Transition}}al language is sparse or simplistic ("\textcolor{rpTypeRule}{\textbf{first}}," "\textcolor{rpTypeRule}{\textbf{second}}," "last"); fluency is limited but the essay remains readable with effort.\par - Anonymized placeholders are present but do not overwhelm the text; they are used as minor contextual substitutes, not as primary \textcolor{rpTypeEvidence}{\textbf{evidence}}.\par - Critical Note: An essay may still earn Score 3 even with multiple placeholders \textcolor{rpTypeRule}{\textbf{if}} they are embedded in a coherent structure and serve as identifiable, contextually grounded \textcolor{rpTypeEvidence}{\textbf{evidence}} (\textcolor{rpTypeEvidence}{\textbf{e.g.}}, "@PERCENT1 say...", "@PERSON1, a researcher...")-even \textcolor{rpTypeRule}{\textbf{if}} \textcolor{rpTypeWriting}{\textbf{grammar}} is poor. Do not penalize for placeholder density alone; penalize only \textcolor{rpTypeRule}{\textbf{when}} placeholders prevent interpretation of the claim's meaning.\par ... [2 lines omitted] ...\par - Presents 2-\textcolor{rpTypeRule}{\textbf{3 reasons}} with adequate elaboration, combining general claims with \textcolor{rpTypeRule}{\textbf{at least}} one specific, concrete example per reason (\textcolor{rpTypeEvidence}{\textbf{e.g.}}, "I use Facebook to ask about homework," "@PERCENT1 of kids are obese due to screen time").\par - Exhibits satisfactory \textcolor{rpTypeWriting}{\textbf{organization}}: clear structure with topic sentences, logical progression, and a conclusion that restates the position.\par - Uses simple \textcolor{rpTypeWriting}{\textbf{transition}}al language ("\textcolor{rpTypeEvidence}{\textbf{for example}}," "another reason," "in conclusion") consistently, though not always sophisticated.\par - Demonstrates adequate awareness of audience (\textcolor{rpTypeEvidence}{\textbf{e.g.}}, direct address to newspaper readers, appropriate tone for public letter).\par ... [1 lines omitted] ...\par - Anonymized placeholders (@CAPS, @PERCENT, @LOCATION, etc.) are used meaningfully as \textcolor{rpTypeEvidence}{\textbf{evidence}} (\textcolor{rpTypeEvidence}{\textbf{e.g.}}, statistics, expert references) and do not disrupt clarity. Even \textcolor{rpTypeRule}{\textbf{if}} placeholders obscure exact identities, the argument's logic and supporting data remain interpretable and persuasive.\par - Critical Note: Score 4 requires that each reason includes \textcolor{rpTypeRule}{\textbf{at least}} one instance of concrete support-whether personal, observational, or anonymized. \textcolor{rpTypeRule}{\textbf{If}} the essay has 2-\textcolor{rpTypeRule}{\textbf{3 reasons}}, each with one clear placeholder-supported example (\textcolor{rpTypeEvidence}{\textbf{e.g.}}, "@PERSON1 says...", "@PERCENT1 of users..."), and the structure is logically organized, it qualifies for Score 4-even with numerous grammatical errors. Do not downgrade for language \textcolor{rpTypeRule}{\textbf{if}} the argument's logic and \textcolor{rpTypeEvidence}{\textbf{evidence}} are intact. Additionally, \textcolor{rpTypeRule}{\textbf{if}} the essay uses rhetorical questions, direct appeals to the reader, or emotional language to strengthen persuasion (\textcolor{rpTypeEvidence}{\textbf{e.g.}}, "Don't you love playing outdoor games?"), this demonstrates persuasive intent beyond basic claims and supports a Score 4-even \textcolor{rpTypeRule}{\textbf{if}} examples are not highly detailed.\par ... [2 lines omitted] ...\par - Presents 3+ well-elaborated reasons with mostly specific, relevant, and varied details (\textcolor{rpTypeEvidence}{\textbf{e.g.}}, personal \textcolor{rpTypeEvidence}{\textbf{anecdote}}s, observable behaviors, anonymized statistics used meaningfully).\par - Exhibits strong \textcolor{rpTypeWriting}{\textbf{organization}}: paragraphs are focused, ideas flow logically, and \textcolor{rpTypeWriting}{\textbf{transition}}s are varied and purposeful (\textcolor{rpTypeEvidence}{\textbf{e.g.}}, "furthermore," "conversely," "as a result").\par ... [1 lines omitted] ...\par - Shows consistent and thoughtful awareness of audience: tone is persuasive, respectful, and appropriate for a newspaper letter; rhetorical strategies (\textcolor{rpTypeEvidence}{\textbf{e.g.}}, rhetorical questions, direct appeals, emotional language) enhance persuasion.\par - Anonymized placeholders are integrated naturally and do not impede clarity or credibility; they serve as valid, context-appropriate \textcolor{rpTypeEvidence}{\textbf{evidence}} (\textcolor{rpTypeEvidence}{\textbf{e.g.}}, "@PERCENT1 of students," "@PERSON1, a researcher at @\textcolor{rpTypeWriting}{\textbf{ORGANIZATION}}1") and are treated as credible substitutes, not distractions.\par - Critical Note: Score 5 is awarded \textcolor{rpTypeRule}{\textbf{when}} the essay demonstrates persuasive intent beyond basic claims-using rhetorical devices, varied \textcolor{rpTypeEvidence}{\textbf{evidence}}, and consistent tone-even \textcolor{rpTypeRule}{\textbf{if}} language is imperfect. Do not require flawless \textcolor{rpTypeWriting}{\textbf{grammar}}. \textcolor{rpTypeRule}{\textbf{If}} anonymized \textcolor{rpTypeEvidence}{\textbf{evidence}} is used repeatedly and meaningfully (\textcolor{rpTypeEvidence}{\textbf{e.g.}}, multiple expert quotes, statistics, location-based observations), and the structure is cohesive, it qualifies for Score 5. A single, well-placed placeholder (\textcolor{rpTypeEvidence}{\textbf{e.g.}}, "@PERSON1 says...") is not enough for Score 5; multiple credible placeholders OR a mix of personal and anonymized \textcolor{rpTypeEvidence}{\textbf{evidence}} across reasons are required. Crucially, essays that use emotional appeals, direct audience engagement, or vivid imagery-even with grammatical flaws-are eligible for Score 5 \textcolor{rpTypeRule}{\textbf{if}} they show layered, intentional support across all reasons. Additionally, to qualify for Score 5, the essay must demonstrate \textcolor{rpTypeRule}{\textbf{at least}} one of the following: (1) a clear contrast or counterpoint acknowledged and addressed, (2) a compelling call to action with emotional weight, or (3) multiple distinct types of \textcolor{rpTypeEvidence}{\textbf{evidence}} (\textcolor{rpTypeEvidence}{\textbf{e.g.}}, one personal \textcolor{rpTypeEvidence}{\textbf{anecdote}} + one statistic + one expert attribution). Do not award Score 5 for merely having many placeholders; they must be meaningfully woven into a persuasive, multi-layered argument.\par ... [2 lines omitted] ...\par - Presents fully elaborated reasons with rich, specific, and insightful details that demonstrate deep understanding of the issue (\textcolor{rpTypeEvidence}{\textbf{e.g.}}, nuanced analysis of social consequences, balanced acknowledgment of counterpoints).\par - Exhibits strong, cohesive \textcolor{rpTypeWriting}{\textbf{organization}}: introduction establishes context and stakes, body paragraphs build logically, and conclusion offers a compelling call to action or broader insight.\par - Is fluent and polished: language is precise, varied, and sophisticated; \textcolor{rpTypeWriting}{\textbf{transition}}s are seamless and enhance rhetorical effect.\par ... [1 lines omitted] ...\par - Anonymized placeholders are used appropriately and do not detract from the argument's strength or clarity; the essay reads as \textcolor{rpTypeRule}{\textbf{if}} it were written without redaction, with placeholders functioning as natural, credible stand-ins for real-world \textcolor{rpTypeEvidence}{\textbf{evidence}}.\par - Critical Note: Score 6 requires sophistication in both reasoning and expression. While placeholders may be present, they must be seamlessly integrated as authoritative \textcolor{rpTypeEvidence}{\textbf{evidence}} (\textcolor{rpTypeEvidence}{\textbf{e.g.}}, "@\textcolor{rpTypeWriting}{\textbf{ORGANIZATION}}1's 2023 report shows...") and the essay must demonstrate a level of rhetorical control and insight that goes beyond mere adequacy. A single strong example is insufficient; multiple layers of \textcolor{rpTypeEvidence}{\textbf{evidence}} and nuanced analysis are required. The essay must not only persuade but also reflect depth of thought-such as recognizing complexity, anticipating objections, or connecting the issue to broader societal values. Placeholders must feel like intentional, credible substitutions, not merely convenient fillers.\par ... [2 lines omitted] ...\par I have made an effort to remove personally identifying information from the essays using the Named Entity Recognizer (NER). The relevant entities are identified in the text and then replaced with a string such as "PERSON", "\textcolor{rpTypeWriting}{\textbf{ORGANIZATION}}", "LOCATION", "DATE", "TIME", "MONEY", "PERCENT", "CAPS" (any capitalized word) and "NUM" (any digits). Please do not penalize the essay because of the anonymizations. \textcolor{rpTypeRule}{\textbf{When}} evaluating, assess whether anonymized placeholders are used in a way that:\par ... [2 lines omitted] ...\par - Are integrated meaningfully (\textcolor{rpTypeEvidence}{\textbf{e.g.}}, "@PERCENT1 of students" used as \textcolor{rpTypeEvidence}{\textbf{evidence}}, "@PERSON1, an expert at @\textcolor{rpTypeWriting}{\textbf{ORGANIZATION}}1" cited as authority) -> treat as valid support (Score 4-6)\par ... [2 lines omitted] ...\par - Do not penalize grammatical errors, awkward phrasing, or minor mis\textcolor{rpTypeWriting}{\textbf{spelling}}s \textcolor{rpTypeRule}{\textbf{if}} the core argument, structure, and use of \textcolor{rpTypeEvidence}{\textbf{evidence}} remain clear and persuasive. The presence of anonymized placeholders should not automatically downgrade an essay; instead, evaluate whether the placeholders enable or obstruct the development of the argument.\par - An essay may still earn Score 5 or 6 even with numerous placeholders \textcolor{rpTypeRule}{\textbf{if}} they are used as credible, contextually grounded \textcolor{rpTypeEvidence}{\textbf{evidence}} (\textcolor{rpTypeEvidence}{\textbf{e.g.}}, statistics, expert attributions) and the reasoning, \textcolor{rpTypeWriting}{\textbf{organization}}, and tone meet the higher-level criteria.\par ... [1 lines omitted] ...\par - Score 4 requires \textcolor{rpTypeRule}{\textbf{at least}} one concrete example per reason, even \textcolor{rpTypeRule}{\textbf{if}} anonymized. Score 5 and 6 require multiple specific, varied examples and \textcolor{rpTypeEvidence}{\textbf{evidence}} that feel intentional and persuasive, not merely inserted.\par - Rhetorical questions, appeals to emotion, direct audience engagement, and vivid imagery are signs of persuasive intent and should be rewarded at Score 4 and above, even \textcolor{rpTypeRule}{\textbf{if}} language is imperfect.\par - Key Revision: \textcolor{rpTypeRule}{\textbf{If}} the essay presents 2-\textcolor{rpTypeRule}{\textbf{3 reasons}}, each supported by \textcolor{rpTypeRule}{\textbf{at least}} one identifiable example (personal or anonymized), and the structure is recognizable (intro, body, conclusion), it should be scored \textcolor{rpTypeRule}{\textbf{at least}} 3-even with poor \textcolor{rpTypeWriting}{\textbf{grammar}}. Do not score 1 or 2 simply because of placeholder density; score 1 only \textcolor{rpTypeRule}{\textbf{when}} the text is completely unintelligible and no claim can be interpreted. Score 2 requires absence of any concrete support, even anonymized.\par - Critical Addition: For Score 5, the presence of multiple rhetorical devices (\textcolor{rpTypeEvidence}{\textbf{e.g.}}, rhetorical questions, emotional appeals, direct address) combined with \textcolor{rpTypeRule}{\textbf{at least}} two distinct types of \textcolor{rpTypeEvidence}{\textbf{evidence}} (\textcolor{rpTypeEvidence}{\textbf{e.g.}}, one personal observation + one anonymized statistic) across the reasons is sufficient-even \textcolor{rpTypeRule}{\textbf{if}} examples are not highly detailed or polished. Do not require flawless expression; prioritize persuasive intent and layered support.\par - Critical Addition: To earn Score 5, the essay must demonstrate not only multiple examples but also \textcolor{rpTypeEvidence}{\textbf{evidence}} of persuasive strategy beyond listing claims-such as a call to action, emotional resonance, acknowledgment of counterarguments, or vivid imagery that deepens the reader's connection to the issue. A Score 4 essay may have adequate support; a Score 5 essay makes the reader feel something or reconsider their view.\par - Critical Addition: For Score 6, the essay must show a level of rhetorical maturity that transforms \textcolor{rpTypeEvidence}{\textbf{evidence}} into insight. It should not just report effects but interpret them-\textcolor{rpTypeEvidence}{\textbf{e.g.}}, "Computers don't just isolate us; they rewire our expectations of human connection." Placeholders must feel like natural, authoritative anchors, not redactions.
\end{tcolorbox}
\caption{Pattern-focused view of the optimized rubric (asap\_1, qwen\_qwen3-next-80b-a3b-instruct, base\_expert\_True\_train100\_iteration5\_top3\_bs4-8-12\_mc4). Colored bold spans indicate regex-matched rubric cues. Color types: \textcolor{rpTypeRule}{\textbf{Rule Structure}} (Explicit decision logic for scoring: conditional branches, boundary tie-breakers, stepwise workflows, and numeric thresholds.); \textcolor{rpTypeEvidence}{\textbf{Evidence Handling}} (How evidence is validated and counted: specific-example requirements, repetition/non-double-count rules, and cap rules for weak evidence.); \textcolor{rpTypeWriting}{\textbf{Writing Quality}} (Language-quality criteria affecting score bands: organization/coherence/transition quality and grammar/mechanics severity.). Matched pattern categories: Conditional Gating (n=28); Stepwise Rating Workflow (n=2); Specific Evidence Requirement (n=52); Off-Topic / Summary Cap (n=1); Organization / Coherence Signal (n=17); Grammar / Mechanics Signal (n=4); Quantitative Threshold (n=12).}
\label{fig:rubric_pattern_asap_1_qwen_qwen3_next_80b_a3b_instruct_base_expert_True_train100_iteration5_top3_bs4_8_12_mc4}
\end{figure*}
