\colorlet{rpTypeRule}{red!80!black}
\colorlet{rpTypeEvidence}{blue!80!black}
\colorlet{rpTypeWriting}{teal!80!black}
\begin{figure*}[t]
\centering
\begin{tcolorbox}[colback=white,colframe=black!25,title=Pattern Legend,fonttitle=\bfseries\small,fontupper=\scriptsize,boxsep=1pt,left=2pt,right=2pt,top=2pt,bottom=2pt]
\textcolor{rpTypeRule}{\textbf{Rule Structure}} (if/threshold/stepwise guidance) \quad \textcolor{rpTypeEvidence}{\textbf{Evidence Handling}} (examples, repetition, and caps) \quad \textcolor{rpTypeWriting}{\textbf{Writing Quality}} (organization and grammar/mechanics)
\end{tcolorbox}
\vspace{2mm}
\begin{minipage}[t]{0.485\textwidth}
\begin{tcolorbox}[colback=white,colframe=black!25,title=Initial Rubric,fonttitle=\bfseries\small,fontupper=\scriptsize,breakable]
\ttfamily
- Contains only general reasons with unelaborated and/or list-like details.\par - Shows little or no \textcolor{rpTypeEvidence}{evidence} of \textcolor{rpTypeWriting}{organization}.\par - May be awkward and confused or simplistic.\par ... [3 lines omitted] ...\par - Has reasons with minimal elaboration and more general than specific details.\par - Shows some \textcolor{rpTypeWriting}{organization}.\par - May be awkward in parts with few \textcolor{rpTypeWriting}{transition}s.\par - Shows some awareness of audience.\par ... [2 lines omitted] ...\par - Has adequately elaborated reasons with a mix of general and specific details.\par - Shows satisfactory \textcolor{rpTypeWriting}{organization}.\par - May be somewhat fluent with some \textcolor{rpTypeWriting}{transition}al language.\par - Shows adequate awareness of audience.\par ... [2 lines omitted] ...\par - Has moderately well elaborated reasons with mostly specific details.\par - Exhibits generally strong \textcolor{rpTypeWriting}{organization}.\par - May be moderately fluent with \textcolor{rpTypeWriting}{transition}al language throughout.\par - May show a consistent awareness of audience.\par ... [2 lines omitted] ...\par - Has fully elaborated reasons with specific details.\par - Exhibits strong \textcolor{rpTypeWriting}{organization}.\par - Is fluent and uses sophisticated \textcolor{rpTypeWriting}{transition}al language.\par - May show a heightened awareness of audience.\par ... [1 lines omitted] ...\par Note: \par I have made an effort to remove personally identifying information from the essays using the Named Entity Recognizer (NER). The relevant entities are identified in the text and then replaced with a string such as "PERSON", "\textcolor{rpTypeWriting}{ORGANIZATION}", "LOCATION", "DATE", "TIME", "MONEY", "PERCENT", "CAPS" (any capitalized word) and "NUM" (any digits). Please do not penalize the essay because of the anonymizations.
\end{tcolorbox}
\end{minipage}
\hfill
\begin{minipage}[t]{0.485\textwidth}
\begin{tcolorbox}[colback=white,colframe=black!25,title=Optimized Rubric,fonttitle=\bfseries\small,fontupper=\scriptsize,breakable]
\ttfamily
- Is fragmented, disjointed, or nearly unintelligible; sentences frequently fail to convey clear meaning.\par - Anonymized placeholders (@CAPS, @\textcolor{rpTypeWriting}{ORGANIZATION}, etc.) dominate the text and prevent coherent interpretation of ideas, rendering key claims unverifiable or unintelligible.\par - Shows no consistent awareness of audience or purpose; letter format, \textcolor{rpTypeRule}{if} present, is incorrectly applied or \textcolor{rpTypeEvidence}{irrelevant}.\par - Fails to develop any reason with even basic elaboration; ideas are either absent, contradictory, or buried in confusion.\par - Critical Note: \textcolor{rpTypeRule}{If} anonymized placeholders are randomly inserted and render core claims unintelligible (\textcolor{rpTypeEvidence}{e.g.}, "@CAPS9 he helps you..."), this qualifies as Score 1. However, \textcolor{rpTypeRule}{if} the structure and intent are discernible despite grammatical errors, and placeholders are used as contextual substitutes (even \textcolor{rpTypeRule}{if} imperfect), do not assign Score 1.\par \par Score Point 2: An under-developed response that may or may not take a position. Typical elements:\par - Contains only broad, general claims with no \textcolor{rpTypeEvidence}{specific example}s or personal context; reasons are listed but not explained.\par - Shows minimal \textcolor{rpTypeWriting}{organization}, with little to no paragraphing or \textcolor{rpTypeWriting}{logical flow}; \textcolor{rpTypeWriting}{transition}s are absent or nonsensical.\par - Language is simplistic, repetitive, or confused, with frequent grammatical errors that impede understanding but do not completely obscure meaning.\par - Anonymized placeholders appear but do not dominate the text; some attempt at audience awareness (\textcolor{rpTypeEvidence}{e.g.}, letter format) is present but ineffective.\par - Ideas are superficial and lack development; no \textcolor{rpTypeEvidence}{evidence} of reflection, analysis, or persuasive intent beyond surface-level statements.\par - Critical Note: \textcolor{rpTypeRule}{If} the essay contains \textcolor{rpTypeRule}{at least} one identifiable personal \textcolor{rpTypeEvidence}{anecdote}, observable behavior, or contextual reference (even \textcolor{rpTypeRule}{if} poorly expressed), and the position is clear, it should not be scored as 1. Score 2 is reserved for responses where the argument is present but entirely unsubstantiated by any concrete detail-even anonymized.\par \par Score Point 3: A minimally-developed response that takes a position with limited but discernible support. Typical elements:\par - Presents 1-\textcolor{rpTypeRule}{2 reasons} with some attempt at elaboration, though details remain general or inconsistently developed.\par - Shows basic \textcolor{rpTypeWriting}{organization}: introduction, body, and conclusion are recognizable, though paragraphing may be weak or uneven.\par - Uses occasional \textcolor{rpTypeEvidence}{specific example}s (\textcolor{rpTypeEvidence}{e.g.}, "I talk to my cousin in Colombia") or anonymized placeholders used contextually as \textcolor{rpTypeEvidence}{evidence} (\textcolor{rpTypeEvidence}{e.g.}, "@PERCENT1 of students," "@PERSON1 says...")-even \textcolor{rpTypeRule}{if} embedded in awkward phrasing or grammatical errors.\par - Shows partial awareness of audience (\textcolor{rpTypeEvidence}{e.g.}, uses letter format, addresses "readers"), but tone is inconsistent or immature.\par - \textcolor{rpTypeWriting}{Transition}al language is sparse or simplistic ("\textcolor{rpTypeRule}{first}," "\textcolor{rpTypeRule}{second}," "last"); fluency is limited but the essay remains readable with effort.\par - Anonymized placeholders are present but do not overwhelm the text; they are used as minor contextual substitutes, not as primary \textcolor{rpTypeEvidence}{evidence}.\par - Critical Note: An essay may still earn Score 3 even with multiple placeholders \textcolor{rpTypeRule}{if} they are embedded in a coherent structure and serve as identifiable, contextually grounded \textcolor{rpTypeEvidence}{evidence} (\textcolor{rpTypeEvidence}{e.g.}, "@PERCENT1 say...", "@PERSON1, a researcher...")-even \textcolor{rpTypeRule}{if} \textcolor{rpTypeWriting}{grammar} is poor. Do not penalize for placeholder density alone; penalize only \textcolor{rpTypeRule}{when} placeholders prevent interpretation of the claim's meaning.\par \par Score Point 4: A somewhat-developed response that takes a clear position and provides adequate support. Typical elements:\par - Presents 2-\textcolor{rpTypeRule}{3 reasons} with adequate elaboration, combining general claims with \textcolor{rpTypeRule}{at least} one specific, concrete example per reason (\textcolor{rpTypeEvidence}{e.g.}, "I use Facebook to ask about homework," "@PERCENT1 of kids are obese due to screen time").\par - Exhibits satisfactory \textcolor{rpTypeWriting}{organization}: clear structure with topic sentences, logical progression, and a conclusion that restates the position.\par - Uses simple \textcolor{rpTypeWriting}{transition}al language ("\textcolor{rpTypeEvidence}{for example}," "another reason," "in conclusion") consistently, though not always sophisticated.\par - Demonstrates adequate awareness of audience (\textcolor{rpTypeEvidence}{e.g.}, direct address to newspaper readers, appropriate tone for public letter).\par - Contains minor grammatical errors or awkward phrasing, but these do not significantly hinder understanding or weaken the argument.\par - Anonymized placeholders (@CAPS, @PERCENT, @LOCATION, etc.) are used meaningfully as \textcolor{rpTypeEvidence}{evidence} (\textcolor{rpTypeEvidence}{e.g.}, statistics, expert references) and do not disrupt clarity. Even \textcolor{rpTypeRule}{if} placeholders obscure exact identities, the argument's logic and supporting data remain interpretable and persuasive.\par - Critical Note: Score 4 requires that each reason includes \textcolor{rpTypeRule}{at least} one instance of concrete support-whether personal, observational, or anonymized. \textcolor{rpTypeRule}{If} the essay has 2-\textcolor{rpTypeRule}{3 reasons}, each with one clear placeholder-supported example (\textcolor{rpTypeEvidence}{e.g.}, "@PERSON1 says...", "@PERCENT1 of users..."), and the structure is logically organized, it qualifies for Score 4-even with numerous grammatical errors. Do not downgrade for language \textcolor{rpTypeRule}{if} the argument's logic and \textcolor{rpTypeEvidence}{evidence} are intact. Additionally, \textcolor{rpTypeRule}{if} the essay uses rhetorical questions, direct appeals to the reader, or emotional language to strengthen persuasion (\textcolor{rpTypeEvidence}{e.g.}, "Don't you love playing outdoor games?"), this demonstrates persuasive intent beyond basic claims and supports a Score 4-even \textcolor{rpTypeRule}{if} examples are not highly detailed.\par \par Score Point 5: A developed response that takes a clear and thoughtful position and provides reasonably persuasive support. Typical elements:\par - Presents 3+ well-elaborated reasons with mostly specific, relevant, and varied details (\textcolor{rpTypeEvidence}{e.g.}, personal \textcolor{rpTypeEvidence}{anecdote}s, observable behaviors, anonymized statistics used meaningfully).\par - Exhibits strong \textcolor{rpTypeWriting}{organization}: paragraphs are focused, ideas flow logically, and \textcolor{rpTypeWriting}{transition}s are varied and purposeful (\textcolor{rpTypeEvidence}{e.g.}, "furthermore," "conversely," "as a result").\par - Demonstrates moderate fluency: language is mostly clear, precise, and controlled, with only occasional errors that do not distract from meaning.\par - Shows consistent and thoughtful awareness of audience: tone is persuasive, respectful, and appropriate for a newspaper letter; rhetorical strategies (\textcolor{rpTypeEvidence}{e.g.}, rhetorical questions, direct appeals, emotional language) enhance persuasion.\par - Anonymized placeholders are integrated naturally and do not impede clarity or credibility; they serve as valid, context-appropriate \textcolor{rpTypeEvidence}{evidence} (\textcolor{rpTypeEvidence}{e.g.}, "@PERCENT1 of students," "@PERSON1, a researcher at @\textcolor{rpTypeWriting}{ORGANIZATION}1") and are treated as credible substitutes, not distractions.\par - Critical Note: Score 5 is awarded \textcolor{rpTypeRule}{when} the essay demonstrates persuasive intent beyond basic claims-using rhetorical devices, varied \textcolor{rpTypeEvidence}{evidence}, and consistent tone-even \textcolor{rpTypeRule}{if} language is imperfect. Do not require flawless \textcolor{rpTypeWriting}{grammar}. \textcolor{rpTypeRule}{If} anonymized \textcolor{rpTypeEvidence}{evidence} is used repeatedly and meaningfully (\textcolor{rpTypeEvidence}{e.g.}, multiple expert quotes, statistics, location-based observations), and the structure is cohesive, it qualifies for Score 5. A single, well-placed placeholder (\textcolor{rpTypeEvidence}{e.g.}, "@PERSON1 says...") is not enough for Score 5; multiple credible placeholders OR a mix of personal and anonymized \textcolor{rpTypeEvidence}{evidence} across reasons are required. Crucially, essays that use emotional appeals, direct audience engagement, or vivid imagery-even with grammatical flaws-are eligible for Score 5 \textcolor{rpTypeRule}{if} they show layered, intentional support across all reasons. Additionally, to qualify for Score 5, the essay must demonstrate \textcolor{rpTypeRule}{at least} one of the following: (1) a clear contrast or counterpoint acknowledged and addressed, (2) a compelling call to action with emotional weight, or (3) multiple distinct types of \textcolor{rpTypeEvidence}{evidence} (\textcolor{rpTypeEvidence}{e.g.}, one personal \textcolor{rpTypeEvidence}{anecdote} + one statistic + one expert attribution). Do not award Score 5 for merely having many placeholders; they must be meaningfully woven into a persuasive, multi-layered argument.\par \par Score Point 6: A well-developed response that takes a clear and thoughtful position and provides persuasive support. Typical elements:\par - Presents fully elaborated reasons with rich, specific, and insightful details that demonstrate deep understanding of the issue (\textcolor{rpTypeEvidence}{e.g.}, nuanced analysis of social consequences, balanced acknowledgment of counterpoints).\par - Exhibits strong, cohesive \textcolor{rpTypeWriting}{organization}: introduction establishes context and stakes, body paragraphs build logically, and conclusion offers a compelling call to action or broader insight.\par - Is fluent and polished: language is precise, varied, and sophisticated; \textcolor{rpTypeWriting}{transition}s are seamless and enhance rhetorical effect.\par - Shows heightened awareness of audience: tone is confident and engaging; writer anticipates reader concerns and responds effectively; persuasive techniques are intentional and well-executed.\par - Anonymized placeholders are used appropriately and do not detract from the argument's strength or clarity; the essay reads as \textcolor{rpTypeRule}{if} it were written without redaction, with placeholders functioning as natural, credible stand-ins for real-world \textcolor{rpTypeEvidence}{evidence}.\par - Critical Note: Score 6 requires sophistication in both reasoning and expression. While placeholders may be present, they must be seamlessly integrated as authoritative \textcolor{rpTypeEvidence}{evidence} (\textcolor{rpTypeEvidence}{e.g.}, "@\textcolor{rpTypeWriting}{ORGANIZATION}1's 2023 report shows...") and the essay must demonstrate a level of rhetorical control and insight that goes beyond mere adequacy. A single strong example is insufficient; multiple layers of \textcolor{rpTypeEvidence}{evidence} and nuanced analysis are required. The essay must not only persuade but also reflect depth of thought-such as recognizing complexity, anticipating objections, or connecting the issue to broader societal values. Placeholders must feel like intentional, credible substitutions, not merely convenient fillers.\par \par Note: \par I have made an effort to remove personally identifying information from the essays using the Named Entity Recognizer (NER). The relevant entities are identified in the text and then replaced with a string such as "PERSON", "\textcolor{rpTypeWriting}{ORGANIZATION}", "LOCATION", "DATE", "TIME", "MONEY", "PERCENT", "CAPS" (any capitalized word) and "NUM" (any digits). Please do not penalize the essay because of the anonymizations. \textcolor{rpTypeRule}{When} evaluating, assess whether anonymized placeholders are used in a way that:\par - Obscures meaning and prevents interpretation -> penalize (Score 1-2)\par - Are sparse and do not interfere with clarity or persuasiveness -> ignore (Score 3-6)\par - Are integrated meaningfully (\textcolor{rpTypeEvidence}{e.g.}, "@PERCENT1 of students" used as \textcolor{rpTypeEvidence}{evidence}, "@PERSON1, an expert at @\textcolor{rpTypeWriting}{ORGANIZATION}1" cited as authority) -> treat as valid support (Score 4-6)\par \par Critical Clarification for Scoring:\par - Do not penalize grammatical errors, awkward phrasing, or minor mis\textcolor{rpTypeWriting}{spelling}s \textcolor{rpTypeRule}{if} the core argument, structure, and use of \textcolor{rpTypeEvidence}{evidence} remain clear and persuasive. The presence of anonymized placeholders should not automatically downgrade an essay; instead, evaluate whether the placeholders enable or obstruct the development of the argument.\par - An essay may still earn Score 5 or 6 even with numerous placeholders \textcolor{rpTypeRule}{if} they are used as credible, contextually grounded \textcolor{rpTypeEvidence}{evidence} (\textcolor{rpTypeEvidence}{e.g.}, statistics, expert attributions) and the reasoning, \textcolor{rpTypeWriting}{organization}, and tone meet the higher-level criteria.\par - Conversely, an essay with few placeholders but incoherent logic, no developed reasons, or unintelligible structure should not exceed Score 2.\par - Score 4 requires \textcolor{rpTypeRule}{at least} one concrete example per reason, even \textcolor{rpTypeRule}{if} anonymized. Score 5 and 6 require multiple specific, varied examples and \textcolor{rpTypeEvidence}{evidence} that feel intentional and persuasive, not merely inserted.\par - Rhetorical questions, appeals to emotion, direct audience engagement, and vivid imagery are signs of persuasive intent and should be rewarded at Score 4 and above, even \textcolor{rpTypeRule}{if} language is imperfect.\par - Key Revision: \textcolor{rpTypeRule}{If} the essay presents 2-\textcolor{rpTypeRule}{3 reasons}, each supported by \textcolor{rpTypeRule}{at least} one identifiable example (personal or anonymized), and the structure is recognizable (intro, body, conclusion), it should be scored \textcolor{rpTypeRule}{at least} 3-even with poor \textcolor{rpTypeWriting}{grammar}. Do not score 1 or 2 simply because of placeholder density; score 1 only \textcolor{rpTypeRule}{when} the text is completely unintelligible and no claim can be interpreted. Score 2 requires absence of any concrete support, even anonymized.\par - Critical Addition: For Score 5, the presence of multiple rhetorical devices (\textcolor{rpTypeEvidence}{e.g.}, rhetorical questions, emotional appeals, direct address) combined with \textcolor{rpTypeRule}{at least} two distinct types of \textcolor{rpTypeEvidence}{evidence} (\textcolor{rpTypeEvidence}{e.g.}, one personal observation + one anonymized statistic) across the reasons is sufficient-even \textcolor{rpTypeRule}{if} examples are not highly detailed or polished. Do not require flawless expression; prioritize persuasive intent and layered support.\par - Critical Addition: To earn Score 5, the essay must demonstrate not only multiple examples but also \textcolor{rpTypeEvidence}{evidence} of persuasive strategy beyond listing claims-such as a call to action, emotional resonance, acknowledgment of counterarguments, or vivid imagery that deepens the reader's connection to the issue. A Score 4 essay may have adequate support; a Score 5 essay makes the reader feel something or reconsider their view.\par - Critical Addition: For Score 6, the essay must show a level of rhetorical maturity that transforms \textcolor{rpTypeEvidence}{evidence} into insight. It should not just report effects but interpret them-\textcolor{rpTypeEvidence}{e.g.}, "Computers don't just isolate us; they rewire our expectations of human connection." Placeholders must feel like natural, authoritative anchors, not redactions.
\end{tcolorbox}
\end{minipage}
\caption{Pattern-highlighted rubric comparison (asap\_1, qwen\_qwen3-next-80b-a3b-instruct, base\_expert\_True\_train100\_iteration5\_top3\_bs4-8-12\_mc4). Matched spans are color-coded by regex pattern. Color types: \textcolor{rpTypeRule}{\textbf{Rule Structure}} (if/threshold/stepwise guidance); \textcolor{rpTypeEvidence}{\textbf{Evidence Handling}} (examples, repetition, and caps); \textcolor{rpTypeWriting}{\textbf{Writing Quality}} (organization and grammar/mechanics).}
\label{fig:rubric_pattern_asap_1_qwen_qwen3_next_80b_a3b_instruct_base_expert_True_train100_iteration5_top3_bs4_8_12_mc4}
\end{figure*}
