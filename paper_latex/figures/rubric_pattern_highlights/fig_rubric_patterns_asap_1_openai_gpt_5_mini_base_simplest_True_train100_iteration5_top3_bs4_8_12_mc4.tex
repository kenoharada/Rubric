\colorlet{rpTypeRule}{red!80!black}
\colorlet{rpTypeEvidence}{blue!80!black}
\colorlet{rpTypeWriting}{teal!80!black}
\begin{figure*}[t]
\centering
\begin{tcolorbox}[colback=white,colframe=black!25,title=Pattern Type Guide,fonttitle=\bfseries\small,fontupper=\scriptsize,boxsep=1pt,left=2pt,right=2pt,top=2pt,bottom=2pt]
\textcolor{rpTypeRule}{\textbf{Rule Structure}}: Explicit decision logic for scoring: conditional branches, boundary tie-breakers, stepwise workflows, and numeric thresholds.\par \textcolor{rpTypeEvidence}{\textbf{Evidence Handling}}: How evidence is validated and counted: specific-example requirements, repetition/non-double-count rules, and cap rules for weak evidence.\par \textcolor{rpTypeWriting}{\textbf{Writing Quality}}: Language-quality criteria affecting score bands: organization/coherence/transition quality and grammar/mechanics severity.
\end{tcolorbox}
\vspace{1mm}
\begin{tcolorbox}[colback=white,colframe=black!25,title=Detailed Pattern Notes,fonttitle=\bfseries\small,fontupper=\scriptsize,boxsep=1pt,left=2pt,right=2pt,top=2pt,bottom=2pt]
\textcolor{rpTypeRule}{\textbf{Rule Structure}}:\par \quad \textcolor{rpTypeRule}{\textbf{Conditional Gating}} [n=34] Captures explicit condition-based rules that switch decisions only when a stated condition is met. Typical cues: if, when.\par \quad \textcolor{rpTypeRule}{\textbf{Boundary / Tie-Break Guidance}} [n=9] Marks criteria used to resolve borderline cases between adjacent score bands (e.g., 4 vs 5). Typical cues: tie-break, borderline, boundary, threshold, 4 vs 5.\par \quad \textcolor{rpTypeRule}{\textbf{Stepwise Rating Workflow}} [n=1] Detects ordered procedures and checklists that standardize how raters walk through scoring decisions. Typical cues: step, checklist, workflow, procedure, first/second/third.\par \quad \textcolor{rpTypeRule}{\textbf{Anti-Mechanical Counting}} [n=5] Finds rules that prevent naive counting and require qualitative judgment before assigning higher scores. Typical cues: do not count, not mechanically, do not equate.\par \quad \textcolor{rpTypeRule}{\textbf{Quantitative Threshold}} [n=8] Marks numeric cutoffs used for consistent decisions (minimum/maximum counts, percentages, explicit count thresholds). Typical cues: at least, at most, <=, >=, \%, N reasons/examples/sentences/words.\par \textcolor{rpTypeEvidence}{\textbf{Evidence Handling}}:\par \quad \textcolor{rpTypeEvidence}{\textbf{Specific Evidence Requirement}} [n=15] Highlights demands for concrete examples and explicit evidence links instead of generic assertions. Typical cues: for example, e.g., specific example, illustration, anecdote, evidence.\par \quad \textcolor{rpTypeEvidence}{\textbf{Off-Topic / Summary Cap}} [n=8] Identifies cap rules that restrict scores when responses are off-topic, irrelevant, or dominated by summary-only content. Typical cues: off-topic, irrelevant, digression, summary-only, cap.\par \quad \textcolor{rpTypeEvidence}{\textbf{Repetition Non-Count Rule}} [n=7] Captures rules that treat repetition/restatement as non-distinct support and prevent double-counting. Typical cues: repetition, restatement, double-count, do not double-count.\par \textcolor{rpTypeWriting}{\textbf{Writing Quality}}:\par \quad \textcolor{rpTypeWriting}{\textbf{Organization / Coherence Signal}} [n=12] Detects explicit references to discourse structure and logical flow as scoring criteria. Typical cues: organization, coherence, logical flow, transition.\par \quad \textcolor{rpTypeWriting}{\textbf{Grammar / Mechanics Signal}} [n=4] Detects references to language-form quality, especially grammar, spelling, punctuation, and mechanics. Typical cues: grammar, mechanics, spelling, punctuation.
\end{tcolorbox}
\vspace{1mm}
\begin{tcolorbox}[colback=white,colframe=black!25,title=Optimized Rubric (Pattern-Highlighted),fonttitle=\bfseries\small,fontupper=\scriptsize]
\ttfamily
Rate essays primarily on (1) clarity of position and direct relevance to the prompt, (2) development and support for that position (number and quality of distinct reasons, and the substantive quality of support for each), (3) \textcolor{rpTypeWriting}{\textbf{organization}}/\textcolor{rpTypeWriting}{\textbf{coherence}}, and (4) control of language (\textcolor{rpTypeWriting}{\textbf{grammar}}, vocabulary, \textcolor{rpTypeWriting}{\textbf{mechanics}}). Prioritize substantive content and meaningful development over surface errors, but require a higher standard of specificity and distinctness for the top bands.\par ... [2 lines omitted] ...\par - Reason: a claim that supports the main position. Count distinct reasons only \textcolor{rpTypeRule}{\textbf{when}} they advance separate lines of argument (\textcolor{rpTypeEvidence}{\textbf{do not double-count}} \textcolor{rpTypeEvidence}{\textbf{restatement}}s or overlapping claims unless each has a distinct supporting point or example).\par - Developed reason: a reason counts as developed only \textcolor{rpTypeRule}{\textbf{when}} the writer provides supporting material that meaningfully advances the claim. Acceptable forms of development include:\par   - A concrete, \textcolor{rpTypeEvidence}{\textbf{specific example}} or brief relevant \textcolor{rpTypeEvidence}{\textbf{anecdote}} tied to the reason.\par ... [2 lines omitted] ...\par - \textcolor{rpTypeRule}{\textbf{Do NOT count}} as development: generic assertions ("they help people learn"), vague generalities, unsupported numeric/statistical claims with no context (\textcolor{rpTypeEvidence}{\textbf{e.g.}}, "\textcolor{rpTypeRule}{\textbf{\%}} more active" without explanation of relevance), strings of placeholders or tokens with no clarifying context, or long rambling passages that never link back to the reason.\par ... [1 lines omitted] ...\par Firm heuristics (primary score \textcolor{rpTypeRule}{\textbf{tiebreak}}ers)\par - Clear position + \textcolor{rpTypeRule}{\textbf{at least}} three genuinely developed, distinct reasons approx score 5 (see higher-band rules for 6).\par ... [1 lines omitted] ...\par - Clear position + one genuinely developed reason (and/or several undeveloped or repetitive assertions) approx score 3 (or 2 \textcolor{rpTypeRule}{\textbf{when}} very short or severely unclear).\par - Very short responses (\textcolor{rpTypeEvidence}{\textbf{e.g.}}, only a sentence or two, or fewer than \textasciitilde{}\textcolor{rpTypeRule}{\textbf{50 words}}) that state a position but offer little/no development should generally be rated 2, not 3.\par ... [3 lines omitted] ...\par 1) Stronger quality \textcolor{rpTypeRule}{\textbf{threshold}} for "developed"\par - Require that each developed reason include \textcolor{rpTypeRule}{\textbf{at least}} one of: a specific concrete example (even a short personal \textcolor{rpTypeEvidence}{\textbf{anecdote}}), a clear logical explanation, or a factual detail tied to relevance. Vague \textcolor{rpTypeEvidence}{\textbf{illustration}}s or mere mention of a category (\textcolor{rpTypeEvidence}{\textbf{e.g.}}, "games improve coordination") without any specific supporting detail should not count.\par - Placeholder tokens are permissible only \textcolor{rpTypeRule}{\textbf{when}} the surrounding text makes the nature and function of the example clear. \textcolor{rpTypeRule}{\textbf{If}} placeholders obscure whether a real example/explanation was provided, \textcolor{rpTypeRule}{\textbf{do NOT count}} that reason as developed.\par - Unsupported statistics count only \textcolor{rpTypeRule}{\textbf{if}} the writer links them to explanation or context that clarifies their meaning and relevance.\par ... [2 lines omitted] ...\par - Do not downgrade a response below 4 solely for grammatical/mechanical errors \textcolor{rpTypeRule}{\textbf{if}} it clearly supplies two developed reasons; likewise do not drop below 5 \textcolor{rpTypeRule}{\textbf{if}} it clearly supplies three developed reasons and development is intelligible.\par - However, reduce leniency \textcolor{rpTypeRule}{\textbf{when}} surface errors combine with vague or generic support: frequent errors + only generic development should not result in 5.\par ... [3 lines omitted] ...\par - Count a partially developed reason \textcolor{rpTypeRule}{\textbf{when}} the essential support is intelligible (\textcolor{rpTypeEvidence}{\textbf{e.g.}}, brief \textcolor{rpTypeEvidence}{\textbf{anecdote}}, coherent one-sentence explanation), even \textcolor{rpTypeRule}{\textbf{if}} wording is garbled.\par - \textcolor{rpTypeRule}{\textbf{Do NOT count}} a long passage that names examples or statistics but never links them coherently to the reason.\par - \textcolor{rpTypeRule}{\textbf{When}} in doubt about \textcolor{rpTypeRule}{\textbf{borderline}} partial examples, prefer the lower adjacent score unless the development convincingly meets the standards above.\par ... [1 lines omitted] ...\par 4) \textcolor{rpTypeEvidence}{\textbf{Repetition}}/overlap rule (strengthened)\par - \textcolor{rpTypeRule}{\textbf{Do not count}} repeated \textcolor{rpTypeEvidence}{\textbf{restatement}}s of the same underlying claim as multiple reasons. \textcolor{rpTypeRule}{\textbf{When}} two apparent reasons overlap substantially, count them as one unless the writer provides distinct supporting points or examples that clearly separate them.\par - \textcolor{rpTypeRule}{\textbf{If}} an essay lists three headings-like reasons but two are essentially the same claim reworded (\textcolor{rpTypeEvidence}{\textbf{e.g.}}, "communication" and "staying in touch" with no distinct support), count them as one for the three-reason \textcolor{rpTypeRule}{\textbf{threshold}}.\par ... [1 lines omitted] ...\par 5) \textcolor{rpTypeEvidence}{\textbf{Cap}} rule and \textcolor{rpTypeRule}{\textbf{when}} to deny a "three-developed" count (new)\par - \textcolor{rpTypeEvidence}{\textbf{Cap}} at 4 (not 5) \textcolor{rpTypeRule}{\textbf{when}} the essay lists three reasons but the development for \textcolor{rpTypeRule}{\textbf{at least}} one is minimal, vague, repetitive, or primarily asserted without specific support.\par - \textcolor{rpTypeEvidence}{\textbf{Cap}} at 4 \textcolor{rpTypeRule}{\textbf{when}} the three "reasons" rely heavily on unsupported statistics, placeholders, or \textcolor{rpTypeEvidence}{\textbf{repetition}} rather than three distinct, meaningful supports.\par - \textcolor{rpTypeRule}{\textbf{If}} one of three reasons is clearly developed and the other two are only generic or mainly assertions, treat the response as a two-developed-reasons case (score 4) or even a one-developed case (score 3) depending on exact quality.\par ... [4 lines omitted] ...\par   - Thorough, persuasive development: multiple (more than three is fine) distinct reasons with well‑explained, \textcolor{rpTypeEvidence}{\textbf{specific example}}s or details for each main reason. Development should be more than one brief sentence per reason; explanations should show logical connection and persuasive depth.\par   - Logical, effective \textcolor{rpTypeWriting}{\textbf{organization}} with fluent, precise language appropriate for top‑quality academic writing for the grade level.\par ... [5 lines omitted] ...\par   - \textcolor{rpTypeRule}{\textbf{At least}} three distinct reasons that are each genuinely developed (each reason includes a \textcolor{rpTypeEvidence}{\textbf{specific example}}, coherent explanation, or concrete detail). Development need not be exhaustive but must move beyond mere assertion.\par   - \textcolor{rpTypeWriting}{\textbf{Organization}} is clear; progression of ideas is coherent.\par   - \textcolor{rpTypeWriting}{\textbf{Grammar}}/\textcolor{rpTypeWriting}{\textbf{spelling}}/mechanical errors may be frequent but do not significantly obscure meaning. Moderate placeholders or garbling are acceptable \textcolor{rpTypeRule}{\textbf{if}} each reason's development remains intelligible.\par - Apply the "\textcolor{rpTypeEvidence}{\textbf{cap}}" rule: do not award 5 \textcolor{rpTypeRule}{\textbf{if}} any one of the three reasons lacks meaningful development (see rule 5).\par ... [3 lines omitted] ...\par   - Clear position is stated and the writer provides \textcolor{rpTypeRule}{\textbf{at least}} two distinct reasons with some supporting detail or examples (i.e., two developed reasons).\par ... [1 lines omitted] ...\par   - \textcolor{rpTypeWriting}{\textbf{Organization}} is evident though \textcolor{rpTypeWriting}{\textbf{transition}}s may be simple.\par ... [6 lines omitted] ...\par   - Examples (\textcolor{rpTypeRule}{\textbf{if}} any) are vague, generic, or only tangentially relevant.\par   - Frequent distracting errors and weak \textcolor{rpTypeWriting}{\textbf{organization}} reduce readability.\par   - Use 3 \textcolor{rpTypeRule}{\textbf{when}} there is some attempt at development but the essay does not meet the two-developed-reasons \textcolor{rpTypeRule}{\textbf{threshold}} for 4.\par ... [4 lines omitted] ...\par   - Few or no meaningful reasons; examples missing, incoherent, or \textcolor{rpTypeEvidence}{\textbf{irrelevant}}. Very short responses usually fall here.\par   - Major \textcolor{rpTypeWriting}{\textbf{organization}} problems or severe language breakdowns that materially impede comprehension.\par ... [4 lines omitted] ...\par - Overcounting three "reasons" \textcolor{rpTypeRule}{\textbf{when}} development is shallow: insist that each of the three reasons include \textcolor{rpTypeRule}{\textbf{at least}} a clear \textcolor{rpTypeEvidence}{\textbf{specific example}} or a coherent explanation that directly links to the claim. \textcolor{rpTypeRule}{\textbf{If}} any one of the three is only a generic assertion or \textcolor{rpTypeEvidence}{\textbf{repetition}}, do not give 5.\par - Under-counting due to placeholders and surface errors: be generous in counting development \textcolor{rpTypeRule}{\textbf{if}} the example's function and meaning are clear despite placeholders or garbling. Partial coherent \textcolor{rpTypeEvidence}{\textbf{anecdote}}s or one-sentence logical explanations should count.\par - Distinguishing quantity vs. quality: three thin assertions do not equal three developed reasons. Two well-developed reasons are preferable to three weak ones; apply the two-developed \textcolor{rpTypeRule}{\textbf{threshold}} for 4 and require substantive specifics for 5.\par ... [1 lines omitted] ...\par   - Example like A/B (three distinct claims each with specific \textcolor{rpTypeEvidence}{\textbf{anecdote}} or example, even with errors/placeholders) -> 5 unless language and expression reach the high standard for 6.\par ... [1 lines omitted] ...\par   - Example like D (explicitly lists three reasons but development is weak/unsupported) -> 4 (\textcolor{rpTypeEvidence}{\textbf{cap}} at 4).\par ... [2 lines omitted] ...\par Final scorer \textcolor{rpTypeRule}{\textbf{checklist}} (practical)\par 1. Is the writer's position clear and responsive to the prompt? \textcolor{rpTypeRule}{\textbf{If}} no -> likely 1-2.\par ... [1 lines omitted] ...\par    - 3+ developed reasons -> consider 5 (or 6 \textcolor{rpTypeRule}{\textbf{if}} development, \textcolor{rpTypeWriting}{\textbf{organization}}, and language are top-tier).\par ... [3 lines omitted] ...\par 3. Are any of the counted reasons actually \textcolor{rpTypeEvidence}{\textbf{repetition}}s/overlaps? \textcolor{rpTypeRule}{\textbf{If}} so, reduce the developed-reason count.\par 4. Are placeholders or errors obscuring whether development exists? \textcolor{rpTypeRule}{\textbf{If}} obscured -> downgrade to 2/1 as needed.\par 5. Apply \textcolor{rpTypeEvidence}{\textbf{cap}} rule: \textcolor{rpTypeRule}{\textbf{if}} three reasons exist but one is shallow/vague/unsupported -> \textcolor{rpTypeEvidence}{\textbf{cap}} at 4.\par 6. Finally, adjust within-band for \textcolor{rpTypeWriting}{\textbf{organization}} and language: strong \textcolor{rpTypeWriting}{\textbf{organization}} and clear polished language can justify moving from 5->6; pervasive errors that still allow understanding do not force downgrades \textcolor{rpTypeRule}{\textbf{if}} development \textcolor{rpTypeRule}{\textbf{threshold}}s are met.\par ... [1 lines omitted] ...\par Summary guidance for adjudicating \textcolor{rpTypeRule}{\textbf{borderline}} cases\par - \textcolor{rpTypeRule}{\textbf{When}} in doubt between adjacent scores, favor the higher score only \textcolor{rpTypeRule}{\textbf{when}} the development clearly meets the heuristic \textcolor{rpTypeRule}{\textbf{threshold}}s (two developed reasons -> 4; three developed reasons with intelligible support -> 5). Do NOT award 5 for three thin or repetitive assertions.\par - Reserve 6 for essays that combine substantive depth for multiple reasons with fluent, precise expression and convincing \textcolor{rpTypeWriting}{\textbf{organization}}.\par - Be strict about counting a "developed" reason: ask, "Would an informed reader be convinced this reason is supported by a concrete example or clear explanation?" \textcolor{rpTypeRule}{\textbf{If}} not, \textcolor{rpTypeRule}{\textbf{do not count}} it.
\end{tcolorbox}
\caption{Pattern-focused view of the optimized rubric (asap\_1, openai\_gpt-5-mini, base\_simplest\_True\_train100\_iteration5\_top3\_bs4-8-12\_mc4). Colored bold spans indicate regex-matched rubric cues. Color types: \textcolor{rpTypeRule}{\textbf{Rule Structure}} (Explicit decision logic for scoring: conditional branches, boundary tie-breakers, stepwise workflows, and numeric thresholds.); \textcolor{rpTypeEvidence}{\textbf{Evidence Handling}} (How evidence is validated and counted: specific-example requirements, repetition/non-double-count rules, and cap rules for weak evidence.); \textcolor{rpTypeWriting}{\textbf{Writing Quality}} (Language-quality criteria affecting score bands: organization/coherence/transition quality and grammar/mechanics severity.). Matched pattern categories: Conditional Gating (n=34); Boundary / Tie-Break Guidance (n=9); Stepwise Rating Workflow (n=1); Anti-Mechanical Counting (n=5); Specific Evidence Requirement (n=15); Off-Topic / Summary Cap (n=8); Organization / Coherence Signal (n=12); Grammar / Mechanics Signal (n=4); Repetition Non-Count Rule (n=7); Quantitative Threshold (n=8).}
\label{fig:rubric_pattern_asap_1_openai_gpt_5_mini_base_simplest_True_train100_iteration5_top3_bs4_8_12_mc4}
\end{figure*}
