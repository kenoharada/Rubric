\colorlet{rpTypeRule}{red!80!black}
\colorlet{rpTypeEvidence}{blue!80!black}
\colorlet{rpTypeWriting}{teal!80!black}
\begin{figure*}[t]
\centering
\begin{tcolorbox}[colback=white,colframe=black!25,title=Pattern Legend,fonttitle=\bfseries\small,fontupper=\scriptsize,boxsep=1pt,left=2pt,right=2pt,top=2pt,bottom=2pt]
\textcolor{rpTypeRule}{\textbf{Rule Structure}} (if/threshold/stepwise guidance) \quad \textcolor{rpTypeEvidence}{\textbf{Evidence Handling}} (examples, repetition, and caps) \quad \textcolor{rpTypeWriting}{\textbf{Writing Quality}} (organization and grammar/mechanics)
\end{tcolorbox}
\vspace{2mm}
\begin{minipage}[t]{0.485\textwidth}
\begin{tcolorbox}[colback=white,colframe=black!25,title=Initial Rubric,fonttitle=\bfseries\small,fontupper=\scriptsize,breakable]
\ttfamily
Based on the response's content, rate the response on a scale of 1 to 6.
\end{tcolorbox}
\end{minipage}
\hfill
\begin{minipage}[t]{0.485\textwidth}
\begin{tcolorbox}[colback=white,colframe=black!25,title=Optimized Rubric,fonttitle=\bfseries\small,fontupper=\scriptsize,breakable]
\ttfamily
Overview (core priorities)\par Rate essays primarily on (1) clarity of position and direct relevance to the prompt, (2) development and support for that position (number and quality of distinct reasons, and the substantive quality of support for each), (3) \textcolor{rpTypeWriting}{organization}/\textcolor{rpTypeWriting}{coherence}, and (4) control of language (\textcolor{rpTypeWriting}{grammar}, vocabulary, \textcolor{rpTypeWriting}{mechanics}). Prioritize substantive content and meaningful development over surface errors, but require a higher standard of specificity and distinctness for the top bands.\par \par Key definitions and counting rules\par - Reason: a claim that supports the main position. Count distinct reasons only \textcolor{rpTypeRule}{when} they advance separate lines of argument (\textcolor{rpTypeEvidence}{do not double-count} \textcolor{rpTypeEvidence}{restatement}s or overlapping claims unless each has a distinct supporting point or example).\par - Developed reason: a reason counts as developed only \textcolor{rpTypeRule}{when} the writer provides supporting material that meaningfully advances the claim. Acceptable forms of development include:\par   - A concrete, \textcolor{rpTypeEvidence}{specific example} or brief relevant \textcolor{rpTypeEvidence}{anecdote} tied to the reason.\par   - A coherent logical explanation showing how/why the reason supports the claim.\par   - Specific factual detail (dates, names, clear contextualized statistics) that is explained or connected to the claim.\par - \textcolor{rpTypeRule}{Do NOT count} as development: generic assertions ("they help people learn"), vague generalities, unsupported numeric/statistical claims with no context (\textcolor{rpTypeEvidence}{e.g.}, "\textcolor{rpTypeRule}{\%} more active" without explanation of relevance), strings of placeholders or tokens with no clarifying context, or long rambling passages that never link back to the reason.\par \par Firm heuristics (primary score \textcolor{rpTypeRule}{tiebreak}ers)\par - Clear position + \textcolor{rpTypeRule}{at least} three genuinely developed, distinct reasons approx score 5 (see higher-band rules for 6).\par - Clear position + two genuinely developed, distinct reasons approx score 4.\par - Clear position + one genuinely developed reason (and/or several undeveloped or repetitive assertions) approx score 3 (or 2 \textcolor{rpTypeRule}{when} very short or severely unclear).\par - Very short responses (\textcolor{rpTypeEvidence}{e.g.}, only a sentence or two, or fewer than \textasciitilde{}\textcolor{rpTypeRule}{50 words}) that state a position but offer little/no development should generally be rated 2, not 3.\par \par ... [1 lines omitted] ...\par \par 1) Stronger quality \textcolor{rpTypeRule}{threshold} for "developed"\par - Require that each developed reason include \textcolor{rpTypeRule}{at least} one of: a specific concrete example (even a short personal \textcolor{rpTypeEvidence}{anecdote}), a clear logical explanation, or a factual detail tied to relevance. Vague \textcolor{rpTypeEvidence}{illustration}s or mere mention of a category (\textcolor{rpTypeEvidence}{e.g.}, "games improve coordination") without any specific supporting detail should not count.\par - Placeholder tokens are permissible only \textcolor{rpTypeRule}{when} the surrounding text makes the nature and function of the example clear. \textcolor{rpTypeRule}{If} placeholders obscure whether a real example/explanation was provided, \textcolor{rpTypeRule}{do NOT count} that reason as developed.\par - Unsupported statistics count only \textcolor{rpTypeRule}{if} the writer links them to explanation or context that clarifies their meaning and relevance.\par \par 2) Handling of language errors vs. development (refined)\par - Do not downgrade a response below 4 solely for grammatical/mechanical errors \textcolor{rpTypeRule}{if} it clearly supplies two developed reasons; likewise do not drop below 5 \textcolor{rpTypeRule}{if} it clearly supplies three developed reasons and development is intelligible.\par - However, reduce leniency \textcolor{rpTypeRule}{when} surface errors combine with vague or generic support: frequent errors + only generic development should not result in 5.\par - Downgrade to 2 (or 1) where language or placeholder use materially obscures the content to the point you cannot reliably infer the writer's intended development.\par ... [1 lines omitted] ...\par 3) Partial development and counting leniency (balanced)\par - Count a partially developed reason \textcolor{rpTypeRule}{when} the essential support is intelligible (\textcolor{rpTypeEvidence}{e.g.}, brief \textcolor{rpTypeEvidence}{anecdote}, coherent one-sentence explanation), even \textcolor{rpTypeRule}{if} wording is garbled.\par - \textcolor{rpTypeRule}{Do NOT count} a long passage that names examples or statistics but never links them coherently to the reason.\par - \textcolor{rpTypeRule}{When} in doubt about \textcolor{rpTypeRule}{borderline} partial examples, prefer the lower adjacent score unless the development convincingly meets the standards above.\par \par 4) \textcolor{rpTypeEvidence}{Repetition}/overlap rule (strengthened)\par - \textcolor{rpTypeRule}{Do not count} repeated \textcolor{rpTypeEvidence}{restatement}s of the same underlying claim as multiple reasons. \textcolor{rpTypeRule}{When} two apparent reasons overlap substantially, count them as one unless the writer provides distinct supporting points or examples that clearly separate them.\par - \textcolor{rpTypeRule}{If} an essay lists three headings-like reasons but two are essentially the same claim reworded (\textcolor{rpTypeEvidence}{e.g.}, "communication" and "staying in touch" with no distinct support), count them as one for the three-reason \textcolor{rpTypeRule}{threshold}.\par \par 5) \textcolor{rpTypeEvidence}{Cap} rule and \textcolor{rpTypeRule}{when} to deny a "three-developed" count (new)\par - \textcolor{rpTypeEvidence}{Cap} at 4 (not 5) \textcolor{rpTypeRule}{when} the essay lists three reasons but the development for \textcolor{rpTypeRule}{at least} one is minimal, vague, repetitive, or primarily asserted without specific support.\par - \textcolor{rpTypeEvidence}{Cap} at 4 \textcolor{rpTypeRule}{when} the three "reasons" rely heavily on unsupported statistics, placeholders, or \textcolor{rpTypeEvidence}{repetition} rather than three distinct, meaningful supports.\par - \textcolor{rpTypeRule}{If} one of three reasons is clearly developed and the other two are only generic or mainly assertions, treat the response as a two-developed-reasons case (score 4) or even a one-developed case (score 3) depending on exact quality.\par \par ... [2 lines omitted] ...\par   - A strong, specific position responding directly to the prompt.\par   - Thorough, persuasive development: multiple (more than three is fine) distinct reasons with well‑explained, \textcolor{rpTypeEvidence}{specific example}s or details for each main reason. Development should be more than one brief sentence per reason; explanations should show logical connection and persuasive depth.\par   - Logical, effective \textcolor{rpTypeWriting}{organization} with fluent, precise language appropriate for top‑quality academic writing for the grade level.\par - Importantly: three intelligible but brief/surface-developed reasons with frequent placeholders or only one-sentence examples usually belong in band 5, not 6. Use 6 for responses that achieve high quality both in content depth and expression.\par ... [3 lines omitted] ...\par   - Clear position directly answers the prompt.\par   - \textcolor{rpTypeRule}{At least} three distinct reasons that are each genuinely developed (each reason includes a \textcolor{rpTypeEvidence}{specific example}, coherent explanation, or concrete detail). Development need not be exhaustive but must move beyond mere assertion.\par   - \textcolor{rpTypeWriting}{Organization} is clear; progression of ideas is coherent.\par   - \textcolor{rpTypeWriting}{Grammar}/\textcolor{rpTypeWriting}{spelling}/mechanical errors may be frequent but do not significantly obscure meaning. Moderate placeholders or garbling are acceptable \textcolor{rpTypeRule}{if} each reason's development remains intelligible.\par - Apply the "\textcolor{rpTypeEvidence}{cap}" rule: do not award 5 \textcolor{rpTypeRule}{if} any one of the three reasons lacks meaningful development (see rule 5).\par \par ... [1 lines omitted] ...\par - Score 4 = Competent/Effective:\par   - Clear position is stated and the writer provides \textcolor{rpTypeRule}{at least} two distinct reasons with some supporting detail or examples (i.e., two developed reasons).\par   - Development may be uneven, somewhat general, or partially undeveloped; examples may be general rather than highly specific.\par   - \textcolor{rpTypeWriting}{Organization} is evident though \textcolor{rpTypeWriting}{transition}s may be simple.\par   - Noticeable errors may distract but do not prevent understanding.\par ... [4 lines omitted] ...\par   - Reasons are simplistic, underdeveloped, or repetitive. There may be one developed reason plus other undeveloped assertions, or two undeveloped reasons.\par   - Examples (\textcolor{rpTypeRule}{if} any) are vague, generic, or only tangentially relevant.\par   - Frequent distracting errors and weak \textcolor{rpTypeWriting}{organization} reduce readability.\par   - Use 3 \textcolor{rpTypeRule}{when} there is some attempt at development but the essay does not meet the two-developed-reasons \textcolor{rpTypeRule}{threshold} for 4.\par \par ... [2 lines omitted] ...\par   - Position unclear, inconsistent, or only implied; supported by little or no meaningful development.\par   - Few or no meaningful reasons; examples missing, incoherent, or \textcolor{rpTypeEvidence}{irrelevant}. Very short responses usually fall here.\par   - Major \textcolor{rpTypeWriting}{organization} problems or severe language breakdowns that materially impede comprehension.\par - Score 1 = Inadequate/Noncommunicative:\par ... [2 lines omitted] ...\par Special guidance to address known mismatch patterns\par - Overcounting three "reasons" \textcolor{rpTypeRule}{when} development is shallow: insist that each of the three reasons include \textcolor{rpTypeRule}{at least} a clear \textcolor{rpTypeEvidence}{specific example} or a coherent explanation that directly links to the claim. \textcolor{rpTypeRule}{If} any one of the three is only a generic assertion or \textcolor{rpTypeEvidence}{repetition}, do not give 5.\par - Under-counting due to placeholders and surface errors: be generous in counting development \textcolor{rpTypeRule}{if} the example's function and meaning are clear despite placeholders or garbling. Partial coherent \textcolor{rpTypeEvidence}{anecdote}s or one-sentence logical explanations should count.\par - Distinguishing quantity vs. quality: three thin assertions do not equal three developed reasons. Two well-developed reasons are preferable to three weak ones; apply the two-developed \textcolor{rpTypeRule}{threshold} for 4 and require substantive specifics for 5.\par - Examples to guide decisions:\par   - Example like A/B (three distinct claims each with specific \textcolor{rpTypeEvidence}{anecdote} or example, even with errors/placeholders) -> 5 unless language and expression reach the high standard for 6.\par   - Example like C (one or one-plus confused reasons) -> 3.\par   - Example like D (explicitly lists three reasons but development is weak/unsupported) -> 4 (\textcolor{rpTypeEvidence}{cap} at 4).\par   - Essays with three distinct reasons each supported by detailed, persuasive examples and polished expression -> 6.\par \par Final scorer \textcolor{rpTypeRule}{checklist} (practical)\par 1. Is the writer's position clear and responsive to the prompt? \textcolor{rpTypeRule}{If} no -> likely 1-2.\par 2. How many genuinely developed, distinct reasons are present? (Use the strict "developed" test above.)\par    - 3+ developed reasons -> consider 5 (or 6 \textcolor{rpTypeRule}{if} development, \textcolor{rpTypeWriting}{organization}, and language are top-tier).\par    - 2 developed reasons -> 4.\par ... [1 lines omitted] ...\par    - 0 developed reasons or extremely brief -> 2.\par 3. Are any of the counted reasons actually \textcolor{rpTypeEvidence}{repetition}s/overlaps? \textcolor{rpTypeRule}{If} so, reduce the developed-reason count.\par 4. Are placeholders or errors obscuring whether development exists? \textcolor{rpTypeRule}{If} obscured -> downgrade to 2/1 as needed.\par 5. Apply \textcolor{rpTypeEvidence}{cap} rule: \textcolor{rpTypeRule}{if} three reasons exist but one is shallow/vague/unsupported -> \textcolor{rpTypeEvidence}{cap} at 4.\par 6. Finally, adjust within-band for \textcolor{rpTypeWriting}{organization} and language: strong \textcolor{rpTypeWriting}{organization} and clear polished language can justify moving from 5->6; pervasive errors that still allow understanding do not force downgrades \textcolor{rpTypeRule}{if} development \textcolor{rpTypeRule}{threshold}s are met.\par \par Summary guidance for adjudicating \textcolor{rpTypeRule}{borderline} cases\par - \textcolor{rpTypeRule}{When} in doubt between adjacent scores, favor the higher score only \textcolor{rpTypeRule}{when} the development clearly meets the heuristic \textcolor{rpTypeRule}{threshold}s (two developed reasons -> 4; three developed reasons with intelligible support -> 5). Do NOT award 5 for three thin or repetitive assertions.\par - Reserve 6 for essays that combine substantive depth for multiple reasons with fluent, precise expression and convincing \textcolor{rpTypeWriting}{organization}.\par - Be strict about counting a "developed" reason: ask, "Would an informed reader be convinced this reason is supported by a concrete example or clear explanation?" \textcolor{rpTypeRule}{If} not, \textcolor{rpTypeRule}{do not count} it.
\end{tcolorbox}
\end{minipage}
\caption{Pattern-highlighted rubric comparison (asap\_1, openai\_gpt-5-mini, base\_simplest\_True\_train100\_iteration5\_top3\_bs4-8-12\_mc4). Matched spans are color-coded by regex pattern. Color types: \textcolor{rpTypeRule}{\textbf{Rule Structure}} (if/threshold/stepwise guidance); \textcolor{rpTypeEvidence}{\textbf{Evidence Handling}} (examples, repetition, and caps); \textcolor{rpTypeWriting}{\textbf{Writing Quality}} (organization and grammar/mechanics).}
\label{fig:rubric_pattern_asap_1_openai_gpt_5_mini_base_simplest_True_train100_iteration5_top3_bs4_8_12_mc4}
\end{figure*}
