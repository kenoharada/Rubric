\colorlet{rpTypeRule}{red!80!black}
\colorlet{rpTypeEvidence}{blue!80!black}
\colorlet{rpTypeWriting}{teal!80!black}
\begin{figure*}[t]
\centering
\begin{tcolorbox}[colback=white,colframe=black!25,title=Pattern Legend,fonttitle=\bfseries\small,fontupper=\scriptsize,boxsep=1pt,left=2pt,right=2pt,top=2pt,bottom=2pt]
\textcolor{rpTypeRule}{\textbf{Rule Structure}} (if/threshold/stepwise guidance) \quad \textcolor{rpTypeEvidence}{\textbf{Evidence Handling}} (examples, repetition, and caps) \quad \textcolor{rpTypeWriting}{\textbf{Writing Quality}} (organization and grammar/mechanics)
\end{tcolorbox}
\vspace{2mm}
\begin{minipage}[t]{0.485\textwidth}
\begin{tcolorbox}[colback=white,colframe=black!25,title=Initial Rubric,fonttitle=\bfseries\small,fontupper=\scriptsize,breakable]
\ttfamily
- is generally well organized and well developed, using appropriate and sufficient explanations, exemplifications, and/or details\par - displays unity, progression, and \textcolor{rpTypeWriting}{coherence}, though it may contain occasional redundancy, \textcolor{rpTypeEvidence}{digression}, or unclear connections\par - displays facility in the use of language, demonstrating syntactic variety and range of vocabulary, though it will probably have occasional noticeable minor errors in structure, word form, or use of idiomatic language that do not interfere with meaning\par ... [3 lines omitted] ...\par - addresses the topic and task using somewhat developed explanations, exemplifications, and/or details\par - displays unity, progression, and \textcolor{rpTypeWriting}{coherence}, though connection of ideas may be occasionally obscured\par - may demonstrate inconsistent facility in sentence formation and word choice that may result in lack of clarity and occasionally obscure meaning\par ... [4 lines omitted] ...\par - limited development in response to the topic and task\par - inadequate \textcolor{rpTypeWriting}{organization} or connection of ideas\par - inappropriate or insufficient exemplifications, explanations, or details to support or illustrate generalizations in response to the task
\end{tcolorbox}
\end{minipage}
\hfill
\begin{minipage}[t]{0.485\textwidth}
\begin{tcolorbox}[colback=white,colframe=black!25,title=Optimized Rubric,fonttitle=\bfseries\small,fontupper=\scriptsize,breakable]
\ttfamily
General instructions (how to apply the three score levels)\par - Primary weighting order (unchanged): Task fulfillment \& development > \textcolor{rpTypeWriting}{Organization} \& \textcolor{rpTypeWriting}{coherence} > Language control.\par - Do not use length alone to determine score.\par - Distinguish error severity precisely and apply \textcolor{rpTypeRule}{threshold}s consistently:\par   - Minor errors: rare \textcolor{rpTypeWriting}{spelling}/typo or small \textcolor{rpTypeWriting}{punctuation} slips that never interrupt comprehension.\par   - Moderate errors: recurring awkward phrasing, incorrect word forms/choices, or sentence problems that sometimes require re-reading but do not generally make meaning impossible to recover.\par   - Severe/accumulated errors: frequent errors across many sentences that regularly interrupt comprehension, force heavy re-reading, or create ambiguous/unclear meaning.\par - Always annotate which of the three dimensions (development, \textcolor{rpTypeWriting}{organization}, language) is the decisive factor for the assigned score, and cite explicit \textcolor{rpTypeEvidence}{evidence} (quote 1-2 exemplar sentences \textcolor{rpTypeRule}{if} language is the decisive issue).\par \par ... [2 lines omitted] ...\par   - Clear position/thesis and sustained development with multiple relevant reasons/examples OR one extended, fully elaborated example that is clearly developed (not a single unsupported assertion).\par   - Addresses counterarguments or shows depth of thought \textcolor{rpTypeRule}{when} relevant.\par   - Clarification added: \textcolor{rpTypeRule}{When} multiple distinct examples are present and each is developed with some explanation/connection to the thesis, favor score 3 even \textcolor{rpTypeRule}{if} language contains many surface-level errors - provided those errors do not regularly obscure meaning or create reader fatigue.\par - \textcolor{rpTypeWriting}{Organization} \& \textcolor{rpTypeWriting}{coherence}:\par   - Clear progression of ideas, logical paragraphing, and explicit links between points. A truncated or slightly abrupt conclusion does not automatically negate a 3 \textcolor{rpTypeRule}{if} multi-paragraph development is sustained and convincing.\par - Language:\par   - Range of vocabulary and syntactic variety appropriate for the task.\par   - Errors, \textcolor{rpTypeRule}{if} present, are minor or occasional moderate slips that do not disrupt reading. Frequent moderate errors are acceptable for 3 only \textcolor{rpTypeRule}{when} (a) they do not force repeated re-construction of meaning, and (b) the essay remains easy to follow without sustained effort.\par - \textcolor{rpTypeRule}{When} to assign 3 (clarified):\par   - Use 3 \textcolor{rpTypeRule}{when} task is fully addressed with well-developed support and clear \textcolor{rpTypeWriting}{organization}-even \textcolor{rpTypeRule}{if} surface-level errors are frequent-ONLY \textcolor{rpTypeRule}{IF} those errors are mostly minor or at worst recurrent-moderate but do not cause reader fatigue or obscure meaning.\par   - Do NOT give 3 \textcolor{rpTypeRule}{when} development is repetitive, thin, or relies on a single undeveloped/example assertion even \textcolor{rpTypeRule}{if} language is good.\par \par ... [2 lines omitted] ...\par   - Some relevant ideas are present but development is uneven, thin, or only partially elaborated.\par   - Typical patterns: general statements without adequate elaboration, brief examples that are not extended, or \textcolor{rpTypeEvidence}{repetition} of ideas rather than deeper development.\par   - Clarification added: \textcolor{rpTypeRule}{If} there are multiple examples but each is only briefly explained and the essay does not show clear, sustained development, assign 2 (even \textcolor{rpTypeRule}{if} language is relatively accurate).\par - \textcolor{rpTypeWriting}{Organization} \& \textcolor{rpTypeWriting}{coherence}:\par   - Overall progression exists but may be disjointed or repetitious; minor \textcolor{rpTypeEvidence}{digression}s or redundancy are common.\par   - Incomplete conclusions or truncated endings that fail to wrap up the argument normally indicate score 2 - but this can be outweighed by exceptionally strong multi-example development (see Score 3 clarification).\par ... [1 lines omitted] ...\par   - Errors are more frequent than in score 3 and may occasionally obscure nuance or require re-reading.\par   - Assign 2 \textcolor{rpTypeRule}{when} moderate or some severe errors occur but communication generally remains recoverable without substantial effort.\par   - Clarification added: Frequent moderate errors that cause reader fatigue (i.e., intermittent re-reading but not constant reconstruction) merit score 2. \textcolor{rpTypeRule}{If} errors are frequent but readers can follow the argument with modest effort, prefer 2 over 1.\par - \textcolor{rpTypeRule}{When} to assign 2 (clarified):\par   - Use 2 \textcolor{rpTypeRule}{when} there is some development and \textcolor{rpTypeWriting}{organization} but support is noticeably limited, OR \textcolor{rpTypeRule}{when} moderate/accumulated errors begin to slow the reader though meaning is still mostly recoverable.\par   - Use 2 (not 3) \textcolor{rpTypeRule}{when} development is repetitive or \textcolor{rpTypeWriting}{organization}ally weak even \textcolor{rpTypeRule}{if} there are no totally unreadable sentences.\par \par ... [1 lines omitted] ...\par - Development \& support:\par   - Very limited development: few or no relevant examples; explanations too brief, \textcolor{rpTypeEvidence}{irrelevant}, or repetitive/general assertions without support.\par   - Explicit rule retained and clarified: a response that relies on a single undeveloped example or single-sentence support that is not extended should be scored 1 (not 2).\par - \textcolor{rpTypeWriting}{Organization} \& \textcolor{rpTypeWriting}{coherence}:\par   - Inadequate \textcolor{rpTypeWriting}{organization}; reader has difficulty following progression; extremely short or fragmentary responses.\par - Language:\par   - Noticeable inappropriate word choices, wrong word forms, or frequent sentence-level errors.\par   - Accumulation of severe errors that often obscure meaning or make the essay hard to understand-score 1 even \textcolor{rpTypeRule}{if} some relevant ideas exist.\par   - Clarification added: To justify score 1 on language grounds, errors must (a) occur frequently across the essay and (b) require repeated reconstruction of meaning (i.e., reader must regularly infer or guess meaning). Isolated severe errors do not by themselves require score 1 \textcolor{rpTypeRule}{if} overall meaning is recoverable.\par - \textcolor{rpTypeRule}{When} to assign 1 (clarified):\par   - Use 1 for perfunctory, very short, or fragmented responses lacking development, OR for essays whose errors are so frequent/severe that they hinder comprehension.\par   - Use 1 \textcolor{rpTypeRule}{when} the reader must frequently re-construct meaning due to severe/accumulated errors, even \textcolor{rpTypeRule}{if} the essay attempts relevant content.\par \par Additional clarifications and \textcolor{rpTypeRule}{borderline} decision rules (to improve inter-rater agreement)\par - Single undeveloped example: Always favor score 1 (not 2). A single brief example that is not extended or connected to a clear line of reasoning indicates inadequate development.\par - Many surface errors but strong development: Favor the higher score (3) only \textcolor{rpTypeRule}{if} errors are mostly minor or occasional-moderate and do not create reading fatigue. \textcolor{rpTypeRule}{If} errors are frequent enough to make sustained reading effortful, downgrade to 2.\par - Frequent moderate errors that cause reader fatigue: Downgrade to 2 even \textcolor{rpTypeRule}{when} development and \textcolor{rpTypeWriting}{organization} are relatively strong.\par - Severe/accumulated errors that obscure meaning: Downgrade to 1 regardless of topic relevance \textcolor{rpTypeRule}{if} comprehension is regularly impeded.\par - Repetitive development: \textcolor{rpTypeRule}{If} the essay repeats the same idea across paragraphs without extending or deepening it, prefer score 2 rather than 3.\par - Incomplete or abrupt conclusion: \textcolor{rpTypeRule}{If} the essay ends mid-thought or with an incomplete conclusion, prefer score 2 unless the rest of the essay demonstrates clear, multi-paragraph, well-elaborated development with multiple examples (in which case 3 may still be warranted).\par - \textcolor{rpTypeWriting}{Organization} outweighs \textcolor{rpTypeWriting}{grammar} in \textcolor{rpTypeRule}{borderline} cases: \textcolor{rpTypeRule}{When} development and logical progression are strong but language contains many minor slips, lean toward 3. \textcolor{rpTypeRule}{When} language issues are moderate/severe and interfere with flow, lean lower.\par - Explicit justification required for scores that contradict primary weighting: \textcolor{rpTypeRule}{If} a language problem (rather than lack of development) is the main reason for assigning a lower score, explicitly note this in the rater's rationale and quote representative problematic sentences.\par - Calibration rule (new): \textcolor{rpTypeRule}{When} in doubt between 2 and 3, ask two questions:\par   1) Does the essay present sustained development (multiple, sufficiently explained examples or one clearly extended example)? \par   2) Do language problems make the text tiring to read or regularly require re-construction of meaning?\par   - \textcolor{rpTypeRule}{If} answer to (1) = yes and (2) = no, assign 3.\par   - \textcolor{rpTypeRule}{If} (1) = partial or no, assign 2 (or 1 \textcolor{rpTypeRule}{if} development is minimal).\par   - \textcolor{rpTypeRule}{If} (1) = yes but (2) = yes (frequent moderate errors causing reader fatigue), assign 2.\par - Calibration rule (new) for 1 vs 2:\par   - \textcolor{rpTypeRule}{When} in doubt between 1 and 2: ask whether there is any meaningful elaboration beyond a thesis sentence. \textcolor{rpTypeRule}{If} not, assign 1. \textcolor{rpTypeRule}{If} there is some elaboration but errors frequently force reconstruction of meaning, assign 1. \textcolor{rpTypeRule}{If} there is some elaboration and meaning is usually recoverable with modest effort, assign 2.\par \par Rater practice reminders (strengthened)\par - For every essay, explicitly identify the decisive dimension (development, \textcolor{rpTypeWriting}{organization}, or language) and cite specific \textcolor{rpTypeEvidence}{evidence} (\textcolor{rpTypeEvidence}{e.g.}, "single undeveloped example," "frequent sentence-level errors causing re-reading," "clear multi-paragraph development with examples X and Y").\par - \textcolor{rpTypeRule}{If} language is decisive, quote 1-2 representative sentences that illustrate the severity and describe whether they require re-reading or guessing.\par - Count and weigh examples: multiple distinct examples with explanation -> strong \textcolor{rpTypeEvidence}{evidence} for 3; single brief example -> \textcolor{rpTypeEvidence}{evidence} for 1.\par - \textcolor{rpTypeRule}{If} development is multi-example and clear but language errors are frequent, ask whether error-induced re-reading is occasional (3) or sustained/frequent (2).\par - \textcolor{rpTypeRule}{If} development is truncated at the end, explicitly weigh the strength of prior development before downgrading-only truncate to 2 \textcolor{rpTypeRule}{if} the truncation meaningfully reduces the essay's overall support.\par - \textcolor{rpTypeRule}{When} rating, prefer concrete justifications (\textcolor{rpTypeEvidence}{e.g.}, "three developed examples: X, Y, Z" or "language: 6 of \textcolor{rpTypeRule}{8 sentences} require re-reading") rather than vague statements.\par \par These refinements emphasize:\par - clearer tolerance rules for frequent surface errors \textcolor{rpTypeRule}{when} development is strong (reduce under-scoring of well-developed but error-prone essays),\par - stricter handling of single undeveloped examples (prevent over-scoring),\par - explicit guidance for truncated conclusions and repetitive development,\par - requirement to quote exemplar problematic sentences \textcolor{rpTypeRule}{when} language drives the score to improve consistency and justification.
\end{tcolorbox}
\end{minipage}
\caption{Pattern-highlighted rubric comparison (ets3, openai\_gpt-5-mini, base\_simplest\_True\_train100\_iteration5\_top3\_bs4-8-12\_mc4). Matched spans are color-coded by regex pattern. Color types: \textcolor{rpTypeRule}{\textbf{Rule Structure}} (if/threshold/stepwise guidance); \textcolor{rpTypeEvidence}{\textbf{Evidence Handling}} (examples, repetition, and caps); \textcolor{rpTypeWriting}{\textbf{Writing Quality}} (organization and grammar/mechanics).}
\label{fig:rubric_pattern_ets3_openai_gpt_5_mini_base_simplest_True_train100_iteration5_top3_bs4_8_12_mc4}
\end{figure*}
