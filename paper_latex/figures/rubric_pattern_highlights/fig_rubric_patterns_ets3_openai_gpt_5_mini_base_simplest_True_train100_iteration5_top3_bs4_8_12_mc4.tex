\colorlet{rpTypeRule}{red!80!black}
\colorlet{rpTypeEvidence}{blue!80!black}
\colorlet{rpTypeWriting}{teal!80!black}
\begin{figure*}[t]
\centering
\begin{tcolorbox}[colback=white,colframe=black!25,title=Pattern Type Guide,fonttitle=\bfseries\small,fontupper=\scriptsize,boxsep=1pt,left=2pt,right=2pt,top=2pt,bottom=2pt]
\textcolor{rpTypeRule}{\textbf{Rule Structure}}: Explicit decision logic for scoring: conditional branches, boundary tie-breakers, stepwise workflows, and numeric thresholds.\par \textcolor{rpTypeEvidence}{\textbf{Evidence Handling}}: How evidence is validated and counted: specific-example requirements, repetition/non-double-count rules, and cap rules for weak evidence.\par \textcolor{rpTypeWriting}{\textbf{Writing Quality}}: Language-quality criteria affecting score bands: organization/coherence/transition quality and grammar/mechanics severity.
\end{tcolorbox}
\vspace{1mm}
\begin{tcolorbox}[colback=white,colframe=black!25,title=Detailed Pattern Notes,fonttitle=\bfseries\small,fontupper=\scriptsize,boxsep=1pt,left=2pt,right=2pt,top=2pt,bottom=2pt]
\textcolor{rpTypeRule}{\textbf{Rule Structure}}:\par \quad \textcolor{rpTypeRule}{\textbf{Conditional Gating}} [n=52] Captures explicit condition-based rules that switch decisions only when a stated condition is met. Typical cues: if, when.\par \quad \textcolor{rpTypeRule}{\textbf{Boundary / Tie-Break Guidance}} [n=3] Marks criteria used to resolve borderline cases between adjacent score bands (e.g., 4 vs 5). Typical cues: tie-break, borderline, boundary, threshold, 4 vs 5.\par \quad \textcolor{rpTypeRule}{\textbf{Quantitative Threshold}} [n=1] Marks numeric cutoffs used for consistent decisions (minimum/maximum counts, percentages, explicit count thresholds). Typical cues: at least, at most, <=, >=, \%, N reasons/examples/sentences/words.\par \textcolor{rpTypeEvidence}{\textbf{Evidence Handling}}:\par \quad \textcolor{rpTypeEvidence}{\textbf{Specific Evidence Requirement}} [n=6] Highlights demands for concrete examples and explicit evidence links instead of generic assertions. Typical cues: for example, e.g., specific example, illustration, anecdote, evidence.\par \quad \textcolor{rpTypeEvidence}{\textbf{Off-Topic / Summary Cap}} [n=2] Identifies cap rules that restrict scores when responses are off-topic, irrelevant, or dominated by summary-only content. Typical cues: off-topic, irrelevant, digression, summary-only, cap.\par \quad \textcolor{rpTypeEvidence}{\textbf{Repetition Non-Count Rule}} [n=1] Captures rules that treat repetition/restatement as non-distinct support and prevent double-counting. Typical cues: repetition, restatement, double-count, do not double-count.\par \textcolor{rpTypeWriting}{\textbf{Writing Quality}}:\par \quad \textcolor{rpTypeWriting}{\textbf{Organization / Coherence Signal}} [n=16] Detects explicit references to discourse structure and logical flow as scoring criteria. Typical cues: organization, coherence, logical flow, transition.\par \quad \textcolor{rpTypeWriting}{\textbf{Grammar / Mechanics Signal}} [n=3] Detects references to language-form quality, especially grammar, spelling, punctuation, and mechanics. Typical cues: grammar, mechanics, spelling, punctuation.
\end{tcolorbox}
\vspace{1mm}
\begin{tcolorbox}[colback=white,colframe=black!25,title=Optimized Rubric (Pattern-Highlighted),fonttitle=\bfseries\small,fontupper=\scriptsize]
\ttfamily
- Primary weighting order (unchanged): Task fulfillment \& development > \textcolor{rpTypeWriting}{\textbf{Organization}} \& \textcolor{rpTypeWriting}{\textbf{coherence}} > Language control.\par ... [1 lines omitted] ...\par - Distinguish error severity precisely and apply \textcolor{rpTypeRule}{\textbf{threshold}}s consistently:\par   - Minor errors: rare \textcolor{rpTypeWriting}{\textbf{spelling}}/typo or small \textcolor{rpTypeWriting}{\textbf{punctuation}} slips that never interrupt comprehension.\par ... [2 lines omitted] ...\par - Always annotate which of the three dimensions (development, \textcolor{rpTypeWriting}{\textbf{organization}}, language) is the decisive factor for the assigned score, and cite explicit \textcolor{rpTypeEvidence}{\textbf{evidence}} (quote 1-2 exemplar sentences \textcolor{rpTypeRule}{\textbf{if}} language is the decisive issue).\par ... [4 lines omitted] ...\par   - Addresses counterarguments or shows depth of thought \textcolor{rpTypeRule}{\textbf{when}} relevant.\par   - Clarification added: \textcolor{rpTypeRule}{\textbf{When}} multiple distinct examples are present and each is developed with some explanation/connection to the thesis, favor score 3 even \textcolor{rpTypeRule}{\textbf{if}} language contains many surface-level errors - provided those errors do not regularly obscure meaning or create reader fatigue.\par - \textcolor{rpTypeWriting}{\textbf{Organization}} \& \textcolor{rpTypeWriting}{\textbf{coherence}}:\par   - Clear progression of ideas, logical paragraphing, and explicit links between points. A truncated or slightly abrupt conclusion does not automatically negate a 3 \textcolor{rpTypeRule}{\textbf{if}} multi-paragraph development is sustained and convincing.\par ... [2 lines omitted] ...\par   - Errors, \textcolor{rpTypeRule}{\textbf{if}} present, are minor or occasional moderate slips that do not disrupt reading. Frequent moderate errors are acceptable for 3 only \textcolor{rpTypeRule}{\textbf{when}} (a) they do not force repeated re-construction of meaning, and (b) the essay remains easy to follow without sustained effort.\par - \textcolor{rpTypeRule}{\textbf{When}} to assign 3 (clarified):\par   - Use 3 \textcolor{rpTypeRule}{\textbf{when}} task is fully addressed with well-developed support and clear \textcolor{rpTypeWriting}{\textbf{organization}}-even \textcolor{rpTypeRule}{\textbf{if}} surface-level errors are frequent-ONLY \textcolor{rpTypeRule}{\textbf{IF}} those errors are mostly minor or at worst recurrent-moderate but do not cause reader fatigue or obscure meaning.\par   - Do NOT give 3 \textcolor{rpTypeRule}{\textbf{when}} development is repetitive, thin, or relies on a single undeveloped/example assertion even \textcolor{rpTypeRule}{\textbf{if}} language is good.\par ... [4 lines omitted] ...\par   - Typical patterns: general statements without adequate elaboration, brief examples that are not extended, or \textcolor{rpTypeEvidence}{\textbf{repetition}} of ideas rather than deeper development.\par   - Clarification added: \textcolor{rpTypeRule}{\textbf{If}} there are multiple examples but each is only briefly explained and the essay does not show clear, sustained development, assign 2 (even \textcolor{rpTypeRule}{\textbf{if}} language is relatively accurate).\par - \textcolor{rpTypeWriting}{\textbf{Organization}} \& \textcolor{rpTypeWriting}{\textbf{coherence}}:\par   - Overall progression exists but may be disjointed or repetitious; minor \textcolor{rpTypeEvidence}{\textbf{digression}}s or redundancy are common.\par ... [3 lines omitted] ...\par   - Assign 2 \textcolor{rpTypeRule}{\textbf{when}} moderate or some severe errors occur but communication generally remains recoverable without substantial effort.\par   - Clarification added: Frequent moderate errors that cause reader fatigue (i.e., intermittent re-reading but not constant reconstruction) merit score 2. \textcolor{rpTypeRule}{\textbf{If}} errors are frequent but readers can follow the argument with modest effort, prefer 2 over 1.\par - \textcolor{rpTypeRule}{\textbf{When}} to assign 2 (clarified):\par   - Use 2 \textcolor{rpTypeRule}{\textbf{when}} there is some development and \textcolor{rpTypeWriting}{\textbf{organization}} but support is noticeably limited, OR \textcolor{rpTypeRule}{\textbf{when}} moderate/accumulated errors begin to slow the reader though meaning is still mostly recoverable.\par   - Use 2 (not 3) \textcolor{rpTypeRule}{\textbf{when}} development is repetitive or \textcolor{rpTypeWriting}{\textbf{organization}}ally weak even \textcolor{rpTypeRule}{\textbf{if}} there are no totally unreadable sentences.\par ... [3 lines omitted] ...\par   - Very limited development: few or no relevant examples; explanations too brief, \textcolor{rpTypeEvidence}{\textbf{irrelevant}}, or repetitive/general assertions without support.\par ... [1 lines omitted] ...\par - \textcolor{rpTypeWriting}{\textbf{Organization}} \& \textcolor{rpTypeWriting}{\textbf{coherence}}:\par   - Inadequate \textcolor{rpTypeWriting}{\textbf{organization}}; reader has difficulty following progression; extremely short or fragmentary responses.\par ... [2 lines omitted] ...\par   - Accumulation of severe errors that often obscure meaning or make the essay hard to understand-score 1 even \textcolor{rpTypeRule}{\textbf{if}} some relevant ideas exist.\par   - Clarification added: To justify score 1 on language grounds, errors must (a) occur frequently across the essay and (b) require repeated reconstruction of meaning (i.e., reader must regularly infer or guess meaning). Isolated severe errors do not by themselves require score 1 \textcolor{rpTypeRule}{\textbf{if}} overall meaning is recoverable.\par - \textcolor{rpTypeRule}{\textbf{When}} to assign 1 (clarified):\par ... [1 lines omitted] ...\par   - Use 1 \textcolor{rpTypeRule}{\textbf{when}} the reader must frequently re-construct meaning due to severe/accumulated errors, even \textcolor{rpTypeRule}{\textbf{if}} the essay attempts relevant content.\par ... [1 lines omitted] ...\par Additional clarifications and \textcolor{rpTypeRule}{\textbf{borderline}} decision rules (to improve inter-rater agreement)\par ... [1 lines omitted] ...\par - Many surface errors but strong development: Favor the higher score (3) only \textcolor{rpTypeRule}{\textbf{if}} errors are mostly minor or occasional-moderate and do not create reading fatigue. \textcolor{rpTypeRule}{\textbf{If}} errors are frequent enough to make sustained reading effortful, downgrade to 2.\par - Frequent moderate errors that cause reader fatigue: Downgrade to 2 even \textcolor{rpTypeRule}{\textbf{when}} development and \textcolor{rpTypeWriting}{\textbf{organization}} are relatively strong.\par - Severe/accumulated errors that obscure meaning: Downgrade to 1 regardless of topic relevance \textcolor{rpTypeRule}{\textbf{if}} comprehension is regularly impeded.\par - Repetitive development: \textcolor{rpTypeRule}{\textbf{If}} the essay repeats the same idea across paragraphs without extending or deepening it, prefer score 2 rather than 3.\par - Incomplete or abrupt conclusion: \textcolor{rpTypeRule}{\textbf{If}} the essay ends mid-thought or with an incomplete conclusion, prefer score 2 unless the rest of the essay demonstrates clear, multi-paragraph, well-elaborated development with multiple examples (in which case 3 may still be warranted).\par - \textcolor{rpTypeWriting}{\textbf{Organization}} outweighs \textcolor{rpTypeWriting}{\textbf{grammar}} in \textcolor{rpTypeRule}{\textbf{borderline}} cases: \textcolor{rpTypeRule}{\textbf{When}} development and logical progression are strong but language contains many minor slips, lean toward 3. \textcolor{rpTypeRule}{\textbf{When}} language issues are moderate/severe and interfere with flow, lean lower.\par - Explicit justification required for scores that contradict primary weighting: \textcolor{rpTypeRule}{\textbf{If}} a language problem (rather than lack of development) is the main reason for assigning a lower score, explicitly note this in the rater's rationale and quote representative problematic sentences.\par - Calibration rule (new): \textcolor{rpTypeRule}{\textbf{When}} in doubt between 2 and 3, ask two questions:\par ... [2 lines omitted] ...\par   - \textcolor{rpTypeRule}{\textbf{If}} answer to (1) = yes and (2) = no, assign 3.\par   - \textcolor{rpTypeRule}{\textbf{If}} (1) = partial or no, assign 2 (or 1 \textcolor{rpTypeRule}{\textbf{if}} development is minimal).\par   - \textcolor{rpTypeRule}{\textbf{If}} (1) = yes but (2) = yes (frequent moderate errors causing reader fatigue), assign 2.\par ... [1 lines omitted] ...\par   - \textcolor{rpTypeRule}{\textbf{When}} in doubt between 1 and 2: ask whether there is any meaningful elaboration beyond a thesis sentence. \textcolor{rpTypeRule}{\textbf{If}} not, assign 1. \textcolor{rpTypeRule}{\textbf{If}} there is some elaboration but errors frequently force reconstruction of meaning, assign 1. \textcolor{rpTypeRule}{\textbf{If}} there is some elaboration and meaning is usually recoverable with modest effort, assign 2.\par ... [2 lines omitted] ...\par - For every essay, explicitly identify the decisive dimension (development, \textcolor{rpTypeWriting}{\textbf{organization}}, or language) and cite specific \textcolor{rpTypeEvidence}{\textbf{evidence}} (\textcolor{rpTypeEvidence}{\textbf{e.g.}}, "single undeveloped example," "frequent sentence-level errors causing re-reading," "clear multi-paragraph development with examples X and Y").\par - \textcolor{rpTypeRule}{\textbf{If}} language is decisive, quote 1-2 representative sentences that illustrate the severity and describe whether they require re-reading or guessing.\par - Count and weigh examples: multiple distinct examples with explanation -> strong \textcolor{rpTypeEvidence}{\textbf{evidence}} for 3; single brief example -> \textcolor{rpTypeEvidence}{\textbf{evidence}} for 1.\par - \textcolor{rpTypeRule}{\textbf{If}} development is multi-example and clear but language errors are frequent, ask whether error-induced re-reading is occasional (3) or sustained/frequent (2).\par - \textcolor{rpTypeRule}{\textbf{If}} development is truncated at the end, explicitly weigh the strength of prior development before downgrading-only truncate to 2 \textcolor{rpTypeRule}{\textbf{if}} the truncation meaningfully reduces the essay's overall support.\par - \textcolor{rpTypeRule}{\textbf{When}} rating, prefer concrete justifications (\textcolor{rpTypeEvidence}{\textbf{e.g.}}, "three developed examples: X, Y, Z" or "language: 6 of \textcolor{rpTypeRule}{\textbf{8 sentences}} require re-reading") rather than vague statements.\par ... [2 lines omitted] ...\par - clearer tolerance rules for frequent surface errors \textcolor{rpTypeRule}{\textbf{when}} development is strong (reduce under-scoring of well-developed but error-prone essays),\par ... [2 lines omitted] ...\par - requirement to quote exemplar problematic sentences \textcolor{rpTypeRule}{\textbf{when}} language drives the score to improve consistency and justification.
\end{tcolorbox}
\caption{Pattern-focused view of the optimized rubric (ets3, openai\_gpt-5-mini, base\_simplest\_True\_train100\_iteration5\_top3\_bs4-8-12\_mc4). Colored bold spans indicate regex-matched rubric cues. Color types: \textcolor{rpTypeRule}{\textbf{Rule Structure}} (Explicit decision logic for scoring: conditional branches, boundary tie-breakers, stepwise workflows, and numeric thresholds.); \textcolor{rpTypeEvidence}{\textbf{Evidence Handling}} (How evidence is validated and counted: specific-example requirements, repetition/non-double-count rules, and cap rules for weak evidence.); \textcolor{rpTypeWriting}{\textbf{Writing Quality}} (Language-quality criteria affecting score bands: organization/coherence/transition quality and grammar/mechanics severity.). Matched pattern categories: Conditional Gating (n=52); Boundary / Tie-Break Guidance (n=3); Specific Evidence Requirement (n=6); Off-Topic / Summary Cap (n=2); Organization / Coherence Signal (n=16); Grammar / Mechanics Signal (n=3); Repetition Non-Count Rule (n=1); Quantitative Threshold (n=1).}
\label{fig:rubric_pattern_ets3_openai_gpt_5_mini_base_simplest_True_train100_iteration5_top3_bs4_8_12_mc4}
\end{figure*}
