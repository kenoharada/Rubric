\colorlet{rpTypeRule}{red!80!black}
\colorlet{rpTypeEvidence}{blue!80!black}
\colorlet{rpTypeWriting}{teal!80!black}
\begin{figure*}[t]
\centering
\begin{tcolorbox}[colback=white,colframe=black!25,title=Pattern Legend,fonttitle=\bfseries\small,fontupper=\scriptsize,boxsep=1pt,left=2pt,right=2pt,top=2pt,bottom=2pt]
\textcolor{rpTypeRule}{\textbf{Rule Structure}} (if/threshold/stepwise guidance) \quad \textcolor{rpTypeEvidence}{\textbf{Evidence Handling}} (examples, repetition, and caps) \quad \textcolor{rpTypeWriting}{\textbf{Writing Quality}} (organization and grammar/mechanics)
\end{tcolorbox}
\vspace{2mm}
\begin{minipage}[t]{0.485\textwidth}
\begin{tcolorbox}[colback=white,colframe=black!25,title=Initial Rubric,fonttitle=\bfseries\small,fontupper=\scriptsize,breakable]
\ttfamily
- Contains only general reasons with unelaborated and/or list-like details.\par - Shows little or no \textcolor{rpTypeEvidence}{evidence} of \textcolor{rpTypeWriting}{organization}.\par - May be awkward and confused or simplistic.\par ... [3 lines omitted] ...\par - Has reasons with minimal elaboration and more general than specific details.\par - Shows some \textcolor{rpTypeWriting}{organization}.\par - May be awkward in parts with few \textcolor{rpTypeWriting}{transition}s.\par - Shows some awareness of audience.\par ... [2 lines omitted] ...\par - Has adequately elaborated reasons with a mix of general and specific details.\par - Shows satisfactory \textcolor{rpTypeWriting}{organization}.\par - May be somewhat fluent with some \textcolor{rpTypeWriting}{transition}al language.\par - Shows adequate awareness of audience.\par ... [2 lines omitted] ...\par - Has moderately well elaborated reasons with mostly specific details.\par - Exhibits generally strong \textcolor{rpTypeWriting}{organization}.\par - May be moderately fluent with \textcolor{rpTypeWriting}{transition}al language throughout.\par - May show a consistent awareness of audience.\par ... [2 lines omitted] ...\par - Has fully elaborated reasons with specific details.\par - Exhibits strong \textcolor{rpTypeWriting}{organization}.\par - Is fluent and uses sophisticated \textcolor{rpTypeWriting}{transition}al language.\par - May show a heightened awareness of audience.\par ... [1 lines omitted] ...\par Note: \par I have made an effort to remove personally identifying information from the essays using the Named Entity Recognizer (NER). The relevant entities are identified in the text and then replaced with a string such as "PERSON", "\textcolor{rpTypeWriting}{ORGANIZATION}", "LOCATION", "DATE", "TIME", "MONEY", "PERCENT", "CAPS" (any capitalized word) and "NUM" (any digits). Please do not penalize the essay because of the anonymizations.
\end{tcolorbox}
\end{minipage}
\hfill
\begin{minipage}[t]{0.485\textwidth}
\begin{tcolorbox}[colback=white,colframe=black!25,title=Optimized Rubric,fonttitle=\bfseries\small,fontupper=\scriptsize,breakable]
\ttfamily
- Shows no awareness of the specific prompt or audience.\par Note: \textcolor{rpTypeRule}{If} a response provides any discernible reasons or specific details, even with severe mechanical errors, move to \textcolor{rpTypeRule}{at least} a Score Point 2.\par \par ... [1 lines omitted] ...\par - Contains only one general reason or a very short, unelaborated list of ideas.\par - Lists items (\textcolor{rpTypeEvidence}{e.g.}, "play games, go on Facebook") without any explanation of why they are good or how they work.\par - Shows little or no \textcolor{rpTypeEvidence}{evidence} of structured \textcolor{rpTypeWriting}{organization}.\par - Highly simplistic or redundant language.\par ... [3 lines omitted] ...\par - Provides a list of reasons with very brief elaboration (usually only one sentence per point).\par - Offers more general than specific details. Even \textcolor{rpTypeRule}{if} proper nouns or specific platforms (\textcolor{rpTypeEvidence}{e.g.}, "Facebook," "Colombia") are mentioned, \textcolor{rpTypeRule}{if} the sentence following them provides no further development of the idea, the response remains a 3.\par - Basic "listing" \textcolor{rpTypeWriting}{organization} (\textcolor{rpTypeEvidence}{e.g.}, "\textcolor{rpTypeRule}{First}, \textcolor{rpTypeRule}{Second}, Finally") with little internal development.\par - Characterized by highly simplistic sentence structures and vocabulary.\par ... [3 lines omitted] ...\par - Addresses several reasons with a mix of general and specific details.\par - Support often feels formulaic or relies heavily on the ideas/phrasing provided in the prompt (\textcolor{rpTypeEvidence}{e.g.}, simply expanding on "exercise" and "nature" without unique hypothetical scenarios).\par - Contains \textcolor{rpTypeEvidence}{anecdote}s or examples that are present but lack multi-layered development or "\textcolor{rpTypeWriting}{mechanics}" (\textcolor{rpTypeEvidence}{e.g.}, mentioning a surgery simulation or a personal fall but not exploring the broader implications).\par - Shows satisfactory \textcolor{rpTypeWriting}{organization} with a clear intro, body, and conclusion.\par - May contain significant mechanical, \textcolor{rpTypeWriting}{spelling}, or syntax errors; however, the argument is logically coherent.\par - Language is functional but lacks fluency. Even \textcolor{rpTypeRule}{if} the essay introduces an original \textcolor{rpTypeRule}{third} point (like safety or jobs), \textcolor{rpTypeRule}{if} the prose is clumsy and the elaboration is limited to 2-\textcolor{rpTypeRule}{3 sentences} per point, it should remain a 4.\par \par Score Point 5: A developed response that takes a clear position and provides moderately persuasive support. Typical elements:\par - Has well-elaborated reasons with specific, original details that explore "\textcolor{rpTypeWriting}{mechanics}" (\textcolor{rpTypeEvidence}{e.g.}, describing the physical sensation of "getting fat," or the specific way a webcam creates "human interaction" compared to a phone).\par - Moves beyond the prompt's suggestions by significantly expanding on how technology functions in personal or professional lives (\textcolor{rpTypeEvidence}{e.g.}, how a nurse uses videos for \textcolor{rpTypeRule}{procedure}s or the emotional impact of staying in touch with a friend who moved).\par - Demonstrates a consistent persuasive tone and a clear attempt to engage the reader personally.\par - While it may use a formulaic structure (\textcolor{rpTypeEvidence}{e.g.}, "My \textcolor{rpTypeRule}{first} reason... My last reason..."), the depth of the elaboration and the use of original scenarios justify the 5. Depth of content overrides repetitive \textcolor{rpTypeWriting}{transition}s at this level.\par - Addresses the counter-argument or looks at the issue from multiple perspectives.\par ... [1 lines omitted] ...\par Score Point 6: A well-developed response that takes a clear and thoughtful position and provides persuasive, in-depth support. Typical elements:\par - Fully elaborated reasons with numerous specific, concrete details or "multi-layered" \textcolor{rpTypeEvidence}{anecdote}s.\par - Goes significantly beyond explaining "why" to explore the "implications" and "consequences" of the position (\textcolor{rpTypeEvidence}{e.g.}, connecting computer use to global warming via power plants, or the specific danger of losing survival skills like lighting matches during a natural disaster).\par - Exhibits sophisticated \textcolor{rpTypeWriting}{organization} or creative framing (\textcolor{rpTypeEvidence}{e.g.}, using a powerful opening quote or a rhetorical "hook").\par - Shows a heightened awareness of audience through a compelling persuasive voice or a strong emotional hook that connects with the reader's daily life.\par ... [4 lines omitted] ...\par - "In-depth development" (Score 5 and 6) is defined by the student's ability to visualize a scenario for the reader or explain a cause-and-effect chain.\par - A \textcolor{rpTypeRule}{4 vs. 5} Distinction: \textcolor{rpTypeRule}{If} an essay provides \textcolor{rpTypeEvidence}{specific example}s (like surgery simulations or medical stats) but fails to explain the *ripple effects* or *human impact* of those examples, it is a 4. \textcolor{rpTypeRule}{If} it explores the "how" and "why" behind the examples (\textcolor{rpTypeEvidence}{e.g.}, the emotional relief of a webcam or the specific process of getting fit), it is a 5.\par - A 5 vs. 6 Distinction: A 6 must explore broader societal or philosophical "implications" (\textcolor{rpTypeEvidence}{e.g.}, global warming, natural disaster survival, the future of the environment) or use highly creative framing and sophisticated voice.\par - Heavy \textcolor{rpTypeEvidence}{repetition} in sentence structure (\textcolor{rpTypeEvidence}{e.g.}, "One reason is... Another reason is...") caps an essay at 4 ONLY \textcolor{rpTypeRule}{IF} the content is also basic/formulaic. \textcolor{rpTypeRule}{If} the content within those paragraphs is deep and explores implications, it should be moved to a 5.\par - Proper nouns or platform names do not automatically grant a higher score; they must be accompanied by an explanation of *how* or *why* that specific item supports the argument.
\end{tcolorbox}
\end{minipage}
\caption{Pattern-highlighted rubric comparison (asap\_1, google\_gemini-3-flash-preview, base\_expert\_True\_train100\_iteration5\_top3\_bs4-8-12\_mc4). Matched spans are color-coded by regex pattern. Color types: \textcolor{rpTypeRule}{\textbf{Rule Structure}} (if/threshold/stepwise guidance); \textcolor{rpTypeEvidence}{\textbf{Evidence Handling}} (examples, repetition, and caps); \textcolor{rpTypeWriting}{\textbf{Writing Quality}} (organization and grammar/mechanics).}
\label{fig:rubric_pattern_asap_1_google_gemini_3_flash_preview_base_expert_True_train100_iteration5_top3_bs4_8_12_mc4}
\end{figure*}
