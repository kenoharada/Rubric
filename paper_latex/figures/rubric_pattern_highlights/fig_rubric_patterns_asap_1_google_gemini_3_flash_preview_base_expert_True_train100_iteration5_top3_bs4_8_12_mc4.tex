\colorlet{rpTypeRule}{red!80!black}
\colorlet{rpTypeEvidence}{blue!80!black}
\colorlet{rpTypeWriting}{teal!80!black}
\begin{figure*}[t]
\centering
\begin{tcolorbox}[colback=white,colframe=black!25,title=Pattern Type Guide,fonttitle=\bfseries\small,fontupper=\scriptsize,boxsep=1pt,left=2pt,right=2pt,top=2pt,bottom=2pt]
\textcolor{rpTypeRule}{\textbf{Rule Structure}}: Explicit decision logic for scoring: conditional branches, boundary tie-breakers, stepwise workflows, and numeric thresholds.\par \textcolor{rpTypeEvidence}{\textbf{Evidence Handling}}: How evidence is validated and counted: specific-example requirements, repetition/non-double-count rules, and cap rules for weak evidence.\par \textcolor{rpTypeWriting}{\textbf{Writing Quality}}: Language-quality criteria affecting score bands: organization/coherence/transition quality and grammar/mechanics severity.
\end{tcolorbox}
\vspace{1mm}
\begin{tcolorbox}[colback=white,colframe=black!25,title=Detailed Pattern Notes,fonttitle=\bfseries\small,fontupper=\scriptsize,boxsep=1pt,left=2pt,right=2pt,top=2pt,bottom=2pt]
\textcolor{rpTypeRule}{\textbf{Rule Structure}}:\par \quad \textcolor{rpTypeRule}{\textbf{Conditional Gating}} [n=9] Captures explicit condition-based rules that switch decisions only when a stated condition is met. Typical cues: if, when.\par \quad \textcolor{rpTypeRule}{\textbf{Boundary / Tie-Break Guidance}} [n=1] Marks criteria used to resolve borderline cases between adjacent score bands (e.g., 4 vs 5). Typical cues: tie-break, borderline, boundary, threshold, 4 vs 5.\par \quad \textcolor{rpTypeRule}{\textbf{Stepwise Rating Workflow}} [n=5] Detects ordered procedures and checklists that standardize how raters walk through scoring decisions. Typical cues: step, checklist, workflow, procedure, first/second/third.\par \quad \textcolor{rpTypeRule}{\textbf{Quantitative Threshold}} [n=2] Marks numeric cutoffs used for consistent decisions (minimum/maximum counts, percentages, explicit count thresholds). Typical cues: at least, at most, <=, >=, \%, N reasons/examples/sentences/words.\par \textcolor{rpTypeEvidence}{\textbf{Evidence Handling}}:\par \quad \textcolor{rpTypeEvidence}{\textbf{Specific Evidence Requirement}} [n=17] Highlights demands for concrete examples and explicit evidence links instead of generic assertions. Typical cues: for example, e.g., specific example, illustration, anecdote, evidence.\par \quad \textcolor{rpTypeEvidence}{\textbf{Repetition Non-Count Rule}} [n=1] Captures rules that treat repetition/restatement as non-distinct support and prevent double-counting. Typical cues: repetition, restatement, double-count, do not double-count.\par \textcolor{rpTypeWriting}{\textbf{Writing Quality}}:\par \quad \textcolor{rpTypeWriting}{\textbf{Organization / Coherence Signal}} [n=5] Detects explicit references to discourse structure and logical flow as scoring criteria. Typical cues: organization, coherence, logical flow, transition.\par \quad \textcolor{rpTypeWriting}{\textbf{Grammar / Mechanics Signal}} [n=3] Detects references to language-form quality, especially grammar, spelling, punctuation, and mechanics. Typical cues: grammar, mechanics, spelling, punctuation.
\end{tcolorbox}
\vspace{1mm}
\begin{tcolorbox}[colback=white,colframe=black!25,title=Optimized Rubric (Pattern-Highlighted),fonttitle=\bfseries\small,fontupper=\scriptsize]
\ttfamily
Note: \textcolor{rpTypeRule}{\textbf{If}} a response provides any discernible reasons or specific details, even with severe mechanical errors, move to \textcolor{rpTypeRule}{\textbf{at least}} a Score Point 2.\par ... [3 lines omitted] ...\par - Lists items (\textcolor{rpTypeEvidence}{\textbf{e.g.}}, "play games, go on Facebook") without any explanation of why they are good or how they work.\par - Shows little or no \textcolor{rpTypeEvidence}{\textbf{evidence}} of structured \textcolor{rpTypeWriting}{\textbf{organization}}.\par ... [5 lines omitted] ...\par - Offers more general than specific details. Even \textcolor{rpTypeRule}{\textbf{if}} proper nouns or specific platforms (\textcolor{rpTypeEvidence}{\textbf{e.g.}}, "Facebook," "Colombia") are mentioned, \textcolor{rpTypeRule}{\textbf{if}} the sentence following them provides no further development of the idea, the response remains a 3.\par - Basic "listing" \textcolor{rpTypeWriting}{\textbf{organization}} (\textcolor{rpTypeEvidence}{\textbf{e.g.}}, "\textcolor{rpTypeRule}{\textbf{First}}, \textcolor{rpTypeRule}{\textbf{Second}}, Finally") with little internal development.\par ... [5 lines omitted] ...\par - Support often feels formulaic or relies heavily on the ideas/phrasing provided in the prompt (\textcolor{rpTypeEvidence}{\textbf{e.g.}}, simply expanding on "exercise" and "nature" without unique hypothetical scenarios).\par - Contains \textcolor{rpTypeEvidence}{\textbf{anecdote}}s or examples that are present but lack multi-layered development or "\textcolor{rpTypeWriting}{\textbf{mechanics}}" (\textcolor{rpTypeEvidence}{\textbf{e.g.}}, mentioning a surgery simulation or a personal fall but not exploring the broader implications).\par - Shows satisfactory \textcolor{rpTypeWriting}{\textbf{organization}} with a clear intro, body, and conclusion.\par - May contain significant mechanical, \textcolor{rpTypeWriting}{\textbf{spelling}}, or syntax errors; however, the argument is logically coherent.\par - Language is functional but lacks fluency. Even \textcolor{rpTypeRule}{\textbf{if}} the essay introduces an original \textcolor{rpTypeRule}{\textbf{third}} point (like safety or jobs), \textcolor{rpTypeRule}{\textbf{if}} the prose is clumsy and the elaboration is limited to 2-\textcolor{rpTypeRule}{\textbf{3 sentences}} per point, it should remain a 4.\par ... [2 lines omitted] ...\par - Has well-elaborated reasons with specific, original details that explore "\textcolor{rpTypeWriting}{\textbf{mechanics}}" (\textcolor{rpTypeEvidence}{\textbf{e.g.}}, describing the physical sensation of "getting fat," or the specific way a webcam creates "human interaction" compared to a phone).\par - Moves beyond the prompt's suggestions by significantly expanding on how technology functions in personal or professional lives (\textcolor{rpTypeEvidence}{\textbf{e.g.}}, how a nurse uses videos for \textcolor{rpTypeRule}{\textbf{procedure}}s or the emotional impact of staying in touch with a friend who moved).\par ... [1 lines omitted] ...\par - While it may use a formulaic structure (\textcolor{rpTypeEvidence}{\textbf{e.g.}}, "My \textcolor{rpTypeRule}{\textbf{first}} reason... My last reason..."), the depth of the elaboration and the use of original scenarios justify the 5. Depth of content overrides repetitive \textcolor{rpTypeWriting}{\textbf{transition}}s at this level.\par ... [3 lines omitted] ...\par - Fully elaborated reasons with numerous specific, concrete details or "multi-layered" \textcolor{rpTypeEvidence}{\textbf{anecdote}}s.\par - Goes significantly beyond explaining "why" to explore the "implications" and "consequences" of the position (\textcolor{rpTypeEvidence}{\textbf{e.g.}}, connecting computer use to global warming via power plants, or the specific danger of losing survival skills like lighting matches during a natural disaster).\par - Exhibits sophisticated \textcolor{rpTypeWriting}{\textbf{organization}} or creative framing (\textcolor{rpTypeEvidence}{\textbf{e.g.}}, using a powerful opening quote or a rhetorical "hook").\par ... [6 lines omitted] ...\par - A \textcolor{rpTypeRule}{\textbf{4 vs. 5}} Distinction: \textcolor{rpTypeRule}{\textbf{If}} an essay provides \textcolor{rpTypeEvidence}{\textbf{specific example}}s (like surgery simulations or medical stats) but fails to explain the *ripple effects* or *human impact* of those examples, it is a 4. \textcolor{rpTypeRule}{\textbf{If}} it explores the "how" and "why" behind the examples (\textcolor{rpTypeEvidence}{\textbf{e.g.}}, the emotional relief of a webcam or the specific process of getting fit), it is a 5.\par - A 5 vs. 6 Distinction: A 6 must explore broader societal or philosophical "implications" (\textcolor{rpTypeEvidence}{\textbf{e.g.}}, global warming, natural disaster survival, the future of the environment) or use highly creative framing and sophisticated voice.\par - Heavy \textcolor{rpTypeEvidence}{\textbf{repetition}} in sentence structure (\textcolor{rpTypeEvidence}{\textbf{e.g.}}, "One reason is... Another reason is...") caps an essay at 4 ONLY \textcolor{rpTypeRule}{\textbf{IF}} the content is also basic/formulaic. \textcolor{rpTypeRule}{\textbf{If}} the content within those paragraphs is deep and explores implications, it should be moved to a 5.
\end{tcolorbox}
\caption{Pattern-focused view of the optimized rubric (asap\_1, google\_gemini-3-flash-preview, base\_expert\_True\_train100\_iteration5\_top3\_bs4-8-12\_mc4). Colored bold spans indicate regex-matched rubric cues. Color types: \textcolor{rpTypeRule}{\textbf{Rule Structure}} (Explicit decision logic for scoring: conditional branches, boundary tie-breakers, stepwise workflows, and numeric thresholds.); \textcolor{rpTypeEvidence}{\textbf{Evidence Handling}} (How evidence is validated and counted: specific-example requirements, repetition/non-double-count rules, and cap rules for weak evidence.); \textcolor{rpTypeWriting}{\textbf{Writing Quality}} (Language-quality criteria affecting score bands: organization/coherence/transition quality and grammar/mechanics severity.). Matched pattern categories: Conditional Gating (n=9); Boundary / Tie-Break Guidance (n=1); Stepwise Rating Workflow (n=5); Specific Evidence Requirement (n=17); Organization / Coherence Signal (n=5); Grammar / Mechanics Signal (n=3); Repetition Non-Count Rule (n=1); Quantitative Threshold (n=2).}
\label{fig:rubric_pattern_asap_1_google_gemini_3_flash_preview_base_expert_True_train100_iteration5_top3_bs4_8_12_mc4}
\end{figure*}
