\documentclass[11pt]{article}

% Change "review" to "final" to generate the final (sometimes called camera-ready) version.
% Change to "preprint" to generate a non-anonymous version with page numbers.
\usepackage[review]{acl}

% Standard package includes
\usepackage{times}
\usepackage{latexsym}

% For proper rendering and hyphenation of words containing Latin characters (including in bib files)
\usepackage[T1]{fontenc}
% For Vietnamese characters
% \usepackage[T5]{fontenc}
% See https://www.latex-project.org/help/documentation/encguide.pdf for other character sets

% This assumes your files are encoded as UTF8
\usepackage[utf8]{inputenc}

% This is not strictly necessary, and may be commented out,
% but it will improve the layout of the manuscript,
% and will typically save some space.
\usepackage{microtype}

% This is also not strictly necessary, and may be commented out.
% However, it will improve the aesthetics of text in
% the typewriter font.
\usepackage{inconsolata}

%Including images in your LaTeX document requires adding
%additional package(s)
\usepackage{graphicx}
\usepackage{amsmath}
\usepackage{amssymb}
\usepackage{booktabs}
\usepackage{algorithm}
\usepackage{algpseudocode}
\usepackage{colortbl}
\usepackage[most]{tcolorbox}
\definecolor{RoyalBlue}{RGB}{65,105,225}
\colorlet{White}{white}
\newcommand{\procname}[1]{\textsc{#1}}
\newcommand{\rubric}{\mathrm{rubric}}

% If the title and author information does not fit in the area allocated, uncomment the following
%
%\setlength\titlebox{<dim>}
%
% and set <dim> to something 5cm or larger.

\title{Automated Refinement of Essay Scoring Rubrics \\for Language Models via Reflect-and-Revise}

\author{
 \textbf{Keno Harada},
 \textbf{Lui Yoshida},
 \textbf{Takeshi Kojima},
 \textbf{Yusuke Iwasawa},
 \textbf{Yutaka Matsuo}
\\
The University of Tokyo
\\
 \small{
 \texttt{{keno.harada@weblab.t.u-tokyo.ac.jp}}}
 }

% Author information can be set in various styles:
% For several authors from the same institution:
% \author{Author 1 \and ... \and Author n \\
%         Address line \\ ... \\ Address line}
% if the names do not fit well on one line use
%         Author 1 \\ {\bf Author 2} \\ ... \\ {\bf Author n} \\
% For authors from different institutions:
% \author{Author 1 \\ Address line \\  ... \\ Address line
%         \And  ... \And
%         Author n \\ Address line \\ ... \\ Address line}
% To start a separate ``row'' of authors use \AND, as in
% \author{Author 1 \\ Address line \\  ... \\ Address line
%         \AND
%         Author 2 \\ Address line \\ ... \\ Address line \And
%         Author 3 \\ Address line \\ ... \\ Address line}

%\author{
%  \textbf{First Author\textsuperscript{1}},
%  \textbf{Second Author\textsuperscript{1,2}},
%  \textbf{Third T. Author\textsuperscript{1}},
%  \textbf{Fourth Author\textsuperscript{1}},
%\\
%  \textbf{Fifth Author\textsuperscript{1,2}},
%  \textbf{Sixth Author\textsuperscript{1}},
%  \textbf{Seventh Author\textsuperscript{1}},
%  \textbf{Eighth Author \textsuperscript{1,2,3,4}},
%\\
%  \textbf{Ninth Author\textsuperscript{1}},
%  \textbf{Tenth Author\textsuperscript{1}},
%  \textbf{Eleventh E. Author\textsuperscript{1,2,3,4,5}},
%  \textbf{Twelfth Author\textsuperscript{1}},
%\\
%  \textbf{Thirteenth Author\textsuperscript{3}},
%  \textbf{Fourteenth F. Author\textsuperscript{2,4}},
%  \textbf{Fifteenth Author\textsuperscript{1}},
%  \textbf{Sixteenth Author\textsuperscript{1}},
%\\
%  \textbf{Seventeenth S. Author\textsuperscript{4,5}},
%  \textbf{Eighteenth Author\textsuperscript{3,4}},
%  \textbf{Nineteenth N. Author\textsuperscript{2,5}},
%  \textbf{Twentieth Author\textsuperscript{1}}
%\\
%\\
%  \textsuperscript{1}Affiliation 1,
%  \textsuperscript{2}Affiliation 2,
%  \textsuperscript{3}Affiliation 3,
%  \textsuperscript{4}Affiliation 4,
%  \textsuperscript{5}Affiliation 5
%\\
%  \small{
%    \textbf{Correspondence:} \href{mailto:email@domain}{email@domain}
%  }
%}

\begin{document}
\maketitle
\begin{abstract}
Large Language Models (LLMs) are increasingly used for Automated Essay Scoring (AES), yet the scoring rubrics they rely on are typically designed for human raters and may not be optimal for LLMs. Inspired by the calibration process that human raters undergo before formal scoring, we propose Reflect-and-Revise, an iterative framework that refines scoring rubrics by prompting models to reflect on their own chain-of-thought rationales and observed score discrepancies with human labels. At each iteration, the model identifies systematic scoring error patterns from sampled mismatches and revises the rubric accordingly. Experiments on three essay scoring benchmarks (ASAP, ASAP 2.0, and TOEFL11) with three LLMs (GPT-5 mini, Gemini 3 Flash, and Qwen3-80B-A3B) show that our method yields substantial improvements in Quadratic Weighted Kappa (QWK), with gains of up to +0.388 over human-authored rubrics on ASAP. Furthermore, starting from a minimal seed rubric that specifies only the score scale, our method can match or exceed the performance of carefully authored expert rubrics in most settings. However, the effectiveness of refinement varies across datasets, with limited gains on TOEFL11 for some models. Our findings demonstrate the potential of iterative rubric refinement for improving LLM-based AES while reducing the burden of manual rubric authoring.\footnote{Optimization and evaluation codes are available at \url{https://anonymous.4open.science/r/ARR2025-1CA3}}
\end{abstract}

\section{Introduction}
Automated Essay Scoring (AES) systems powered by Large Language Models (LLMs) are increasingly expected to provide real-time, scalable feedback for students and alleviate the grading burden on instructors~\citep{mizumoto2023aes,yancey-etal-2023-rating,naismith-etal-2023-automated,pack2024validity}. Typically, these systems employ static, pre-defined rubrics to guide the evaluation. However, it remains an open question whether rubrics designed for human raters are optimal for LLMs. When human raters use a rubric, they often engage in a collaborative calibration process: they score sample essays, discuss discrepancies in their judgments, and refine their shared understanding of the criteria to ensure consistency~\citep{trace2016rubricsnegotiate,ozfidan2022rubricdev,yoo-etal-2025-dress}. This iterative, reflective practice is overlooked in current LLM-based AES, potentially limiting their alignment with human scoring patterns.

Recent studies show that LLMs have the ability to refine their own outputs especially when there is reliable external feedback~\citep{madaan2023selfrefine,kamoi2024selfcorrect}. Prompt optimization techniques leverage these capabilities to update prompts to maximize a targeted metric and show performance improvements in various tasks such as multi-hop reasoning, instruction following and privacy-aware delegation~\citep{khattab2023dspy,opsahl-ong-etal-2024-optimizing,agrawal2025gepa}.

Inspired by these developments and the calibration process of human raters, we propose an iterative Reflect-and-Revise approach for refining rubrics in LLM-based AES.
Specifically, given a small set of 100 sample essays with human scores, the model iteratively refines the rubric by reflecting on its own scoring rationales and the discrepancies between its predicted scores and human labels, with the objective of maximizing Quadratic Weighted Kappa (QWK) between model and human scores.

We evaluate our method on three datasets (ASAP, ASAP 2.0, and TOEFL11) with three models (\texttt{GPT-5 mini}, \texttt{Qwen3-80B-A3B}, and \texttt{Gemini 3 Flash}). Our main findings are as follows:
\begin{itemize}
    \item Iterative rubric refinement with chain-of-thought rationales yields substantial QWK improvements, with gains of up to +0.388 on ASAP P1, consistently outperforming single-pass revision without rationales.
    \item Even starting from a minimal seed rubric that specifies only the score scale, our method achieves comparable or better performance than carefully authored expert rubrics in most settings, substantially reducing the need for manual rubric authoring.
    \item Qualitative analysis reveals that the refinement process introduces explicit decision rules, concrete scoring heuristics, and boundary-case guidance that are absent from the original human-authored rubrics.
    \item However, the effectiveness of refinement varies across datasets, with limited or negative gains on TOEFL11 for some models, indicating room for improvement in cross-dataset robustness.
\end{itemize}

\begin{figure*}[t]
\centering
\includegraphics[width=0.75\textwidth]{figures/confusion_matrices.pdf}
\caption{Confusion matrices for Qwen3-80B-A3B with the human expert rubric (top) and our refined rubric (bottom) across three datasets. Cell colors indicate row-normalized proportions; numbers show raw counts. Our refined rubric produces predictions more concentrated along the diagonal, indicating better agreement with human annotations.}
\label{fig:confusion}
\end{figure*}

\begin{table*}[t]
  \centering
  \resizebox{\textwidth}{!}{%
  \begin{tabular}{lcccccc}
    \toprule
    \textbf{Study} &
    \textbf{Scores} &
    \textbf{Texts} &
    \textbf{Sources} &
    \textbf{Predictions} &
    \textbf{Rationales} &
    \textbf{Iterative} \\
    \midrule
    HD-Eval (\citet{liu-etal-2024-hd}) & $\checkmark$ &  &  &  &  &  $\checkmark$\\
    MTS (\citet{lee-etal-2024-unleashing}) &  &  & $\checkmark$ &  &  &  \\
    \citet{xie2024gradelikehuman} & $\checkmark$ & $\checkmark$ &  &  &  &  \\
    ActiveCritic (\citet{xu2025activecritics}) & $\checkmark$ & $\checkmark$ & $\checkmark$ &  &  &  \\
    AutoCalibrate (\citet{liu-etal-2024-calibrating}) & $\checkmark$ & $\checkmark$ & $\checkmark$ & $\checkmark$ &  &  \\
    \midrule
    \textbf{Ours} &
    \textbf{$\checkmark$} & \textbf{$\checkmark$} & \textbf{$\checkmark$} & \textbf{$\checkmark$} & \textbf{$\checkmark$} & \textbf{$\checkmark$} \\
    \bottomrule
  \end{tabular}}
  \caption{Comparison of signals and strategies used to refine evaluation rubrics. \textbf{Scores}: human-labeled scores. \textbf{Texts}: texts being evaluated (essays/answers/responses that receive scores). \textbf{Sources}: source passages used to compose responses (e.g., source documents, provided essay themes). \textbf{Predictions}: model-predicted scores. \textbf{Rationales}: model-generated justifications accompanying predicted scores. \textbf{Iterative}: whether the method iteratively refines rubrics over multiple rounds. Our method uniquely leverages all five signals and performs iterative refinement.}
\label{tab:rubric_rewrite_data_requirements}
\end{table*}



\section{Related Work}
For non-verifiable tasks, where judging success is not as straightforward as in math or code, recent research has focused on LLM-based automatic evaluation using checklists and rubrics in prompts~\citep{min-etal-2023-factscore,qin-etal-2024-infobench,lin2024wildbench,wu-etal-2025-lifbench,cook2025ticking,huang2025rubricanchors,gunjal2025rubricsasrewards,viswanathan2025checklistsbetterrewardmodels,lee2025checkeval,xu2025activecritics,liu-etal-2024-calibrating,liu-etal-2024-hd,wen-etal-2025-hpss}. AES is an example of such a non-verifiable task, and various techniques have been proposed~\citep{mizumoto2023aes,xie2024gradelikehuman,lee-etal-2024-unleashing}.

\subsection{Rubric Design for LLM Evaluation}
Recent studies suggest that the relationship between rubric design and LLM evaluation quality is not straightforward. \citet{yoshida2025rubrics} found that making rubrics more detailed does not always lead to performance gains in AES: three out of four models maintained similar scoring accuracy with a simplified rubric, and one model even showed decreased performance with more detailed rubrics. Similarly, \citet{furuhashi2025checklist} identified ``negative items,'' rubric components that are valid for human evaluators but do not improve LLM performance, and showed that removing such items can even boost accuracy. These findings suggest that there remains room to find rubric formulations better suited for LLMs.

\subsection{LLM-based Rubric Refinement}
A growing body of work explores methods for generating or refining evaluation rubrics to improve agreement with human scores~\citep{liu-etal-2024-hd,lee-etal-2024-unleashing,xie2024gradelikehuman,xu2025activecritics,liu-etal-2024-calibrating}. Some methods generate rubrics in a single pass without subsequent revision: for example, generating rubrics from source passages~\citep{lee-etal-2024-unleashing}, from few input--score examples~\citep{xu2025activecritics}, or by rewriting existing rubrics using human-labeled scores and evaluated texts~\citep{xie2024gradelikehuman}. 

Among these, \citet{liu-etal-2024-calibrating} proposed the pipeline closest to ours. Their method samples candidate evaluation criteria from human-scored data, scores a held-out set, and refines the best-performing criteria by analyzing error cases where model scores diverge from human labels. However, their approach differs from ours in two key respects: (1) the refinement loop is executed only once rather than iteratively, and (2) the refinement step does not incorporate the model's chain-of-thought rationales, limiting the diagnostic signal available for rubric revision.

Our method extends this line of work by enabling the LLM to reflect on its own scoring output, including its chain-of-thought rationales, to iteratively refine the rubric. By feeding back not only human-labeled and model-predicted scores but also the model's justifications for those predictions, our approach provides richer diagnostic information for identifying why the current rubric leads to scoring errors. This process effectively mimics the calibration sessions of human evaluators, who refine their interpretations and build shared understanding before formal scoring~\citep{trace2016rubricsnegotiate,ozfidan2022rubricdev,ouyang2022instructgpt,yoo-etal-2025-dress}. Table~\ref{tab:rubric_rewrite_data_requirements} summarizes the signals used for rubric refinement across prior work and our method.


\section{Iterative Rubric Refinement}
\label{sec:method}
Our method iteratively refines rubric text from score mismatches between LLM predictions and human labels, and model's chain-of-thought rationales. We provide the full algorithm in Appendix~\ref{sec:appendix}.

\subsection{Preliminaries}
Let \(\mathcal{D}_{\mathrm{train}}=\{(x_i,y_i)\}_{i=1}^{N}\) be the training set, where \(x_i\) is an essay response and \(y_i\) is the corresponding human score. Rubric search starts from an initial seed rubric \(\rubric_0\) (from coarse to detailed variants) and iteratively updates it over \(T\) iterations, maintaining a candidate pool of at most \(K\) rubrics. At each iteration, \(M\) Monte Carlo trials are run per candidate, each drawing error batches of sizes \(b \in \mathcal{B}\).

Given a rubric \(\rubric\), the evaluator LLM scores each training essay \(x_i\), producing a predicted score \(\hat{y}_i\) and a chain-of-thought rationale \(z_i\):
\begin{equation}
(\hat{y}_i,\, z_i) = \mathrm{LLM}(\rubric,\, x_i).
\end{equation}
We measure rubric quality by Quadratic Weighted Kappa (QWK) between the predicted scores \(\hat{\mathbf{y}}_{\rubric}=(\hat{y}_1,\dots,\hat{y}_N)\) and the human scores \(\mathbf{y}=(y_1,\dots,y_N)\), and seek:
\begin{equation}
\rubric_{\mathrm{best}} = \arg\max_{\rubric}\;\mathrm{QWK}(\hat{\mathbf{y}}_{\rubric},\,\mathbf{y}).
\end{equation}
For brevity, we hereafter write \(\mathrm{QWK}(\rubric)\) to denote \(\mathrm{QWK}(\hat{\mathbf{y}}_{\rubric},\mathbf{y})\) on the training set. QWK is a standard metric in automated essay scoring for measuring agreement with human raters~\citep{ijcai2019p879}. It is an agreement metric for ordinal labels and penalizes larger score gaps more heavily than smaller ones:
\begin{equation}
\mathrm{QWK} = 1 - \frac{\sum_{i,j} w_{ij} O_{ij}}{\sum_{i,j} w_{ij} E_{ij}}, \quad
w_{ij}=\frac{(i-j)^2}{(L-1)^2},
\end{equation}
where \(O_{ij}\) and \(E_{ij}\) are observed and expected confusion counts, and \(L\) is the number of score levels. QWK ranges from \(-1\) to \(1\) and higher QWK is better: \(1\) indicates perfect agreement, \(0\) indicates chance-level agreement, and \(-1\) indicates complete disagreement.

\subsection{Reflect-and-Revise}
At iteration \(t\), we maintain a candidate pool \(\mathcal{C}_{t-1}\) of size at most \(K\). For each candidate rubric \(\rubric \in \mathcal{C}_{t-1}\), we first score all training samples and collect failed examples:
\begin{equation}
\mathcal{E}(\rubric)=\{(x_i,y_i,\hat{y}_i,z_i)\mid \hat{y}_i \neq y_i\}.
\end{equation}

For each Monte Carlo trial \(m\in\{1,\dots,M\}\) and each batch size \(b \in \mathcal{B}\), we draw a balanced subsample \(\tilde{\mathcal{E}} \subset \mathcal{E}(\rubric)\) of size \(b\), stratified across score levels so that each human-score value is represented as equally as possible. The model then rewrites the rubric by reflecting on the sampled errors and their rationales:
\begin{equation}
\rubric' = \procname{ReviseRubric}(\rubric, \tilde{\mathcal{E}}).
\end{equation}
Concretely, the revision prompt presents the current rubric together with each error case in \(\tilde{\mathcal{E}}\)---including the essay text, the model's predicted score and rationale, and the human score---and asks the model to identify systematic scoring-error patterns and propose targeted rubric modifications (see Appendix~\ref{sec:appendix_prompts} for the full prompt templates).

\subsection{Iterative Update}
Let \(\mathcal{N}_t\) be the union of all newly revised rubrics and the previous top candidates \(\mathcal{C}_{t-1}\). Every rubric in \(\mathcal{N}_t\) is re-evaluated on \(\mathcal{D}_{\mathrm{train}}\), and we retain the top \(K\) by QWK:
\begin{equation}
\mathcal{C}_t=\mathrm{TopK}_{\rubric\in\mathcal{N}_t}\ \mathrm{QWK}(\rubric).
\end{equation}
The global best rubric \(\rubric_{\mathrm{best}}\) is updated only when the best candidate of \(\mathcal{C}_t\) improves over the previous best training QWK. We repeat this process for \(T\) iterations and return \(\rubric_{\mathrm{best}}\).

\section{Experiments}
\label{sec:experiments}

\subsection{Datasets}
We evaluate on three essay scoring benchmarks. The Automated Student Assessment Prize (ASAP) dataset~\citep{ben2012asap} consists of student essays from U.S.\ standardized tests; we use essay set~1 (P1), which contains persuasive essays scored on an integer scale from 1 to 6 by human raters. ASAP 2.0~\citep{Crossley2025ASAP2} is a corpus of source-based argumentative essays written by U.S.\ secondary students across seven prompts, scored on a 1--6 integer scale with consistent rubrics and accompanying source texts; we use the ``Exploring Venus'' subset in our experiments. The TOEFL11 corpus~\citep{blanchard2013toefl} contains English essays written by non-native speakers across eight essay prompts, labeled at three proficiency levels (high, medium, low). For each dataset, we use 100 training samples for rubric refinement and evaluate on 100 held-out test samples. The expert and simplest seed rubrics of ASAP are provided in Appendix~\ref{sec:appendix_rubrics}.

For TOEFL11, the original rubric defines five proficiency levels (scores 1--5), but the dataset labels use three levels (high, medium, low). We adopt the rubric descriptions for score 4 as \textit{high} (mapped to score 3), score 3 as \textit{medium} (mapped to score 2), and score 2 as \textit{low} (mapped to score 1). 

\subsection{Experimental Setup}
We compare our Reflect-and-Revise method against two baselines. The first is the \textbf{Human Rubric} baseline, which directly uses the original human-authored rubric without any refinement. The second is \textbf{AutoCalibrate}, which follows AutoCalibrate~\citep{liu-etal-2024-calibrating} in performing a single-pass rubric revision using score mismatches between model predictions and human labels, without incorporating the model's chain-of-thought rationales. As shown in \autoref{tab:rubric_rewrite_data_requirements}, AutoCalibrate uses the most signals among prior methods (human-labeled scores, evaluated texts, source passages, and model-predicted scores), making it the closest baseline to our approach. We chose AutoCalibrate as our primary baseline also because many rubric refinement methods discussed in \autoref{tab:rubric_rewrite_data_requirements} incorporate post-processing steps beyond rubric modification (e.g., score calibration or aggregation), making it difficult to isolate the effect of rubric refinement itself. To further ensure a fair comparison, we initialize AutoCalibrate from the same human-authored rubric as our method, rather than generating an initial rubric from scratch with an LLM as in the original work.

We evaluate using three frontier LLMs accessed via the OpenRouter API~\citep{openrouter}: \texttt{GPT-5-mini}~\citep{gpt5}, \texttt{Gemini~3~Flash}~\citep{Google2025GeminiFlash}, and \texttt{Qwen3-Next-80B-A3B-Instruct}~\citep{qwen3-next}. 
We report QWK between model predictions and human scores on the held-out test set in a zero-shot setting. We experiment with two seed rubric variants: an \textit{expert} rubric, which is the full human-authored rubric provided with the dataset, and a \textit{simplest} rubric, which is a minimal instruction specifying only the score scale (e.g., ``Based on the response's content, rate the response on a scale of 1 to 6.''). Detailed hyperparameters, including model-specific generation parameters, are provided in Appendix~\ref{sec:appendix_hparams}.

\section{Experimental Results}
\label{sec:results}

\subsection{Main Results}
\label{sec:main_results}
Results on ASAP, ASAP 2.0 and TOEFL11 are provided in Tables~\ref{tab:asap1},~\ref{tab:asap2} and~\ref{tab:toefl} respectively. Bold and underlined values indicate the best and second-best scores for each model, respectively.

\begin{table*}[t]
\centering
\begin{tabular}{llrrrrr}
\toprule
LLM & Rubric written by & QWK & Accuracy & Spearman & Macro-F1 & MAE \\
\midrule
Gemini 3 Flash & Human & 0.427 & 0.404 & 0.568 & 0.307 & 0.697 \\
Gemini 3 Flash & AutoCalibrate & 0.489 & 0.590 & 0.493 & \underline{0.382} & 0.460 \\
Gemini 3 Flash & \textbf{Ours} & \textbf{0.646} & \textbf{0.680} & \textbf{0.656} & \textbf{0.595} & \textbf{0.330} \\
Gemini 3 Flash & \textbf{Ours from simplest} & \underline{0.580} & \underline{0.650} & \underline{0.619} & 0.338 & \underline{0.360} \\
\midrule
GPT-5 mini & Human & 0.042 & 0.110 & 0.151 & 0.058 & 1.430 \\
GPT-5 mini & AutoCalibrate & \underline{0.390} & \underline{0.410} & \underline{0.396} & \underline{0.231} & \underline{0.620} \\
GPT-5 mini & \textbf{Ours} & \textbf{0.480} & \textbf{0.450} & \textbf{0.510} & \textbf{0.271} & \textbf{0.600} \\
GPT-5 mini & \textbf{Ours from simplest} & 0.314 & 0.360 & 0.282 & 0.219 & 0.700 \\
\midrule
Qwen3-80B-A3B & Human & 0.121 & 0.130 & 0.323 & 0.123 & 1.540 \\
Qwen3-80B-A3B & AutoCalibrate & \underline{0.240} & 0.310 & \underline{0.458} & \underline{0.233} & 1.100 \\
Qwen3-80B-A3B & \textbf{Ours} & \textbf{0.473} & \textbf{0.520} & \textbf{0.490} & \textbf{0.402} & \textbf{0.570} \\
Qwen3-80B-A3B & \textbf{Ours from simplest} & 0.154 & \underline{0.320} & 0.188 & 0.158 & \underline{0.920} \\
\bottomrule
\end{tabular}
\caption{Zero-shot evaluation results on ASAP. Bold and underlined values indicate the best and second-best scores for each model.}
\label{tab:asap1}
\end{table*}

\begin{table*}[t]
\centering
\begin{tabular}{llrrrrr}
\toprule
LLM & Rubric written by & QWK & Accuracy & Spearman & Macro-F1 & MAE \\
\midrule
Gemini 3 Flash & Human & 0.613 & 0.440 & 0.657 & 0.270 & 0.610 \\
Gemini 3 Flash & AutoCalibrate & 0.646 & \underline{0.490} & \textbf{0.716} & 0.297 & \underline{0.550} \\
Gemini 3 Flash & \textbf{Ours} & \textbf{0.700} & \textbf{0.540} & \underline{0.716} & \textbf{0.393} & \textbf{0.520} \\
Gemini 3 Flash & \textbf{Ours from simplest} & \underline{0.651} & 0.470 & 0.682 & \underline{0.310} & 0.580 \\
\midrule
GPT-5 mini & Human & 0.358 & \underline{0.350} & \underline{0.533} & 0.148 & 0.780 \\
GPT-5 mini & AutoCalibrate & 0.338 & 0.340 & 0.514 & 0.142 & 0.790 \\
GPT-5 mini & \textbf{Ours} & \textbf{0.537} & \textbf{0.440} & \textbf{0.618} & \textbf{0.239} & \textbf{0.620} \\
GPT-5 mini & \textbf{Ours from simplest} & \underline{0.452} & \underline{0.350} & 0.454 & \underline{0.176} & \underline{0.750} \\
\midrule
Qwen3-80B-A3B & Human & 0.575 & 0.270 & \textbf{0.668} & 0.272 & 0.890 \\
Qwen3-80B-A3B & AutoCalibrate & \underline{0.595} & \underline{0.290} & 0.643 & \textbf{0.310} & 0.850 \\
Qwen3-80B-A3B & \textbf{Ours} & \textbf{0.636} & \textbf{0.410} & \underline{0.645} & \underline{0.295} & \textbf{0.660} \\
Qwen3-80B-A3B & \textbf{Ours from simplest} & 0.509 & \underline{0.290} & 0.558 & 0.214 & \underline{0.820} \\
\bottomrule
\end{tabular}
\caption{Zero-shot evaluation results on ASAP 2.0. Bold and underlined values indicate the best and second-best scores for each model.}
\label{tab:asap2}
\end{table*}

\begin{table*}[t]
\centering
\begin{tabular}{llrrrrr}
\toprule
LLM & Rubric written by & QWK & Accuracy & Spearman & Macro-F1 & MAE \\
\midrule
Gemini 3 Flash & Human & \underline{0.574} & 0.690 & \underline{0.621} & \underline{0.636} & 0.320 \\
Gemini 3 Flash & AutoCalibrate & 0.510 & 0.650 & 0.528 & 0.591 & 0.360 \\
Gemini 3 Flash & \textbf{Ours} & 0.557 & \underline{0.720} & 0.568 & 0.630 & \underline{0.280} \\
Gemini 3 Flash & \textbf{Ours from simplest} & \textbf{0.663} & \textbf{0.770} & \textbf{0.672} & \textbf{0.712} & \textbf{0.230} \\
\midrule
GPT-5 mini & Human & \underline{0.447} & \underline{0.680} & \underline{0.475} & 0.528 & \underline{0.320} \\
GPT-5 mini & AutoCalibrate & \textbf{0.539} & \textbf{0.730} & \textbf{0.580} & \textbf{0.578} & \textbf{0.270} \\
GPT-5 mini & \textbf{Ours} & 0.348 & 0.650 & 0.367 & 0.493 & 0.360 \\
GPT-5 mini & \textbf{Ours from simplest} & 0.394 & 0.640 & 0.391 & \underline{0.545} & 0.370 \\
\midrule
Qwen3-80B-A3B & Human & 0.364 & 0.470 & \underline{0.476} & 0.423 & 0.560 \\
Qwen3-80B-A3B & AutoCalibrate & \underline{0.456} & 0.580 & 0.464 & \textbf{0.545} & 0.440 \\
Qwen3-80B-A3B & \textbf{Ours} & 0.452 & \underline{0.590} & 0.468 & 0.516 & \underline{0.420} \\
Qwen3-80B-A3B & \textbf{Ours from simplest} & \textbf{0.493} & \textbf{0.600} & \textbf{0.502} & \underline{0.544} & \textbf{0.400} \\
\bottomrule
\end{tabular}
\caption{Zero-shot evaluation results on TOEFL11. Bold and underlined values indicate the best and second-best scores for each model.}
\label{tab:toefl}
\end{table*}

\subsection{Refinement from Minimal Seed Rubrics}
\label{sec:simplest_seed}
A key practical question is whether rubric refinement can reduce the burden of manual rubric authoring for LLMs. To investigate this, we apply our method starting from a \textit{simplest} seed rubric and compare the resulting test QWK against the unrefined human expert rubric. Table~\ref{tab:simplest} reports these results.

\begin{table}[t]
\centering
\small
\resizebox{\columnwidth}{!}{%
\begin{tabular}{llrr}
\toprule
\textbf{Dataset} & \textbf{LLM} & \textbf{Ours from} & \textbf{$\Delta$ vs.\ expert} \\
 & & \textbf{simplest} & \textbf{rubric} \\
\midrule
ASAP & GPT-5 mini & 0.314 & +0.272 \\
 & Gemini 3 Flash & 0.580 & +0.153 \\
 & Qwen3-80B-A3B & 0.154 & +0.034 \\
\midrule
ASAP 2.0 & GPT-5 mini & 0.452 & +0.094 \\
 & Gemini 3 Flash & 0.651 & +0.038 \\
 & Qwen3-80B-A3B & 0.509 & $-$0.066 \\
\midrule
TOEFL11 & GPT-5 mini & 0.394 & $-$0.053 \\
 & Gemini 3 Flash & 0.663 & +0.088 \\
 & Qwen3-80B-A3B & 0.493 & +0.129 \\
\bottomrule
\end{tabular}%
}
\caption{QWK comparison: rubrics refined from the simplest seed (``Based on the response's content, rate the response on a scale of 1 to 6.'') vs.\ the rubric written by human experts. $\Delta$ denotes QWK improvement over the baseline using human expert rubrics. Our method achieves comparable or better performance than the human expert rubric in most settings, demonstrating that iterative refinement can substantially reduce the need for manual rubric authoring.}
\label{tab:simplest}
\end{table}

\section{Analysis}
\label{sec:analysis}

\paragraph{Quantitative and Qualitative Analysis of Refined Rubrics}
\label{sec:qualitative}
% To understand how refinement changes rubric content

\paragraph{Iterative Refinement and Monte Carlo Effects}
To analyze the effect of iterative refinement and Monte Carlo trials, which controls depth and breadth of exploration, respectively, we track the best QWK in training dataset at each iteration. Figure~\ref{fig:training_qwk} shows these trajectories of Gemini 3 Flash on ASAP 2.0. By using large number of Monte Carlo trials, our method explores a wide variety of rubric modifications at each iteration, which leads to steady improvements in training QWK across iterations for all three models. 

About gains from iteration, we calculate the stepwise improvement in QWK at each iteration, defined as the difference in best training QWK between steps. Figure~\ref{fig:qwk_gains} plots these stepwise improvements across iterations for all three datasets averaging across models. The largest gains tend to occur in early iterations, with diminishing returns in later iterations.

\begin{figure}[t]
\centering
\includegraphics[width=0.9\linewidth]{figures/fig_iterative_refinement_mc_ASAP2.pdf}
\caption{Best training QWK across iterations of Gemini 3 Flash on ASAP 2.0. MC stands for Monte Carlo trials. By using a large number of Monte Carlo trials, our method explores a wide variety of rubric modifications at each iteration, which leads to improvements in training QWK across iterations.}
\label{fig:training_qwk}
\end{figure}

\begin{figure}[t]
\centering
\includegraphics[width=0.9\linewidth]{figures/fig_stepwise_improvement.pdf}
\caption{Best training QWK improvement between refinement steps, averaged across models for each dataset. $s_{t-n}\!\to\!s_{t}$ denotes the relative improvement from step $t-n$ to step $t$. The largest gains tend to occur in early iterations.}
\label{fig:qwk_gains}
\end{figure}

\paragraph{Ablation on Iterative and Rationale Components}
To quantify the contribution of iterative refinement and chain-of-thought rationales, we perform an ablation study using ASAP 2.0 dataset. We compare our full method against two ablated variants: (1) \textbf{w/o Iteration}, which performs only a single revision step (i.e., \(T=1\)) without further iterations, and (2) \textbf{w/o Rationale}, which performs iterative refinement but omits the model's chain-of-thought rationales from the revision prompt, relying solely on score mismatches as in AutoCalibrate. Table~\ref{tab:ablation} reports the QWK results for these variants. The full method outperforms both ablated variants across all three models, confirming that both iterative refinement and chain-of-thought rationales contribute to performance improvements.

\begin{table}[t]
\centering
\small
\begin{tabular}{llr}
\toprule
LLM & Method & QWK \\
\midrule
Gemini 3 Flash & Ours & \textbf{0.700} \\
Gemini 3 Flash & w/o Iteration ($T=1$) & 0.616 \\
Gemini 3 Flash & w/o Rationale & \underline{0.665} \\
\midrule
GPT-5 mini & Ours & \textbf{0.537} \\
GPT-5 mini & w/o Iteration ($T=1$) & 0.325 \\
GPT-5 mini & w/o Rationale & \underline{0.527} \\
\midrule
Qwen3-80B-A3B & Ours & \textbf{0.636} \\
Qwen3-80B-A3B & w/o Iteration ($T=1$) & 0.522 \\
Qwen3-80B-A3B & w/o Rationale & \underline{0.561} \\
\bottomrule
\end{tabular}
\caption{The full method (Ours) outperforms both ablated variants, confirming the importance of both iterative refinement and chain-of-thought rationales for maximizing rubric effectiveness.}
\label{tab:ablation}
\end{table}

\paragraph{Scoring distribution analysis}
To visualize how rubric refinement changes scoring behavior, Figure~\ref{fig:confusion} shows confusion matrices for Qwen3-80B-A3B with the human expert rubric versus our refined rubric across all three datasets. On ASAP~P1, the human rubric causes severe under-scoring: predictions cluster around scores 2--3 regardless of the true score. Our refined rubric shifts the distribution toward the diagonal, concentrating predictions around score~4---the most frequent true score---and dramatically improving accuracy from 13\% to 40\%. On ASAP~2.0, a similar pattern emerges: the human rubric scatters predictions across scores 1--3, while our rubric aligns predictions with the true score distribution, improving accuracy from 27\% to 51\%. On TOEFL11, the human rubric never predicts the ``High'' category, collapsing all predictions into Low and Medium. Our refined rubric recovers the ability to predict all three levels, improving accuracy from 48\% to 61\%, although some Low-scored essays are misclassified as Medium. These patterns suggest that the primary mechanism of improvement is better calibration of score boundaries to match the target distribution.

% \paragraph{Cross-model rubric transferability}
% An important question for practical deployment is whether rubrics optimized for one model can benefit other models. To investigate this, we evaluate rubrics refined by each model on the other two models. Table~\ref{tab:cross_model} shows cross-model QWK for expert-seed rubrics refined with our method.

% \begin{table*}[t]
% \centering
% \small
% \begin{tabular}{lrrr}
% \toprule
% \textbf{Source $\backslash$ Evaluator} & \textbf{GPT-5 mini} & \textbf{Gemini 3 Flash} & \textbf{Qwen3-80B-A3B} \\
% \multicolumn{4}{l}{\textit{ASAP P1}} \\
% Human & 0.042 & 0.427 & 0.121 \\
% GPT-5 mini & \textbf{\cellcolor{gray!20}0.429} & 0.424 & \textbf{0.369} \\
% Gemini 3 Flash & 0.235 & \textbf{\cellcolor{gray!20}0.643} & 0.355 \\
% Qwen3-80B-A3B & 0.060 & 0.297 & \cellcolor{gray!20}0.339 \\
% \midrule
% \multicolumn{4}{l}{\textit{ASAP 2.0}} \\
% Human & 0.358 & 0.613 & \textbf{0.575} \\
% GPT-5 mini & \cellcolor{gray!20}0.398 & \textbf{0.651} & 0.329 \\
% Gemini 3 Flash & \textbf{0.399} & \cellcolor{gray!20}0.639 & 0.312 \\
% Qwen3-80B-A3B & 0.321 & 0.644 & \cellcolor{gray!20}0.558 \\
% \midrule
% \multicolumn{4}{l}{\textit{TOEFL11}} \\
% Human & 0.447 & \textbf{0.574} & 0.364 \\
% GPT-5 mini & \textbf{\cellcolor{gray!20}0.485} & 0.557 & 0.341 \\
% Gemini 3 Flash & 0.360 & \cellcolor{gray!20}0.477 & 0.483 \\
% Qwen3-80B-A3B & 0.107 & 0.376 & \textbf{\cellcolor{gray!20}0.499} \\
% \bottomrule
% \end{tabular}
% \caption{Cross-model rubric transferability (QWK). Each cell shows QWK when the rubric optimized by the \textit{source} model (row) is used by the \textit{evaluator} model (column). Shaded diagonal cells denote same-model results. ``Human'' row shows the unrefined human rubric baseline. Bold values indicate the best QWK per evaluator model within each dataset.}
% \label{tab:cross_model}
% \end{table*}

% The results reveal that transferability is highly asymmetric and dataset-dependent. On ASAP P1, the rubric refined by Gemini 3 Flash yields the best QWK not only for Gemini itself (0.576) but also for Qwen3-80B-A3B (0.524), substantially outperforming Qwen's own refined rubric (0.213). Similarly, on ASAP 2.0, the rubric refined by GPT-5 mini achieves the highest QWK when used by Gemini 3 Flash (0.688), surpassing Gemini's own rubric (0.639). These cases suggest that some models produce rubrics that capture more generalizable scoring criteria. However, the reverse transfers are often ineffective: on ASAP P1, the Qwen-refined rubric yields near-zero QWK for GPT-5 mini (0.001). On TOEFL11, the unrefined human rubric remains the strongest option for GPT-5 mini and Gemini 3 Flash, indicating that refined rubrics can overfit to model-specific scoring patterns on certain datasets. Overall, while cross-model transfer can sometimes be beneficial, rubrics are generally most effective when optimized for the target model.

\section{Conclusion}

We proposed an iterative Reflect-and-Revise approach for refining scoring rubrics in LLM-based automated essay scoring. By prompting models to reflect on their own chain-of-thought rationales and observed score discrepancies with human labels, our method enables targeted rubric revisions that improve alignment with human scoring. Experiments on three datasets and three models demonstrate that our approach yields substantial QWK improvements on ASAP P1 (up to +0.388) and consistent gains on ASAP 2.0, while also showing that rubrics refined from minimal seed specifications can match or exceed human expert rubrics. However, the effectiveness varies across datasets, with limited or negative gains on TOEFL11 for some models. These findings suggest that iterative rubric refinement is a promising direction for reducing manual rubric authoring effort and improving LLM-based AES, while also highlighting the need for future work on preventing overfitting to training distributions and improving cross-dataset robustness.

\section*{Limitations}

Our study has several limitations. First, our evaluation covers three essay scoring benchmarks with a single essay set per dataset. The generalizability of our findings to other scoring tasks, writing genres, or educational contexts remains to be verified through broader evaluation.

Second, the effectiveness of our method varies considerably across datasets and models. On TOEFL11, the refined rubrics underperform the original human rubric for some models, suggesting that the refinement process can introduce biases that do not generalize beyond the training samples. This inconsistency indicates that our method may be less effective when the initial rubric is already well-calibrated for the target model or when the training and test distributions differ substantially.

Third, the iterative refinement process incurs non-trivial computational cost, as each iteration requires scoring all training samples with the current rubric candidates. With five iterations, three candidates, and multiple batch sizes, the total number of LLM calls scales considerably, which may limit practical applicability in resource-constrained settings.

Fourth, our experiments use only 100 training samples for rubric refinement. While this reflects realistic scenarios where labeled data is scarce, the small sample size may increase the risk of overfitting the rubric to idiosyncratic properties of the training set rather than capturing generalizable scoring patterns.

Finally, our cross-model transfer analysis reveals that rubrics optimized for one model do not reliably benefit other models, and in some cases substantially degrade performance. This limits the reusability of refined rubrics and suggests that separate optimization may be required for each target model.

\bibliography{custom}
\appendix

\section{Algorithm}
\label{sec:appendix}

Algorithm~\ref{alg:rubric_refinement} gives the full pseudocode for iterative rubric refinement.

\begin{algorithm*}[t]
\caption{Iterative Rubric Refinement}
\label{alg:rubric_refinement}
\begin{algorithmic}[1]
\Require Training set $\mathcal{D}_{\mathrm{train}} = \{(x_i, y_i)\}_{i=1}^{N}$, seed rubric $\rubric_0$, iterations $T$, pool size $K$, Monte Carlo trials $M$, batch sizes $\mathcal{B}$, evaluator LLM
\Ensure Best rubric $\rubric_{\mathrm{best}}$
\State $\mathcal{C}_0 \gets \{\rubric_0\}$;\; $\rubric_{\mathrm{best}} \gets \rubric_0$;\; $q_{\mathrm{best}} \gets \mathrm{QWK}(\rubric_0)$
\For{$t = 1$ \textbf{to} $T$}
    \State $\mathcal{N}_t \gets \emptyset$
    \For{\textbf{each} $\rubric \in \mathcal{C}_{t-1}$}
        \State Score all $(x_i,y_i) \in \mathcal{D}_{\mathrm{train}}$ with $\rubric$ $\to$ $(\hat{y}_i, z_i)$
        \State $\mathcal{E}(\rubric) \gets \{(x_i,y_i,\hat{y}_i,z_i) \mid \hat{y}_i \neq y_i\}$
        \For{$m = 1$ \textbf{to} $M$}
            \For{\textbf{each} $b \in \mathcal{B}$}
                \State $\tilde{\mathcal{E}} \gets \procname{BalancedSample}(\mathcal{E}(\rubric),\, b)$
                % \Comment{Stratified by score level}
                \State $\rubric' \gets \procname{ReviseRubric}(\rubric,\, \tilde{\mathcal{E}})$
                % \Comment{LLM revises rubric}
                \State $\mathcal{N}_t \gets \mathcal{N}_t \cup \{\rubric'\}$
            \EndFor
        \EndFor
    \EndFor
    \State $\mathcal{N}_t \gets \mathcal{N}_t \cup \mathcal{C}_{t-1}$
    % \Comment{Retain previous top candidates}
    \State Re-evaluate all $\rubric \in \mathcal{N}_t$ on $\mathcal{D}_{\mathrm{train}}$
    \State $\mathcal{C}_t \gets \mathrm{TopK}_{\rubric \in \mathcal{N}_t}\;\mathrm{QWK}(\rubric)$
    \If{$\max_{\rubric \in \mathcal{C}_t} \mathrm{QWK}(\rubric) > q_{\mathrm{best}}$}
        \State $\rubric_{\mathrm{best}} \gets \arg\max_{\rubric \in \mathcal{C}_t} \mathrm{QWK}(\rubric)$;\; $q_{\mathrm{best}} \gets \mathrm{QWK}(\rubric_{\mathrm{best}})$
    \EndIf
\EndFor
\State \Return $\rubric_{\mathrm{best}}$
\end{algorithmic}
\end{algorithm*}

\procname{BalancedSample}$(\mathcal{E}, b)$ partitions the error set $\mathcal{E}$ into buckets by human-score level, then draws $\lfloor b / L' \rfloor$ examples from each bucket (where $L'$ is the number of non-empty buckets), ensuring that each score level is represented as equally as possible.

\section{Prompt Templates}
\label{sec:appendix_prompts}

This section lists the exact prompt templates used in our pipeline. Placeholders enclosed in braces (e.g., \texttt{\{rubric\}}) are filled at runtime.

\subsection{Evaluation Prompt}
\label{sec:eval_prompt}
The following prompt is used to score each essay with the current rubric. The LLM is instructed to output a rationale followed by an integer score.

\begin{tcolorbox}[colback=white, colframe=RoyalBlue, title=Evaluation Prompt, fonttitle=\bfseries\small, fontupper=\small, breakable]
You are an expert rater for a high-stakes English writing exam for second-language learners.
Evaluate the response strictly using the scoring guideline. Choose exactly one score from the scoring guideline's score points.

\texttt{\# Essay Prompt}\\
\texttt{"""\{essay\_prompt\}"""}

\texttt{\# Response}\\
\texttt{"""\{response\}"""}

\texttt{\# Scoring Guideline}\\
\texttt{"""\{rubric\}"""}

\texttt{\# Output format (follow exactly)}\\
\texttt{Rationale: [<<<Brief evidence-based rationale.>>>]}\\
\texttt{Rating: [<<<One integer score only.>>>]}
\end{tcolorbox}

\subsection{Error-Case Format for Rubric Revision}
\label{sec:error_format}
For each incorrectly scored example fed into the revision prompt, the following format is used to present the error case to the model. When rationales are included (\texttt{with\_rationale=True}), the model's chain-of-thought is shown alongside its predicted score.

\begin{tcolorbox}[colback=white, colframe=RoyalBlue, title=Error-Case Format (with rationale), fonttitle=\bfseries\small, fontupper=\small, breakable]
\texttt{Assistant input:}\\
\texttt{Essay prompt:}\\
\texttt{"""\{essay\_prompt\}"""}\\
\texttt{Essay to be rated:}\\
\texttt{"""\{response\}"""}\\
\texttt{Assistant rationale:}\\
\texttt{"""\{rationale\}"""}\\
\texttt{Assistant score:}\\
\texttt{"""\{rating\}"""}\\
\texttt{Desired score:}\\
\texttt{"""\{desired\_rating\}"""}
\end{tcolorbox}

\subsection{Rubric Revision Prompt}
\label{sec:revision_prompt}
The revision prompt wraps the current rubric and the sampled error cases, and instructs the model to output a revised rubric.

\begin{tcolorbox}[colback=white, colframe=RoyalBlue, title=Rubric Revision Prompt (with rationale), fonttitle=\bfseries\small, fontupper=\small, breakable]
I asked an assistant to grade essays using the scoring guideline below:

\texttt{```}\\
\texttt{\{current\_rubric\}}\\
\texttt{```}

Here are grading examples that include the assistant input, the assistant rationale, the assistant score, and the desired score:

\texttt{```}\\
\texttt{\{examples\}}\\
\texttt{```}

Revise the scoring guideline to improve score agreement so the assistant's future ratings align more closely with the desired scores.

Requirements:\\
1. Use the rationale patterns to identify why the assistant over-scored or under-scored, and improve the scoring guideline guidance accordingly to reduce score mismatches.

Output rules:\\
-- Return only the revised scoring guideline.\\
-- Use exactly one fenced code block with triple backticks.\\
-- Do not include any text before or after the code block.
\end{tcolorbox}

When rationales are not used (\texttt{with\_rationale=False}), the error-case format omits the \texttt{Assistant rationale} field, and the requirement in the revision prompt is replaced with: ``Use score mismatch patterns to identify where the scoring guideline guidance is insufficient or ambiguous, and revise it to reduce score mismatches.''

\section{Hyperparameters}
\label{sec:appendix_hparams}

Table~\ref{tab:hyperparams} summarizes the optimization hyperparameters. We adopt the number of Monte Carlo trials ($M{=}4$) from AutoCalibrate. For error batch sizes, AutoCalibrate originally used $\mathcal{B}{=}\{1,2,4\}$, but considering the improved capacity of recent LLMs to process longer contexts, we increase them to $\mathcal{B}{=}\{4,8,12\}$. For the number of iterations, the original self-refinement loop in \citet{madaan2023selfrefine} used $T{=}4$; we set $T{=}5$, expecting that the stronger reasoning and instruction-following capabilities of recent models would allow productive refinement over additional cycles. The AutoCalibrate baseline uses $T{=}1$ (single-pass revision) without rationales, while our Reflect-and-Revise method performs iterative refinement with rationale-based feedback. Both methods maintain $K{=}3$ top candidates and use $N{=}100$ training samples.

\begin{table}[h]
  \centering
  \small
  \begin{tabular}{lcc}
    \toprule
    \textbf{Hyperparameter} & \textbf{Ours} & \textbf{AutoCalibrate} \\
    \midrule
    Iterations ($T$) & 5 & 1 \\
    Top-$K$ candidates & 3 & 3 \\
    Monte Carlo trials ($M$) & 4 & 4 \\
    Batch sizes ($\mathcal{B}$) & \{4, 8, 12\} & \{4, 8, 12\} \\
    Training samples ($N$) & 100 & 100 \\
    With rationale & Yes & No \\
    \bottomrule
  \end{tabular}
  \caption{Optimization hyperparameters for rubric refinement.}
  \label{tab:hyperparams}
\end{table}

For all models, we set the maximum output tokens to 8192. For GPT-5-mini and Gemini 3 Flash, we set reasoning effort to low. All other generation parameters were left at their default values for each model.


\section{Seed Rubrics}
\label{sec:appendix_rubrics}

This section lists the expert (human-authored) and simplest seed rubrics of the ASAP dataset used in our experiments. The expert rubric is the original human-authored rubric provided with the dataset, while the simplest rubric is a minimal description specifying only the score scale.

\subsection{Expert rubric}

\begin{tcolorbox}[colback=white, colframe=gray, fontupper=\small, breakable]
Score Point 1: An undeveloped response that may take a position but offers no more than very minimal support. Typical elements:
- Contains few or vague details.
- Is awkward and fragmented.
- May be difficult to read and understand.
- May show no awareness of audience.

Score Point 2: An under-developed response that may or may not take a position. Typical elements:
- Contains only general reasons with unelaborated and/or list-like details.
- Shows little or no evidence of organization.
- May be awkward and confused or simplistic.
- May show little awareness of audience.

Score Point 3: A minimally-developed response that may take a position, but with inadequate support and details. Typical elements:
- Has reasons with minimal elaboration and more general than specific details.
- Shows some organization.
- May be awkward in parts with few transitions.
- Shows some awareness of audience.

Score Point 4: A somewhat-developed response that takes a position and provides adequate support. Typical elements:
- Has adequately elaborated reasons with a mix of general and specific details.
- Shows satisfactory organization.
- May be somewhat fluent with some transitional language.
- Shows adequate awareness of audience.

Score Point 5: A developed response that takes a clear position and provides reasonably persuasive support. Typical elements:
- Has moderately well elaborated reasons with mostly specific details.
- Exhibits generally strong organization.
- May be moderately fluent with transitional language throughout.
- May show a consistent awareness of audience.

Score Point 6: A well-developed response that takes a clear and thoughtful position and provides persuasive support. Typical elements:
- Has fully elaborated reasons with specific details.
- Exhibits strong organization.
- Is fluent and uses sophisticated transitional language.
- May show a heightened awareness of audience.

Note: 
I have made an effort to remove personally identifying information from the essays using the Named Entity Recognizer (NER). The relevant entities are identified in the text and then replaced with a string such as "PERSON", "ORGANIZATION", "LOCATION", "DATE", "TIME", "MONEY", "PERCENT", "CAPS" (any capitalized word) and "NUM" (any digits). Please do not penalize the essay because of the anonymizations.
\end{tcolorbox}

\subsection{Simplest rubric}

\begin{tcolorbox}[colback=white, colframe=gray, fontupper=\small]
Based on the response's content, rate the response on a scale of 1 to 6.
\end{tcolorbox}

\subsection{Refined rubric (GPT-5~mini, ASAP P1, expert seed)}
\label{sec:appendix_refined_rubric}

The following is the full text of the rubric refined by GPT-5~mini on ASAP P1 starting from the expert seed rubric, corresponding to the excerpts shown in Figure~\ref{fig:rubric_pattern_asap_1_openai_gpt_5_mini_base_expert_True_train100_iteration5_top3_bs4_8_12_mc4}.

\begin{tcolorbox}[colback=white, colframe=gray, fontupper=\small, breakable]
Revised Scoring Guideline (with added clarifications to improve rater alignment and reduce common mismatches)

\textbf{General notes} (apply to all scores)
\begin{itemize}
\setlength{\itemsep}{1pt}
\item Do not penalize anonymization (PERSON, ORGANIZATION, LOCATION, DATE, TIME, MONEY, PERCENT, CAPS, NUM).
\item Distinguish surface errors (spelling, grammar, typos) from communicative breakdowns. Surface errors are expected and should only lower a score if they substantially impede meaning, coherence, or readability.
\item Consider four core dimensions when placing a response: Position/claim clarity, Development/Specificity of support, Organization/Coherence/Transitions, and Readability/Fluency. A single weak dimension can lower score by at most one point unless it causes major incomprehensibility.
\item Use ``examples'' to mean concrete, specific instances, anecdotes, or explained scenarios---not mere lists of general categories.
\item When in doubt between adjacent scores, use the development and specificity of examples as the primary tiebreaker: more distinct, explained examples $\to$ higher score.
\end{itemize}

\textbf{New guidance to address recurring rater mismatches}
\begin{itemize}
\setlength{\itemsep}{1pt}
\item Treat brief anecdotes or named personal examples (e.g., ``my neighbor,'' ``a study found,'' short story about a person) as concrete examples even when they are imperfectly written or poorly sourced. Such examples should count toward the essay's level of development, so long as they are distinct and clearly connected to the claim.
\item Multiple concrete examples that are each briefly explained (even if explanations are awkward or contain errors) should generally move a response up one score level relative to essays that only list general reasons. Specifically: multiple distinct examples + brief explanations $\to$ prefer 5 over 4; multiple named anecdotes/examples across domains $\to$ prefer 5 over 4.
\item Repetition of the same point with minor variation does not equal multiple elaborations. Distinctness matters: different domains, different types of evidence, or separate illustrative anecdotes count as multiple examples; restatements do not.
\item Poor mechanics/surface errors alone should not prevent an essay from earning 4 or 5 if the claim, organization, and development meet those criteria. Reserve downgrades for errors only when they create frequent confusion or break comprehension.
\item Organization: Even basic intro/body/conclusion structure with clear grouping of reasons should satisfy the ``organization'' requirement for scores 3--5. Only very fragmented or missing structure should lead to a 1 or 2.
\end{itemize}

\textbf{Score Point 1:} Undeveloped / Minimal communicative content\\
Typical elements (choose 1--3 to justify a 1):
\begin{itemize}
\setlength{\itemsep}{1pt}
\item Takes little or no defensible position, or the position is incomprehensible.
\item Contains extremely few details or only vague, fragmentary statements.
\item Organization is absent; writing may be so fragmented it is difficult to follow.
\item Errors seriously impede understanding.
\end{itemize}
Rater guidance: Assign 1 when the essay provides almost no development and the reader cannot follow an argument beyond a bare claim.

\textbf{Score Point 2:} Under-developed / General or list-like reasons only\\
Typical elements (choose 2--4 to justify a 2):
\begin{itemize}
\setlength{\itemsep}{1pt}
\item May take a position, but support consists primarily of general assertions or list-like reasons with little to no elaboration.
\item Little or no effective organization; ideas may be loosely connected.
\item Language is often awkward or confused; errors frequent but meaning largely recoverable.
\item No real attempt at addressing counterarguments or providing specific examples.
\end{itemize}
Rater guidance: Use 2 for responses that go beyond fragments but still present primarily undeveloped, general claims with minimal cohesion.

\textbf{Score Point 3:} Minimally developed / Some organization, minimal elaboration\\
Typical elements (choose 2--4 to justify a 3):
\begin{itemize}
\setlength{\itemsep}{1pt}
\item Takes a position and supplies one or more reasons, but elaboration is minimal and details are mostly general rather than specific.
\item Shows some organization (intro, body, conclusion or paragraphing) though transitions may be weak and sequencing simplistic.
\item Examples, if present, are few, repetitive, or only loosely connected to claims (e.g., listing benefits without explanation).
\item Errors and awkward phrasing are evident but do not severely block comprehension.
\end{itemize}
Rater guidance: Assign 3 when the essay shows a clear position and basic organizational structure but lacks adequate, specific development. If there are only general reasons and the examples are mainly restatements or repeats, prefer 3 over 4. A couple of short, distinct examples with almost no explanation can still be 3; require at least brief explanation to move above 3.

\textbf{Score Point 4:} Somewhat developed / Adequate support and organization\\
Typical elements (choose 3--5 to justify a 4):
\begin{itemize}
\setlength{\itemsep}{1pt}
\item Takes a clear position and provides adequately elaborated reasons with a mix of general and specific details.
\item Presents multiple distinct supporting points and at least some specific examples or brief explanations (e.g., named applications, short anecdotes, simple illustrative scenarios).
\item Organization is satisfactory: logical sequence, basic use of transitions, and paragraphs that group ideas.
\item Fluency is moderate; errors present but do not significantly impede persuasiveness or clarity.
\item May briefly acknowledge or respond to a counterargument or limitation.
\end{itemize}
Rater guidance: Use 4 when the response is reasonably persuasive and shows adequate breadth of support with at least some specific examples or short elaborations---even if surface errors or awkwardness are frequent. If the essay offers several reasons with a few concrete examples but the examples are not well explained or are uneven in quality, score 4. If organization is weak but there are multiple specific examples with some explanation, prefer 4 over 3.

\textbf{Score Point 5:} Developed / Clear position with persuasive, specific support\\
Typical elements (choose 3--5 to justify a 5):
\begin{itemize}
\setlength{\itemsep}{1pt}
\item Takes a clear, consistent position and provides reasonably persuasive support with mostly specific details and concrete examples.
\item Reasons are moderately well elaborated: cause/effect, implications, or examples explained beyond mere mention.
\item Organization is strong: clear progression of ideas, reasonably consistent paragraph structure, and functional transitions.
\item Writing is generally fluent; surface errors are present but do not hinder persuasiveness or clarity.
\item May include a limited but relevant counterargument and rebuttal.
\end{itemize}
Rater guidance: Assign 5 when the essay offers multiple distinct, concrete examples that are explained or linked to the claim, has clear logical organization, and is persuasive overall despite some errors. Multiple concrete examples across different domains (e.g., health, environment, social/family) each with brief explanation $\to$ prefer 5 rather than 4. Named anecdotes or brief case reports count as concrete examples when they illustrate the claim and are connected to the reasoning.

\textbf{Score Point 6:} Well-developed / Thoughtful, persuasive, and fluent\\
Typical elements (choose 3--5 to justify a 6):
\begin{itemize}
\setlength{\itemsep}{1pt}
\item Takes a clear and thoughtful position and provides fully elaborated reasons with specific, vivid details and well-chosen examples.
\item Exhibits strong organization and coherence: effective introduction, logically developed paragraphs, clear transitions, and a persuasive conclusion.
\item Writing is fluent and displays sophisticated use of language; errors are rare and do not distract.
\item Demonstrates heightened awareness of audience and purpose (tone, rhetorical devices, effective counterargument/refutation).
\end{itemize}
Rater guidance: Reserve 6 for essays that are persuasive and mature in thought and expression---detailed, well-explained examples, clear reasoning, and strong control of language.

\textbf{Additional decision rules to reduce inconsistency}
\begin{itemize}
\setlength{\itemsep}{1pt}
\item Count distinct examples: 0--1 distinct examples $\to$ less likely than 4; 2--3 distinct examples briefly explained $\to$ 4; 3+ distinct examples with brief explanation across different domains or backed by named anecdotes/studies $\to$ 5.
\item Distinctness beats quantity: three distinct, separate examples (different domains or different types of evidence) that are each at least briefly explained should lift an essay to 5 even if writing is error-prone.
\item Repetition detection: if support is largely the same point repeated (e.g., ``computers cause obesity'' restated many times with minor wording changes), do not count as multiple elaborations.
\item Anecdotes: treat properly connected anecdotes as evidence. If an essay includes several personal or named anecdotes that illustrate separate facets of the claim, these count toward higher development.
\item Organization vs.\ development: When organization is weak but multiple specific, relevant examples are present and explained, prefer 4 or 5 (depending on number/explainedness) rather than 3.
\item Surface errors: Frequent mechanical errors by themselves should not block a 4 or 5. Downgrade more for error-driven incomprehensibility than for presence of many errors alone.
\item When uncertain between 3 and 4: ask whether there are any concrete or named examples that go beyond mere lists. If yes $\to$ 4. When uncertain between 4 and 5: count distinct, explained examples across domains; 3+ $\to$ 5.
\end{itemize}

Illustrative heuristics (short mapping):
\begin{itemize}
\setlength{\itemsep}{1pt}
\item Score 3 $\to$ clear claim + several general reasons, little elaboration, weak transitions (e.g., listing benefits or harms with minimal examples).
\item Score 4 $\to$ clear claim + multiple reasons with at least some specific examples or brief illustrations and satisfactory organization; persuasive but not consistently developed.
\item Score 5 $\to$ clear, persuasive claim + multiple distinct, specific examples that are explained and connected to the thesis; strong organization; readers are likely convinced.
\end{itemize}
\end{tcolorbox}

\begin{table}[t]
\centering
\small
\resizebox{\columnwidth}{!}{%
\begin{tabular}{llrrrrr}
\toprule
\textbf{Dataset} & \textbf{LLM} & \textbf{Baseline} & \textbf{$N=10$} & \textbf{$N=20$} & \textbf{$N=50$} & \textbf{$N=100$} \\
\midrule
ASAP & Gemini 3 Flash & 0.427 & 0.382 & 0.267 & 0.538 & \textbf{0.646} \\
 & GPT-5 mini & 0.042 & 0.294 & 0.249 & 0.409 & \textbf{0.480} \\
 & Qwen3-80B-A3B & 0.121 & 0.302 & 0.395 & 0.383 & \textbf{0.473} \\
\midrule
ASAP 2.0 & Gemini 3 Flash & 0.613 & 0.366 & \textbf{0.724} & 0.619 & 0.700 \\
 & GPT-5 mini & 0.358 & 0.322 & \textbf{0.544} & 0.540 & 0.537 \\
 & Qwen3-80B-A3B & 0.575 & 0.426 & 0.616 & 0.557 & \textbf{0.636} \\
\bottomrule
\end{tabular}%
}
\caption{Effect of training sample size ($N$) on test QWK. Baseline uses the human expert rubric without optimization. Bold indicates the best QWK among training sizes for each model.}
\label{tab:train_size}
\end{table}

\begin{table*}[t]
\centering
\small
\caption{Quantitative comparison between seed and refined rubrics. The \textit{Seed} rubric is the human-authored rubric used as the starting point for optimization; \textit{Refined} is the rubric after iterative refinement. Word counts are shown for each, with \textit{$\Delta$\,Words} indicating the change. \textit{Overlap} is the percentage of refined rubric words that also appear in the seed rubric (order-preserving longest common subsequence). $\Delta$\,QWK is the improvement in test-set Quadratic Weighted Kappa after refinement (Refined $-$ Seed).}
\label{tab:rubric_comparison}
\begin{tabular}{ll rrr r r}
\toprule
Dataset & Model & Seed & Refined & $\Delta$\,Words & Overlap (\%) & $\Delta$\,QWK \\
\midrule
  ASAP & Gemini 3 Flash & 352 & 920 & +568 & 19.8 & +0.219 \\
  ASAP & GPT-5 mini & 352 & 1,696 & +1,344 & 5.1 & +0.438 \\
  ASAP & Qwen3 Next & 352 & 1,925 & +1,573 & 12.3 & +0.352 \\
\midrule
  ASAP 2.0 & Gemini 3 Flash & 688 & 624 & -64 & 41.0 & +0.087 \\
  ASAP 2.0 & GPT-5 mini & 688 & 1,632 & +944 & 1.1 & +0.179 \\
  ASAP 2.0 & Qwen3 Next & 688 & 1,758 & +1,070 & 38.5 & +0.061 \\
\midrule
  TOEFL11 & Gemini 3 Flash & 250 & 430 & +180 & 30.9 & -0.018 \\
  TOEFL11 & GPT-5 mini & 250 & 2,384 & +2,134 & 3.3 & -0.099 \\
  TOEFL11 & Qwen3 Next & 250 & 637 & +387 & 18.8 & +0.088 \\
\midrule
  \multicolumn{2}{l}{\textbf{Average}} & 430 & 1,334 & +904 & 19.0 & +0.145 \\
\bottomrule
\end{tabular}
\end{table*}

% \colorlet{rpTypeRule}{red!80!black}
\colorlet{rpTypeEvidence}{blue!80!black}
\colorlet{rpTypeWriting}{teal!80!black}
\begin{figure*}[t]
\centering
\begin{tcolorbox}[colback=white,colframe=black!25,title=Pattern Type Guide,fonttitle=\bfseries\small,fontupper=\scriptsize,boxsep=1pt,left=2pt,right=2pt,top=2pt,bottom=2pt]
\textcolor{rpTypeRule}{\textbf{Rule Structure}}: Explicit decision logic for scoring: conditional branches, boundary tie-breakers, stepwise workflows, and numeric thresholds.\par \textcolor{rpTypeEvidence}{\textbf{Evidence Handling}}: How evidence is validated and counted: specific-example requirements, repetition/non-double-count rules, and cap rules for weak evidence.\par \textcolor{rpTypeWriting}{\textbf{Writing Quality}}: Language-quality criteria affecting score bands: organization/coherence/transition quality and grammar/mechanics severity.
\end{tcolorbox}
\vspace{1mm}
\begin{tcolorbox}[colback=white,colframe=black!25,title=Detailed Pattern Notes,fonttitle=\bfseries\small,fontupper=\scriptsize,boxsep=1pt,left=2pt,right=2pt,top=2pt,bottom=2pt]
\textcolor{rpTypeRule}{\textbf{Rule Structure}}:\par \quad \textcolor{rpTypeRule}{\textbf{Conditional Gating}} [n=4] Captures explicit condition-based rules that switch decisions only when a stated condition is met. Typical cues: if, when.\par \quad \textcolor{rpTypeRule}{\textbf{Quantitative Threshold}} [n=1] Marks numeric cutoffs used for consistent decisions (minimum/maximum counts, percentages, explicit count thresholds). Typical cues: at least, at most, <=, >=, \%, N reasons/examples/sentences/words.\par \textcolor{rpTypeEvidence}{\textbf{Evidence Handling}}:\par \quad \textcolor{rpTypeEvidence}{\textbf{Specific Evidence Requirement}} [n=9] Highlights demands for concrete examples and explicit evidence links instead of generic assertions. Typical cues: for example, e.g., specific example, illustration, anecdote, evidence.\par \textcolor{rpTypeWriting}{\textbf{Writing Quality}}:\par \quad \textcolor{rpTypeWriting}{\textbf{Organization / Coherence Signal}} [n=3] Detects explicit references to discourse structure and logical flow as scoring criteria. Typical cues: organization, coherence, logical flow, transition.\par \quad \textcolor{rpTypeWriting}{\textbf{Grammar / Mechanics Signal}} [n=2] Detects references to language-form quality, especially grammar, spelling, punctuation, and mechanics. Typical cues: grammar, mechanics, spelling, punctuation.
\end{tcolorbox}
\vspace{1mm}
\begin{tcolorbox}[colback=white,colframe=black!25,title=Optimized Rubric (Pattern-Highlighted),fonttitle=\bfseries\small,fontupper=\scriptsize]
\ttfamily
After reading each essay and completing the analytical rating form, assign a holistic score based on the rubric below. For the following evaluations you will need to use a grading scale between 1 (minimum) and 6 (maximum). The distance between each grade should be considered equal. \textcolor{rpTypeRule}{\textbf{When}} scoring, prioritize the quality of critical thinking and the student's ability to use \textcolor{rpTypeEvidence}{\textbf{evidence}} over surface-level mechanical errors, unless those errors obscure meaning.\par ... [1 lines omitted] ...\par SCORE OF 6: Outstanding Mastery. An essay in this category demonstrates clear and consistent mastery. It effectively and insightfully develops a point of view, showing sophisticated critical thinking (\textcolor{rpTypeEvidence}{\textbf{e.g.}}, analyzing tone, diction, or complex contradictions). It uses clearly appropriate examples and \textcolor{rpTypeEvidence}{\textbf{evidence}} from the source text to support its position; it is well-organized, focused, and demonstrates clear \textcolor{rpTypeWriting}{\textbf{coherence}} and smooth progression of ideas. It exhibits skillful use of language, varied vocabulary, and meaningful variety in sentence structure.\par ... [1 lines omitted] ...\par SCORE OF 5: Strong Mastery. An essay in this category demonstrates reasonably consistent mastery, though it will have occasional errors or lapses in quality. The essay effectively develops a point of view and demonstrates strong critical thinking; it generally uses appropriate \textcolor{rpTypeEvidence}{\textbf{evidence}} from the source text; it is well-organized and focused, demonstrating \textcolor{rpTypeWriting}{\textbf{coherence}} and progression; it exhibits facility in language, using appropriate vocabulary and varied sentence structure.\par ... [1 lines omitted] ...\par SCORE OF 4: Adequate Mastery. An essay in this category demonstrates adequate mastery. The essay develops a clear point of view and demonstrates competent critical thinking by connecting the text to a broader argument, synthesizing several parts of the text to support a theme, or by analyzing the author's methods (\textcolor{rpTypeEvidence}{\textbf{e.g.}}, how the author uses facts or examples to persuade). \textcolor{rpTypeRule}{\textbf{If}} the essay moves beyond a chronological retelling to a thematic \textcolor{rpTypeWriting}{\textbf{organization}}-even \textcolor{rpTypeRule}{\textbf{if}} it contains frequent errors in \textcolor{rpTypeWriting}{\textbf{grammar}} and \textcolor{rpTypeWriting}{\textbf{mechanics}}-it should receive a 4. The student must demonstrate an understanding of the text's construction rather than just its content.\par ... [1 lines omitted] ...\par SCORE OF 3: Developing Mastery. An essay in this category demonstrates developing mastery and is marked by one or more of the following: it develops a point of view but does so inconsistently; it shows some attempt at analysis but is limited in depth or relies more on paraphrasing than critical evaluation. A 3 may identify the author's argument and provide \textcolor{rpTypeEvidence}{\textbf{evidence}} but fails to explain *how* or *why* the \textcolor{rpTypeEvidence}{\textbf{evidence}} supports the argument in a meaningful way. It may follow the text's chronology too closely (\textcolor{rpTypeEvidence}{\textbf{e.g.}}, "In paragraph 1... in paragraph 2...") but must offer \textcolor{rpTypeRule}{\textbf{at least}} some original interpretation or evaluation of the ideas to remain in this category.\par ... [1 lines omitted] ...\par SCORE OF 2: Little Mastery. An essay in this category demonstrates little mastery and is flawed by one or more of the following: it develops a vague, simplistic, or seriously limited point of view; it relies almost exclusively on listing facts or providing a sequential summary of the text. Common traits of a 2 include substituting conversational fillers, repetitive rhetorical questions (\textcolor{rpTypeEvidence}{\textbf{e.g.}}, "Don't you want to find out?"), or a simple "I agree" for an actual argument. Even \textcolor{rpTypeRule}{\textbf{if}} the writing is relatively clear and the structure is logical, an essay that is essentially a summary with a brief personal opinion tacked on belongs in this category.
\end{tcolorbox}
\caption{Pattern-focused view of the optimized rubric (ASAP2, google\_gemini-3-flash-preview, base\_expert\_True\_train100\_iteration5\_top3\_bs4-8-12\_mc4). Colored bold spans indicate regex-matched rubric cues. Color types: \textcolor{rpTypeRule}{\textbf{Rule Structure}} (Explicit decision logic for scoring: conditional branches, boundary tie-breakers, stepwise workflows, and numeric thresholds.); \textcolor{rpTypeEvidence}{\textbf{Evidence Handling}} (How evidence is validated and counted: specific-example requirements, repetition/non-double-count rules, and cap rules for weak evidence.); \textcolor{rpTypeWriting}{\textbf{Writing Quality}} (Language-quality criteria affecting score bands: organization/coherence/transition quality and grammar/mechanics severity.). Matched pattern categories: Conditional Gating (n=4); Specific Evidence Requirement (n=9); Organization / Coherence Signal (n=3); Grammar / Mechanics Signal (n=2); Quantitative Threshold (n=1).}
\label{fig:rubric_pattern_ASAP2_google_gemini_3_flash_preview_base_expert_True_train100_iteration5_top3_bs4_8_12_mc4}
\end{figure*}


\colorlet{rpTypeRule}{red!80!black}
\colorlet{rpTypeEvidence}{blue!80!black}
\colorlet{rpTypeWriting}{teal!80!black}
\begin{figure*}[t]
\centering
\begin{tcolorbox}[colback=white,colframe=black!25,title=Pattern Legend,fonttitle=\bfseries\small,fontupper=\scriptsize,boxsep=1pt,left=2pt,right=2pt,top=2pt,bottom=2pt]
\textcolor{rpTypeRule}{\textbf{Rule Structure}} (if/threshold/stepwise guidance) \quad \textcolor{rpTypeEvidence}{\textbf{Evidence Handling}} (examples, repetition, and caps) \quad \textcolor{rpTypeWriting}{\textbf{Writing Quality}} (organization and grammar/mechanics)
\end{tcolorbox}
\vspace{2mm}
\begin{minipage}[t]{0.485\textwidth}
\begin{tcolorbox}[colback=white,colframe=black!25,title=Initial Rubric,fonttitle=\bfseries\small,fontupper=\scriptsize,breakable]
\ttfamily
Based on the response's content, rate the response on a scale of 1 to 6.
\end{tcolorbox}
\end{minipage}
\hfill
\begin{minipage}[t]{0.485\textwidth}
\begin{tcolorbox}[colback=white,colframe=black!25,title=Optimized Rubric,fonttitle=\bfseries\small,fontupper=\scriptsize,breakable]
\ttfamily
\par 1 - Minimal/Inadequate: The response provides little to no original input. It may consist of a few isolated facts, a wall of direct quotes without connecting thought, or be so brief and error-ridden (\textcolor{rpTypeWriting}{punctuation}, capitalization, \textcolor{rpTypeWriting}{spelling}) that it remains undeveloped or incoherent. Reasoning is often logically flawed or absent. Even \textcolor{rpTypeRule}{if} it attempts to follow the prompt, the lack of an independent summary or a clear voice places it here.\par \par 2 - Limited: The response shows a basic understanding but is primarily a chronological, paragraph-by-paragraph retelling or a list-like collection of facts and reactions. It relies heavily on the source's phrasing and \textcolor{rpTypeWriting}{organization}al structure. While there may be a hint of an evaluative stance or a simple conclusion, the "analysis" is almost entirely composed of strings of quotes. It lacks the cohesive structure of a functional summary.\par \par 3 - Functional: The response demonstrates a clear understanding and provides an independent summary that captures the essence of the text rather than just listing facts. It follows a logical structure (usually the source's chrono\textcolor{rpTypeWriting}{logical flow}). The student's voice is present in connecting ideas, though the reasoning may be simplistic or naive. It uses relevant \textcolor{rpTypeEvidence}{evidence} but does not yet effectively group that \textcolor{rpTypeEvidence}{evidence} by theme or concept (\textcolor{rpTypeEvidence}{e.g.}, it summarizes the text from beginning to end rather than by "pros vs. cons").\par \par 4 - Developing: The response provides a clear position and a complete, purposeful structure (intro, body, conclusion). The defining characteristic is the shift from a chronological retelling to a thematic grouping of \textcolor{rpTypeEvidence}{evidence} (\textcolor{rpTypeEvidence}{e.g.}, categorizing by "hazards," "geology," or "solutions"). While the student's own \textcolor{rpTypeWriting}{organization}al intent drives the response, it may still rely heavily on direct quotes to fill out the themes, contain repetitive points, or feature simplistic analysis. \par \par 5 - Advanced: The response provides a strong, purposeful analysis where the student's voice and thematic structure guide the response entirely. It consistently and effectively categorizes different types of \textcolor{rpTypeEvidence}{evidence} to explain how the author supports their claim. At this level, the depth of synthesis and the logical progression of the argument outweigh frequent mechanical or \textcolor{rpTypeWriting}{spelling} errors. The response demonstrates a clear command of how the \textcolor{rpTypeEvidence}{evidence} functions to support the author's broader purpose.\par \par 6 - Sophisticated: The response takes a critical, analytical stance, evaluating the rhetorical effectiveness of the text (\textcolor{rpTypeEvidence}{e.g.}, analyzing diction, tone, or logical contradictions). It is highly effective and well-organized, demonstrating deep insight into the nuances of the author's argument. It moves beyond simple categorization to synthesize information with high-level proficiency. Minor mechanical errors do not detract from the sophisticated evaluation and synthesis.
\end{tcolorbox}
\end{minipage}
\caption{Pattern-highlighted rubric comparison (ASAP2, google\_gemini-3-flash-preview, base\_simplest\_True\_train100\_iteration5\_top3\_bs4-8-12\_mc4). Matched spans are color-coded by regex pattern. Color types: \textcolor{rpTypeRule}{\textbf{Rule Structure}} (if/threshold/stepwise guidance); \textcolor{rpTypeEvidence}{\textbf{Evidence Handling}} (examples, repetition, and caps); \textcolor{rpTypeWriting}{\textbf{Writing Quality}} (organization and grammar/mechanics).}
\label{fig:rubric_pattern_ASAP2_google_gemini_3_flash_preview_base_simplest_True_train100_iteration5_top3_bs4_8_12_mc4}
\end{figure*}


\colorlet{rpTypeRule}{red!80!black}
\colorlet{rpTypeEvidence}{blue!80!black}
\colorlet{rpTypeWriting}{teal!80!black}
\begin{figure*}[t]
\centering
\begin{tcolorbox}[colback=white,colframe=black!25,title=Pattern Type Guide,fonttitle=\bfseries\small,fontupper=\scriptsize,boxsep=1pt,left=2pt,right=2pt,top=2pt,bottom=2pt]
\textcolor{rpTypeRule}{\textbf{Rule Structure}}: Explicit decision logic for scoring: conditional branches, boundary tie-breakers, stepwise workflows, and numeric thresholds.\par \textcolor{rpTypeEvidence}{\textbf{Evidence Handling}}: How evidence is validated and counted: specific-example requirements, repetition/non-double-count rules, and cap rules for weak evidence.\par \textcolor{rpTypeWriting}{\textbf{Writing Quality}}: Language-quality criteria affecting score bands: organization/coherence/transition quality and grammar/mechanics severity.
\end{tcolorbox}
\vspace{1mm}
\begin{tcolorbox}[colback=white,colframe=black!25,title=Detailed Pattern Notes,fonttitle=\bfseries\small,fontupper=\scriptsize,boxsep=1pt,left=2pt,right=2pt,top=2pt,bottom=2pt]
\textcolor{rpTypeRule}{\textbf{Rule Structure}}:\par \quad \textcolor{rpTypeRule}{\textbf{Conditional Gating}} [n=34] Captures explicit condition-based rules that switch decisions only when a stated condition is met. Typical cues: if, when.\par \quad \textcolor{rpTypeRule}{\textbf{Boundary / Tie-Break Guidance}} [n=7] Marks criteria used to resolve borderline cases between adjacent score bands (e.g., 4 vs 5). Typical cues: tie-break, borderline, boundary, threshold, 4 vs 5.\par \quad \textcolor{rpTypeRule}{\textbf{Stepwise Rating Workflow}} [n=18] Detects ordered procedures and checklists that standardize how raters walk through scoring decisions. Typical cues: step, checklist, workflow, procedure, first/second/third.\par \quad \textcolor{rpTypeRule}{\textbf{Quantitative Threshold}} [n=8] Marks numeric cutoffs used for consistent decisions (minimum/maximum counts, percentages, explicit count thresholds). Typical cues: at least, at most, <=, >=, \%, N reasons/examples/sentences/words.\par \textcolor{rpTypeEvidence}{\textbf{Evidence Handling}}:\par \quad \textcolor{rpTypeEvidence}{\textbf{Specific Evidence Requirement}} [n=9] Highlights demands for concrete examples and explicit evidence links instead of generic assertions. Typical cues: for example, e.g., specific example, illustration, anecdote, evidence.\par \quad \textcolor{rpTypeEvidence}{\textbf{Off-Topic / Summary Cap}} [n=30] Identifies cap rules that restrict scores when responses are off-topic, irrelevant, or dominated by summary-only content. Typical cues: off-topic, irrelevant, digression, summary-only, cap.\par \quad \textcolor{rpTypeEvidence}{\textbf{Repetition Non-Count Rule}} [n=1] Captures rules that treat repetition/restatement as non-distinct support and prevent double-counting. Typical cues: repetition, restatement, double-count, do not double-count.\par \textcolor{rpTypeWriting}{\textbf{Writing Quality}}:\par \quad \textcolor{rpTypeWriting}{\textbf{Organization / Coherence Signal}} [n=3] Detects explicit references to discourse structure and logical flow as scoring criteria. Typical cues: organization, coherence, logical flow, transition.\par \quad \textcolor{rpTypeWriting}{\textbf{Grammar / Mechanics Signal}} [n=10] Detects references to language-form quality, especially grammar, spelling, punctuation, and mechanics. Typical cues: grammar, mechanics, spelling, punctuation.
\end{tcolorbox}
\vspace{1mm}
\begin{tcolorbox}[colback=white,colframe=black!25,title=Optimized Rubric (Pattern-Highlighted),fonttitle=\bfseries\small,fontupper=\scriptsize]
\ttfamily
  - Under-penalizing \textcolor{rpTypeEvidence}{\textbf{summary-only}} responses and pervasive mechanical errors (assistant sometimes gave too-high scores).\par   - Inconsistent application of the \textcolor{rpTypeEvidence}{\textbf{Summary-only}} \textcolor{rpTypeEvidence}{\textbf{cap}}, \textcolor{rpTypeEvidence}{\textbf{Evidence}}-count rule, and Mechanical-distortion penalty.\par ... [3 lines omitted] ...\par 1. Prioritize Development (B) and \textcolor{rpTypeWriting}{\textbf{Mechanics}} (E) \textcolor{rpTypeRule}{\textbf{first}}: B determines maximum possible band (3 vs \textcolor{rpTypeRule}{\textbf{4 vs 5}}/6), then apply E penalties. A and C then refine placement; D refines ties within a band.\par 2. Use explicit, ordered scoring steps below rather than global judgment \textcolor{rpTypeRule}{\textbf{first}}.\par 3. Always record which caps/penalties were applied \textcolor{rpTypeRule}{\textbf{when}} scoring.\par ... [2 lines omitted] ...\par - "\textcolor{rpTypeEvidence}{\textbf{Summary-only}}" (B = Weak/Insufficient): The response predominantly restates source points or lists details with no explanatory linkage to the claim. Indicators: repeated paraphrase phrases, no cause/effect/implication/interpretation, no explanation of why \textcolor{rpTypeEvidence}{\textbf{evidence}} supports the claim.\par - "Shallow Adequate" (B = Adequate but paraphrase-heavy): \textcolor{rpTypeRule}{\textbf{At least}} one attempt to link \textcolor{rpTypeEvidence}{\textbf{evidence}} to claim is present but explanation is superficial, repetitive, or circular.\par - "Strong development" (B = Strong): Minimum of two distinct, specific, text-based examples/reasons. Each example must be explicitly explained (not only quoted) and tied to the claim through interpretation (cause/effect, implication) or analysis. Single detailed example = potentially Strong only \textcolor{rpTypeRule}{\textbf{if}} explanation is sustained and clearly tied to claim; otherwise counts as Adequate.\par - \textcolor{rpTypeWriting}{\textbf{Mechanics}} severity (E):\par ... [5 lines omitted] ...\par \textcolor{rpTypeRule}{\textbf{Step}} 1 - Assess Development (B) and impose \textcolor{rpTypeEvidence}{\textbf{Summary-only}} caps immediately:\par   - \textcolor{rpTypeRule}{\textbf{If}} B = Weak/Insufficient due to \textcolor{rpTypeEvidence}{\textbf{summary-only}} (paraphrase/\textcolor{rpTypeEvidence}{\textbf{repetition}} with no explanation), set MAX\_POSSIBLE = 3 and continue to \textcolor{rpTypeRule}{\textbf{Step}} 2.\par   - Else \textcolor{rpTypeRule}{\textbf{if}} B = Adequate but paraphrase-heavy with only one shallow link to claim, set MAX\_POSSIBLE = 4 and continue to \textcolor{rpTypeRule}{\textbf{Step}} 2.\par ... [2 lines omitted] ...\par \textcolor{rpTypeRule}{\textbf{Step}} 2 - Assess \textcolor{rpTypeWriting}{\textbf{Mechanics}} (E) and apply Mechanical-distortion penalty:\par   - \textcolor{rpTypeRule}{\textbf{If}} E = Severe and errors frequently obscure meaning: reduce MAX\_POSSIBLE by 1.\par   - \textcolor{rpTypeRule}{\textbf{If}} E = Severe and meaning is largely unintelligible: reduce MAX\_POSSIBLE by 2.\par   - \textcolor{rpTypeRule}{\textbf{If}} E = Noticeable: do not automatically reduce MAX\_POSSIBLE; note for \textcolor{rpTypeRule}{\textbf{tie-break}}ing and possible downward adjustment in \textcolor{rpTypeRule}{\textbf{Step}} 4.\par   - \textcolor{rpTypeRule}{\textbf{If}} E = Minor: no reduction.\par ... [1 lines omitted] ...\par \textcolor{rpTypeRule}{\textbf{Step}} 3 - Assess Central Claim (A), \textcolor{rpTypeWriting}{\textbf{Organization}} (C), Language (D)\par ... [11 lines omitted] ...\par \textcolor{rpTypeRule}{\textbf{Step}} 4 - Enforce MAX\_POSSIBLE and apply \textcolor{rpTypeRule}{\textbf{tie-break}}ers / final adjustments\par ... [1 lines omitted] ...\par   - \textcolor{rpTypeRule}{\textbf{Tie-break}}ers (used only \textcolor{rpTypeRule}{\textbf{if}} candidate <= MAX\_POSSIBLE and multiple close alternatives):\par     1. Favor essays with more and higher-quality explained \textcolor{rpTypeEvidence}{\textbf{evidence}} (B) - a Strong B pushes toward higher score within the band.\par     2. \textcolor{rpTypeRule}{\textbf{If}} E = Noticeable and candidate is 5, downgrade to 4 unless B = Strong with multiple explained examples and C = Strong.\par     3. \textcolor{rpTypeRule}{\textbf{If}} E = Severe but MAX\_POSSIBLE reduction already applied in \textcolor{rpTypeRule}{\textbf{Step}} 2, allow no further upward adjustment.\par     4. \textcolor{rpTypeRule}{\textbf{If}} A and C are Strong but B = Adequate (with one convincing explained piece), and E = Minor, candidate can be 5 (only \textcolor{rpTypeRule}{\textbf{if}} MAX\_POSSIBLE >= 5).\par ... [1 lines omitted] ...\par     - \textcolor{rpTypeRule}{\textbf{If}} candidate = 5 but B = Adequate with only one shallow link, force Final\_score <= 4.\par     - \textcolor{rpTypeRule}{\textbf{If}} candidate = 4 but B = Adequate mainly paraphrase and (D = Weak or E = Noticeable/Severe), consider Final\_score = 3.\par ... [2 lines omitted] ...\par 1. \textcolor{rpTypeEvidence}{\textbf{Summary-only}} \textcolor{rpTypeEvidence}{\textbf{cap}} (refined):\par    - B = Weak due to \textcolor{rpTypeEvidence}{\textbf{summary-only}} -> MAX\_POSSIBLE = 3 always (no exception).\par ... [3 lines omitted] ...\par 2. \textcolor{rpTypeEvidence}{\textbf{Evidence}}-count requirement for 5-6 (clarified):\par ... [2 lines omitted] ...\par      - Either B = Strong (>=2 distinct explained pieces) OR (B = Adequate with \textcolor{rpTypeRule}{\textbf{at least}} one sustained, convincing, well-explained example that clearly ties to claim and C = Strong). Single brief example without sustained explanation cannot support 5.\par ... [2 lines omitted] ...\par 3. Mechanical-distortion penalty (refined \textcolor{rpTypeRule}{\textbf{threshold}}s):\par    - E = Severe -> reduce MAX\_POSSIBLE by 1 point as default; \textcolor{rpTypeRule}{\textbf{if}} meaning largely unintelligible reduce by 2 points.\par    - E = Noticeable -> no automatic reduction from MAX\_POSSIBLE, but treat Noticeable as negative \textcolor{rpTypeRule}{\textbf{tie-break}}er preventing upward movement: do not promote an essay above 4 \textcolor{rpTypeRule}{\textbf{if}} E = Noticeable unless B = Strong and multiple content strengths exist.\par ... [1 lines omitted] ...\par 4. Development vs. \textcolor{rpTypeWriting}{\textbf{Mechanics}} trade-off clarified numerically:\par    - \textcolor{rpTypeRule}{\textbf{If}} B = Strong and C = Strong, allow up to +1 above what E = Noticeable would otherwise permit (i.e., \textcolor{rpTypeRule}{\textbf{if}} candidate = 5 but E = Noticeable, allow 5 only \textcolor{rpTypeRule}{\textbf{if}} both B and C are Strong).\par    - Conversely, \textcolor{rpTypeRule}{\textbf{if}} B = Adequate or Weak, strong D/E cannot move essay upward past MAX\_POSSIBLE.\par ... [3 lines omitted] ...\par - Use \textcolor{rpTypeRule}{\textbf{Step}} 3 mapping to candidate score, then enforce MAX\_POSSIBLE from \textcolor{rpTypeRule}{\textbf{Step}} 1 and penalties from \textcolor{rpTypeRule}{\textbf{Step}} 2.\par ... [1 lines omitted] ...\par   - 5 dimensions Strong -> 6 (\textcolor{rpTypeRule}{\textbf{if}} E = Minor). \textcolor{rpTypeRule}{\textbf{If}} E = Noticeable, still can be 5; \textcolor{rpTypeRule}{\textbf{if}} E = Severe, apply penalty and re-evaluate.\par   - 4 Strong + 1 Adequate -> 5 \textcolor{rpTypeRule}{\textbf{if}} Adequate is not B; \textcolor{rpTypeRule}{\textbf{if}} the Adequate is B and it is paraphrase-heavy, \textcolor{rpTypeEvidence}{\textbf{cap}} at 4.\par   - 3 Strong, 1 Adequate, 1 Weak -> normally 4. Upgrade to 5 only \textcolor{rpTypeRule}{\textbf{if}} B = Strong or B = Adequate WITH one strong explained example AND E = Minor.\par   - 2 Strong, rest Adequate -> normally 3. Upgrade to 4 only \textcolor{rpTypeRule}{\textbf{if}} E = Minor and B shows \textcolor{rpTypeRule}{\textbf{at least}} one clear, well-explained example (not mere paraphrase).\par   - 1 Strong or 0 Strong -> 1 or 2 depending on E severity and whether any coherent claim/\textcolor{rpTypeEvidence}{\textbf{evidence}} exists.\par ... [2 lines omitted] ...\par - Pitfall A: Scoring a paraphrase-heavy essay as 4 \textcolor{rpTypeRule}{\textbf{when}} development is shallow and \textcolor{rpTypeWriting}{\textbf{mechanics}} are Noticeable.\par   - Fix: \textcolor{rpTypeRule}{\textbf{If}} B = Adequate but primarily paraphrase and E = Noticeable, default to 3 unless there is \textcolor{rpTypeRule}{\textbf{at least}} one convincingly explained example (sustained explanation) -> then 4.\par - Pitfall B: Scoring a fragmented essay with Severe \textcolor{rpTypeWriting}{\textbf{mechanics}} too highly (assistant sometimes was too high).\par   - Fix: \textcolor{rpTypeRule}{\textbf{If}} E = Severe, reduce MAX\_POSSIBLE by \textcolor{rpTypeRule}{\textbf{at least}} 1; \textcolor{rpTypeRule}{\textbf{if}} E = Severe + B = Weak or A absent, final should be 1 or 2.\par - Pitfall C: Failing to \textcolor{rpTypeEvidence}{\textbf{cap}} at 3 for clear \textcolor{rpTypeEvidence}{\textbf{summary-only}} essays (assistant sometimes gave 4).\par   - Fix: Apply \textcolor{rpTypeEvidence}{\textbf{Summary-only}} \textcolor{rpTypeEvidence}{\textbf{cap}} in \textcolor{rpTypeRule}{\textbf{Step}} 1 strictly: B = Weak (\textcolor{rpTypeEvidence}{\textbf{summary-only}}) -> MAX\_POSSIBLE = 3 regardless of other apparent strengths.\par ... [1 lines omitted] ...\par   - Fix: Enforce \textcolor{rpTypeEvidence}{\textbf{Evidence}}-count: \textcolor{rpTypeRule}{\textbf{at least}} two distinct explained evidentiary elements (or one sustained, convincing analysis + strong \textcolor{rpTypeWriting}{\textbf{organization}}) for 5; two for 6 with other dimensions Strong.\par - Pitfall E: Letting good \textcolor{rpTypeWriting}{\textbf{mechanics}} rescue very shallow content.\par   - Fix: B and A must be \textcolor{rpTypeRule}{\textbf{at least}} Adequate for score >= 4. Strong \textcolor{rpTypeWriting}{\textbf{mechanics}} and language without development should not push above MAX\_POSSIBLE from \textcolor{rpTypeRule}{\textbf{Step}} 1.\par ... [3 lines omitted] ...\par   - Example: "A=Adequate; B=Adequate (paraphrase-heavy); C=Adequate; D=Weak; E=Noticeable; Capped at 3 due to B=\textcolor{rpTypeEvidence}{\textbf{summary-only}}" (adapt as appropriate).\par - \textcolor{rpTypeRule}{\textbf{If}} any automatic \textcolor{rpTypeEvidence}{\textbf{cap}} or penalty was applied (\textcolor{rpTypeEvidence}{\textbf{Summary-only}} \textcolor{rpTypeEvidence}{\textbf{cap}}, Mechanical-distortion penalty), state it explicitly.\par ... [2 lines omitted] ...\par 1. Is development \textcolor{rpTypeEvidence}{\textbf{summary-only}}? Yes -> \textcolor{rpTypeEvidence}{\textbf{cap}} = 3. No -> continue.\par 2. Is B Adequate but paraphrase-heavy with single shallow link? Yes -> \textcolor{rpTypeEvidence}{\textbf{cap}} = 4. No -> no \textcolor{rpTypeEvidence}{\textbf{cap}}.\par 3. Is E Severe? Yes -> reduce \textcolor{rpTypeEvidence}{\textbf{cap}} by 1 (or 2 \textcolor{rpTypeRule}{\textbf{if}} largely unintelligible).\par ... [1 lines omitted] ...\par 5. Final\_score = min(candidate, \textcolor{rpTypeEvidence}{\textbf{cap}}); apply \textcolor{rpTypeRule}{\textbf{tie-break}}ers per \textcolor{rpTypeRule}{\textbf{Step}} 4.\par ... [2 lines omitted] ...\par - Essays that summarize source with no analysis -> final = 3 (apply \textcolor{rpTypeEvidence}{\textbf{summary-only}} \textcolor{rpTypeEvidence}{\textbf{cap}}).\par - Essays with multiple paraphrased details and noticeable \textcolor{rpTypeWriting}{\textbf{mechanics}} but \textcolor{rpTypeRule}{\textbf{at least}} one clear linkage -> usually 4.\par - Essays with many facts but Severe mechanical distortion (obscuring meaning) and \textcolor{rpTypeEvidence}{\textbf{summary-only}} development -> 1-2 depending on unintelligibility.\par - Essays with clear claim, multiple explained references, good \textcolor{rpTypeWriting}{\textbf{organization}}, and Minor \textcolor{rpTypeWriting}{\textbf{mechanics}} -> 5 or 6 depending on number/quality of explained \textcolor{rpTypeEvidence}{\textbf{evidence}} and strength of language.\par ... [2 lines omitted] ...\par - Always follow the Stepwise algorithm; do not skip the \textcolor{rpTypeEvidence}{\textbf{Summary-only}} \textcolor{rpTypeEvidence}{\textbf{cap}} in \textcolor{rpTypeRule}{\textbf{Step}} 1.\par - B and E decide bands \textcolor{rpTypeRule}{\textbf{first}}; A/C/D refine position within band.\par ... [1 lines omitted] ...\par - Use the \textcolor{rpTypeEvidence}{\textbf{Evidence}}-count requirement strictly for 5-6 placements.
\end{tcolorbox}
\caption{Pattern-focused view of the optimized rubric (ASAP2, openai\_gpt-5-mini, base\_expert\_True\_train100\_iteration5\_top3\_bs4-8-12\_mc4). Colored bold spans indicate regex-matched rubric cues. Color types: \textcolor{rpTypeRule}{\textbf{Rule Structure}} (Explicit decision logic for scoring: conditional branches, boundary tie-breakers, stepwise workflows, and numeric thresholds.); \textcolor{rpTypeEvidence}{\textbf{Evidence Handling}} (How evidence is validated and counted: specific-example requirements, repetition/non-double-count rules, and cap rules for weak evidence.); \textcolor{rpTypeWriting}{\textbf{Writing Quality}} (Language-quality criteria affecting score bands: organization/coherence/transition quality and grammar/mechanics severity.). Matched pattern categories: Conditional Gating (n=34); Boundary / Tie-Break Guidance (n=7); Stepwise Rating Workflow (n=18); Specific Evidence Requirement (n=9); Off-Topic / Summary Cap (n=30); Organization / Coherence Signal (n=3); Grammar / Mechanics Signal (n=10); Repetition Non-Count Rule (n=1); Quantitative Threshold (n=8).}
\label{fig:rubric_pattern_ASAP2_openai_gpt_5_mini_base_expert_True_train100_iteration5_top3_bs4_8_12_mc4}
\end{figure*}


\colorlet{rpTypeRule}{red!80!black}
\colorlet{rpTypeEvidence}{blue!80!black}
\colorlet{rpTypeWriting}{teal!80!black}
\begin{figure*}[t]
\centering
\begin{tcolorbox}[colback=white,colframe=black!25,title=Pattern Type Guide,fonttitle=\bfseries\small,fontupper=\scriptsize,boxsep=1pt,left=2pt,right=2pt,top=2pt,bottom=2pt]
\textcolor{rpTypeRule}{\textbf{Rule Structure}}: Explicit decision logic for scoring: conditional branches, boundary tie-breakers, stepwise workflows, and numeric thresholds.\par \textcolor{rpTypeEvidence}{\textbf{Evidence Handling}}: How evidence is validated and counted: specific-example requirements, repetition/non-double-count rules, and cap rules for weak evidence.\par \textcolor{rpTypeWriting}{\textbf{Writing Quality}}: Language-quality criteria affecting score bands: organization/coherence/transition quality and grammar/mechanics severity.
\end{tcolorbox}
\vspace{1mm}
\begin{tcolorbox}[colback=white,colframe=black!25,title=Detailed Pattern Notes,fonttitle=\bfseries\small,fontupper=\scriptsize,boxsep=1pt,left=2pt,right=2pt,top=2pt,bottom=2pt]
\textcolor{rpTypeRule}{\textbf{Rule Structure}}:\par \quad \textcolor{rpTypeRule}{\textbf{Conditional Gating}} [n=20] Captures explicit condition-based rules that switch decisions only when a stated condition is met. Typical cues: if, when.\par \quad \textcolor{rpTypeRule}{\textbf{Boundary / Tie-Break Guidance}} [n=17] Marks criteria used to resolve borderline cases between adjacent score bands (e.g., 4 vs 5). Typical cues: tie-break, borderline, boundary, threshold, 4 vs 5.\par \quad \textcolor{rpTypeRule}{\textbf{Stepwise Rating Workflow}} [n=6] Detects ordered procedures and checklists that standardize how raters walk through scoring decisions. Typical cues: step, checklist, workflow, procedure, first/second/third.\par \quad \textcolor{rpTypeRule}{\textbf{Quantitative Threshold}} [n=11] Marks numeric cutoffs used for consistent decisions (minimum/maximum counts, percentages, explicit count thresholds). Typical cues: at least, at most, <=, >=, \%, N reasons/examples/sentences/words.\par \textcolor{rpTypeEvidence}{\textbf{Evidence Handling}}:\par \quad \textcolor{rpTypeEvidence}{\textbf{Specific Evidence Requirement}} [n=34] Highlights demands for concrete examples and explicit evidence links instead of generic assertions. Typical cues: for example, e.g., specific example, illustration, anecdote, evidence.\par \quad \textcolor{rpTypeEvidence}{\textbf{Off-Topic / Summary Cap}} [n=4] Identifies cap rules that restrict scores when responses are off-topic, irrelevant, or dominated by summary-only content. Typical cues: off-topic, irrelevant, digression, summary-only, cap.\par \quad \textcolor{rpTypeEvidence}{\textbf{Repetition Non-Count Rule}} [n=1] Captures rules that treat repetition/restatement as non-distinct support and prevent double-counting. Typical cues: repetition, restatement, double-count, do not double-count.\par \textcolor{rpTypeWriting}{\textbf{Writing Quality}}:\par \quad \textcolor{rpTypeWriting}{\textbf{Organization / Coherence Signal}} [n=15] Detects explicit references to discourse structure and logical flow as scoring criteria. Typical cues: organization, coherence, logical flow, transition.\par \quad \textcolor{rpTypeWriting}{\textbf{Grammar / Mechanics Signal}} [n=2] Detects references to language-form quality, especially grammar, spelling, punctuation, and mechanics. Typical cues: grammar, mechanics, spelling, punctuation.
\end{tcolorbox}
\vspace{1mm}
\begin{tcolorbox}[colback=white,colframe=black!25,title=Optimized Rubric (Pattern-Highlighted),fonttitle=\bfseries\small,fontupper=\scriptsize]
\ttfamily
- Assign a single holistic score (1-6) reflecting overall performance across five dimensions: Comprehension \& Analysis, Development \& \textcolor{rpTypeWriting}{\textbf{Organization}}, Use of \textcolor{rpTypeEvidence}{\textbf{Evidence}}, Clarity \& Control of Language, and Overall Communicative Effectiveness.\par - For every score, provide a brief written justification naming the primary driver(s) (\textcolor{rpTypeEvidence}{\textbf{e.g.}}, "lowered to 2 due to pervasive sentence‑level errors" or "capped at 4 because \textcolor{rpTypeEvidence}{\textbf{evidence}} is quoted but not explained").\par ... [2 lines omitted] ...\par - To separate 1-2 from 3-6: place heavier weight on Clarity \& Control of Language (apply the strengthened Mechanical‑error rule \textcolor{rpTypeRule}{\textbf{first}}) and on Comprehension \& Analysis (including Factual‑misunderstanding rule).\par - To separate 3-4 from 5-6: place heavier weight on Use of \textcolor{rpTypeEvidence}{\textbf{Evidence}} and Development \& \textcolor{rpTypeWriting}{\textbf{Organization}} (especially \textcolor{rpTypeEvidence}{\textbf{evidence}} integration and explicit explanation).\par - To separate adjacent mid-scores (3 vs 4, \textcolor{rpTypeRule}{\textbf{4 vs 5}}): apply the three tie‑questions (see Conservatism in midpoints) but with concrete \textcolor{rpTypeRule}{\textbf{threshold}}s (see "Operational \textcolor{rpTypeRule}{\textbf{threshold}}s" below).\par ... [1 lines omitted] ...\par Operational \textcolor{rpTypeRule}{\textbf{threshold}}s and clarified decision rules\par ... [1 lines omitted] ...\par   - "Pervasive" errors = \textcolor{rpTypeWriting}{\textbf{spelling}}/word-choice/\textcolor{rpTypeWriting}{\textbf{punctuation}} or sentence‑level breakdowns in more than \textasciitilde{}25\textcolor{rpTypeRule}{\textbf{\%}} of sentences, OR repeated sentence fragments/run-ons that force reader to infer basic claims repeatedly.\par   - Assign 1 \textcolor{rpTypeRule}{\textbf{when}} meaning is largely or frequently incomprehensible even to a careful reader (multiple sentences/paragraphs where core claims cannot be reliably extracted).\par   - Assign 2 \textcolor{rpTypeRule}{\textbf{when}} pervasive errors often impede comprehension and the reader must repeatedly infer basic claims (meaning extractable only with repeated effort).\par   - Assign 3 \textcolor{rpTypeRule}{\textbf{when}} mechanical errors are frequent but the reader can generally extract the intended meaning without repeated inference (errors are distracting but not obstructive).\par   - Assign 4-6 only \textcolor{rpTypeRule}{\textbf{when}} meaning is consistently extractable and mechanical issues do not repeatedly force inference. (4 may have noticeable errors; 5-6 should have minor/no errors.)\par   - Raters must count error frequency roughly: \textcolor{rpTypeRule}{\textbf{if}} \textasciitilde{}1 in \textcolor{rpTypeRule}{\textbf{4 sentences}} or more contains a serious mechanical problem that impacts understanding, consider 2 rather than 3.\par ... [3 lines omitted] ...\par   - One significant factual inaccuracy that contradicts a core claim of the source (\textcolor{rpTypeEvidence}{\textbf{e.g.}}, saying "no spacecraft have ever landed on Venus" \textcolor{rpTypeRule}{\textbf{when}} the passage or known facts indicate otherwise) -> lower Comprehension \& Analysis \textcolor{rpTypeRule}{\textbf{at least}} one score band (\textcolor{rpTypeEvidence}{\textbf{e.g.}}, 4->3, 3->2).\par ... [3 lines omitted] ...\par - \textcolor{rpTypeEvidence}{\textbf{Evidence}}‑integration rule (strengthened):\par   - To award 5 or 6, the essay must (a) include relevant \textcolor{rpTypeEvidence}{\textbf{evidence}} for each major claim and (b) explicitly interpret or explain how each key piece of \textcolor{rpTypeEvidence}{\textbf{evidence}} supports the claim (linking, not just quoting).\par   - Essays that predominantly summarize the passage or string quotations with little explanation should \textcolor{rpTypeEvidence}{\textbf{cap}} at 4.\par ... [2 lines omitted] ...\par - Conservatism in midpoints (3 vs 4 and \textcolor{rpTypeRule}{\textbf{4 vs 5}}) - concrete tie questions and \textcolor{rpTypeRule}{\textbf{threshold}}s:\par   For each pair (3 vs 4 and \textcolor{rpTypeRule}{\textbf{4 vs 5}}), apply these three questions and score "yes" \textcolor{rpTypeRule}{\textbf{when}} the criterion is met clearly:\par ... [1 lines omitted] ...\par   (b) Is \textcolor{rpTypeEvidence}{\textbf{evidence}} explicitly linked to that claim (i.e., each major supporting point has \textcolor{rpTypeEvidence}{\textbf{evidence}} followed by explanation of how it supports the thesis)? (Yes = explicit linkage for most major claims; No = mostly paraphrase/quotation without explanation.)\par   (c) Does \textcolor{rpTypeWriting}{\textbf{organization}} show purposeful progression with clear paragraphing and \textcolor{rpTypeWriting}{\textbf{transition}}s (not mere \textcolor{rpTypeEvidence}{\textbf{repetition}} of passage sequence)? (Yes = purposeful development; No = repetitive summary or disjointed paragraphs.)\par   - For 3 vs 4: \textcolor{rpTypeRule}{\textbf{if}} \textcolor{rpTypeRule}{\textbf{at least}} two of three are "Yes", favor 4; otherwise favor 3. But \textcolor{rpTypeRule}{\textbf{if}} mechanical errors meet the "pervasive" \textcolor{rpTypeRule}{\textbf{threshold}} for 2, assign 2 instead.\par   - For \textcolor{rpTypeRule}{\textbf{4 vs 5}}: require all three "Yes" to favor 5. \textcolor{rpTypeRule}{\textbf{If}} only two are "Yes", remain at 4 unless \textcolor{rpTypeEvidence}{\textbf{evidence}} integration is strongly analytic and interpretive (in which case consider 5).\par ... [5 lines omitted] ...\par - Development \& \textcolor{rpTypeWriting}{\textbf{Organization}}: Logical, purposeful progression; distinct paragraphs and smooth \textcolor{rpTypeWriting}{\textbf{transition}}s.\par - Use of \textcolor{rpTypeEvidence}{\textbf{Evidence}}: \textcolor{rpTypeEvidence}{\textbf{Evidence}} is consistently relevant, fully integrated, and each key piece is explicitly explained in support of claims.\par ... [1 lines omitted] ...\par - Signals to award 6: All tie questions met; \textcolor{rpTypeEvidence}{\textbf{evidence}} exemplifies interpretation rather than summary; only isolated minor errors \textcolor{rpTypeRule}{\textbf{at most}}.\par ... [3 lines omitted] ...\par - Development \& \textcolor{rpTypeWriting}{\textbf{Organization}}: Coherent and substantive development; minor lapses possible.\par - Use of \textcolor{rpTypeEvidence}{\textbf{Evidence}}: \textcolor{rpTypeEvidence}{\textbf{Evidence}} usually integrated and explained; most major claims linked to textual support.\par - Clarity \& Control: Mechanical errors may be present but do not impede comprehension; error frequency well below "pervasive" (\textcolor{rpTypeEvidence}{\textbf{e.g.}}, <25\textcolor{rpTypeRule}{\textbf{\%}} of sentences affected).\par - Signals to award 5: All three tie questions largely met; explanation of \textcolor{rpTypeEvidence}{\textbf{evidence}} is analytic and not just paraphrase.\par ... [3 lines omitted] ...\par - Development \& \textcolor{rpTypeWriting}{\textbf{Organization}}: \textcolor{rpTypeWriting}{\textbf{Organization}} apparent but may be repetitive or underdeveloped.\par - Use of \textcolor{rpTypeEvidence}{\textbf{Evidence}}: Uses relevant \textcolor{rpTypeEvidence}{\textbf{evidence}}, but integration/explanation is superficial; heavy reliance on summary or quotes without interpretation.\par - Clarity \& Control: Noticeable errors and awkward phrasing but meaning is generally extractable without repeated inference (errors below "pervasive" \textcolor{rpTypeRule}{\textbf{threshold}}).\par - Signals to award 4: Typically answers \textcolor{rpTypeRule}{\textbf{at least}} two tie questions affirmatively for 3/4 decision; \textcolor{rpTypeEvidence}{\textbf{evidence}} linked superficially. Do not raise beyond 4 for quotation-heavy essays lacking interpretation.\par ... [3 lines omitted] ...\par - Development \& \textcolor{rpTypeWriting}{\textbf{Organization}}: Weak structure; uneven development or some disjointed paragraphs.\par - Use of \textcolor{rpTypeEvidence}{\textbf{Evidence}}: \textcolor{rpTypeEvidence}{\textbf{Evidence}} inconsistently linked; often used as summary or is minimal.\par ... [1 lines omitted] ...\par - Signals to award 3: One or none of the tie questions clearly met; errors present but do not reach "pervasive" \textcolor{rpTypeRule}{\textbf{threshold}} that would force a 2.\par ... [3 lines omitted] ...\par - Development \& \textcolor{rpTypeWriting}{\textbf{Organization}}: Poor, fragmented, illogical structure.\par - Use of \textcolor{rpTypeEvidence}{\textbf{Evidence}}: Little or no effective use of \textcolor{rpTypeEvidence}{\textbf{evidence}}; quotations are present but unexplained or misused.\par ... [1 lines omitted] ...\par - Signals to award 2: Error frequency meets "pervasive" \textcolor{rpTypeRule}{\textbf{threshold}} (>\textasciitilde{}25\textcolor{rpTypeRule}{\textbf{\%}} of sentences seriously problematic) and/or a significant factual misunderstanding exists (see Factual‑misunderstanding rule) that reduces comprehension by \textcolor{rpTypeRule}{\textbf{at least}} one band.\par ... [2 lines omitted] ...\par - Comprehension \& Analysis: Little to no comprehension; incoherent, nonsensical, or \textcolor{rpTypeEvidence}{\textbf{irrelevant}} content.\par - Development \& \textcolor{rpTypeWriting}{\textbf{Organization}}: No usable \textcolor{rpTypeWriting}{\textbf{organization}}; ideas largely incomprehensible.\par - Use of \textcolor{rpTypeEvidence}{\textbf{Evidence}}: No meaningful \textcolor{rpTypeEvidence}{\textbf{evidence}} usage.\par ... [5 lines omitted] ...\par   - \textcolor{rpTypeRule}{\textbf{If}} the essay has a clear central claim, reasonable paragraphing, and meaning is extractable without repeated effort, favor 4 (Competent) rather than 3-even \textcolor{rpTypeRule}{\textbf{if}} errors are frequent-so long as errors do not meet the "pervasive" \textcolor{rpTypeRule}{\textbf{threshold}}. Rationale must note mechanical issues as secondary drivers.\par   - Reserve 3 \textcolor{rpTypeRule}{\textbf{when}} \textcolor{rpTypeWriting}{\textbf{organization}}/analysis is weak in addition to frequent (but non‑pervasive) mechanical errors.\par ... [2 lines omitted] ...\par   - \textcolor{rpTypeEvidence}{\textbf{Cap}} at 4 unless the writer interprets, connects, and explains quotations to support a central claim. Quantity of accurate facts/quotations alone is insufficient for 5-6.\par ... [2 lines omitted] ...\par   - \textcolor{rpTypeRule}{\textbf{If}} a single significant factual contradiction to the passage or core knowledge appears, reduce the overall score by \textcolor{rpTypeRule}{\textbf{at least}} one band from what it would otherwise be. Cite the factual error in the justification.\par ... [3 lines omitted] ...\par   - \textcolor{rpTypeRule}{\textbf{If}} topical content is present but sentence‑level errors force repeated inference of basic claims (reader must re‑read repeatedly), assign 2 (Minimal). Only assign 3 \textcolor{rpTypeRule}{\textbf{when}} meaning is generally extractable without repeated inference.\par ... [1 lines omitted] ...\par Reviewer \textcolor{rpTypeRule}{\textbf{workflow}} \textcolor{rpTypeRule}{\textbf{checklist}} (to enforce consistent application)\par 1. Apply Mechanical‑error rule \textcolor{rpTypeRule}{\textbf{first}}: does the essay meet "pervasive" error \textcolor{rpTypeRule}{\textbf{threshold}}? \textcolor{rpTypeRule}{\textbf{If}} yes, consider 1-2 per definitions above.\par 2. Check for significant factual misunderstandings: do any errors contradict the passage's core claims? \textcolor{rpTypeRule}{\textbf{If}} yes, lower \textcolor{rpTypeRule}{\textbf{at least}} one band and document it.\par 3. Evaluate central claim and \textcolor{rpTypeWriting}{\textbf{organization}}: apply tie‑questions and operational \textcolor{rpTypeRule}{\textbf{threshold}}s.\par 4. Evaluate \textcolor{rpTypeEvidence}{\textbf{evidence}} integration: are quotations/paraphrases interpreted and linked to claims? \textcolor{rpTypeRule}{\textbf{If}} not, \textcolor{rpTypeEvidence}{\textbf{cap}} at 4.\par 5. Assign score and write a concise justification naming primary drivers (language problems, \textcolor{rpTypeEvidence}{\textbf{evidence}} integration, factual misunderstanding, or development).\par 6. \textcolor{rpTypeRule}{\textbf{If}} \textcolor{rpTypeRule}{\textbf{borderline}}, explicitly state which dimension tipped the decision.\par ... [2 lines omitted] ...\par - For every score, include a one‑sentence annotation listing the primary reason(s) for the score (\textcolor{rpTypeEvidence}{\textbf{e.g.}}, "Capped at 4 because \textcolor{rpTypeEvidence}{\textbf{evidence}} is quoted but not explained; frequent but non‑pervasive mechanical errors"). This is required to make disagreements traceable.\par ... [1 lines omitted] ...\par This revised guidance focuses on clearer operational \textcolor{rpTypeRule}{\textbf{threshold}}s for mechanical errors, stricter enforcement of factual‑misunderstanding penalties, and firmer rules for \textcolor{rpTypeEvidence}{\textbf{evidence}} integration and caps on quotation‑heavy essays. Applying the \textcolor{rpTypeRule}{\textbf{workflow}} \textcolor{rpTypeRule}{\textbf{checklist}} should reduce the common over- and under‑scoring patterns observed in past ratings.
\end{tcolorbox}
\caption{Pattern-focused view of the optimized rubric (ASAP2, openai\_gpt-5-mini, base\_simplest\_True\_train100\_iteration5\_top3\_bs4-8-12\_mc4). Colored bold spans indicate regex-matched rubric cues. Color types: \textcolor{rpTypeRule}{\textbf{Rule Structure}} (Explicit decision logic for scoring: conditional branches, boundary tie-breakers, stepwise workflows, and numeric thresholds.); \textcolor{rpTypeEvidence}{\textbf{Evidence Handling}} (How evidence is validated and counted: specific-example requirements, repetition/non-double-count rules, and cap rules for weak evidence.); \textcolor{rpTypeWriting}{\textbf{Writing Quality}} (Language-quality criteria affecting score bands: organization/coherence/transition quality and grammar/mechanics severity.). Matched pattern categories: Conditional Gating (n=20); Boundary / Tie-Break Guidance (n=17); Stepwise Rating Workflow (n=6); Specific Evidence Requirement (n=34); Off-Topic / Summary Cap (n=4); Organization / Coherence Signal (n=15); Grammar / Mechanics Signal (n=2); Repetition Non-Count Rule (n=1); Quantitative Threshold (n=11).}
\label{fig:rubric_pattern_ASAP2_openai_gpt_5_mini_base_simplest_True_train100_iteration5_top3_bs4_8_12_mc4}
\end{figure*}


\colorlet{rpTypeRule}{red!80!black}
\colorlet{rpTypeEvidence}{blue!80!black}
\colorlet{rpTypeWriting}{teal!80!black}
\begin{figure*}[t]
\centering
\begin{tcolorbox}[colback=white,colframe=black!25,title=Pattern Legend,fonttitle=\bfseries\small,fontupper=\scriptsize,boxsep=1pt,left=2pt,right=2pt,top=2pt,bottom=2pt]
\textcolor{rpTypeRule}{\textbf{Rule Structure}} (if/threshold/stepwise guidance) \quad \textcolor{rpTypeEvidence}{\textbf{Evidence Handling}} (examples, repetition, and caps) \quad \textcolor{rpTypeWriting}{\textbf{Writing Quality}} (organization and grammar/mechanics)
\end{tcolorbox}
\vspace{2mm}
\begin{minipage}[t]{0.485\textwidth}
\begin{tcolorbox}[colback=white,colframe=black!25,title=Initial Rubric,fonttitle=\bfseries\small,fontupper=\scriptsize,breakable]
\ttfamily
After reading each essay and completing the analytical rating form, assign a holistic score based on the rubric below. For the following evaluations you will need to use a grading scale between 1 (minimum) and 6 (maximum). As with the analytical rating form, the distance between each grade (\textcolor{rpTypeEvidence}{e.g.}, 1-2, 3-4, 4-5) should be considered equal.\par \par SCORE OF 6: An essay in this category demonstrates clear and consistent mastery, although it may have a few minor errors. A typical essay effectively and insightfully develops a point of view on the issue and demonstrates outstanding critical thinking; the essay uses clearly appropriate examples, reasons, and other \textcolor{rpTypeEvidence}{evidence} taken from the source text(s) to support its position; the essay is well organized and clearly focused, demonstrating clear \textcolor{rpTypeWriting}{coherence} and smooth progression of ideas; the essay exhibits skillful use of language, using a varied, accurate, and apt vocabulary and demonstrates meaningful variety in sentence structure; the essay is free of most errors in \textcolor{rpTypeWriting}{grammar}, usage, and \textcolor{rpTypeWriting}{mechanics}.\par \par SCORE OF 5: An essay in this category demonstrates reasonably consistent mastery, although it will have occasional errors or lapses in quality. A typical essay effectively develops a point of view on the issue and demonstrates strong critical thinking; the essay generally using appropriate examples, reasons, and other \textcolor{rpTypeEvidence}{evidence} taken from the source text(s) to support its position; the essay is well organized and focused, demonstrating \textcolor{rpTypeWriting}{coherence} and progression of ideas; the essay exhibits facility in the use of language, using appropriate vocabulary demonstrates variety in sentence structure; the essay is generally free of most errors in \textcolor{rpTypeWriting}{grammar}, usage, and \textcolor{rpTypeWriting}{mechanics}.\par \par SCORE OF 4: An essay in this category demonstrates adequate mastery, although it will have lapses in quality. A typical essay develops a point of view on the issue and demonstrates competent critical thinking; the essay using adequate examples, reasons, and other \textcolor{rpTypeEvidence}{evidence} taken from the source text(s) to support its position; the essay is generally organized and focused, demonstrating some \textcolor{rpTypeWriting}{coherence} and progression of ideas exhibits adequate; the essay may demonstrate inconsistent facility in the use of language, using generally appropriate vocabulary demonstrates some variety in sentence structure; the essay may have some errors in \textcolor{rpTypeWriting}{grammar}, usage, and \textcolor{rpTypeWriting}{mechanics}.\par \par SCORE OF 3: An essay in this category demonstrates developing mastery, and is marked by ONE OR MORE of the following weaknesses: develops a point of view on the issue, demonstrating some critical thinking, but may do so inconsistently or use inadequate examples, reasons, or other \textcolor{rpTypeEvidence}{evidence} taken from the source texts to support its position; the essay is limited in its \textcolor{rpTypeWriting}{organization} or focus, or may demonstrate some lapses in \textcolor{rpTypeWriting}{coherence} or progression of ideas displays; the essay may demonstrate facility in the use of language, but sometimes uses weak vocabulary or inappropriate word choice and/or lacks variety or demonstrates problems in sentence structure; the essay may contain an accumulation of errors in \textcolor{rpTypeWriting}{grammar}, usage, and \textcolor{rpTypeWriting}{mechanics}.\par \par SCORE OF 2: An essay in this category demonstrates little mastery, and is flawed by ONE OR MORE of the following weaknesses: develops a point of view on the issue that is vague or seriously limited, and demonstrates weak critical thinking; the essay providesinappropriate or insufficient examples, reasons, or other \textcolor{rpTypeEvidence}{evidence} taken from the source text to support its position; the essay is poorly organized and/or focused, or demonstrates serious problems with \textcolor{rpTypeWriting}{coherence} or progression of ideas; the essay displays very little facility in the use of language, using very limited vocabulary or incorrect word choice and/or demonstrates frequent problems in sentence structure; the essay contains errors in \textcolor{rpTypeWriting}{grammar}, usage, and \textcolor{rpTypeWriting}{mechanics} so serious that meaning is somewhat obscured.\par \par SCORE OF 1: An essay in this category demonstrates very little or no mastery, and is severely flawed by ONE OR MORE of the following weaknesses: develops no viable point of view on the issue, or provides little or no \textcolor{rpTypeEvidence}{evidence} to support its position; the essay is disorganized or unfocused, resulting in a disjointed orincoherent essay; the essay displays fundamental errors in vocabulary and/or demonstrates severe flaws in sentence structure; the essay contains pervasive errors in \textcolor{rpTypeWriting}{grammar}, usage, or \textcolor{rpTypeWriting}{mechanics} that persistently interfere with meaning.
\end{tcolorbox}
\end{minipage}
\hfill
\begin{minipage}[t]{0.485\textwidth}
\begin{tcolorbox}[colback=white,colframe=black!25,title=Optimized Rubric,fonttitle=\bfseries\small,fontupper=\scriptsize,breakable]
\ttfamily
After reading each essay and completing the analytical rating form, assign a holistic score based on the rubric below. For the following evaluations you will need to use a grading scale between 1 (minimum) and 6 (maximum). As with the analytical rating form, the distance between each grade (\textcolor{rpTypeEvidence}{e.g.}, 1-2, 3-4, 4-5) should be considered equal.\par \par SCORE OF 6: An essay in this category demonstrates clear and consistent mastery, although it may have a few minor errors. A typical essay effectively and insightfully develops a point of view on the issue and demonstrates outstanding critical thinking; the essay uses clearly appropriate examples, reasons, and other \textcolor{rpTypeEvidence}{evidence} taken from the source text(s) to support its position; the essay is well organized and clearly focused, demonstrating clear \textcolor{rpTypeWriting}{coherence} and smooth progression of ideas; the essay exhibits skillful use of language, using a varied, accurate, and apt vocabulary and demonstrates meaningful variety in sentence structure; the essay is free of most errors in \textcolor{rpTypeWriting}{grammar}, usage, and \textcolor{rpTypeWriting}{mechanics}. A score of 6 requires the essay to be a fully developed, original argument that engages deeply with the source material-not merely summarizing, restating, or superficially agreeing with it. The argument must show independent insight, synthesis of ideas, and a distinctive voice that goes beyond paraphrasing the source. The essay must not only cite \textcolor{rpTypeEvidence}{evidence} but interpret its significance, connect multiple ideas across the text, and offer a novel perspective that could not be derived solely from the source. Crucially, a score of 6 demands that the insight be substantive, original, and central to the argument-not merely implied, tangential, or rephrased from the source. Essays that accurately summarize the source, correctly identify its claims, and even correctly critique its reasoning, but fail to generate a new interpretive framework, unexpected connection, or distinctive evaluative stance, do not qualify for a 6.\par \par SCORE OF 5: An essay in this category demonstrates reasonably consistent mastery, although it will have occasional errors or lapses in quality. A typical essay effectively develops a point of view on the issue and demonstrates strong critical thinking; the essay generally uses appropriate examples, reasons, and other \textcolor{rpTypeEvidence}{evidence} taken from the source text(s) to support its position; the essay is well organized and focused, demonstrating \textcolor{rpTypeWriting}{coherence} and progression of ideas; the essay exhibits facility in the use of language, using appropriate vocabulary and demonstrating variety in sentence structure; the essay is generally free of most errors in \textcolor{rpTypeWriting}{grammar}, usage, and \textcolor{rpTypeWriting}{mechanics}. A score of 5 requires the essay to present a clear, substantive position with thoughtful development and \textcolor{rpTypeEvidence}{evidence}-based reasoning that goes beyond simple agreement or summary. The essay must interpret, connect, or extend ideas from the source-\textcolor{rpTypeEvidence}{e.g.}, explaining why a detail matters, comparing multiple claims, evaluating implications, or identifying an unstated assumption-not merely listing facts or repeating the author's claims. Language errors must be infrequent and never obscure meaning. Essays that summarize the source effectively but fail to add interpretive depth, even with good \textcolor{rpTypeWriting}{organization} and language, do not qualify for a 5. Do not award a 5 \textcolor{rpTypeRule}{if} the essay's analysis is dominated by paraphrasing or \textcolor{rpTypeRule}{if} the insight is implied rather than explicitly developed. Importantly, \textcolor{rpTypeRule}{if} the essay's position directly contradicts the source's intent (\textcolor{rpTypeEvidence}{e.g.}, arguing the author's point is invalid \textcolor{rpTypeRule}{when} the author is clearly advocating for it), the essay may still earn a 5 \textcolor{rpTypeRule}{if} it offers a well-supported, coherent, and analytically rich counter-argument grounded in the text.\par \par SCORE OF 4: An essay in this category demonstrates adequate mastery, although it will have lapses in quality. A typical essay develops a point of view on the issue and demonstrates competent critical thinking; the essay uses adequate examples, reasons, and other \textcolor{rpTypeEvidence}{evidence} taken from the source text(s) to support its position; the essay is generally organized and focused, demonstrating some \textcolor{rpTypeWriting}{coherence} and progression of ideas; the essay may demonstrate inconsistent facility in the use of language, using generally appropriate vocabulary and demonstrating some variety in sentence structure; the essay may have some errors in \textcolor{rpTypeWriting}{grammar}, usage, and \textcolor{rpTypeWriting}{mechanics}. A score of 4 requires the essay to assert a discernible position that responds directly to the prompt and to go beyond mere description or summary by offering \textcolor{rpTypeRule}{at least} one clear, developed interpretive insight-not just a single phrase or label. The essay must explain why the \textcolor{rpTypeEvidence}{evidence} matters, how ideas relate, or what the implications are-not just state that the author is "for" or "against" something. \textcolor{rpTypeEvidence}{For example}, saying "the author shows Venus is dangerous" is summary; saying "the author uses Venus's extreme conditions to argue that human ingenuity, not advanced tech, may be the key to exploration" is analysis. Essays that rely heavily on paraphrasing the source without adding analysis, or that repeat the same point without development, should not exceed a score of 3. Language errors may be noticeable but must not prevent understanding of the argument. \textcolor{rpTypeRule}{If} the essay's analysis is superficial (\textcolor{rpTypeEvidence}{e.g.}, "the author doesn't support his claim well enough") without explaining why or how, or \textcolor{rpTypeRule}{if} the evaluation merely repeats source content with no added insight, it should be scored 3 or lower. Do not award a 4 \textcolor{rpTypeRule}{if} the essay's position is vague, its \textcolor{rpTypeEvidence}{evidence} is disconnected, or its insight is buried in summary. Crucially, essays that misinterpret the source (\textcolor{rpTypeEvidence}{e.g.}, claiming the author uses "suspense" \textcolor{rpTypeRule}{when} the text is factual) but still offer a clear, developed interpretation of their own misunderstanding may still receive a 4 \textcolor{rpTypeRule}{if} the analysis is internally consistent and grounded in the text as they read it.\par \par SCORE OF 3: An essay in this category demonstrates developing mastery, and is marked by ONE OR MORE of the following weaknesses: develops a point of view on the issue, demonstrating some critical thinking, but may do so inconsistently or use inadequate examples, reasons, or other \textcolor{rpTypeEvidence}{evidence} taken from the source texts to support its position; the essay is limited in its \textcolor{rpTypeWriting}{organization} or focus, or may demonstrate some lapses in \textcolor{rpTypeWriting}{coherence} or progression of ideas; the essay may demonstrate facility in the use of language, but sometimes uses weak vocabulary or inappropriate word choice and/or lacks variety or demonstrates problems in sentence structure; the essay may contain an accumulation of errors in \textcolor{rpTypeWriting}{grammar}, usage, and \textcolor{rpTypeWriting}{mechanics}. A score of 3 is appropriate \textcolor{rpTypeRule}{when} the essay attempts to take a position but struggles to sustain it, or \textcolor{rpTypeRule}{when} the position is implied rather than clearly stated. Language errors may be frequent but do not consistently obscure meaning. Essays that are primarily descriptive, summarize the source without asserting a clear claim of their own, or that show only surface-level engagement (\textcolor{rpTypeEvidence}{e.g.}, restating source claims without analysis) may still receive a 3 \textcolor{rpTypeRule}{if} they show emerging critical thinking, such as a tentative connection or a single interpretive insight. However, \textcolor{rpTypeRule}{if} the essay is dominated by summary, contains no discernible independent claim, or is so confused in structure that the argument cannot be reconstructed, it must be scored lower. Importantly, essays that merely label the author's argument as "weak" or "not well supported" without explaining the basis for that judgment, or that repeat source phrases verbatim without interpretation, should not receive a 3 unless they contain \textcolor{rpTypeRule}{at least} one clear, albeit underdeveloped, evaluative insight. Do not award a 3 \textcolor{rpTypeRule}{if} the essay contains no discernible position or \textcolor{rpTypeRule}{if} the only "analysis" is restating source claims with minor rewording. Essays that misinterpret the source (\textcolor{rpTypeEvidence}{e.g.}, confusing "suspense" for factual description) but still attempt to build a coherent, \textcolor{rpTypeRule}{if} flawed, argument from that misreading may qualify for a 3 \textcolor{rpTypeRule}{if} the reasoning is internally consistent and shows an attempt to interpret-not just report-the text.\par \par SCORE OF 2: An essay in this category demonstrates little mastery, and is flawed by ONE OR MORE of the following weaknesses: develops a point of view on the issue that is vague or seriously limited, and demonstrates weak critical thinking; the essay provides inappropriate or insufficient examples, reasons, or other \textcolor{rpTypeEvidence}{evidence} taken from the source text to support its position; the essay is poorly organized and/or focused, or demonstrates serious problems with \textcolor{rpTypeWriting}{coherence} or progression of ideas; the essay displays very little facility in the use of language, using very limited vocabulary or incorrect word choice and/or demonstrates frequent problems in sentence structure; the essay contains errors in \textcolor{rpTypeWriting}{grammar}, usage, and \textcolor{rpTypeWriting}{mechanics} so serious that meaning is somewhat obscured. A score of 2 is reserved for essays that either fail to articulate a clear position, rely almost entirely on summary without analysis, or are so linguistically flawed that understanding is significantly impaired-even \textcolor{rpTypeRule}{if} some relevant content is present. Essays that repeat source phrases without interpretation, contain no original claim, and are marred by persistent grammatical errors that require effort to decode should be scored 2. Do not award a 3 \textcolor{rpTypeRule}{if} the essay does not demonstrate even minimal critical thinking or an attempt to move beyond summary. \textcolor{rpTypeRule}{If} the essay's language is so fragmented, misspelled, or syntactically broken that the intended argument must be inferred with difficulty, and the analysis is nonexistent or reduced to isolated phrases (\textcolor{rpTypeEvidence}{e.g.}, "it shows that scientists are studying"), it belongs in this category. \textcolor{rpTypeRule}{If} the essay's only "point" is a vague \textcolor{rpTypeEvidence}{restatement} of the prompt ("Venus is dangerous but interesting") without any explanation, it is a 2. Crucially, essays that misinterpret the source but offer no coherent argument or insight-even \textcolor{rpTypeRule}{if} they use source phrases-should be scored 2 \textcolor{rpTypeRule}{if} the reasoning cannot be reconstructed into a meaningful claim.\par \par SCORE OF 1: An essay in this category demonstrates very little or no mastery, and is severely flawed by ONE OR MORE of the following weaknesses: develops no viable point of view on the issue, or provides little or no \textcolor{rpTypeEvidence}{evidence} to support its position; the essay is disorganized or unfocused, resulting in a disjointed or incoherent essay; the essay displays fundamental errors in vocabulary and/or demonstrates severe flaws in sentence structure; the essay contains pervasive errors in \textcolor{rpTypeWriting}{grammar}, usage, or \textcolor{rpTypeWriting}{mechanics} that persistently interfere with meaning. A score of 1 is reserved for essays that are essentially non-responsive: they may be entirely \textcolor{rpTypeEvidence}{off-topic}, contain no discernible argument, or are so garbled by errors that no coherent meaning can be extracted-even \textcolor{rpTypeRule}{if} isolated phrases reference the source. Essays that are unintelligible, contain no identifiable stance, or consist only of fragmented phrases with no logical connection should receive a 1. \textcolor{rpTypeRule}{If} the essay contains only incoherent \textcolor{rpTypeEvidence}{repetition}s of source phrases, grammatical errors that prevent identification of any claim, or no discernible structure beyond a string of disconnected sentences, it must be scored 1-even \textcolor{rpTypeRule}{if} keywords from the prompt appear.
\end{tcolorbox}
\end{minipage}
\caption{Pattern-highlighted rubric comparison (ASAP2, qwen\_qwen3-next-80b-a3b-instruct, base\_expert\_True\_train100\_iteration5\_top3\_bs4-8-12\_mc4). Matched spans are color-coded by regex pattern. Color types: \textcolor{rpTypeRule}{\textbf{Rule Structure}} (if/threshold/stepwise guidance); \textcolor{rpTypeEvidence}{\textbf{Evidence Handling}} (examples, repetition, and caps); \textcolor{rpTypeWriting}{\textbf{Writing Quality}} (organization and grammar/mechanics).}
\label{fig:rubric_pattern_ASAP2_qwen_qwen3_next_80b_a3b_instruct_base_expert_True_train100_iteration5_top3_bs4_8_12_mc4}
\end{figure*}


\colorlet{rpTypeRule}{red!80!black}
\colorlet{rpTypeEvidence}{blue!80!black}
\colorlet{rpTypeWriting}{teal!80!black}
\begin{figure*}[t]
\centering
\begin{tcolorbox}[colback=white,colframe=black!25,title=Pattern Legend,fonttitle=\bfseries\small,fontupper=\scriptsize,boxsep=1pt,left=2pt,right=2pt,top=2pt,bottom=2pt]
\textcolor{rpTypeEvidence}{\textbf{Evidence Handling}} (examples, repetition, and caps) \quad \textcolor{rpTypeWriting}{\textbf{Writing Quality}} (organization and grammar/mechanics)
\end{tcolorbox}
\vspace{2mm}
\begin{minipage}[t]{0.485\textwidth}
\begin{tcolorbox}[colback=white,colframe=black!25,title=Initial Rubric,fonttitle=\bfseries\small,fontupper=\scriptsize,breakable]
\ttfamily
Based on the response's content, rate the response on a scale of 1 to 6.
\end{tcolorbox}
\end{minipage}
\hfill
\begin{minipage}[t]{0.485\textwidth}
\begin{tcolorbox}[colback=white,colframe=black!25,title=Optimized Rubric,fonttitle=\bfseries\small,fontupper=\scriptsize,breakable]
\ttfamily
\par Score 1: The response is severely deficient in all areas. It contains pervasive, uncorrected grammatical and \textcolor{rpTypeWriting}{spelling} errors that render most or all of the content unintelligible. Ideas are either absent, completely incoherent, or consist of fragmented phrases with no logical progression. Factual inaccuracies are numerous and fundamental. The response fails to demonstrate any meaningful understanding of the prompt or topic. Language issues are so severe that even basic comprehension of intent is impossible, and no coherent argument or relevant content can be discerned.\par \par Score 2: The response attempts to address the topic but is severely hampered by language issues. Grammatical errors, mis\textcolor{rpTypeWriting}{spelling}s, and awkward phrasing are frequent and persistent, significantly impeding clarity and comprehension. Ideas are partially present but poorly organized, underdeveloped, or contradictory. While some relevant content may be identifiable, it is buried in linguistic noise. The response may contain factual distortions, incomplete sentences, or illogical \textcolor{rpTypeWriting}{transition}s that prevent meaningful interpretation. Analysis is absent or reduced to isolated, disconnected phrases. The response does not demonstrate consistent understanding of the prompt, and its structure fails to support any coherent line of reasoning.\par \par Score 3: The response demonstrates a basic understanding of the topic with some relevant ideas, though language control is weak. Errors in \textcolor{rpTypeWriting}{grammar}, \textcolor{rpTypeWriting}{spelling}, and syntax are common but do not completely obscure meaning. Ideas are loosely connected and may lack depth or sophistication, but the central argument or observations are discernible. The response may include limited \textcolor{rpTypeEvidence}{evidence} or examples, though their use is imprecise or poorly integrated. Structure is rudimentary, and tone is inconsistent. While the response meets minimal expectations for \textcolor{rpTypeWriting}{coherence}, it lacks analytical development, nuanced interpretation, or effective integration of \textcolor{rpTypeEvidence}{evidence}.\par \par Score 4: The response presents a clear, mostly coherent argument with relevant ideas and sufficient development. Language use is generally effective, with occasional errors that do not impede understanding. Structure is logical, and \textcolor{rpTypeEvidence}{evidence} is used to support claims, though analysis may be superficial or uneven. The response demonstrates solid comprehension of the prompt, engages with the text meaningfully, and maintains an appropriate academic tone. While insight may be limited or predictable, the response avoids major factual inaccuracies, repetitive phrasing, or structural confusion. Language issues are minor and do not distract from the argument's clarity.\par \par Score 5: The response is well-developed, clearly organized, and effectively communicates a nuanced understanding of the topic. Language is precise and mostly error-free, with sophisticated sentence structure and appropriate academic tone. \textcolor{rpTypeEvidence}{Evidence} is well-chosen and thoughtfully integrated. Analysis goes beyond description to offer insight, connection, or critical evaluation. The response demonstrates original thinking, consistent depth, and strong command of rhetorical conventions. Minor stylistic imperfections may exist but do not detract from overall effectiveness.\par 
\end{tcolorbox}
\end{minipage}
\caption{Pattern-highlighted rubric comparison (ASAP2, qwen\_qwen3-next-80b-a3b-instruct, base\_simplest\_True\_train100\_iteration5\_top3\_bs4-8-12\_mc4). Matched spans are color-coded by regex pattern. Color types: \textcolor{rpTypeEvidence}{\textbf{Evidence Handling}} (examples, repetition, and caps); \textcolor{rpTypeWriting}{\textbf{Writing Quality}} (organization and grammar/mechanics).}
\label{fig:rubric_pattern_ASAP2_qwen_qwen3_next_80b_a3b_instruct_base_simplest_True_train100_iteration5_top3_bs4_8_12_mc4}
\end{figure*}


\colorlet{rpTypeRule}{red!80!black}
\colorlet{rpTypeEvidence}{blue!80!black}
\colorlet{rpTypeWriting}{teal!80!black}
\begin{figure*}[t]
\centering
\begin{tcolorbox}[colback=white,colframe=black!25,title=Pattern Type Guide,fonttitle=\bfseries\small,fontupper=\scriptsize,boxsep=1pt,left=2pt,right=2pt,top=2pt,bottom=2pt]
\textcolor{rpTypeRule}{\textbf{Rule Structure}}: Explicit decision logic for scoring: conditional branches, boundary tie-breakers, stepwise workflows, and numeric thresholds.\par \textcolor{rpTypeEvidence}{\textbf{Evidence Handling}}: How evidence is validated and counted: specific-example requirements, repetition/non-double-count rules, and cap rules for weak evidence.\par \textcolor{rpTypeWriting}{\textbf{Writing Quality}}: Language-quality criteria affecting score bands: organization/coherence/transition quality and grammar/mechanics severity.
\end{tcolorbox}
\vspace{1mm}
\begin{tcolorbox}[colback=white,colframe=black!25,title=Detailed Pattern Notes,fonttitle=\bfseries\small,fontupper=\scriptsize,boxsep=1pt,left=2pt,right=2pt,top=2pt,bottom=2pt]
\textcolor{rpTypeRule}{\textbf{Rule Structure}}:\par \quad \textcolor{rpTypeRule}{\textbf{Conditional Gating}} [n=9] Captures explicit condition-based rules that switch decisions only when a stated condition is met. Typical cues: if, when.\par \quad \textcolor{rpTypeRule}{\textbf{Boundary / Tie-Break Guidance}} [n=1] Marks criteria used to resolve borderline cases between adjacent score bands (e.g., 4 vs 5). Typical cues: tie-break, borderline, boundary, threshold, 4 vs 5.\par \quad \textcolor{rpTypeRule}{\textbf{Stepwise Rating Workflow}} [n=5] Detects ordered procedures and checklists that standardize how raters walk through scoring decisions. Typical cues: step, checklist, workflow, procedure, first/second/third.\par \quad \textcolor{rpTypeRule}{\textbf{Quantitative Threshold}} [n=2] Marks numeric cutoffs used for consistent decisions (minimum/maximum counts, percentages, explicit count thresholds). Typical cues: at least, at most, <=, >=, \%, N reasons/examples/sentences/words.\par \textcolor{rpTypeEvidence}{\textbf{Evidence Handling}}:\par \quad \textcolor{rpTypeEvidence}{\textbf{Specific Evidence Requirement}} [n=17] Highlights demands for concrete examples and explicit evidence links instead of generic assertions. Typical cues: for example, e.g., specific example, illustration, anecdote, evidence.\par \quad \textcolor{rpTypeEvidence}{\textbf{Repetition Non-Count Rule}} [n=1] Captures rules that treat repetition/restatement as non-distinct support and prevent double-counting. Typical cues: repetition, restatement, double-count, do not double-count.\par \textcolor{rpTypeWriting}{\textbf{Writing Quality}}:\par \quad \textcolor{rpTypeWriting}{\textbf{Organization / Coherence Signal}} [n=5] Detects explicit references to discourse structure and logical flow as scoring criteria. Typical cues: organization, coherence, logical flow, transition.\par \quad \textcolor{rpTypeWriting}{\textbf{Grammar / Mechanics Signal}} [n=3] Detects references to language-form quality, especially grammar, spelling, punctuation, and mechanics. Typical cues: grammar, mechanics, spelling, punctuation.
\end{tcolorbox}
\vspace{1mm}
\begin{tcolorbox}[colback=white,colframe=black!25,title=Optimized Rubric (Pattern-Highlighted),fonttitle=\bfseries\small,fontupper=\scriptsize]
\ttfamily
Note: \textcolor{rpTypeRule}{\textbf{If}} a response provides any discernible reasons or specific details, even with severe mechanical errors, move to \textcolor{rpTypeRule}{\textbf{at least}} a Score Point 2.\par ... [3 lines omitted] ...\par - Lists items (\textcolor{rpTypeEvidence}{\textbf{e.g.}}, "play games, go on Facebook") without any explanation of why they are good or how they work.\par - Shows little or no \textcolor{rpTypeEvidence}{\textbf{evidence}} of structured \textcolor{rpTypeWriting}{\textbf{organization}}.\par ... [5 lines omitted] ...\par - Offers more general than specific details. Even \textcolor{rpTypeRule}{\textbf{if}} proper nouns or specific platforms (\textcolor{rpTypeEvidence}{\textbf{e.g.}}, "Facebook," "Colombia") are mentioned, \textcolor{rpTypeRule}{\textbf{if}} the sentence following them provides no further development of the idea, the response remains a 3.\par - Basic "listing" \textcolor{rpTypeWriting}{\textbf{organization}} (\textcolor{rpTypeEvidence}{\textbf{e.g.}}, "\textcolor{rpTypeRule}{\textbf{First}}, \textcolor{rpTypeRule}{\textbf{Second}}, Finally") with little internal development.\par ... [5 lines omitted] ...\par - Support often feels formulaic or relies heavily on the ideas/phrasing provided in the prompt (\textcolor{rpTypeEvidence}{\textbf{e.g.}}, simply expanding on "exercise" and "nature" without unique hypothetical scenarios).\par - Contains \textcolor{rpTypeEvidence}{\textbf{anecdote}}s or examples that are present but lack multi-layered development or "\textcolor{rpTypeWriting}{\textbf{mechanics}}" (\textcolor{rpTypeEvidence}{\textbf{e.g.}}, mentioning a surgery simulation or a personal fall but not exploring the broader implications).\par - Shows satisfactory \textcolor{rpTypeWriting}{\textbf{organization}} with a clear intro, body, and conclusion.\par - May contain significant mechanical, \textcolor{rpTypeWriting}{\textbf{spelling}}, or syntax errors; however, the argument is logically coherent.\par - Language is functional but lacks fluency. Even \textcolor{rpTypeRule}{\textbf{if}} the essay introduces an original \textcolor{rpTypeRule}{\textbf{third}} point (like safety or jobs), \textcolor{rpTypeRule}{\textbf{if}} the prose is clumsy and the elaboration is limited to 2-\textcolor{rpTypeRule}{\textbf{3 sentences}} per point, it should remain a 4.\par ... [2 lines omitted] ...\par - Has well-elaborated reasons with specific, original details that explore "\textcolor{rpTypeWriting}{\textbf{mechanics}}" (\textcolor{rpTypeEvidence}{\textbf{e.g.}}, describing the physical sensation of "getting fat," or the specific way a webcam creates "human interaction" compared to a phone).\par - Moves beyond the prompt's suggestions by significantly expanding on how technology functions in personal or professional lives (\textcolor{rpTypeEvidence}{\textbf{e.g.}}, how a nurse uses videos for \textcolor{rpTypeRule}{\textbf{procedure}}s or the emotional impact of staying in touch with a friend who moved).\par ... [1 lines omitted] ...\par - While it may use a formulaic structure (\textcolor{rpTypeEvidence}{\textbf{e.g.}}, "My \textcolor{rpTypeRule}{\textbf{first}} reason... My last reason..."), the depth of the elaboration and the use of original scenarios justify the 5. Depth of content overrides repetitive \textcolor{rpTypeWriting}{\textbf{transition}}s at this level.\par ... [3 lines omitted] ...\par - Fully elaborated reasons with numerous specific, concrete details or "multi-layered" \textcolor{rpTypeEvidence}{\textbf{anecdote}}s.\par - Goes significantly beyond explaining "why" to explore the "implications" and "consequences" of the position (\textcolor{rpTypeEvidence}{\textbf{e.g.}}, connecting computer use to global warming via power plants, or the specific danger of losing survival skills like lighting matches during a natural disaster).\par - Exhibits sophisticated \textcolor{rpTypeWriting}{\textbf{organization}} or creative framing (\textcolor{rpTypeEvidence}{\textbf{e.g.}}, using a powerful opening quote or a rhetorical "hook").\par ... [6 lines omitted] ...\par - A \textcolor{rpTypeRule}{\textbf{4 vs. 5}} Distinction: \textcolor{rpTypeRule}{\textbf{If}} an essay provides \textcolor{rpTypeEvidence}{\textbf{specific example}}s (like surgery simulations or medical stats) but fails to explain the *ripple effects* or *human impact* of those examples, it is a 4. \textcolor{rpTypeRule}{\textbf{If}} it explores the "how" and "why" behind the examples (\textcolor{rpTypeEvidence}{\textbf{e.g.}}, the emotional relief of a webcam or the specific process of getting fit), it is a 5.\par - A 5 vs. 6 Distinction: A 6 must explore broader societal or philosophical "implications" (\textcolor{rpTypeEvidence}{\textbf{e.g.}}, global warming, natural disaster survival, the future of the environment) or use highly creative framing and sophisticated voice.\par - Heavy \textcolor{rpTypeEvidence}{\textbf{repetition}} in sentence structure (\textcolor{rpTypeEvidence}{\textbf{e.g.}}, "One reason is... Another reason is...") caps an essay at 4 ONLY \textcolor{rpTypeRule}{\textbf{IF}} the content is also basic/formulaic. \textcolor{rpTypeRule}{\textbf{If}} the content within those paragraphs is deep and explores implications, it should be moved to a 5.
\end{tcolorbox}
\caption{Pattern-focused view of the optimized rubric (asap\_1, google\_gemini-3-flash-preview, base\_expert\_True\_train100\_iteration5\_top3\_bs4-8-12\_mc4). Colored bold spans indicate regex-matched rubric cues. Color types: \textcolor{rpTypeRule}{\textbf{Rule Structure}} (Explicit decision logic for scoring: conditional branches, boundary tie-breakers, stepwise workflows, and numeric thresholds.); \textcolor{rpTypeEvidence}{\textbf{Evidence Handling}} (How evidence is validated and counted: specific-example requirements, repetition/non-double-count rules, and cap rules for weak evidence.); \textcolor{rpTypeWriting}{\textbf{Writing Quality}} (Language-quality criteria affecting score bands: organization/coherence/transition quality and grammar/mechanics severity.). Matched pattern categories: Conditional Gating (n=9); Boundary / Tie-Break Guidance (n=1); Stepwise Rating Workflow (n=5); Specific Evidence Requirement (n=17); Organization / Coherence Signal (n=5); Grammar / Mechanics Signal (n=3); Repetition Non-Count Rule (n=1); Quantitative Threshold (n=2).}
\label{fig:rubric_pattern_asap_1_google_gemini_3_flash_preview_base_expert_True_train100_iteration5_top3_bs4_8_12_mc4}
\end{figure*}


\colorlet{rpTypeRule}{red!80!black}
\colorlet{rpTypeEvidence}{blue!80!black}
\colorlet{rpTypeWriting}{teal!80!black}
\begin{figure*}[t]
\centering
\begin{tcolorbox}[colback=white,colframe=black!25,title=Pattern Type Guide,fonttitle=\bfseries\small,fontupper=\scriptsize,boxsep=1pt,left=2pt,right=2pt,top=2pt,bottom=2pt]
\textcolor{rpTypeRule}{\textbf{Rule Structure}}: Explicit decision logic for scoring: conditional branches, boundary tie-breakers, stepwise workflows, and numeric thresholds.\par \textcolor{rpTypeEvidence}{\textbf{Evidence Handling}}: How evidence is validated and counted: specific-example requirements, repetition/non-double-count rules, and cap rules for weak evidence.\par \textcolor{rpTypeWriting}{\textbf{Writing Quality}}: Language-quality criteria affecting score bands: organization/coherence/transition quality and grammar/mechanics severity.
\end{tcolorbox}
\vspace{1mm}
\begin{tcolorbox}[colback=white,colframe=black!25,title=Detailed Pattern Notes,fonttitle=\bfseries\small,fontupper=\scriptsize,boxsep=1pt,left=2pt,right=2pt,top=2pt,bottom=2pt]
\textcolor{rpTypeRule}{\textbf{Rule Structure}}:\par \quad \textcolor{rpTypeRule}{\textbf{Conditional Gating}} [n=5] Captures explicit condition-based rules that switch decisions only when a stated condition is met. Typical cues: if, when.\par \quad \textcolor{rpTypeRule}{\textbf{Stepwise Rating Workflow}} [n=1] Detects ordered procedures and checklists that standardize how raters walk through scoring decisions. Typical cues: step, checklist, workflow, procedure, first/second/third.\par \textcolor{rpTypeEvidence}{\textbf{Evidence Handling}}:\par \quad \textcolor{rpTypeEvidence}{\textbf{Specific Evidence Requirement}} [n=9] Highlights demands for concrete examples and explicit evidence links instead of generic assertions. Typical cues: for example, e.g., specific example, illustration, anecdote, evidence.\par \quad \textcolor{rpTypeEvidence}{\textbf{Off-Topic / Summary Cap}} [n=1] Identifies cap rules that restrict scores when responses are off-topic, irrelevant, or dominated by summary-only content. Typical cues: off-topic, irrelevant, digression, summary-only, cap.\par \textcolor{rpTypeWriting}{\textbf{Writing Quality}}:\par \quad \textcolor{rpTypeWriting}{\textbf{Organization / Coherence Signal}} [n=3] Detects explicit references to discourse structure and logical flow as scoring criteria. Typical cues: organization, coherence, logical flow, transition.\par \quad \textcolor{rpTypeWriting}{\textbf{Grammar / Mechanics Signal}} [n=3] Detects references to language-form quality, especially grammar, spelling, punctuation, and mechanics. Typical cues: grammar, mechanics, spelling, punctuation.
\end{tcolorbox}
\vspace{1mm}
\begin{tcolorbox}[colback=white,colframe=black!25,title=Optimized Rubric (Pattern-Highlighted),fonttitle=\bfseries\small,fontupper=\scriptsize]
\ttfamily
Score the essay on a scale of 1 to 6 based on the following criteria. The primary determinants of the score are the writer's ability to address the prompt with original thought, the depth/thickness of supporting details, and the clarity of the \textcolor{rpTypeWriting}{\textbf{organization}}al structure.\par ... [1 lines omitted] ...\par CRITICAL SCORING PRINCIPLE: Surface-level errors in \textcolor{rpTypeWriting}{\textbf{spelling}}, \textcolor{rpTypeWriting}{\textbf{grammar}}, syntax, and \textcolor{rpTypeWriting}{\textbf{punctuation}}-even \textcolor{rpTypeRule}{\textbf{if}} frequent, severe, or making the text difficult to read-should NOT prevent a high score (4, 5, or 6) \textcolor{rpTypeRule}{\textbf{if}} the logic is discernible and the content is developed. Focus on the "voice" and the richness of the \textcolor{rpTypeEvidence}{\textbf{evidence}} provided rather than mechanical accuracy.\par ... [1 lines omitted] ...\par - 6 (Superior): The response provides a sophisticated, persuasive argument that goes beyond standard responses by offering unique perspectives or creative logic (\textcolor{rpTypeEvidence}{\textbf{e.g.}}, reframing a common disadvantage into a situational advantage or providing vivid, sensory imagery). It is highly organized with a clear, rhythmic progression of ideas. It is distinguished by "authoritative" voice and layered, original supporting details that feel authentic/personal rather than just meeting a length requirement.\par ... [1 lines omitted] ...\par - 5 (Strong): The response takes a clear stance and is well-developed with "thick" body paragraphs. It is distinguished from a 4 by its use of varied \textcolor{rpTypeWriting}{\textbf{transition}}s and nuanced reasoning (\textcolor{rpTypeEvidence}{\textbf{e.g.}}, comparing digital experiences to physical reality or discussing long-term psychological impacts). While it may rely on more common arguments, it supports them with extensive personal \textcolor{rpTypeEvidence}{\textbf{anecdote}}s or specific, varied examples that show a depth of reflection beyond a simple list.\par ... [1 lines omitted] ...\par - 4 (Competent): The response is the baseline for a "successful" essay. It addresses the prompt with a clear opinion and a discernible structure. To earn a 4, the writer MUST provide some original expansion or \textcolor{rpTypeEvidence}{\textbf{specific example}}s (\textcolor{rpTypeEvidence}{\textbf{e.g.}}, a specific personal story, naming a specific website, or a unique hypothetical scenario). Even \textcolor{rpTypeRule}{\textbf{if}} the language is "broken," \textcolor{rpTypeRule}{\textbf{if}} the writer moves beyond simply repeating/listing the prompt's ideas and provides a clear "why" or "how" for their points, it earns a 4.\par ... [1 lines omitted] ...\par - 3 (Developing): The response is limited and feels "thin." While it may have an introduction, body, and conclusion, it relies heavily on repeating the prompt's own language or listing common reasons (\textcolor{rpTypeEvidence}{\textbf{e.g.}}, "you can talk to friends," "it helps with homework") without providing original details or unique \textcolor{rpTypeEvidence}{\textbf{evidence}}. A 3 often feels like a \textcolor{rpTypeRule}{\textbf{checklist}} of the prompt's suggestions. \textcolor{rpTypeRule}{\textbf{If}} the essay is long but repetitive or lacks specific, original \textcolor{rpTypeEvidence}{\textbf{anecdote}}s, it stays at a 3.\par ... [1 lines omitted] ...\par - 2 (Limited): The response shows minimal control of language and \textcolor{rpTypeWriting}{\textbf{organization}}. Ideas are thin, highly fragmented, or consist of only a few repetitive sentences. It fails to build a coherent argument. It may be very short or comprise a list of disjointed thoughts that barely move beyond the prompt's own words.\par ... [1 lines omitted] ...\par - 1 (Inadequate): The response is \textcolor{rpTypeEvidence}{\textbf{off-topic}}, too brief to evaluate, or largely unintelligible due to a total lack of language control that prevents any logic from emerging.
\end{tcolorbox}
\caption{Pattern-focused view of the optimized rubric (asap\_1, google\_gemini-3-flash-preview, base\_simplest\_True\_train100\_iteration5\_top3\_bs4-8-12\_mc4). Colored bold spans indicate regex-matched rubric cues. Color types: \textcolor{rpTypeRule}{\textbf{Rule Structure}} (Explicit decision logic for scoring: conditional branches, boundary tie-breakers, stepwise workflows, and numeric thresholds.); \textcolor{rpTypeEvidence}{\textbf{Evidence Handling}} (How evidence is validated and counted: specific-example requirements, repetition/non-double-count rules, and cap rules for weak evidence.); \textcolor{rpTypeWriting}{\textbf{Writing Quality}} (Language-quality criteria affecting score bands: organization/coherence/transition quality and grammar/mechanics severity.). Matched pattern categories: Conditional Gating (n=5); Stepwise Rating Workflow (n=1); Specific Evidence Requirement (n=9); Off-Topic / Summary Cap (n=1); Organization / Coherence Signal (n=3); Grammar / Mechanics Signal (n=3).}
\label{fig:rubric_pattern_asap_1_google_gemini_3_flash_preview_base_simplest_True_train100_iteration5_top3_bs4_8_12_mc4}
\end{figure*}


\colorlet{rpTypeRule}{red!80!black}
\colorlet{rpTypeEvidence}{blue!80!black}
\colorlet{rpTypeWriting}{teal!80!black}
\begin{figure*}[t]
\centering
\begin{tcolorbox}[colback=white,colframe=black!25,title=Pattern Legend,fonttitle=\bfseries\small,fontupper=\scriptsize,boxsep=1pt,left=2pt,right=2pt,top=2pt,bottom=2pt]
\textcolor{rpTypeRule}{\textbf{Rule Structure}} (if/threshold/stepwise guidance) \quad \textcolor{rpTypeEvidence}{\textbf{Evidence Handling}} (examples, repetition, and caps) \quad \textcolor{rpTypeWriting}{\textbf{Writing Quality}} (organization and grammar/mechanics)
\end{tcolorbox}
\vspace{2mm}
\begin{minipage}[t]{0.485\textwidth}
\begin{tcolorbox}[colback=white,colframe=black!25,title=Initial Rubric,fonttitle=\bfseries\small,fontupper=\scriptsize,breakable]
\ttfamily
- Contains only general reasons with unelaborated and/or list-like details.\par - Shows little or no \textcolor{rpTypeEvidence}{evidence} of \textcolor{rpTypeWriting}{organization}.\par - May be awkward and confused or simplistic.\par ... [3 lines omitted] ...\par - Has reasons with minimal elaboration and more general than specific details.\par - Shows some \textcolor{rpTypeWriting}{organization}.\par - May be awkward in parts with few \textcolor{rpTypeWriting}{transition}s.\par - Shows some awareness of audience.\par ... [2 lines omitted] ...\par - Has adequately elaborated reasons with a mix of general and specific details.\par - Shows satisfactory \textcolor{rpTypeWriting}{organization}.\par - May be somewhat fluent with some \textcolor{rpTypeWriting}{transition}al language.\par - Shows adequate awareness of audience.\par ... [2 lines omitted] ...\par - Has moderately well elaborated reasons with mostly specific details.\par - Exhibits generally strong \textcolor{rpTypeWriting}{organization}.\par - May be moderately fluent with \textcolor{rpTypeWriting}{transition}al language throughout.\par - May show a consistent awareness of audience.\par ... [2 lines omitted] ...\par - Has fully elaborated reasons with specific details.\par - Exhibits strong \textcolor{rpTypeWriting}{organization}.\par - Is fluent and uses sophisticated \textcolor{rpTypeWriting}{transition}al language.\par - May show a heightened awareness of audience.\par ... [1 lines omitted] ...\par Note: \par I have made an effort to remove personally identifying information from the essays using the Named Entity Recognizer (NER). The relevant entities are identified in the text and then replaced with a string such as "PERSON", "\textcolor{rpTypeWriting}{ORGANIZATION}", "LOCATION", "DATE", "TIME", "MONEY", "PERCENT", "CAPS" (any capitalized word) and "NUM" (any digits). Please do not penalize the essay because of the anonymizations.
\end{tcolorbox}
\end{minipage}
\hfill
\begin{minipage}[t]{0.485\textwidth}
\begin{tcolorbox}[colback=white,colframe=black!25,title=Optimized Rubric,fonttitle=\bfseries\small,fontupper=\scriptsize,breakable]
\ttfamily
General note about anonymization:\par - Do NOT deduct points for named-entity placeholders (PERSON, LOCATION, NUM, PERCENT, etc.). Treat them as neutral substitutions for real details; evaluate the presence, clarity, and specificity of ideas rather than the literal labels. Count placeholders as specific details \textcolor{rpTypeRule}{when} the writer clearly intends a concrete fact, example, or \textcolor{rpTypeEvidence}{anecdote}.\par \par ... [2 lines omitted] ...\par 2. Number and specificity of supporting reasons/examples (count distinct reasons and whether each has elaboration).\par 3. \textcolor{rpTypeWriting}{Organization}/\textcolor{rpTypeWriting}{coherence} (intro-body-conclusion, paragraphing, \textcolor{rpTypeWriting}{transition}s).\par 4. Development depth and insight (analysis, counterargument).\par ... [5 lines omitted] ...\par   b. Minimal elaboration (general explanation, vague example) -> "min-elab".\par   c. Specific elaboration/example or clear personal \textcolor{rpTypeEvidence}{anecdote}/data -> "specific".\par - Treat personal \textcolor{rpTypeEvidence}{anecdote}s and clearly-intended placeholders-as-\textcolor{rpTypeEvidence}{evidence} as valid "specific".\par - \textcolor{rpTypeEvidence}{Do not double-count} repeated \textcolor{rpTypeEvidence}{restatement}s of the same reason. Different examples that support the same reason count as strengthening that one reason (do not convert them into separate reasons unless they support a genuinely different claim).\par - \textcolor{rpTypeRule}{When} in doubt about whether two supports are distinct reasons or sub-points of the same reason, prefer to count them as the same reason unless they address different effects, audiences, or mechanisms.\par \par ... [1 lines omitted] ...\par - Position: May state a position or may be off-task; little or no purposeful response to the prompt.\par - Development: Few or no reasons; \textcolor{rpTypeRule}{if} present they are list-only or \textcolor{rpTypeEvidence}{irrelevant}. No meaningful examples or elaboration.\par - \textcolor{rpTypeWriting}{Organization} \& \textcolor{rpTypeWriting}{coherence}: Fragmented, chaotic, or extremely hard to follow; may be one or two disjointed sentences.\par - Language: \textcolor{rpTypeWriting}{Grammar} and usage may prevent comprehension.\par - Use \textcolor{rpTypeRule}{when} the essay essentially fails to form an argument or provide any supporting content.\par \par ... [2 lines omitted] ...\par - Development: Mostly list-only reasons or 1-2 general reasons with no or minimal elaboration (classified as "list-only" or mostly "min-elab"). Distinctness of reasons is low.\par - \textcolor{rpTypeWriting}{Organization} \& \textcolor{rpTypeWriting}{coherence}: Little or no logical \textcolor{rpTypeWriting}{organization}; \textcolor{rpTypeWriting}{transition}s absent and sequencing is weak.\par - Language: Frequent errors that sometimes impede comprehension.\par - Use \textcolor{rpTypeRule}{when} the response is more than a sentence or two but lacks development, explanation, and clear structure.\par \par Score Point 3 - "Minimally developed / some \textcolor{rpTypeWriting}{organization}"\par - Position: Takes a position that is generally clear.\par - Development: Provides basic reasons with minimal elaboration. Typical patterns:\par   - 1-2 distinct reasons where \textcolor{rpTypeRule}{at least} one has min-elab; or\par   - 2+ reasons but most are list-only or repetitive \textcolor{rpTypeEvidence}{restatement}s.\par   - May include one brief \textcolor{rpTypeEvidence}{specific example} or \textcolor{rpTypeEvidence}{anecdote}, but it is not well developed or persuasive.\par - \textcolor{rpTypeWriting}{Organization} \& \textcolor{rpTypeWriting}{coherence}: Some sense of \textcolor{rpTypeWriting}{organization} (intro, body, conclusion or paragraphing) though progression may be weak; limited \textcolor{rpTypeWriting}{transition}s.\par - Language: Errors are frequent but overall meaning is still understandable.\par - Use \textcolor{rpTypeRule}{when} the essay demonstrates a clear stance and rudimentary structure with limited support and few specific, distinct examples.\par \par ... [1 lines omitted] ...\par - Position: Clearly stated position throughout.\par - Development: Offers adequately elaborated reasons with a mix of general and specific details/examples. Typical \textcolor{rpTypeRule}{threshold}:\par   - \textcolor{rpTypeRule}{At least} 2 distinct reasons each with \textcolor{rpTypeRule}{at least} min-elab AND \textcolor{rpTypeRule}{at least} one clear \textcolor{rpTypeEvidence}{specific example} supporting any one of the reasons; OR\par   - 2-3 distinct reasons with mostly min-elab development and \textcolor{rpTypeRule}{at least} one specific that meaningfully strengthens the argument.\par - \textcolor{rpTypeWriting}{Organization} \& \textcolor{rpTypeWriting}{coherence}: Satisfactory \textcolor{rpTypeWriting}{organization} with clear paragraphing and some \textcolor{rpTypeWriting}{transition}s; readers can follow the argument.\par - Language: Noticeable errors may be present but do not substantially obscure meaning.\par - \textcolor{rpTypeRule}{Tie-break}er to promote consistency: \textcolor{rpTypeRule}{If} an essay has 2+ distinct reasons each with min-elab and \textcolor{rpTypeRule}{at least} one clearly functioning specific (including placeholders or brief \textcolor{rpTypeEvidence}{anecdote}s), prefer 4 over 3-even \textcolor{rpTypeRule}{when} language is weak.\par \par ... [2 lines omitted] ...\par - Development: Stronger elaboration than 4. Typical patterns (choose the rule that fits):\par   - 3+ distinct reasons with \textcolor{rpTypeRule}{at least} two being "specific" examples/\textcolor{rpTypeEvidence}{anecdote}s; OR\par   - 2 distinct reasons each with robust, specific elaboration and varied supporting details; OR\par   - 2+ reasons plus multiple concrete personal \textcolor{rpTypeEvidence}{anecdote}s or data points that combine to make the argument persuasive.\par - \textcolor{rpTypeWriting}{Organization} \& \textcolor{rpTypeWriting}{coherence}: Generally strong \textcolor{rpTypeWriting}{organization} and logical progression; effective paragraphing and \textcolor{rpTypeWriting}{transition}al language throughout.\par - Language: Generally fluent; errors present but not distracting.\par - Important constraint to reduce over-scoring: Do NOT award a 5 \textcolor{rpTypeRule}{if} the "specific" supports are repetitive \textcolor{rpTypeEvidence}{restatement}s or the same example reused to pad counts. Multiple \textcolor{rpTypeEvidence}{specific example}s must be distinct in content or context (different data points, different \textcolor{rpTypeEvidence}{anecdote}s, different illustrative scenarios).\par \par Score Point 6 - "Well-developed / thoughtful \& sophisticated"\par - Position: Takes a thoughtful, nuanced, and compelling position that goes beyond the obvious OR demonstrates exceptional development through multiple distinct specifics and strong \textcolor{rpTypeWriting}{organization}.\par - Development: One of the following should be true:\par   - The essay presents a nuanced/complex stance (acknowledges trade-offs, limits, or a qualified position) AND offers multiple specific, relevant examples and explanation; OR\par   - The essay contains 3+ distinct reasons each supported by clear, separate "specific" examples (not merely repeated \textcolor{rpTypeEvidence}{restatement}s), combined with cohesive \textcolor{rpTypeWriting}{organization} and some synthesis (linking reasons, explaining implications) even \textcolor{rpTypeRule}{if} explicit counterargument is brief or implicit.\par - \textcolor{rpTypeWriting}{Organization} \& \textcolor{rpTypeWriting}{coherence}: Strong, logical \textcolor{rpTypeWriting}{organization} with clear, effective \textcolor{rpTypeWriting}{transition}s and paragraphing; the argument builds cohesively.\par - Language: Fluent and controlled with varied sentence structure and vocabulary; minor errors may occur but do not interfere with meaning.\par - Audience awareness: Heightened-persuasive techniques tailored to the intended audience.\par - Use 6 \textcolor{rpTypeRule}{when} the essay demonstrates either clear analytic depth (counterargument, trade-offs, synthesis) OR very strong breadth and specificity of development (three distinct, well-supported reasons) plus clear \textcolor{rpTypeWriting}{organization}.\par \par ... [2 lines omitted] ...\par    - Use the three-tier label (list-only / min-elab / specific) per reason.\par    - 2+ adequately elaborated reasons (min-elab) with \textcolor{rpTypeRule}{at least} one specific -> lean 4.\par    - 3+ distinct reasons with 2+ \textcolor{rpTypeEvidence}{specific example}s/\textcolor{rpTypeEvidence}{anecdote}s -> lean 5; \textcolor{rpTypeRule}{if} those 3+ specifics are present and \textcolor{rpTypeWriting}{organization} is strong, consider 6 (see 6's alternate path).\par    - 2 strongly specific, well-connected reasons with varied \textcolor{rpTypeEvidence}{evidence} or a clear rebuttal -> can justify 5.\par    - Counterargument or analytical depth -> consider 6.\par 2. Favor development over surface fluency:\par    - Frequent mechanical errors should lower the fluency descriptor but should not automatically move a piece from 4/5/6 down to 2/3 \textcolor{rpTypeRule}{if} the essay contains clear \textcolor{rpTypeWriting}{organization} and multiple \textcolor{rpTypeEvidence}{specific example}s.\par    - However, severe breakdowns in \textcolor{rpTypeWriting}{grammar} that impede comprehension of key supports should lower the score.\par 3. Treat personal \textcolor{rpTypeEvidence}{anecdote}s and placeholder-based "studies" as valid \textcolor{rpTypeEvidence}{evidence}:\par    - \textcolor{rpTypeRule}{If} the writer provides a personal story or clearly intended study/example (even with placeholders), count it as a "specific" example for development-unless the placeholder is so vague that it does not function as \textcolor{rpTypeEvidence}{evidence}.\par 4. Distinguish \textcolor{rpTypeEvidence}{repetition} vs. distinct \textcolor{rpTypeEvidence}{evidence}:\par    - \textcolor{rpTypeEvidence}{Repetition} or rephrasing of the same example should not be counted as multiple specifics.\par    - Different contexts or different concrete examples (even \textcolor{rpTypeRule}{if} they support the same general reason) strengthen that reason but \textcolor{rpTypeRule}{do not count} as additional distinct reasons.\par    - To move from 4->5 using the "3+ reasons" route, ensure the \textcolor{rpTypeRule}{third} reason is independent (addresses a different effect or mechanism) and has a \textcolor{rpTypeEvidence}{specific example}.\par 5. Use \textcolor{rpTypeWriting}{organization} to resolve close calls:\par    - Clear intro/body/conclusion and logical paragraphing can raise a \textcolor{rpTypeRule}{borderline} 3 to a 4 even \textcolor{rpTypeRule}{when} details are modest.\par    - Conversely, poor \textcolor{rpTypeWriting}{organization} can keep a richly supported response from reaching 6 \textcolor{rpTypeRule}{if} the argument fails to cohere.\par 6. Holistic \textcolor{rpTypeRule}{tie-break}ers (final arbitration):\par    - \textcolor{rpTypeRule}{If} features point to different scores, prioritize in order: (a) specificity \& number of supporting details (b) clarity of position (c) \textcolor{rpTypeWriting}{organization}/cohesion.\par    - \textcolor{rpTypeRule}{When} in doubt between 4 and 5, count \textcolor{rpTypeEvidence}{specific example}s carefully-require distinctiveness and substantive support. \textcolor{rpTypeRule}{If} you count 2 full- strength specifics (distinct content) and either a \textcolor{rpTypeRule}{third} reason or \textcolor{rpTypeRule}{second} reason with strong specifics, prefer 5.\par    - \textcolor{rpTypeRule}{When} in doubt between 5 and 6, require either explicit nuance/counterargument OR 3+ distinct reasons each with clear specifics AND strong \textcolor{rpTypeWriting}{organization} for 6.\par 7. Examples of \textcolor{rpTypeRule}{boundary} judgments (updated heuristics):\par    - Several distinct reasons but only list-like, no examples -> Score 2.\par    - Clear stance, 1-\textcolor{rpTypeRule}{2 reasons} with brief examples or \textcolor{rpTypeEvidence}{anecdote}s; some \textcolor{rpTypeWriting}{organization} -> Score 3.\par    - Clear stance, 2+ distinct reasons each with \textcolor{rpTypeRule}{at least} some elaboration and \textcolor{rpTypeRule}{at least} one \textcolor{rpTypeEvidence}{specific example} -> Score 4.\par    - Clear stance, either (a) 3+ reasons with mostly specific, relevant examples (distinct and non-repetitive) OR (b) \textcolor{rpTypeRule}{2 reasons} each with substantial specific elaboration and persuasive flow -> Score 5.\par    - Nuanced stance, thorough analysis, counterargument, or 3+ distinct, well-supported reasons with cohesive synthesis -> Score 6.\par \par Practical scoring \textcolor{rpTypeRule}{checklist} (use \textcolor{rpTypeRule}{when} assigning a score):\par - Is the position clear and consistent? (Yes -> continue; No -> lean 1-2)\par ... [2 lines omitted] ...\par - Count specifics: how many distinct "specific" supports? (Are they different in content/context or repetitive?)\par - Evaluate \textcolor{rpTypeWriting}{organization}: clear paragraphs and \textcolor{rpTypeWriting}{transition}s? (Yes raises \textcolor{rpTypeRule}{borderline} 3->4)\par - Is there counterargument, synthesis, or analytic depth? (Yes -> consider 6)\par - Check language: are errors limiting comprehension? (\textcolor{rpTypeRule}{If} comprehension fails, lower to 1-2; otherwise, do not heavily penalize development)\par - Apply \textcolor{rpTypeRule}{tie-break}er rules above (prioritize specificity, then clarity, then \textcolor{rpTypeWriting}{organization}).\par \par Rationale summary for raters:\par - Increase scores \textcolor{rpTypeRule}{when} multiple distinct, \textcolor{rpTypeEvidence}{specific example}s or \textcolor{rpTypeEvidence}{anecdote}s are present even \textcolor{rpTypeRule}{if} the essay is marred by grammatical errors or placeholders.\par - Do not let surface-level \textcolor{rpTypeEvidence}{repetition} or weak phrasing obscure counting of distinct supports-explicitly count and label supports.\par - Reserve the top score (6) for essays that either add analytical depth (counterargument, synthesis) OR supply broad, distinct, and specific development across 3+ independent reasons with cohesive \textcolor{rpTypeWriting}{organization}.\par - Be stricter about distinctiveness of specifics \textcolor{rpTypeRule}{when} moving between 4, 5, and 6-require different content/context for each counted specific support.
\end{tcolorbox}
\end{minipage}
\caption{Pattern-highlighted rubric comparison (asap\_1, openai\_gpt-5-mini, base\_expert\_True\_train100\_iteration5\_top3\_bs4-8-12\_mc4). Matched spans are color-coded by regex pattern. Color types: \textcolor{rpTypeRule}{\textbf{Rule Structure}} (if/threshold/stepwise guidance); \textcolor{rpTypeEvidence}{\textbf{Evidence Handling}} (examples, repetition, and caps); \textcolor{rpTypeWriting}{\textbf{Writing Quality}} (organization and grammar/mechanics).}
\label{fig:rubric_pattern_asap_1_openai_gpt_5_mini_base_expert_True_train100_iteration5_top3_bs4_8_12_mc4}
\end{figure*}


\colorlet{rpTypeRule}{red!80!black}
\colorlet{rpTypeEvidence}{blue!80!black}
\colorlet{rpTypeWriting}{teal!80!black}
\begin{figure*}[t]
\centering
\begin{tcolorbox}[colback=white,colframe=black!25,title=Pattern Legend,fonttitle=\bfseries\small,fontupper=\scriptsize,boxsep=1pt,left=2pt,right=2pt,top=2pt,bottom=2pt]
\textcolor{rpTypeRule}{\textbf{Rule Structure}} (if/threshold/stepwise guidance) \quad \textcolor{rpTypeEvidence}{\textbf{Evidence Handling}} (examples, repetition, and caps) \quad \textcolor{rpTypeWriting}{\textbf{Writing Quality}} (organization and grammar/mechanics)
\end{tcolorbox}
\vspace{2mm}
\begin{minipage}[t]{0.485\textwidth}
\begin{tcolorbox}[colback=white,colframe=black!25,title=Initial Rubric,fonttitle=\bfseries\small,fontupper=\scriptsize,breakable]
\ttfamily
Based on the response's content, rate the response on a scale of 1 to 6.
\end{tcolorbox}
\end{minipage}
\hfill
\begin{minipage}[t]{0.485\textwidth}
\begin{tcolorbox}[colback=white,colframe=black!25,title=Optimized Rubric,fonttitle=\bfseries\small,fontupper=\scriptsize,breakable]
\ttfamily
Overview (core priorities)\par Rate essays primarily on (1) clarity of position and direct relevance to the prompt, (2) development and support for that position (number and quality of distinct reasons, and the substantive quality of support for each), (3) \textcolor{rpTypeWriting}{organization}/\textcolor{rpTypeWriting}{coherence}, and (4) control of language (\textcolor{rpTypeWriting}{grammar}, vocabulary, \textcolor{rpTypeWriting}{mechanics}). Prioritize substantive content and meaningful development over surface errors, but require a higher standard of specificity and distinctness for the top bands.\par \par Key definitions and counting rules\par - Reason: a claim that supports the main position. Count distinct reasons only \textcolor{rpTypeRule}{when} they advance separate lines of argument (\textcolor{rpTypeEvidence}{do not double-count} \textcolor{rpTypeEvidence}{restatement}s or overlapping claims unless each has a distinct supporting point or example).\par - Developed reason: a reason counts as developed only \textcolor{rpTypeRule}{when} the writer provides supporting material that meaningfully advances the claim. Acceptable forms of development include:\par   - A concrete, \textcolor{rpTypeEvidence}{specific example} or brief relevant \textcolor{rpTypeEvidence}{anecdote} tied to the reason.\par   - A coherent logical explanation showing how/why the reason supports the claim.\par   - Specific factual detail (dates, names, clear contextualized statistics) that is explained or connected to the claim.\par - \textcolor{rpTypeRule}{Do NOT count} as development: generic assertions ("they help people learn"), vague generalities, unsupported numeric/statistical claims with no context (\textcolor{rpTypeEvidence}{e.g.}, "\textcolor{rpTypeRule}{\%} more active" without explanation of relevance), strings of placeholders or tokens with no clarifying context, or long rambling passages that never link back to the reason.\par \par Firm heuristics (primary score \textcolor{rpTypeRule}{tiebreak}ers)\par - Clear position + \textcolor{rpTypeRule}{at least} three genuinely developed, distinct reasons approx score 5 (see higher-band rules for 6).\par - Clear position + two genuinely developed, distinct reasons approx score 4.\par - Clear position + one genuinely developed reason (and/or several undeveloped or repetitive assertions) approx score 3 (or 2 \textcolor{rpTypeRule}{when} very short or severely unclear).\par - Very short responses (\textcolor{rpTypeEvidence}{e.g.}, only a sentence or two, or fewer than \textasciitilde{}\textcolor{rpTypeRule}{50 words}) that state a position but offer little/no development should generally be rated 2, not 3.\par \par ... [1 lines omitted] ...\par \par 1) Stronger quality \textcolor{rpTypeRule}{threshold} for "developed"\par - Require that each developed reason include \textcolor{rpTypeRule}{at least} one of: a specific concrete example (even a short personal \textcolor{rpTypeEvidence}{anecdote}), a clear logical explanation, or a factual detail tied to relevance. Vague \textcolor{rpTypeEvidence}{illustration}s or mere mention of a category (\textcolor{rpTypeEvidence}{e.g.}, "games improve coordination") without any specific supporting detail should not count.\par - Placeholder tokens are permissible only \textcolor{rpTypeRule}{when} the surrounding text makes the nature and function of the example clear. \textcolor{rpTypeRule}{If} placeholders obscure whether a real example/explanation was provided, \textcolor{rpTypeRule}{do NOT count} that reason as developed.\par - Unsupported statistics count only \textcolor{rpTypeRule}{if} the writer links them to explanation or context that clarifies their meaning and relevance.\par \par 2) Handling of language errors vs. development (refined)\par - Do not downgrade a response below 4 solely for grammatical/mechanical errors \textcolor{rpTypeRule}{if} it clearly supplies two developed reasons; likewise do not drop below 5 \textcolor{rpTypeRule}{if} it clearly supplies three developed reasons and development is intelligible.\par - However, reduce leniency \textcolor{rpTypeRule}{when} surface errors combine with vague or generic support: frequent errors + only generic development should not result in 5.\par - Downgrade to 2 (or 1) where language or placeholder use materially obscures the content to the point you cannot reliably infer the writer's intended development.\par ... [1 lines omitted] ...\par 3) Partial development and counting leniency (balanced)\par - Count a partially developed reason \textcolor{rpTypeRule}{when} the essential support is intelligible (\textcolor{rpTypeEvidence}{e.g.}, brief \textcolor{rpTypeEvidence}{anecdote}, coherent one-sentence explanation), even \textcolor{rpTypeRule}{if} wording is garbled.\par - \textcolor{rpTypeRule}{Do NOT count} a long passage that names examples or statistics but never links them coherently to the reason.\par - \textcolor{rpTypeRule}{When} in doubt about \textcolor{rpTypeRule}{borderline} partial examples, prefer the lower adjacent score unless the development convincingly meets the standards above.\par \par 4) \textcolor{rpTypeEvidence}{Repetition}/overlap rule (strengthened)\par - \textcolor{rpTypeRule}{Do not count} repeated \textcolor{rpTypeEvidence}{restatement}s of the same underlying claim as multiple reasons. \textcolor{rpTypeRule}{When} two apparent reasons overlap substantially, count them as one unless the writer provides distinct supporting points or examples that clearly separate them.\par - \textcolor{rpTypeRule}{If} an essay lists three headings-like reasons but two are essentially the same claim reworded (\textcolor{rpTypeEvidence}{e.g.}, "communication" and "staying in touch" with no distinct support), count them as one for the three-reason \textcolor{rpTypeRule}{threshold}.\par \par 5) \textcolor{rpTypeEvidence}{Cap} rule and \textcolor{rpTypeRule}{when} to deny a "three-developed" count (new)\par - \textcolor{rpTypeEvidence}{Cap} at 4 (not 5) \textcolor{rpTypeRule}{when} the essay lists three reasons but the development for \textcolor{rpTypeRule}{at least} one is minimal, vague, repetitive, or primarily asserted without specific support.\par - \textcolor{rpTypeEvidence}{Cap} at 4 \textcolor{rpTypeRule}{when} the three "reasons" rely heavily on unsupported statistics, placeholders, or \textcolor{rpTypeEvidence}{repetition} rather than three distinct, meaningful supports.\par - \textcolor{rpTypeRule}{If} one of three reasons is clearly developed and the other two are only generic or mainly assertions, treat the response as a two-developed-reasons case (score 4) or even a one-developed case (score 3) depending on exact quality.\par \par ... [2 lines omitted] ...\par   - A strong, specific position responding directly to the prompt.\par   - Thorough, persuasive development: multiple (more than three is fine) distinct reasons with well‑explained, \textcolor{rpTypeEvidence}{specific example}s or details for each main reason. Development should be more than one brief sentence per reason; explanations should show logical connection and persuasive depth.\par   - Logical, effective \textcolor{rpTypeWriting}{organization} with fluent, precise language appropriate for top‑quality academic writing for the grade level.\par - Importantly: three intelligible but brief/surface-developed reasons with frequent placeholders or only one-sentence examples usually belong in band 5, not 6. Use 6 for responses that achieve high quality both in content depth and expression.\par ... [3 lines omitted] ...\par   - Clear position directly answers the prompt.\par   - \textcolor{rpTypeRule}{At least} three distinct reasons that are each genuinely developed (each reason includes a \textcolor{rpTypeEvidence}{specific example}, coherent explanation, or concrete detail). Development need not be exhaustive but must move beyond mere assertion.\par   - \textcolor{rpTypeWriting}{Organization} is clear; progression of ideas is coherent.\par   - \textcolor{rpTypeWriting}{Grammar}/\textcolor{rpTypeWriting}{spelling}/mechanical errors may be frequent but do not significantly obscure meaning. Moderate placeholders or garbling are acceptable \textcolor{rpTypeRule}{if} each reason's development remains intelligible.\par - Apply the "\textcolor{rpTypeEvidence}{cap}" rule: do not award 5 \textcolor{rpTypeRule}{if} any one of the three reasons lacks meaningful development (see rule 5).\par \par ... [1 lines omitted] ...\par - Score 4 = Competent/Effective:\par   - Clear position is stated and the writer provides \textcolor{rpTypeRule}{at least} two distinct reasons with some supporting detail or examples (i.e., two developed reasons).\par   - Development may be uneven, somewhat general, or partially undeveloped; examples may be general rather than highly specific.\par   - \textcolor{rpTypeWriting}{Organization} is evident though \textcolor{rpTypeWriting}{transition}s may be simple.\par   - Noticeable errors may distract but do not prevent understanding.\par ... [4 lines omitted] ...\par   - Reasons are simplistic, underdeveloped, or repetitive. There may be one developed reason plus other undeveloped assertions, or two undeveloped reasons.\par   - Examples (\textcolor{rpTypeRule}{if} any) are vague, generic, or only tangentially relevant.\par   - Frequent distracting errors and weak \textcolor{rpTypeWriting}{organization} reduce readability.\par   - Use 3 \textcolor{rpTypeRule}{when} there is some attempt at development but the essay does not meet the two-developed-reasons \textcolor{rpTypeRule}{threshold} for 4.\par \par ... [2 lines omitted] ...\par   - Position unclear, inconsistent, or only implied; supported by little or no meaningful development.\par   - Few or no meaningful reasons; examples missing, incoherent, or \textcolor{rpTypeEvidence}{irrelevant}. Very short responses usually fall here.\par   - Major \textcolor{rpTypeWriting}{organization} problems or severe language breakdowns that materially impede comprehension.\par - Score 1 = Inadequate/Noncommunicative:\par ... [2 lines omitted] ...\par Special guidance to address known mismatch patterns\par - Overcounting three "reasons" \textcolor{rpTypeRule}{when} development is shallow: insist that each of the three reasons include \textcolor{rpTypeRule}{at least} a clear \textcolor{rpTypeEvidence}{specific example} or a coherent explanation that directly links to the claim. \textcolor{rpTypeRule}{If} any one of the three is only a generic assertion or \textcolor{rpTypeEvidence}{repetition}, do not give 5.\par - Under-counting due to placeholders and surface errors: be generous in counting development \textcolor{rpTypeRule}{if} the example's function and meaning are clear despite placeholders or garbling. Partial coherent \textcolor{rpTypeEvidence}{anecdote}s or one-sentence logical explanations should count.\par - Distinguishing quantity vs. quality: three thin assertions do not equal three developed reasons. Two well-developed reasons are preferable to three weak ones; apply the two-developed \textcolor{rpTypeRule}{threshold} for 4 and require substantive specifics for 5.\par - Examples to guide decisions:\par   - Example like A/B (three distinct claims each with specific \textcolor{rpTypeEvidence}{anecdote} or example, even with errors/placeholders) -> 5 unless language and expression reach the high standard for 6.\par   - Example like C (one or one-plus confused reasons) -> 3.\par   - Example like D (explicitly lists three reasons but development is weak/unsupported) -> 4 (\textcolor{rpTypeEvidence}{cap} at 4).\par   - Essays with three distinct reasons each supported by detailed, persuasive examples and polished expression -> 6.\par \par Final scorer \textcolor{rpTypeRule}{checklist} (practical)\par 1. Is the writer's position clear and responsive to the prompt? \textcolor{rpTypeRule}{If} no -> likely 1-2.\par 2. How many genuinely developed, distinct reasons are present? (Use the strict "developed" test above.)\par    - 3+ developed reasons -> consider 5 (or 6 \textcolor{rpTypeRule}{if} development, \textcolor{rpTypeWriting}{organization}, and language are top-tier).\par    - 2 developed reasons -> 4.\par ... [1 lines omitted] ...\par    - 0 developed reasons or extremely brief -> 2.\par 3. Are any of the counted reasons actually \textcolor{rpTypeEvidence}{repetition}s/overlaps? \textcolor{rpTypeRule}{If} so, reduce the developed-reason count.\par 4. Are placeholders or errors obscuring whether development exists? \textcolor{rpTypeRule}{If} obscured -> downgrade to 2/1 as needed.\par 5. Apply \textcolor{rpTypeEvidence}{cap} rule: \textcolor{rpTypeRule}{if} three reasons exist but one is shallow/vague/unsupported -> \textcolor{rpTypeEvidence}{cap} at 4.\par 6. Finally, adjust within-band for \textcolor{rpTypeWriting}{organization} and language: strong \textcolor{rpTypeWriting}{organization} and clear polished language can justify moving from 5->6; pervasive errors that still allow understanding do not force downgrades \textcolor{rpTypeRule}{if} development \textcolor{rpTypeRule}{threshold}s are met.\par \par Summary guidance for adjudicating \textcolor{rpTypeRule}{borderline} cases\par - \textcolor{rpTypeRule}{When} in doubt between adjacent scores, favor the higher score only \textcolor{rpTypeRule}{when} the development clearly meets the heuristic \textcolor{rpTypeRule}{threshold}s (two developed reasons -> 4; three developed reasons with intelligible support -> 5). Do NOT award 5 for three thin or repetitive assertions.\par - Reserve 6 for essays that combine substantive depth for multiple reasons with fluent, precise expression and convincing \textcolor{rpTypeWriting}{organization}.\par - Be strict about counting a "developed" reason: ask, "Would an informed reader be convinced this reason is supported by a concrete example or clear explanation?" \textcolor{rpTypeRule}{If} not, \textcolor{rpTypeRule}{do not count} it.
\end{tcolorbox}
\end{minipage}
\caption{Pattern-highlighted rubric comparison (asap\_1, openai\_gpt-5-mini, base\_simplest\_True\_train100\_iteration5\_top3\_bs4-8-12\_mc4). Matched spans are color-coded by regex pattern. Color types: \textcolor{rpTypeRule}{\textbf{Rule Structure}} (if/threshold/stepwise guidance); \textcolor{rpTypeEvidence}{\textbf{Evidence Handling}} (examples, repetition, and caps); \textcolor{rpTypeWriting}{\textbf{Writing Quality}} (organization and grammar/mechanics).}
\label{fig:rubric_pattern_asap_1_openai_gpt_5_mini_base_simplest_True_train100_iteration5_top3_bs4_8_12_mc4}
\end{figure*}


\colorlet{rpTypeRule}{red!80!black}
\colorlet{rpTypeEvidence}{blue!80!black}
\colorlet{rpTypeWriting}{teal!80!black}
\begin{figure*}[t]
\centering
\begin{tcolorbox}[colback=white,colframe=black!25,title=Pattern Legend,fonttitle=\bfseries\small,fontupper=\scriptsize,boxsep=1pt,left=2pt,right=2pt,top=2pt,bottom=2pt]
\textcolor{rpTypeRule}{\textbf{Rule Structure}} (if/threshold/stepwise guidance) \quad \textcolor{rpTypeEvidence}{\textbf{Evidence Handling}} (examples, repetition, and caps) \quad \textcolor{rpTypeWriting}{\textbf{Writing Quality}} (organization and grammar/mechanics)
\end{tcolorbox}
\vspace{2mm}
\begin{minipage}[t]{0.485\textwidth}
\begin{tcolorbox}[colback=white,colframe=black!25,title=Initial Rubric,fonttitle=\bfseries\small,fontupper=\scriptsize,breakable]
\ttfamily
- Contains only general reasons with unelaborated and/or list-like details.\par - Shows little or no \textcolor{rpTypeEvidence}{evidence} of \textcolor{rpTypeWriting}{organization}.\par - May be awkward and confused or simplistic.\par ... [3 lines omitted] ...\par - Has reasons with minimal elaboration and more general than specific details.\par - Shows some \textcolor{rpTypeWriting}{organization}.\par - May be awkward in parts with few \textcolor{rpTypeWriting}{transition}s.\par - Shows some awareness of audience.\par ... [2 lines omitted] ...\par - Has adequately elaborated reasons with a mix of general and specific details.\par - Shows satisfactory \textcolor{rpTypeWriting}{organization}.\par - May be somewhat fluent with some \textcolor{rpTypeWriting}{transition}al language.\par - Shows adequate awareness of audience.\par ... [2 lines omitted] ...\par - Has moderately well elaborated reasons with mostly specific details.\par - Exhibits generally strong \textcolor{rpTypeWriting}{organization}.\par - May be moderately fluent with \textcolor{rpTypeWriting}{transition}al language throughout.\par - May show a consistent awareness of audience.\par ... [2 lines omitted] ...\par - Has fully elaborated reasons with specific details.\par - Exhibits strong \textcolor{rpTypeWriting}{organization}.\par - Is fluent and uses sophisticated \textcolor{rpTypeWriting}{transition}al language.\par - May show a heightened awareness of audience.\par ... [1 lines omitted] ...\par Note: \par I have made an effort to remove personally identifying information from the essays using the Named Entity Recognizer (NER). The relevant entities are identified in the text and then replaced with a string such as "PERSON", "\textcolor{rpTypeWriting}{ORGANIZATION}", "LOCATION", "DATE", "TIME", "MONEY", "PERCENT", "CAPS" (any capitalized word) and "NUM" (any digits). Please do not penalize the essay because of the anonymizations.
\end{tcolorbox}
\end{minipage}
\hfill
\begin{minipage}[t]{0.485\textwidth}
\begin{tcolorbox}[colback=white,colframe=black!25,title=Optimized Rubric,fonttitle=\bfseries\small,fontupper=\scriptsize,breakable]
\ttfamily
- Is fragmented, disjointed, or nearly unintelligible; sentences frequently fail to convey clear meaning.\par - Anonymized placeholders (@CAPS, @\textcolor{rpTypeWriting}{ORGANIZATION}, etc.) dominate the text and prevent coherent interpretation of ideas, rendering key claims unverifiable or unintelligible.\par - Shows no consistent awareness of audience or purpose; letter format, \textcolor{rpTypeRule}{if} present, is incorrectly applied or \textcolor{rpTypeEvidence}{irrelevant}.\par - Fails to develop any reason with even basic elaboration; ideas are either absent, contradictory, or buried in confusion.\par - Critical Note: \textcolor{rpTypeRule}{If} anonymized placeholders are randomly inserted and render core claims unintelligible (\textcolor{rpTypeEvidence}{e.g.}, "@CAPS9 he helps you..."), this qualifies as Score 1. However, \textcolor{rpTypeRule}{if} the structure and intent are discernible despite grammatical errors, and placeholders are used as contextual substitutes (even \textcolor{rpTypeRule}{if} imperfect), do not assign Score 1.\par \par Score Point 2: An under-developed response that may or may not take a position. Typical elements:\par - Contains only broad, general claims with no \textcolor{rpTypeEvidence}{specific example}s or personal context; reasons are listed but not explained.\par - Shows minimal \textcolor{rpTypeWriting}{organization}, with little to no paragraphing or \textcolor{rpTypeWriting}{logical flow}; \textcolor{rpTypeWriting}{transition}s are absent or nonsensical.\par - Language is simplistic, repetitive, or confused, with frequent grammatical errors that impede understanding but do not completely obscure meaning.\par - Anonymized placeholders appear but do not dominate the text; some attempt at audience awareness (\textcolor{rpTypeEvidence}{e.g.}, letter format) is present but ineffective.\par - Ideas are superficial and lack development; no \textcolor{rpTypeEvidence}{evidence} of reflection, analysis, or persuasive intent beyond surface-level statements.\par - Critical Note: \textcolor{rpTypeRule}{If} the essay contains \textcolor{rpTypeRule}{at least} one identifiable personal \textcolor{rpTypeEvidence}{anecdote}, observable behavior, or contextual reference (even \textcolor{rpTypeRule}{if} poorly expressed), and the position is clear, it should not be scored as 1. Score 2 is reserved for responses where the argument is present but entirely unsubstantiated by any concrete detail-even anonymized.\par \par Score Point 3: A minimally-developed response that takes a position with limited but discernible support. Typical elements:\par - Presents 1-\textcolor{rpTypeRule}{2 reasons} with some attempt at elaboration, though details remain general or inconsistently developed.\par - Shows basic \textcolor{rpTypeWriting}{organization}: introduction, body, and conclusion are recognizable, though paragraphing may be weak or uneven.\par - Uses occasional \textcolor{rpTypeEvidence}{specific example}s (\textcolor{rpTypeEvidence}{e.g.}, "I talk to my cousin in Colombia") or anonymized placeholders used contextually as \textcolor{rpTypeEvidence}{evidence} (\textcolor{rpTypeEvidence}{e.g.}, "@PERCENT1 of students," "@PERSON1 says...")-even \textcolor{rpTypeRule}{if} embedded in awkward phrasing or grammatical errors.\par - Shows partial awareness of audience (\textcolor{rpTypeEvidence}{e.g.}, uses letter format, addresses "readers"), but tone is inconsistent or immature.\par - \textcolor{rpTypeWriting}{Transition}al language is sparse or simplistic ("\textcolor{rpTypeRule}{first}," "\textcolor{rpTypeRule}{second}," "last"); fluency is limited but the essay remains readable with effort.\par - Anonymized placeholders are present but do not overwhelm the text; they are used as minor contextual substitutes, not as primary \textcolor{rpTypeEvidence}{evidence}.\par - Critical Note: An essay may still earn Score 3 even with multiple placeholders \textcolor{rpTypeRule}{if} they are embedded in a coherent structure and serve as identifiable, contextually grounded \textcolor{rpTypeEvidence}{evidence} (\textcolor{rpTypeEvidence}{e.g.}, "@PERCENT1 say...", "@PERSON1, a researcher...")-even \textcolor{rpTypeRule}{if} \textcolor{rpTypeWriting}{grammar} is poor. Do not penalize for placeholder density alone; penalize only \textcolor{rpTypeRule}{when} placeholders prevent interpretation of the claim's meaning.\par \par Score Point 4: A somewhat-developed response that takes a clear position and provides adequate support. Typical elements:\par - Presents 2-\textcolor{rpTypeRule}{3 reasons} with adequate elaboration, combining general claims with \textcolor{rpTypeRule}{at least} one specific, concrete example per reason (\textcolor{rpTypeEvidence}{e.g.}, "I use Facebook to ask about homework," "@PERCENT1 of kids are obese due to screen time").\par - Exhibits satisfactory \textcolor{rpTypeWriting}{organization}: clear structure with topic sentences, logical progression, and a conclusion that restates the position.\par - Uses simple \textcolor{rpTypeWriting}{transition}al language ("\textcolor{rpTypeEvidence}{for example}," "another reason," "in conclusion") consistently, though not always sophisticated.\par - Demonstrates adequate awareness of audience (\textcolor{rpTypeEvidence}{e.g.}, direct address to newspaper readers, appropriate tone for public letter).\par - Contains minor grammatical errors or awkward phrasing, but these do not significantly hinder understanding or weaken the argument.\par - Anonymized placeholders (@CAPS, @PERCENT, @LOCATION, etc.) are used meaningfully as \textcolor{rpTypeEvidence}{evidence} (\textcolor{rpTypeEvidence}{e.g.}, statistics, expert references) and do not disrupt clarity. Even \textcolor{rpTypeRule}{if} placeholders obscure exact identities, the argument's logic and supporting data remain interpretable and persuasive.\par - Critical Note: Score 4 requires that each reason includes \textcolor{rpTypeRule}{at least} one instance of concrete support-whether personal, observational, or anonymized. \textcolor{rpTypeRule}{If} the essay has 2-\textcolor{rpTypeRule}{3 reasons}, each with one clear placeholder-supported example (\textcolor{rpTypeEvidence}{e.g.}, "@PERSON1 says...", "@PERCENT1 of users..."), and the structure is logically organized, it qualifies for Score 4-even with numerous grammatical errors. Do not downgrade for language \textcolor{rpTypeRule}{if} the argument's logic and \textcolor{rpTypeEvidence}{evidence} are intact. Additionally, \textcolor{rpTypeRule}{if} the essay uses rhetorical questions, direct appeals to the reader, or emotional language to strengthen persuasion (\textcolor{rpTypeEvidence}{e.g.}, "Don't you love playing outdoor games?"), this demonstrates persuasive intent beyond basic claims and supports a Score 4-even \textcolor{rpTypeRule}{if} examples are not highly detailed.\par \par Score Point 5: A developed response that takes a clear and thoughtful position and provides reasonably persuasive support. Typical elements:\par - Presents 3+ well-elaborated reasons with mostly specific, relevant, and varied details (\textcolor{rpTypeEvidence}{e.g.}, personal \textcolor{rpTypeEvidence}{anecdote}s, observable behaviors, anonymized statistics used meaningfully).\par - Exhibits strong \textcolor{rpTypeWriting}{organization}: paragraphs are focused, ideas flow logically, and \textcolor{rpTypeWriting}{transition}s are varied and purposeful (\textcolor{rpTypeEvidence}{e.g.}, "furthermore," "conversely," "as a result").\par - Demonstrates moderate fluency: language is mostly clear, precise, and controlled, with only occasional errors that do not distract from meaning.\par - Shows consistent and thoughtful awareness of audience: tone is persuasive, respectful, and appropriate for a newspaper letter; rhetorical strategies (\textcolor{rpTypeEvidence}{e.g.}, rhetorical questions, direct appeals, emotional language) enhance persuasion.\par - Anonymized placeholders are integrated naturally and do not impede clarity or credibility; they serve as valid, context-appropriate \textcolor{rpTypeEvidence}{evidence} (\textcolor{rpTypeEvidence}{e.g.}, "@PERCENT1 of students," "@PERSON1, a researcher at @\textcolor{rpTypeWriting}{ORGANIZATION}1") and are treated as credible substitutes, not distractions.\par - Critical Note: Score 5 is awarded \textcolor{rpTypeRule}{when} the essay demonstrates persuasive intent beyond basic claims-using rhetorical devices, varied \textcolor{rpTypeEvidence}{evidence}, and consistent tone-even \textcolor{rpTypeRule}{if} language is imperfect. Do not require flawless \textcolor{rpTypeWriting}{grammar}. \textcolor{rpTypeRule}{If} anonymized \textcolor{rpTypeEvidence}{evidence} is used repeatedly and meaningfully (\textcolor{rpTypeEvidence}{e.g.}, multiple expert quotes, statistics, location-based observations), and the structure is cohesive, it qualifies for Score 5. A single, well-placed placeholder (\textcolor{rpTypeEvidence}{e.g.}, "@PERSON1 says...") is not enough for Score 5; multiple credible placeholders OR a mix of personal and anonymized \textcolor{rpTypeEvidence}{evidence} across reasons are required. Crucially, essays that use emotional appeals, direct audience engagement, or vivid imagery-even with grammatical flaws-are eligible for Score 5 \textcolor{rpTypeRule}{if} they show layered, intentional support across all reasons. Additionally, to qualify for Score 5, the essay must demonstrate \textcolor{rpTypeRule}{at least} one of the following: (1) a clear contrast or counterpoint acknowledged and addressed, (2) a compelling call to action with emotional weight, or (3) multiple distinct types of \textcolor{rpTypeEvidence}{evidence} (\textcolor{rpTypeEvidence}{e.g.}, one personal \textcolor{rpTypeEvidence}{anecdote} + one statistic + one expert attribution). Do not award Score 5 for merely having many placeholders; they must be meaningfully woven into a persuasive, multi-layered argument.\par \par Score Point 6: A well-developed response that takes a clear and thoughtful position and provides persuasive support. Typical elements:\par - Presents fully elaborated reasons with rich, specific, and insightful details that demonstrate deep understanding of the issue (\textcolor{rpTypeEvidence}{e.g.}, nuanced analysis of social consequences, balanced acknowledgment of counterpoints).\par - Exhibits strong, cohesive \textcolor{rpTypeWriting}{organization}: introduction establishes context and stakes, body paragraphs build logically, and conclusion offers a compelling call to action or broader insight.\par - Is fluent and polished: language is precise, varied, and sophisticated; \textcolor{rpTypeWriting}{transition}s are seamless and enhance rhetorical effect.\par - Shows heightened awareness of audience: tone is confident and engaging; writer anticipates reader concerns and responds effectively; persuasive techniques are intentional and well-executed.\par - Anonymized placeholders are used appropriately and do not detract from the argument's strength or clarity; the essay reads as \textcolor{rpTypeRule}{if} it were written without redaction, with placeholders functioning as natural, credible stand-ins for real-world \textcolor{rpTypeEvidence}{evidence}.\par - Critical Note: Score 6 requires sophistication in both reasoning and expression. While placeholders may be present, they must be seamlessly integrated as authoritative \textcolor{rpTypeEvidence}{evidence} (\textcolor{rpTypeEvidence}{e.g.}, "@\textcolor{rpTypeWriting}{ORGANIZATION}1's 2023 report shows...") and the essay must demonstrate a level of rhetorical control and insight that goes beyond mere adequacy. A single strong example is insufficient; multiple layers of \textcolor{rpTypeEvidence}{evidence} and nuanced analysis are required. The essay must not only persuade but also reflect depth of thought-such as recognizing complexity, anticipating objections, or connecting the issue to broader societal values. Placeholders must feel like intentional, credible substitutions, not merely convenient fillers.\par \par Note: \par I have made an effort to remove personally identifying information from the essays using the Named Entity Recognizer (NER). The relevant entities are identified in the text and then replaced with a string such as "PERSON", "\textcolor{rpTypeWriting}{ORGANIZATION}", "LOCATION", "DATE", "TIME", "MONEY", "PERCENT", "CAPS" (any capitalized word) and "NUM" (any digits). Please do not penalize the essay because of the anonymizations. \textcolor{rpTypeRule}{When} evaluating, assess whether anonymized placeholders are used in a way that:\par - Obscures meaning and prevents interpretation -> penalize (Score 1-2)\par - Are sparse and do not interfere with clarity or persuasiveness -> ignore (Score 3-6)\par - Are integrated meaningfully (\textcolor{rpTypeEvidence}{e.g.}, "@PERCENT1 of students" used as \textcolor{rpTypeEvidence}{evidence}, "@PERSON1, an expert at @\textcolor{rpTypeWriting}{ORGANIZATION}1" cited as authority) -> treat as valid support (Score 4-6)\par \par Critical Clarification for Scoring:\par - Do not penalize grammatical errors, awkward phrasing, or minor mis\textcolor{rpTypeWriting}{spelling}s \textcolor{rpTypeRule}{if} the core argument, structure, and use of \textcolor{rpTypeEvidence}{evidence} remain clear and persuasive. The presence of anonymized placeholders should not automatically downgrade an essay; instead, evaluate whether the placeholders enable or obstruct the development of the argument.\par - An essay may still earn Score 5 or 6 even with numerous placeholders \textcolor{rpTypeRule}{if} they are used as credible, contextually grounded \textcolor{rpTypeEvidence}{evidence} (\textcolor{rpTypeEvidence}{e.g.}, statistics, expert attributions) and the reasoning, \textcolor{rpTypeWriting}{organization}, and tone meet the higher-level criteria.\par - Conversely, an essay with few placeholders but incoherent logic, no developed reasons, or unintelligible structure should not exceed Score 2.\par - Score 4 requires \textcolor{rpTypeRule}{at least} one concrete example per reason, even \textcolor{rpTypeRule}{if} anonymized. Score 5 and 6 require multiple specific, varied examples and \textcolor{rpTypeEvidence}{evidence} that feel intentional and persuasive, not merely inserted.\par - Rhetorical questions, appeals to emotion, direct audience engagement, and vivid imagery are signs of persuasive intent and should be rewarded at Score 4 and above, even \textcolor{rpTypeRule}{if} language is imperfect.\par - Key Revision: \textcolor{rpTypeRule}{If} the essay presents 2-\textcolor{rpTypeRule}{3 reasons}, each supported by \textcolor{rpTypeRule}{at least} one identifiable example (personal or anonymized), and the structure is recognizable (intro, body, conclusion), it should be scored \textcolor{rpTypeRule}{at least} 3-even with poor \textcolor{rpTypeWriting}{grammar}. Do not score 1 or 2 simply because of placeholder density; score 1 only \textcolor{rpTypeRule}{when} the text is completely unintelligible and no claim can be interpreted. Score 2 requires absence of any concrete support, even anonymized.\par - Critical Addition: For Score 5, the presence of multiple rhetorical devices (\textcolor{rpTypeEvidence}{e.g.}, rhetorical questions, emotional appeals, direct address) combined with \textcolor{rpTypeRule}{at least} two distinct types of \textcolor{rpTypeEvidence}{evidence} (\textcolor{rpTypeEvidence}{e.g.}, one personal observation + one anonymized statistic) across the reasons is sufficient-even \textcolor{rpTypeRule}{if} examples are not highly detailed or polished. Do not require flawless expression; prioritize persuasive intent and layered support.\par - Critical Addition: To earn Score 5, the essay must demonstrate not only multiple examples but also \textcolor{rpTypeEvidence}{evidence} of persuasive strategy beyond listing claims-such as a call to action, emotional resonance, acknowledgment of counterarguments, or vivid imagery that deepens the reader's connection to the issue. A Score 4 essay may have adequate support; a Score 5 essay makes the reader feel something or reconsider their view.\par - Critical Addition: For Score 6, the essay must show a level of rhetorical maturity that transforms \textcolor{rpTypeEvidence}{evidence} into insight. It should not just report effects but interpret them-\textcolor{rpTypeEvidence}{e.g.}, "Computers don't just isolate us; they rewire our expectations of human connection." Placeholders must feel like natural, authoritative anchors, not redactions.
\end{tcolorbox}
\end{minipage}
\caption{Pattern-highlighted rubric comparison (asap\_1, qwen\_qwen3-next-80b-a3b-instruct, base\_expert\_True\_train100\_iteration5\_top3\_bs4-8-12\_mc4). Matched spans are color-coded by regex pattern. Color types: \textcolor{rpTypeRule}{\textbf{Rule Structure}} (if/threshold/stepwise guidance); \textcolor{rpTypeEvidence}{\textbf{Evidence Handling}} (examples, repetition, and caps); \textcolor{rpTypeWriting}{\textbf{Writing Quality}} (organization and grammar/mechanics).}
\label{fig:rubric_pattern_asap_1_qwen_qwen3_next_80b_a3b_instruct_base_expert_True_train100_iteration5_top3_bs4_8_12_mc4}
\end{figure*}


\colorlet{rpTypeRule}{red!80!black}
\colorlet{rpTypeEvidence}{blue!80!black}
\colorlet{rpTypeWriting}{teal!80!black}
\begin{figure*}[t]
\centering
\begin{tcolorbox}[colback=white,colframe=black!25,title=Pattern Legend,fonttitle=\bfseries\small,fontupper=\scriptsize,boxsep=1pt,left=2pt,right=2pt,top=2pt,bottom=2pt]
\textcolor{rpTypeRule}{\textbf{Rule Structure}} (if/threshold/stepwise guidance) \quad \textcolor{rpTypeEvidence}{\textbf{Evidence Handling}} (examples, repetition, and caps) \quad \textcolor{rpTypeWriting}{\textbf{Writing Quality}} (organization and grammar/mechanics)
\end{tcolorbox}
\vspace{2mm}
\begin{minipage}[t]{0.485\textwidth}
\begin{tcolorbox}[colback=white,colframe=black!25,title=Initial Rubric,fonttitle=\bfseries\small,fontupper=\scriptsize,breakable]
\ttfamily
Based on the response's content, rate the response on a scale of 1 to 6.
\end{tcolorbox}
\end{minipage}
\hfill
\begin{minipage}[t]{0.485\textwidth}
\begin{tcolorbox}[colback=white,colframe=black!25,title=Optimized Rubric,fonttitle=\bfseries\small,fontupper=\scriptsize,breakable]
\ttfamily
\par A score of 6 requires a well-organized, compelling argument with clear, credible, and \textcolor{rpTypeEvidence}{specific example}s; effective persuasion grounded in logical reasoning; and minimal language errors that do not distract from the message. Placeholders (\textcolor{rpTypeEvidence}{e.g.}, @NUM1, @CAPS1) are acceptable only \textcolor{rpTypeRule}{if} they are sparse, contextually clear, and do not replace substantive \textcolor{rpTypeEvidence}{evidence} or personal insight. Fabricated statistics, fictional names, or unsupported claims presented as fact-even \textcolor{rpTypeRule}{if} embedded in placeholders-undermine credibility and disqualify a score of 6.  \par \par A score of 5 requires a strong, coherent argument with relevant, developed examples and effective persuasion. Language errors may be occasional and non-obstructive. Placeholders are acceptable \textcolor{rpTypeRule}{if} they are clearly intended as stand-ins for real-world references (\textcolor{rpTypeEvidence}{e.g.}, @LOCATION1 for a city name, @\textcolor{rpTypeWriting}{ORGANIZATION}1 for a known institution) and do not replace meaningful analysis, personal experience, or verifiable reasoning. A response may achieve a score of 5 even \textcolor{rpTypeRule}{if} it contains multiple placeholders, provided: (1) the core claims are reasonable and grounded in plausible real-world phenomena; (2) the placeholders serve as convenient abbreviations or anonymizations rather than fabrications (\textcolor{rpTypeEvidence}{e.g.}, @PERSON2 for "a friend" or "a teacher"); (3) the argument's logic, structure, and persuasive intent remain intact and credible; and (4) the reader can reasonably infer the intended meaning without being misled into believing the placeholders represent falsified data. Fabricated statistics, fictional \textcolor{rpTypeEvidence}{anecdote}s, or implausible events presented as fact-even \textcolor{rpTypeRule}{if} labeled with placeholders-do not automatically disqualify a score of 5 \textcolor{rpTypeRule}{if} they are clearly used as illustrative proxies for real trends (\textcolor{rpTypeEvidence}{e.g.}, "@PERCENT1 of teens" meaning "many teens") and the overall reasoning remains internally consistent and persuasive. The key is whether the response functions as a credible, thoughtful letter to the editor, not whether every detail is verifiable.  \par \par A score of 4 requires a clear position with logical reasoning and sufficient supporting details, even \textcolor{rpTypeRule}{if} the language contains frequent but understandable errors (\textcolor{rpTypeEvidence}{e.g.}, mis\textcolor{rpTypeWriting}{spelling}s, awkward phrasing, minor \textcolor{rpTypeWriting}{grammar} issues). The argument remains comprehensible and persuasive despite imperfections. Placeholders are acceptable \textcolor{rpTypeRule}{if} they are limited in number, contextually clear, and do not substitute for core claims or \textcolor{rpTypeEvidence}{evidence}. Responses that rely on fabricated or implausible \textcolor{rpTypeEvidence}{anecdote}s (\textcolor{rpTypeEvidence}{e.g.}, "@PERSON3 overdosed because of cyberbullying") or unverifiable statistics (\textcolor{rpTypeEvidence}{e.g.}, "@PERCENT1 of teenagers") are scored no higher than 4 \textcolor{rpTypeRule}{if} the placeholders dominate the \textcolor{rpTypeEvidence}{evidence} base or \textcolor{rpTypeRule}{if} the claims are presented as factual rather than illustrative. However, \textcolor{rpTypeRule}{if} the core reasoning is sound, the structure is coherent, and the placeholders are used minimally to represent generic real-world phenomena (\textcolor{rpTypeEvidence}{e.g.}, "@CAPS1" for "the internet"), a score of 4 is appropriate.  \par \par A score of 3 indicates a partially developed argument with some relevance to the prompt but significant language errors or dis\textcolor{rpTypeWriting}{organization} that hinder clarity. Ideas may be fragmented or inconsistently supported. Placeholders are used excessively or in ways that obscure meaning, but the core intent to persuade is discernible.  \par \par A score of 2 indicates a weak or confused argument with severe language problems, incoherent structure, or nonsensical phrasing that makes the intent difficult to discern. Some relevance may be present, but the response fails to function as a persuasive letter. Excessive placeholder use that replaces meaningful content (\textcolor{rpTypeEvidence}{e.g.}, @CAPS1 used as a generic substitute for "computer" in nearly every sentence) or reliance on absurd, unexplained claims (\textcolor{rpTypeEvidence}{e.g.}, "@CAPS1 destroys the ozone layer") qualifies for a score of 2 \textcolor{rpTypeRule}{if} the tone is unprofessional or the logic is too broken to follow. Additionally, responses that present only superficial, repetitive, or trivial claims without meaningful development-such as listing benefits or harms in a simplistic, unsupported manner (\textcolor{rpTypeEvidence}{e.g.}, "computers are good because you can play games and do homework")-and fail to engage with the complexity of the issue, even \textcolor{rpTypeRule}{if} language errors are present, must be scored as 2. Persuasive intent requires substantive reasoning, not just enumeration of obvious or shallow points.  \par \par A score of 1 indicates a response that is largely \textcolor{rpTypeEvidence}{irrelevant}, incoherent, or dominated by meaningless placeholders, fabricated data, or random text with no discernible argument or purpose. Placeholders are used as the primary content, replacing all substantive ideas, or the response is filled with nonsensical claims, fictional tragedies, or random symbols that convey no persuasive intent.  \par \par Note: Do not penalize heavily for \textcolor{rpTypeWriting}{spelling} or grammatical errors \textcolor{rpTypeRule}{if} the core argument is clear, logically structured, and persuasively developed. Conversely, do not award high scores for structure or ideas \textcolor{rpTypeRule}{if} the argument relies on fabricated \textcolor{rpTypeEvidence}{evidence}, implausible \textcolor{rpTypeEvidence}{anecdote}s, or excessive placeholders that replace authentic content and undermine credibility-even \textcolor{rpTypeRule}{if} the language is otherwise fluent. A persuasive letter must be credible as well as coherent. However, credibility is assessed holistically: a response may still be persuasive and worthy of a score of 5 \textcolor{rpTypeRule}{if} its core claims are reasonable, its structure is sound, and its use of placeholders is clearly meant to represent real-world references rather than to fabricate false authority or data. Crucially, responses that merely list surface-level observations without analysis, connection to broader implications, or personal insight-regardless of grammatical correctness-lack the depth required for a score above 2. A score of 4 or higher requires \textcolor{rpTypeEvidence}{evidence} of critical thinking, not just assertion.  \par \par ... [1 lines omitted] ...\par - Placeholders are not inherently disqualifying.  \par - A response with multiple placeholders (\textcolor{rpTypeEvidence}{e.g.}, @\textcolor{rpTypeWriting}{ORGANIZATION}1, @PERSON2, @PERCENT1) may still earn a 5 \textcolor{rpTypeRule}{if} the underlying argument is logically structured, the examples are plausible, and the placeholders clearly represent common, real-world entities or trends (\textcolor{rpTypeEvidence}{e.g.}, "@PERCENT1" = "a large percentage," "@\textcolor{rpTypeWriting}{ORGANIZATION}1" = "a major tech company," "@PERSON2" = "a teacher or expert").  \par - A score of 5 is awarded \textcolor{rpTypeRule}{when} the reader can reasonably infer the intended real-world reference and the argument's persuasive power is not dependent on the placeholder's specificity.  \par - A score of 4 is assigned \textcolor{rpTypeRule}{when} placeholders replace substantive \textcolor{rpTypeEvidence}{evidence} in a way that weakens credibility, but the argument's structure and intent remain clear.  \par - A score of 2 or 1 is reserved only \textcolor{rpTypeRule}{when} placeholders are used so pervasively or absurdly that the response becomes unintelligible, nonsensical, or devoid of meaningful reasoning.
\end{tcolorbox}
\end{minipage}
\caption{Pattern-highlighted rubric comparison (asap\_1, qwen\_qwen3-next-80b-a3b-instruct, base\_simplest\_True\_train100\_iteration5\_top3\_bs4-8-12\_mc4). Matched spans are color-coded by regex pattern. Color types: \textcolor{rpTypeRule}{\textbf{Rule Structure}} (if/threshold/stepwise guidance); \textcolor{rpTypeEvidence}{\textbf{Evidence Handling}} (examples, repetition, and caps); \textcolor{rpTypeWriting}{\textbf{Writing Quality}} (organization and grammar/mechanics).}
\label{fig:rubric_pattern_asap_1_qwen_qwen3_next_80b_a3b_instruct_base_simplest_True_train100_iteration5_top3_bs4_8_12_mc4}
\end{figure*}


\colorlet{rpTypeRule}{red!80!black}
\colorlet{rpTypeEvidence}{blue!80!black}
\colorlet{rpTypeWriting}{teal!80!black}
\begin{figure*}[t]
\centering
\begin{tcolorbox}[colback=white,colframe=black!25,title=Pattern Type Guide,fonttitle=\bfseries\small,fontupper=\scriptsize,boxsep=1pt,left=2pt,right=2pt,top=2pt,bottom=2pt]
\textcolor{rpTypeRule}{\textbf{Rule Structure}}: Explicit decision logic for scoring: conditional branches, boundary tie-breakers, stepwise workflows, and numeric thresholds.\par \textcolor{rpTypeEvidence}{\textbf{Evidence Handling}}: How evidence is validated and counted: specific-example requirements, repetition/non-double-count rules, and cap rules for weak evidence.\par \textcolor{rpTypeWriting}{\textbf{Writing Quality}}: Language-quality criteria affecting score bands: organization/coherence/transition quality and grammar/mechanics severity.
\end{tcolorbox}
\vspace{1mm}
\begin{tcolorbox}[colback=white,colframe=black!25,title=Detailed Pattern Notes,fonttitle=\bfseries\small,fontupper=\scriptsize,boxsep=1pt,left=2pt,right=2pt,top=2pt,bottom=2pt]
\textcolor{rpTypeRule}{\textbf{Rule Structure}}:\par \quad \textcolor{rpTypeRule}{\textbf{Conditional Gating}} [n=3] Captures explicit condition-based rules that switch decisions only when a stated condition is met. Typical cues: if, when.\par \textcolor{rpTypeEvidence}{\textbf{Evidence Handling}}:\par \quad \textcolor{rpTypeEvidence}{\textbf{Specific Evidence Requirement}} [n=5] Highlights demands for concrete examples and explicit evidence links instead of generic assertions. Typical cues: for example, e.g., specific example, illustration, anecdote, evidence.\par \quad \textcolor{rpTypeEvidence}{\textbf{Off-Topic / Summary Cap}} [n=2] Identifies cap rules that restrict scores when responses are off-topic, irrelevant, or dominated by summary-only content. Typical cues: off-topic, irrelevant, digression, summary-only, cap.\par \textcolor{rpTypeWriting}{\textbf{Writing Quality}}:\par \quad \textcolor{rpTypeWriting}{\textbf{Organization / Coherence Signal}} [n=3] Detects explicit references to discourse structure and logical flow as scoring criteria. Typical cues: organization, coherence, logical flow, transition.\par \quad \textcolor{rpTypeWriting}{\textbf{Grammar / Mechanics Signal}} [n=5] Detects references to language-form quality, especially grammar, spelling, punctuation, and mechanics. Typical cues: grammar, mechanics, spelling, punctuation.
\end{tcolorbox}
\vspace{1mm}
\begin{tcolorbox}[colback=white,colframe=black!25,title=Optimized Rubric (Pattern-Highlighted),fonttitle=\bfseries\small,fontupper=\scriptsize]
\ttfamily
- addresses the topic and task well, providing specific, relevant, and sophisticated details or examples; an unfinished conclusion or mechanical errors (\textcolor{rpTypeWriting}{\textbf{spelling}}, typos, word form) do not automatically disqualify an essay \textcolor{rpTypeRule}{\textbf{if}} the preceding development is thorough and the logic is compelling\par ... [1 lines omitted] ...\par - displays unity and \textcolor{rpTypeWriting}{\textbf{coherence}}; the argument is sustained throughout the response, and any minor errors do not interfere with the reader's ability to follow sophisticated reasoning\par - displays facility in the use of language, demonstrating syntactic variety and range of vocabulary; even \textcolor{rpTypeRule}{\textbf{if}} there is a noticeable "accumulation" of surface-level errors (\textcolor{rpTypeEvidence}{\textbf{e.g.}}, "patern," "nowdays," "thining," "Eddison"), the essay remains at this level \textcolor{rpTypeRule}{\textbf{if}} the writer maintains a consistent rhythm and control of complex sentence structures\par ... [4 lines omitted] ...\par - displays unity and \textcolor{rpTypeWriting}{\textbf{coherence}}, but the connection of ideas is occasionally obscured, or the essay moves mechanically from one point to another without deep exploration\par - demonstrates inconsistent facility in sentence formation; while the meaning is generally clear, frequent errors in \textcolor{rpTypeWriting}{\textbf{grammar}}, \textcolor{rpTypeWriting}{\textbf{spelling}}, or word choice (\textcolor{rpTypeEvidence}{\textbf{e.g.}}, "more risk more yield," "out of different reasons," "there were made several surveys") indicate a lack of sustained control over the language or a reliance on overly simple sentence structures\par - may stray into \textcolor{rpTypeEvidence}{\textbf{irrelevant}} \textcolor{rpTypeEvidence}{\textbf{digression}}s, personal \textcolor{rpTypeEvidence}{\textbf{anecdote}}s that provide only tangential support, or meta-commentary about the test-taking process\par ... [4 lines omitted] ...\par - inadequate \textcolor{rpTypeWriting}{\textbf{organization}} or connection of ideas; the essay may struggle to move beyond a few basic thoughts or may end abruptly mid-sentence (\textcolor{rpTypeEvidence}{\textbf{e.g.}}, "maybe we would be who")\par - a pervasive accumulation of errors in sentence structure, usage, \textcolor{rpTypeWriting}{\textbf{spelling}}, and \textcolor{rpTypeWriting}{\textbf{grammar}} (\textcolor{rpTypeEvidence}{\textbf{e.g.}}, "it can be happen," "how I can happy," "rock question," "anwear") that seriously interrupts the flow of reading and suggests a lack of basic linguistic competence
\end{tcolorbox}
\caption{Pattern-focused view of the optimized rubric (ets3, google\_gemini-3-flash-preview, base\_expert\_True\_train100\_iteration5\_top3\_bs4-8-12\_mc4). Colored bold spans indicate regex-matched rubric cues. Color types: \textcolor{rpTypeRule}{\textbf{Rule Structure}} (Explicit decision logic for scoring: conditional branches, boundary tie-breakers, stepwise workflows, and numeric thresholds.); \textcolor{rpTypeEvidence}{\textbf{Evidence Handling}} (How evidence is validated and counted: specific-example requirements, repetition/non-double-count rules, and cap rules for weak evidence.); \textcolor{rpTypeWriting}{\textbf{Writing Quality}} (Language-quality criteria affecting score bands: organization/coherence/transition quality and grammar/mechanics severity.). Matched pattern categories: Conditional Gating (n=3); Specific Evidence Requirement (n=5); Off-Topic / Summary Cap (n=2); Organization / Coherence Signal (n=3); Grammar / Mechanics Signal (n=5).}
\label{fig:rubric_pattern_ets3_google_gemini_3_flash_preview_base_expert_True_train100_iteration5_top3_bs4_8_12_mc4}
\end{figure*}


\colorlet{rpTypeRule}{red!80!black}
\colorlet{rpTypeEvidence}{blue!80!black}
\colorlet{rpTypeWriting}{teal!80!black}
\begin{figure*}[t]
\centering
\begin{tcolorbox}[colback=white,colframe=black!25,title=Pattern Legend,fonttitle=\bfseries\small,fontupper=\scriptsize,boxsep=1pt,left=2pt,right=2pt,top=2pt,bottom=2pt]
\textcolor{rpTypeRule}{\textbf{Rule Structure}} (if/threshold/stepwise guidance) \quad \textcolor{rpTypeEvidence}{\textbf{Evidence Handling}} (examples, repetition, and caps) \quad \textcolor{rpTypeWriting}{\textbf{Writing Quality}} (organization and grammar/mechanics)
\end{tcolorbox}
\vspace{2mm}
\begin{minipage}[t]{0.485\textwidth}
\begin{tcolorbox}[colback=white,colframe=black!25,title=Initial Rubric,fonttitle=\bfseries\small,fontupper=\scriptsize,breakable]
\ttfamily
- is generally well organized and well developed, using appropriate and sufficient explanations, exemplifications, and/or details\par - displays unity, progression, and \textcolor{rpTypeWriting}{coherence}, though it may contain occasional redundancy, \textcolor{rpTypeEvidence}{digression}, or unclear connections\par - displays facility in the use of language, demonstrating syntactic variety and range of vocabulary, though it will probably have occasional noticeable minor errors in structure, word form, or use of idiomatic language that do not interfere with meaning\par ... [3 lines omitted] ...\par - addresses the topic and task using somewhat developed explanations, exemplifications, and/or details\par - displays unity, progression, and \textcolor{rpTypeWriting}{coherence}, though connection of ideas may be occasionally obscured\par - may demonstrate inconsistent facility in sentence formation and word choice that may result in lack of clarity and occasionally obscure meaning\par ... [4 lines omitted] ...\par - limited development in response to the topic and task\par - inadequate \textcolor{rpTypeWriting}{organization} or connection of ideas\par - inappropriate or insufficient exemplifications, explanations, or details to support or illustrate generalizations in response to the task
\end{tcolorbox}
\end{minipage}
\hfill
\begin{minipage}[t]{0.485\textwidth}
\begin{tcolorbox}[colback=white,colframe=black!25,title=Optimized Rubric,fonttitle=\bfseries\small,fontupper=\scriptsize,breakable]
\ttfamily
An essay at this level largely accomplishes all of the following:\par - addresses the topic and task effectively, providing specific reasons and well-developed, concrete examples (\textcolor{rpTypeEvidence}{e.g.}, citing specific historical figures, detailed personal \textcolor{rpTypeEvidence}{anecdote}s, or specific societal trends) that directly support the thesis; the development shows an ability to handle complex ideas and nuanced perspectives\par - is well-organized and displays a clear, logical progression of ideas; while the structure may rely on standard \textcolor{rpTypeWriting}{transition} words (\textcolor{rpTypeEvidence}{e.g.}, "\textcolor{rpTypeRule}{First}," "Secondly"), they are used effectively to guide a coherent argument rather than appearing purely mechanical\par - displays a high facility in the use of language and a range of syntactic variety; although it may contain frequent minor errors in \textcolor{rpTypeWriting}{grammar}, \textcolor{rpTypeWriting}{spelling}, or \textcolor{rpTypeWriting}{mechanics} (\textcolor{rpTypeEvidence}{e.g.}, "doesnot," "yound," "popluare," "well-arouned"), these errors do not obscure meaning or significantly interfere with the strength and flow of the argument\par - demonstrates a level of detail and depth of thought that goes beyond simple observations; the writing feels substantial and the author's control over the argument remains strong despite linguistic imperfections\par ... [2 lines omitted] ...\par An essay at this level is marked by one or more of the following:\par - addresses the topic and task but development is limited or uneven; the essay may rely on generalities, \textcolor{rpTypeEvidence}{repetition} of ideas, or explanations that lack grounded, specific detail (\textcolor{rpTypeEvidence}{e.g.}, repeating that a situation is "boring" or "hard" without further elaboration)\par - displays unity and \textcolor{rpTypeWriting}{coherence}, but the \textcolor{rpTypeWriting}{organization} may feel overly mechanical or the connection of ideas may be occasionally obscured by linguistic limitations; the essay may resemble a list of points rather than a progression of an argument\par - demonstrates grammatical, \textcolor{rpTypeWriting}{spelling}, or word-choice errors that are persistent enough to distract the reader or suggest a lack of range (\textcolor{rpTypeEvidence}{e.g.}, "earn money hardly," "is important trying," "specialising on"); while the general meaning is discernible, the phrasing often feels unnatural or lacks the complexity/nuance of a higher-level response\par - the essay may end abruptly or contain sections where the author's intent is clear but the execution is significantly hindered by a limited vocabulary or repetitive sentence structures\par ... [1 lines omitted] ...\par \#\# Score 1\par An essay at this level reveals a significant lack of competence \textcolor{rpTypeEvidence}{evidence}d by one or more of the following:\par - limited or very poor development in response to the topic; the essay may be significantly short (under \textcolor{rpTypeRule}{200 words}), fail to address key parts of the prompt, or rely almost entirely on vague, hypothetical "\textcolor{rpTypeRule}{if}/then" scenarios and generalities ("\textcolor{rpTypeRule}{if} we help then world is good") without any concrete \textcolor{rpTypeEvidence}{evidence}\par - inadequate \textcolor{rpTypeWriting}{organization} or connection of ideas, where the reader must frequently pause to reconstruct the author's logic or piece together the relationship between sentences\par - inappropriate, insufficient, or \textcolor{rpTypeEvidence}{irrelevant} examples that fail to support the generalizations made, often resulting in a response that feels "\textcolor{rpTypeEvidence}{off-topic}," purely philosophical, or superficial\par - a pervasive accumulation of serious errors in sentence structure, usage, and \textcolor{rpTypeWriting}{spelling} (\textcolor{rpTypeEvidence}{e.g.}, "cheeting," "funny and rock question," "past the goal") that frequently obscures meaning or results in a lack of clarity throughout the majority of the essay
\end{tcolorbox}
\end{minipage}
\caption{Pattern-highlighted rubric comparison (ets3, google\_gemini-3-flash-preview, base\_simplest\_True\_train100\_iteration5\_top3\_bs4-8-12\_mc4). Matched spans are color-coded by regex pattern. Color types: \textcolor{rpTypeRule}{\textbf{Rule Structure}} (if/threshold/stepwise guidance); \textcolor{rpTypeEvidence}{\textbf{Evidence Handling}} (examples, repetition, and caps); \textcolor{rpTypeWriting}{\textbf{Writing Quality}} (organization and grammar/mechanics).}
\label{fig:rubric_pattern_ets3_google_gemini_3_flash_preview_base_simplest_True_train100_iteration5_top3_bs4_8_12_mc4}
\end{figure*}


\colorlet{rpTypeRule}{red!80!black}
\colorlet{rpTypeEvidence}{blue!80!black}
\colorlet{rpTypeWriting}{teal!80!black}
\begin{figure*}[t]
\centering
\begin{tcolorbox}[colback=white,colframe=black!25,title=Pattern Legend,fonttitle=\bfseries\small,fontupper=\scriptsize,boxsep=1pt,left=2pt,right=2pt,top=2pt,bottom=2pt]
\textcolor{rpTypeRule}{\textbf{Rule Structure}} (if/threshold/stepwise guidance) \quad \textcolor{rpTypeEvidence}{\textbf{Evidence Handling}} (examples, repetition, and caps) \quad \textcolor{rpTypeWriting}{\textbf{Writing Quality}} (organization and grammar/mechanics)
\end{tcolorbox}
\vspace{2mm}
\begin{minipage}[t]{0.485\textwidth}
\begin{tcolorbox}[colback=white,colframe=black!25,title=Initial Rubric,fonttitle=\bfseries\small,fontupper=\scriptsize,breakable]
\ttfamily
- is generally well organized and well developed, using appropriate and sufficient explanations, exemplifications, and/or details\par - displays unity, progression, and \textcolor{rpTypeWriting}{coherence}, though it may contain occasional redundancy, \textcolor{rpTypeEvidence}{digression}, or unclear connections\par - displays facility in the use of language, demonstrating syntactic variety and range of vocabulary, though it will probably have occasional noticeable minor errors in structure, word form, or use of idiomatic language that do not interfere with meaning\par ... [3 lines omitted] ...\par - addresses the topic and task using somewhat developed explanations, exemplifications, and/or details\par - displays unity, progression, and \textcolor{rpTypeWriting}{coherence}, though connection of ideas may be occasionally obscured\par - may demonstrate inconsistent facility in sentence formation and word choice that may result in lack of clarity and occasionally obscure meaning\par ... [4 lines omitted] ...\par - limited development in response to the topic and task\par - inadequate \textcolor{rpTypeWriting}{organization} or connection of ideas\par - inappropriate or insufficient exemplifications, explanations, or details to support or illustrate generalizations in response to the task
\end{tcolorbox}
\end{minipage}
\hfill
\begin{minipage}[t]{0.485\textwidth}
\begin{tcolorbox}[colback=white,colframe=black!25,title=Optimized Rubric,fonttitle=\bfseries\small,fontupper=\scriptsize,breakable]
\ttfamily
Revised Scoring Guideline (Version 5 - tightened \textcolor{rpTypeRule}{threshold}s and quantitative guidance to reduce rater disagreement)\par \par ... [1 lines omitted] ...\par - Read the entire essay for overall communicative effectiveness (how clearly the writer expresses and supports a position).\par - Use the concrete indicators below (development, \textcolor{rpTypeWriting}{organization}/\textcolor{rpTypeWriting}{coherence}, language accuracy and clarity) to place the essay into one of three score bands.\par - Priority rule \textcolor{rpTypeRule}{when} indicators conflict: prioritize intelligibility and development. \textcolor{rpTypeRule}{If} meaning is frequently or repeatedly obscured, lower the score even \textcolor{rpTypeRule}{if} some \textcolor{rpTypeWriting}{organization} or vocabulary are present.\par - Apply demotion rules strictly: any essay whose meaning is often or repeatedly obscured by errors must not receive Score 3. Any essay with only minimal, token, repetitive, or fragmentary development must not receive Score 3. Essays that end abruptly such that one or more required elaborations are missing should be downgraded (see Incomplete/abrupt ending rule).\par - Count both quantity and quality of supporting detail. Long essays with \textcolor{rpTypeEvidence}{repetition}, vagueness, or circular \textcolor{rpTypeEvidence}{restatement} of the same point are not sufficient for higher scores.\par \par ... [1 lines omitted] ...\par \par Score 3 - "Accomplished response" (use this only \textcolor{rpTypeRule}{when} the essay clearly meets most or all items below)\par - Task response and development\par   - Clearly and directly addresses the prompt and establishes a clear, sustained position or central idea.\par   - Provides \textcolor{rpTypeRule}{at least} two distinct supporting points/reasons. Distinct = different aspects or lines of reasoning (\textcolor{rpTypeEvidence}{e.g.}, cause vs. effect, two separate causes, two separate outcomes). \textcolor{rpTypeEvidence}{Repetition} or rewording of the same reason does not count as a \textcolor{rpTypeRule}{second} reason.\par   - For each supporting point, provides specific, relevant elaboration: at minimum 1) an explicit reason sentence, and 2) \textcolor{rpTypeRule}{at least} one additional sentence that explains relevance, consequence, or provides a concrete example tied to the reason.\par   - Example-linking requirement (strict): each example or \textcolor{rpTypeEvidence}{anecdote} must include an explicit, readable link back to the reason or claim. Acceptable links include short phrases such as "this shows that," "because," "therefore," "so," or a brief clause that ties the example to the reason. \textcolor{rpTypeRule}{If} the example is anecdotal or personal, it must include \textcolor{rpTypeRule}{at least} one concrete detail (time, place, specific action or outcome) and an explicit tie-back explaining how it supports the reason.\par   - Development depth requirement: expect roughly 2-\textcolor{rpTypeRule}{3 sentences} of elaboration per reason in a typical short essay (reason + explanation/link ± brief example). Single-sentence reasons with no explicit tie-back do not qualify for Score 3.\par - \textcolor{rpTypeWriting}{Organization} and \textcolor{rpTypeWriting}{coherence}\par   - Logical progression of ideas with paragraphing and \textcolor{rpTypeWriting}{transition}s; relationships among ideas are easy to trace.\par   - Has an adequately coherent introduction and conclusion. A brief or truncated conclusion is acceptable only \textcolor{rpTypeRule}{if} all required development (two distinct reasons, explicit tie-back, adequate elaboration) is already present in the body.\par - Language accuracy and range\par ... [1 lines omitted] ...\par   - No recurring error patterns that regularly obscure sense.\par - Intelligibility \textcolor{rpTypeRule}{threshold} (quantified)\par   - Overall meaning is easy to understand; the reader rarely needs to infer missing information.\par   - Quantitative guidance: in a typical short essay (approx200-\textcolor{rpTypeRule}{300 words}), the reader should need to re-read or infer meaning for no more than \textcolor{rpTypeRule}{3 sentences}, and these should constitute no more than \textasciitilde{}10\textcolor{rpTypeRule}{\%} of total sentences. \textcolor{rpTypeRule}{If} inference is required on more than \textcolor{rpTypeRule}{3 sentences} or >10\textcolor{rpTypeRule}{\%} of sentences, the essay cannot be Score 3.\par \par Score 2 - "Limited to partial response" (use \textcolor{rpTypeRule}{when} development and/or language prevent full accomplishment)\par - Task response and development\par ... [2 lines omitted] ...\par   - Essays that include examples with only implicit links (no explicit tie-back) are Score 2 unless other signs of clear, specific development compensate strongly.\par   - Personal \textcolor{rpTypeEvidence}{anecdote}s that lack concrete detail (\textcolor{rpTypeRule}{when}, where, what specifically happened) and/or lack explicit linkage to the claim should be treated as weak \textcolor{rpTypeEvidence}{evidence} and typically count only as partial support.\par   - Clarified \textcolor{rpTypeRule}{threshold}: To assign Score 2, the essay should show at minimum one clearly articulated reason with \textcolor{rpTypeRule}{at least} one sentence of elaboration beyond the reason statement (explanation, consequence, or a linked example). \textcolor{rpTypeRule}{If} the essay presents only assertions or general statements with no elaboration beyond repeating the claim, treat as Score 1.\par - \textcolor{rpTypeWriting}{Organization} and \textcolor{rpTypeWriting}{coherence}\par   - Basic \textcolor{rpTypeWriting}{organization} exists, but connections among ideas may be occasionally unclear or abrupt; paragraphs/\textcolor{rpTypeWriting}{transition}s may be weak.\par   - Reader can follow the argument with some effort; occasional rereading or inference is needed to understand links.\par ... [2 lines omitted] ...\par   - Demonstrates inconsistent control of sentence formation and word choice. Errors are frequent enough to be noticeable and sometimes obscure meaning, but the essay remains broadly recoverable.\par - Intelligibility \textcolor{rpTypeRule}{threshold}\par   - Meaning is generally recoverable; the reader sometimes must infer or re-read portions to understand intent.\par   - Quantified guidance: inference/re-reading needed on more than \textcolor{rpTypeRule}{3 sentences} but not pervasive across the essay, or inference is needed occasionally but does not regularly interrupt comprehension.\par \par Score 1 - "Weak or minimal response" (use \textcolor{rpTypeRule}{when} the response is minimal, disorganized, or frequently unintelligible)\par - Task response and development\par   - Response is minimal, only loosely related to the prompt, lacks a clear position, or presents only fragmented ideas.\par   - Very limited development: few or no relevant reasons, explanations, or examples; ideas are superficial, largely undeveloped, repetitive, or \textcolor{rpTypeEvidence}{off-topic}.\par   - Essays that end abruptly with an unfinished thought or literally cut off mid-sentence/paragraph leaving core claims undeveloped should be Score 1.\par   - Clarified \textcolor{rpTypeRule}{threshold}: \textcolor{rpTypeRule}{If} the essay contains only assertions, a stated opinion without any discernible supporting reason beyond \textcolor{rpTypeEvidence}{repetition}, or only one undeveloped reason with no elaboration, assign Score 1.\par - \textcolor{rpTypeWriting}{Organization} and \textcolor{rpTypeWriting}{coherence}\par   - \textcolor{rpTypeWriting}{Organization} is inadequate or absent. Ideas may be disjointed, digressive, or lack cohesion.\par   - \textcolor{rpTypeWriting}{Transition}s are missing and the reader cannot follow a clear line of reasoning without substantial effort.\par - Language accuracy and range\par   - Frequent and/or serious errors in sentence structure, \textcolor{rpTypeWriting}{grammar}, word form, or word choice that often obscure meaning.\par   - Recurring error patterns that regularly make propositions unclear warrant Score 1 even \textcolor{rpTypeRule}{if} a position is stated.\par - Intelligibility \textcolor{rpTypeRule}{threshold}\par   - Meaning is often unclear or frequently obscured; the reader has significant difficulty recovering the writer's intended message.\par   - Quantitative guidance: inference or re-reading is required on a substantial portion of sentences (\textcolor{rpTypeEvidence}{e.g.}, >25\textcolor{rpTypeRule}{\%} of sentences or more than \textasciitilde{}\textcolor{rpTypeRule}{8 sentences} in a typical short essay), or multiple sentences are unintelligible.\par \par Additional clarifying and procedural rules (to reduce disagreement)\par - Specificity and example-quality rule (strengthened): For Score 3, require \textcolor{rpTypeRule}{at least} two distinct supporting points, and for each point require:\par     1) an explicit reason sentence,\par     2) \textcolor{rpTypeRule}{at least} one explicit linking sentence or clause tying any example to the reason, and\par     3) either a concrete example with specific detail or a clear explanation of consequence/relevance. Vague \textcolor{rpTypeEvidence}{anecdote}s or generic statements without specifics count as weak \textcolor{rpTypeEvidence}{evidence}.\par - Example-linking rule (clarified \& enforced): Award development credit for an example only \textcolor{rpTypeRule}{when} the writer explicitly connects the example to the reason/claim. \textcolor{rpTypeRule}{If} a reason is supported only by an implicit \textcolor{rpTypeEvidence}{anecdote} or generic example with no explicit tie-back, treat it as partial support (Score 2).\par - Density-of-\textcolor{rpTypeEvidence}{evidence} rule (strengthened): \textcolor{rpTypeRule}{If} an essay is long but provides few real supporting details (lots of \textcolor{rpTypeEvidence}{repetition}, vague generalities, or circular \textcolor{rpTypeEvidence}{restatement}), prefer Score 1 or 2. \textcolor{rpTypeEvidence}{Repetition} of the same reason in multiple paragraphs does not constitute multiplicity of reasons.\par - Error-severity and consistency/demotion rules (quantified \& clarified):\par   - Occasional minor errors that never obscure the main idea should not lower a Score 3 rating.\par   - \textcolor{rpTypeRule}{If} errors sometimes obscure meaning (reader must infer), the essay cannot be Score 3; assign Score 2 or Score 1 depending on frequency and severity.\par   - To reduce subjective disagreement about "sometimes" vs "occasional," use the quantitative guidance above (Score 3: inference needed on <=\textcolor{rpTypeRule}{3 sentences} and <=10\textcolor{rpTypeRule}{\%} of sentences; Score 2: inference needed on >3 but not pervasive; Score 1: inference pervasive).\par   - \textcolor{rpTypeRule}{If} there is a recurring error pattern that regularly interferes with comprehension (many sentences where \textcolor{rpTypeWriting}{grammar} or word choice make meaning unclear), lower to Score 2 or Score 1. Frequent re-reading required = Score 1.\par   - Frequent surface errors that do not obscure core claims but substantially reduce clarity and fluency (many mis\textcolor{rpTypeWriting}{spelling}s, wrong word choices, sentence fragments) are grounds for Score 2 rather than Score 3. \textcolor{rpTypeRule}{If} such errors are pervasive across the essay and make main points shaky or effortful to recover, use Score 1.\par - Incomplete/abrupt ending rule (refined and made more consistent):\par   - \textcolor{rpTypeRule}{If} an essay terminates abruptly or ends mid-idea and thereby leaves core reasoning undeveloped, downgrade one band.\par   - Exceptions: \textcolor{rpTypeRule}{If} an essay otherwise meets Score 3 except for a truncated concluding sentence but both reasons are fully developed with explicit links in the body, maintain Score 3.\par   - \textcolor{rpTypeRule}{If} truncation leaves one required elaboration missing (\textcolor{rpTypeEvidence}{e.g.}, one reason only partially developed or an example not tied back), downgrade one band (usually Score 3 -> Score 2; Score 2 -> Score 1) rather than automatically to the lowest band.\par   - \textcolor{rpTypeRule}{If} truncation plus language errors make recovery of the missing elaboration difficult, consider further downgrade to the lowest band appropriate (Score 1).\par - \textcolor{rpTypeEvidence}{Repetition} vs. multiplicity rule (re-emphasized): Repeating the same reason with slightly different wording is not equivalent to providing multiple distinct reasons. Score 3 requires distinct reasons; repetitive \textcolor{rpTypeEvidence}{restatement} counts as one reason and should limit the score.\par - Personal \textcolor{rpTypeEvidence}{anecdote} handling: Personal experience can count as \textcolor{rpTypeEvidence}{evidence} but is often weaker than a concrete, \textcolor{rpTypeEvidence}{specific example} that is linked analytically to the claim. \textcolor{rpTypeRule}{When} a personal \textcolor{rpTypeEvidence}{anecdote} lacks detail or an explicit analytic tie-back, treat it as partial support (Score 2 or 1 depending on other factors).\par - Minimum-development \textcolor{rpTypeRule}{threshold}s (practical guidance for raters):\par   - To award Score 3: require \textcolor{rpTypeRule}{at least} two distinct reasons, each with explicit relevant elaboration and an explicit tie-back sentence/clause. Typical short essays should show \textasciitilde{}2-\textcolor{rpTypeRule}{3 sentences} per reason (reason + explanation/link + optional brief example). \textcolor{rpTypeRule}{If} either reason lacks this minimal elaboration and explicit tie-back, prefer Score 2.\par   - To award Score 2: the essay should show a clear position and \textcolor{rpTypeRule}{at least} one discernible supporting reason with some attempt at elaboration (one full sentence beyond the reason), or two weak reasons with thin support. \textcolor{rpTypeRule}{If} there is only assertion without elaboration or only one undeveloped reason, assign Score 1.\par   - To award Score 1: the essay has no clear position OR only isolated, undeveloped statements, or frequent unintelligibility caused by pervasive errors or severe underdevelopment.\par - \textcolor{rpTypeRule}{Borderline} handling (refined with explicit \textcolor{rpTypeRule}{checklist} and \textcolor{rpTypeRule}{tie-break}ers): \textcolor{rpTypeRule}{When} between adjacent scores, ask these questions in order:\par   1) Are there \textcolor{rpTypeRule}{at least} two distinct reasons? \textcolor{rpTypeRule}{If} not -> cannot be Score 3.\par   2) For each reason present, is there an explicit tie-back linking any example/explanation to that reason? \textcolor{rpTypeRule}{If} a reason lacks an explicit tie-back -> cannot be Score 3.\par   3) For each reason, is there \textcolor{rpTypeRule}{at least} one sentence of development beyond the reason statement (explanation, consequence, or concrete detail)? \textcolor{rpTypeRule}{If} not for one or more reasons -> cannot be Score 3.\par   4) Does error frequency/severity force the reader to infer or re-read often? Use quantitative \textcolor{rpTypeRule}{threshold}s: \textcolor{rpTypeRule}{if} inference is needed on <=\textcolor{rpTypeRule}{3 sentences} and <=10\textcolor{rpTypeRule}{\%} of sentences -> may still be Score 3; \textcolor{rpTypeRule}{if} inference is needed on >\textcolor{rpTypeRule}{3 sentences} but not pervasive -> prefer Score 2; \textcolor{rpTypeRule}{if} inference is pervasive -> Score 1.\par   5) Is the essay truncated or abruptly ended such that required elaboration is missing? \textcolor{rpTypeRule}{If} yes -> downgrade one band (usually to the adjacent lower band).\par   Use these ordered checks as decisive \textcolor{rpTypeRule}{tie-break}ers rather than holistic instinct \textcolor{rpTypeRule}{when} uncertain.\par - Calibration notes addressing common rater pitfalls (explicit guidance derived from mismatch cases):\par   - Do not award Score 3 solely because an essay contains two named examples; each example must be explicitly connected to a distinct reason and show relevant elaboration. \textcolor{rpTypeRule}{If} examples are undeveloped or linkage is implicit, prefer Score 2.\par   - Do not let a coherent introduction and some understandable sentences override severe underdevelopment or pervasive errors. A readable intro is not a substitute for full development.\par   - Penalize essays that appear long but are largely \textcolor{rpTypeEvidence}{repetition} or vague generalities; length is not \textcolor{rpTypeEvidence}{evidence} of sufficient development.\par   - \textcolor{rpTypeRule}{If} many surface errors exist but content is broadly understandable, Score 2 is expected unless development clearly meets the Score 3 \textcolor{rpTypeRule}{threshold}s. \textcolor{rpTypeRule}{If} errors make recovery effortful or the reasoning shaky, use Score 1.\par   - Personal experience statements like "I have realized" or "from my experience" without specifics and an explicit tie-back are weak \textcolor{rpTypeEvidence}{evidence}-treat them as partial support.\par   - Use the quantitative inference \textcolor{rpTypeRule}{threshold}s to reduce subjective disagreement about whether errors "sometimes" obscure meaning.\par - Practical quick \textcolor{rpTypeRule}{checklist} before choosing a score\par   - Is there a clear position? \textcolor{rpTypeRule}{If} no -> Score 1.\par   - How many distinct supporting reasons are given? 0 -> Score 1; 1 (with weak or no elaboration) -> Score 1; 1 with \textcolor{rpTypeRule}{at least} one sentence of elaboration beyond the reason -> Score 2; 2+ with explicit tie-back for each and reasonable elaboration (\textasciitilde{}\textcolor{rpTypeRule}{2 sentences} per reason) -> consider Score 3.\par   - For each reason, is there a concrete example/explanation and an explicit link to the claim? \textcolor{rpTypeRule}{If} not for one or more reasons -> do not assign Score 3.\par   - Are errors frequent enough that the reader must re-read or infer meaning often? \textcolor{rpTypeRule}{If} inference needed on <=\textcolor{rpTypeRule}{3 sentences}/<=10\textcolor{rpTypeRule}{\%} -> Score 3 possible; \textcolor{rpTypeRule}{if} inference needed on >3 but not pervasive -> Score 2; \textcolor{rpTypeRule}{if} inference pervasive or many unintelligible sentences -> Score 1.\par   - Is the ending abruptly cut such that one or more required elaborations are missing? \textcolor{rpTypeRule}{If} yes -> downgrade one band (usually to the adjacent lower band).\par - Documentation: For any Score 1 or 2 rating where the essay contains some apparent development, raters must annotate (briefly) which specific requirements were missing (\textcolor{rpTypeEvidence}{e.g.}, "only one reason developed; \textcolor{rpTypeRule}{second} reason absent," "examples lack explicit tie-backs," or "frequent error pattern obscures meaning") to aid later calibration.
\end{tcolorbox}
\end{minipage}
\caption{Pattern-highlighted rubric comparison (ets3, openai\_gpt-5-mini, base\_expert\_True\_train100\_iteration5\_top3\_bs4-8-12\_mc4). Matched spans are color-coded by regex pattern. Color types: \textcolor{rpTypeRule}{\textbf{Rule Structure}} (if/threshold/stepwise guidance); \textcolor{rpTypeEvidence}{\textbf{Evidence Handling}} (examples, repetition, and caps); \textcolor{rpTypeWriting}{\textbf{Writing Quality}} (organization and grammar/mechanics).}
\label{fig:rubric_pattern_ets3_openai_gpt_5_mini_base_expert_True_train100_iteration5_top3_bs4_8_12_mc4}
\end{figure*}


\colorlet{rpTypeRule}{red!80!black}
\colorlet{rpTypeEvidence}{blue!80!black}
\colorlet{rpTypeWriting}{teal!80!black}
\begin{figure*}[t]
\centering
\begin{tcolorbox}[colback=white,colframe=black!25,title=Pattern Type Guide,fonttitle=\bfseries\small,fontupper=\scriptsize,boxsep=1pt,left=2pt,right=2pt,top=2pt,bottom=2pt]
\textcolor{rpTypeRule}{\textbf{Rule Structure}}: Explicit decision logic for scoring: conditional branches, boundary tie-breakers, stepwise workflows, and numeric thresholds.\par \textcolor{rpTypeEvidence}{\textbf{Evidence Handling}}: How evidence is validated and counted: specific-example requirements, repetition/non-double-count rules, and cap rules for weak evidence.\par \textcolor{rpTypeWriting}{\textbf{Writing Quality}}: Language-quality criteria affecting score bands: organization/coherence/transition quality and grammar/mechanics severity.
\end{tcolorbox}
\vspace{1mm}
\begin{tcolorbox}[colback=white,colframe=black!25,title=Detailed Pattern Notes,fonttitle=\bfseries\small,fontupper=\scriptsize,boxsep=1pt,left=2pt,right=2pt,top=2pt,bottom=2pt]
\textcolor{rpTypeRule}{\textbf{Rule Structure}}:\par \quad \textcolor{rpTypeRule}{\textbf{Conditional Gating}} [n=52] Captures explicit condition-based rules that switch decisions only when a stated condition is met. Typical cues: if, when.\par \quad \textcolor{rpTypeRule}{\textbf{Boundary / Tie-Break Guidance}} [n=3] Marks criteria used to resolve borderline cases between adjacent score bands (e.g., 4 vs 5). Typical cues: tie-break, borderline, boundary, threshold, 4 vs 5.\par \quad \textcolor{rpTypeRule}{\textbf{Quantitative Threshold}} [n=1] Marks numeric cutoffs used for consistent decisions (minimum/maximum counts, percentages, explicit count thresholds). Typical cues: at least, at most, <=, >=, \%, N reasons/examples/sentences/words.\par \textcolor{rpTypeEvidence}{\textbf{Evidence Handling}}:\par \quad \textcolor{rpTypeEvidence}{\textbf{Specific Evidence Requirement}} [n=6] Highlights demands for concrete examples and explicit evidence links instead of generic assertions. Typical cues: for example, e.g., specific example, illustration, anecdote, evidence.\par \quad \textcolor{rpTypeEvidence}{\textbf{Off-Topic / Summary Cap}} [n=2] Identifies cap rules that restrict scores when responses are off-topic, irrelevant, or dominated by summary-only content. Typical cues: off-topic, irrelevant, digression, summary-only, cap.\par \quad \textcolor{rpTypeEvidence}{\textbf{Repetition Non-Count Rule}} [n=1] Captures rules that treat repetition/restatement as non-distinct support and prevent double-counting. Typical cues: repetition, restatement, double-count, do not double-count.\par \textcolor{rpTypeWriting}{\textbf{Writing Quality}}:\par \quad \textcolor{rpTypeWriting}{\textbf{Organization / Coherence Signal}} [n=16] Detects explicit references to discourse structure and logical flow as scoring criteria. Typical cues: organization, coherence, logical flow, transition.\par \quad \textcolor{rpTypeWriting}{\textbf{Grammar / Mechanics Signal}} [n=3] Detects references to language-form quality, especially grammar, spelling, punctuation, and mechanics. Typical cues: grammar, mechanics, spelling, punctuation.
\end{tcolorbox}
\vspace{1mm}
\begin{tcolorbox}[colback=white,colframe=black!25,title=Optimized Rubric (Pattern-Highlighted),fonttitle=\bfseries\small,fontupper=\scriptsize]
\ttfamily
- Primary weighting order (unchanged): Task fulfillment \& development > \textcolor{rpTypeWriting}{\textbf{Organization}} \& \textcolor{rpTypeWriting}{\textbf{coherence}} > Language control.\par ... [1 lines omitted] ...\par - Distinguish error severity precisely and apply \textcolor{rpTypeRule}{\textbf{threshold}}s consistently:\par   - Minor errors: rare \textcolor{rpTypeWriting}{\textbf{spelling}}/typo or small \textcolor{rpTypeWriting}{\textbf{punctuation}} slips that never interrupt comprehension.\par ... [2 lines omitted] ...\par - Always annotate which of the three dimensions (development, \textcolor{rpTypeWriting}{\textbf{organization}}, language) is the decisive factor for the assigned score, and cite explicit \textcolor{rpTypeEvidence}{\textbf{evidence}} (quote 1-2 exemplar sentences \textcolor{rpTypeRule}{\textbf{if}} language is the decisive issue).\par ... [4 lines omitted] ...\par   - Addresses counterarguments or shows depth of thought \textcolor{rpTypeRule}{\textbf{when}} relevant.\par   - Clarification added: \textcolor{rpTypeRule}{\textbf{When}} multiple distinct examples are present and each is developed with some explanation/connection to the thesis, favor score 3 even \textcolor{rpTypeRule}{\textbf{if}} language contains many surface-level errors - provided those errors do not regularly obscure meaning or create reader fatigue.\par - \textcolor{rpTypeWriting}{\textbf{Organization}} \& \textcolor{rpTypeWriting}{\textbf{coherence}}:\par   - Clear progression of ideas, logical paragraphing, and explicit links between points. A truncated or slightly abrupt conclusion does not automatically negate a 3 \textcolor{rpTypeRule}{\textbf{if}} multi-paragraph development is sustained and convincing.\par ... [2 lines omitted] ...\par   - Errors, \textcolor{rpTypeRule}{\textbf{if}} present, are minor or occasional moderate slips that do not disrupt reading. Frequent moderate errors are acceptable for 3 only \textcolor{rpTypeRule}{\textbf{when}} (a) they do not force repeated re-construction of meaning, and (b) the essay remains easy to follow without sustained effort.\par - \textcolor{rpTypeRule}{\textbf{When}} to assign 3 (clarified):\par   - Use 3 \textcolor{rpTypeRule}{\textbf{when}} task is fully addressed with well-developed support and clear \textcolor{rpTypeWriting}{\textbf{organization}}-even \textcolor{rpTypeRule}{\textbf{if}} surface-level errors are frequent-ONLY \textcolor{rpTypeRule}{\textbf{IF}} those errors are mostly minor or at worst recurrent-moderate but do not cause reader fatigue or obscure meaning.\par   - Do NOT give 3 \textcolor{rpTypeRule}{\textbf{when}} development is repetitive, thin, or relies on a single undeveloped/example assertion even \textcolor{rpTypeRule}{\textbf{if}} language is good.\par ... [4 lines omitted] ...\par   - Typical patterns: general statements without adequate elaboration, brief examples that are not extended, or \textcolor{rpTypeEvidence}{\textbf{repetition}} of ideas rather than deeper development.\par   - Clarification added: \textcolor{rpTypeRule}{\textbf{If}} there are multiple examples but each is only briefly explained and the essay does not show clear, sustained development, assign 2 (even \textcolor{rpTypeRule}{\textbf{if}} language is relatively accurate).\par - \textcolor{rpTypeWriting}{\textbf{Organization}} \& \textcolor{rpTypeWriting}{\textbf{coherence}}:\par   - Overall progression exists but may be disjointed or repetitious; minor \textcolor{rpTypeEvidence}{\textbf{digression}}s or redundancy are common.\par ... [3 lines omitted] ...\par   - Assign 2 \textcolor{rpTypeRule}{\textbf{when}} moderate or some severe errors occur but communication generally remains recoverable without substantial effort.\par   - Clarification added: Frequent moderate errors that cause reader fatigue (i.e., intermittent re-reading but not constant reconstruction) merit score 2. \textcolor{rpTypeRule}{\textbf{If}} errors are frequent but readers can follow the argument with modest effort, prefer 2 over 1.\par - \textcolor{rpTypeRule}{\textbf{When}} to assign 2 (clarified):\par   - Use 2 \textcolor{rpTypeRule}{\textbf{when}} there is some development and \textcolor{rpTypeWriting}{\textbf{organization}} but support is noticeably limited, OR \textcolor{rpTypeRule}{\textbf{when}} moderate/accumulated errors begin to slow the reader though meaning is still mostly recoverable.\par   - Use 2 (not 3) \textcolor{rpTypeRule}{\textbf{when}} development is repetitive or \textcolor{rpTypeWriting}{\textbf{organization}}ally weak even \textcolor{rpTypeRule}{\textbf{if}} there are no totally unreadable sentences.\par ... [3 lines omitted] ...\par   - Very limited development: few or no relevant examples; explanations too brief, \textcolor{rpTypeEvidence}{\textbf{irrelevant}}, or repetitive/general assertions without support.\par ... [1 lines omitted] ...\par - \textcolor{rpTypeWriting}{\textbf{Organization}} \& \textcolor{rpTypeWriting}{\textbf{coherence}}:\par   - Inadequate \textcolor{rpTypeWriting}{\textbf{organization}}; reader has difficulty following progression; extremely short or fragmentary responses.\par ... [2 lines omitted] ...\par   - Accumulation of severe errors that often obscure meaning or make the essay hard to understand-score 1 even \textcolor{rpTypeRule}{\textbf{if}} some relevant ideas exist.\par   - Clarification added: To justify score 1 on language grounds, errors must (a) occur frequently across the essay and (b) require repeated reconstruction of meaning (i.e., reader must regularly infer or guess meaning). Isolated severe errors do not by themselves require score 1 \textcolor{rpTypeRule}{\textbf{if}} overall meaning is recoverable.\par - \textcolor{rpTypeRule}{\textbf{When}} to assign 1 (clarified):\par ... [1 lines omitted] ...\par   - Use 1 \textcolor{rpTypeRule}{\textbf{when}} the reader must frequently re-construct meaning due to severe/accumulated errors, even \textcolor{rpTypeRule}{\textbf{if}} the essay attempts relevant content.\par ... [1 lines omitted] ...\par Additional clarifications and \textcolor{rpTypeRule}{\textbf{borderline}} decision rules (to improve inter-rater agreement)\par ... [1 lines omitted] ...\par - Many surface errors but strong development: Favor the higher score (3) only \textcolor{rpTypeRule}{\textbf{if}} errors are mostly minor or occasional-moderate and do not create reading fatigue. \textcolor{rpTypeRule}{\textbf{If}} errors are frequent enough to make sustained reading effortful, downgrade to 2.\par - Frequent moderate errors that cause reader fatigue: Downgrade to 2 even \textcolor{rpTypeRule}{\textbf{when}} development and \textcolor{rpTypeWriting}{\textbf{organization}} are relatively strong.\par - Severe/accumulated errors that obscure meaning: Downgrade to 1 regardless of topic relevance \textcolor{rpTypeRule}{\textbf{if}} comprehension is regularly impeded.\par - Repetitive development: \textcolor{rpTypeRule}{\textbf{If}} the essay repeats the same idea across paragraphs without extending or deepening it, prefer score 2 rather than 3.\par - Incomplete or abrupt conclusion: \textcolor{rpTypeRule}{\textbf{If}} the essay ends mid-thought or with an incomplete conclusion, prefer score 2 unless the rest of the essay demonstrates clear, multi-paragraph, well-elaborated development with multiple examples (in which case 3 may still be warranted).\par - \textcolor{rpTypeWriting}{\textbf{Organization}} outweighs \textcolor{rpTypeWriting}{\textbf{grammar}} in \textcolor{rpTypeRule}{\textbf{borderline}} cases: \textcolor{rpTypeRule}{\textbf{When}} development and logical progression are strong but language contains many minor slips, lean toward 3. \textcolor{rpTypeRule}{\textbf{When}} language issues are moderate/severe and interfere with flow, lean lower.\par - Explicit justification required for scores that contradict primary weighting: \textcolor{rpTypeRule}{\textbf{If}} a language problem (rather than lack of development) is the main reason for assigning a lower score, explicitly note this in the rater's rationale and quote representative problematic sentences.\par - Calibration rule (new): \textcolor{rpTypeRule}{\textbf{When}} in doubt between 2 and 3, ask two questions:\par ... [2 lines omitted] ...\par   - \textcolor{rpTypeRule}{\textbf{If}} answer to (1) = yes and (2) = no, assign 3.\par   - \textcolor{rpTypeRule}{\textbf{If}} (1) = partial or no, assign 2 (or 1 \textcolor{rpTypeRule}{\textbf{if}} development is minimal).\par   - \textcolor{rpTypeRule}{\textbf{If}} (1) = yes but (2) = yes (frequent moderate errors causing reader fatigue), assign 2.\par ... [1 lines omitted] ...\par   - \textcolor{rpTypeRule}{\textbf{When}} in doubt between 1 and 2: ask whether there is any meaningful elaboration beyond a thesis sentence. \textcolor{rpTypeRule}{\textbf{If}} not, assign 1. \textcolor{rpTypeRule}{\textbf{If}} there is some elaboration but errors frequently force reconstruction of meaning, assign 1. \textcolor{rpTypeRule}{\textbf{If}} there is some elaboration and meaning is usually recoverable with modest effort, assign 2.\par ... [2 lines omitted] ...\par - For every essay, explicitly identify the decisive dimension (development, \textcolor{rpTypeWriting}{\textbf{organization}}, or language) and cite specific \textcolor{rpTypeEvidence}{\textbf{evidence}} (\textcolor{rpTypeEvidence}{\textbf{e.g.}}, "single undeveloped example," "frequent sentence-level errors causing re-reading," "clear multi-paragraph development with examples X and Y").\par - \textcolor{rpTypeRule}{\textbf{If}} language is decisive, quote 1-2 representative sentences that illustrate the severity and describe whether they require re-reading or guessing.\par - Count and weigh examples: multiple distinct examples with explanation -> strong \textcolor{rpTypeEvidence}{\textbf{evidence}} for 3; single brief example -> \textcolor{rpTypeEvidence}{\textbf{evidence}} for 1.\par - \textcolor{rpTypeRule}{\textbf{If}} development is multi-example and clear but language errors are frequent, ask whether error-induced re-reading is occasional (3) or sustained/frequent (2).\par - \textcolor{rpTypeRule}{\textbf{If}} development is truncated at the end, explicitly weigh the strength of prior development before downgrading-only truncate to 2 \textcolor{rpTypeRule}{\textbf{if}} the truncation meaningfully reduces the essay's overall support.\par - \textcolor{rpTypeRule}{\textbf{When}} rating, prefer concrete justifications (\textcolor{rpTypeEvidence}{\textbf{e.g.}}, "three developed examples: X, Y, Z" or "language: 6 of \textcolor{rpTypeRule}{\textbf{8 sentences}} require re-reading") rather than vague statements.\par ... [2 lines omitted] ...\par - clearer tolerance rules for frequent surface errors \textcolor{rpTypeRule}{\textbf{when}} development is strong (reduce under-scoring of well-developed but error-prone essays),\par ... [2 lines omitted] ...\par - requirement to quote exemplar problematic sentences \textcolor{rpTypeRule}{\textbf{when}} language drives the score to improve consistency and justification.
\end{tcolorbox}
\caption{Pattern-focused view of the optimized rubric (ets3, openai\_gpt-5-mini, base\_simplest\_True\_train100\_iteration5\_top3\_bs4-8-12\_mc4). Colored bold spans indicate regex-matched rubric cues. Color types: \textcolor{rpTypeRule}{\textbf{Rule Structure}} (Explicit decision logic for scoring: conditional branches, boundary tie-breakers, stepwise workflows, and numeric thresholds.); \textcolor{rpTypeEvidence}{\textbf{Evidence Handling}} (How evidence is validated and counted: specific-example requirements, repetition/non-double-count rules, and cap rules for weak evidence.); \textcolor{rpTypeWriting}{\textbf{Writing Quality}} (Language-quality criteria affecting score bands: organization/coherence/transition quality and grammar/mechanics severity.). Matched pattern categories: Conditional Gating (n=52); Boundary / Tie-Break Guidance (n=3); Specific Evidence Requirement (n=6); Off-Topic / Summary Cap (n=2); Organization / Coherence Signal (n=16); Grammar / Mechanics Signal (n=3); Repetition Non-Count Rule (n=1); Quantitative Threshold (n=1).}
\label{fig:rubric_pattern_ets3_openai_gpt_5_mini_base_simplest_True_train100_iteration5_top3_bs4_8_12_mc4}
\end{figure*}


\colorlet{rpTypeRule}{red!80!black}
\colorlet{rpTypeEvidence}{blue!80!black}
\colorlet{rpTypeWriting}{teal!80!black}
\begin{figure*}[t]
\centering
\begin{tcolorbox}[colback=white,colframe=black!25,title=Pattern Type Guide,fonttitle=\bfseries\small,fontupper=\scriptsize,boxsep=1pt,left=2pt,right=2pt,top=2pt,bottom=2pt]
\textcolor{rpTypeRule}{\textbf{Rule Structure}}: Explicit decision logic for scoring: conditional branches, boundary tie-breakers, stepwise workflows, and numeric thresholds.\par \textcolor{rpTypeEvidence}{\textbf{Evidence Handling}}: How evidence is validated and counted: specific-example requirements, repetition/non-double-count rules, and cap rules for weak evidence.\par \textcolor{rpTypeWriting}{\textbf{Writing Quality}}: Language-quality criteria affecting score bands: organization/coherence/transition quality and grammar/mechanics severity.
\end{tcolorbox}
\vspace{1mm}
\begin{tcolorbox}[colback=white,colframe=black!25,title=Detailed Pattern Notes,fonttitle=\bfseries\small,fontupper=\scriptsize,boxsep=1pt,left=2pt,right=2pt,top=2pt,bottom=2pt]
\textcolor{rpTypeRule}{\textbf{Rule Structure}}:\par \quad \textcolor{rpTypeRule}{\textbf{Conditional Gating}} [n=6] Captures explicit condition-based rules that switch decisions only when a stated condition is met. Typical cues: if, when.\par \textcolor{rpTypeEvidence}{\textbf{Evidence Handling}}:\par \quad \textcolor{rpTypeEvidence}{\textbf{Specific Evidence Requirement}} [n=3] Highlights demands for concrete examples and explicit evidence links instead of generic assertions. Typical cues: for example, e.g., specific example, illustration, anecdote, evidence.\par \quad \textcolor{rpTypeEvidence}{\textbf{Off-Topic / Summary Cap}} [n=5] Identifies cap rules that restrict scores when responses are off-topic, irrelevant, or dominated by summary-only content. Typical cues: off-topic, irrelevant, digression, summary-only, cap.\par \textcolor{rpTypeWriting}{\textbf{Writing Quality}}:\par \quad \textcolor{rpTypeWriting}{\textbf{Organization / Coherence Signal}} [n=5] Detects explicit references to discourse structure and logical flow as scoring criteria. Typical cues: organization, coherence, logical flow, transition.\par \quad \textcolor{rpTypeWriting}{\textbf{Grammar / Mechanics Signal}} [n=3] Detects references to language-form quality, especially grammar, spelling, punctuation, and mechanics. Typical cues: grammar, mechanics, spelling, punctuation.
\end{tcolorbox}
\vspace{1mm}
\begin{tcolorbox}[colback=white,colframe=black!25,title=Optimized Rubric (Pattern-Highlighted),fonttitle=\bfseries\small,fontupper=\scriptsize]
\ttfamily
- addresses the topic and task with clear, thorough, and well-developed explanations, exemplifications, and details that directly and meaningfully support the position; examples are specific, concrete, fully elaborated, and explicitly connected to the argument - not merely named, vaguely referenced, or superficially mentioned; even \textcolor{rpTypeRule}{\textbf{if}} minor factual inaccuracies exist (\textcolor{rpTypeEvidence}{\textbf{e.g.}}, misspelled names like "Eddison" instead of "Edison"), they do not undermine the logical \textcolor{rpTypeWriting}{\textbf{coherence}} or depth of the analysis\par - is logically organized with strong unity, clear progression, and coherent \textcolor{rpTypeWriting}{\textbf{transition}}s between ideas; no significant \textcolor{rpTypeEvidence}{\textbf{digression}}s, redundancy, or unclear connections exist; structure supports argumentative flow, even \textcolor{rpTypeRule}{\textbf{if}} the conclusion is slightly truncated or implied rather than formally stated\par - displays consistent and effective facility in language use, including syntactic variety and a broad range of precise vocabulary; grammatical, word form, or idiomatic errors are extremely rare, minor, and do not impede clarity, fluency, or meaning in any way - even \textcolor{rpTypeRule}{\textbf{when}} present, they are isolated and do not affect comprehension or flow; \textcolor{rpTypeWriting}{\textbf{spelling}} errors or non-standard word forms (\textcolor{rpTypeEvidence}{\textbf{e.g.}}, "popolation," "essectial") are acceptable \textcolor{rpTypeRule}{\textbf{if}} they are isolated and do not obscure meaning or disrupt the reader's ability to follow the argument\par ... [3 lines omitted] ...\par - addresses the topic and task with partially developed explanations, examples, or details that may be vague, incomplete, repetitive, or only indirectly relevant, but the central position remains discernible; examples may be mentioned but not fully explained, or their connection to the argument is unclear or underdeveloped - however, the writer demonstrates intent to support the claim and the core logic is present, even \textcolor{rpTypeRule}{\textbf{if}} execution is flawed\par - demonstrates adequate but inconsistent \textcolor{rpTypeWriting}{\textbf{organization}}; ideas may be connected in a general way, but \textcolor{rpTypeWriting}{\textbf{transition}}s may be awkward, missing, repetitive, or overly simplistic; there may be some \textcolor{rpTypeEvidence}{\textbf{digression}} or redundancy that disrupts flow, though the overall structure is recognizable and the progression of ideas is discernible\par - shows limited or inconsistent facility in language use: vocabulary may be repetitive, imprecise, or occasionally inappropriate; syntactic structures may be simplistic, faulty, or awkward; and errors in \textcolor{rpTypeWriting}{\textbf{grammar}}, word form, or usage are frequent enough to reduce fluency or occasionally obscure meaning, but the main ideas remain comprehensible despite these flaws - errors are systemic but not so severe as to prevent understanding of the core argument; mis\textcolor{rpTypeWriting}{\textbf{spelling}}s, non-standard phrasing, or awkward constructions (\textcolor{rpTypeEvidence}{\textbf{e.g.}}, "popolation," "it going with the light of the Sun") are tolerated \textcolor{rpTypeRule}{\textbf{if}} the intended meaning can be reasonably inferred without requiring significant reconstruction\par ... [3 lines omitted] ...\par - fails to adequately address the topic or task; response is \textcolor{rpTypeEvidence}{\textbf{off-topic}}, superficial, lacks a clear position, or is largely \textcolor{rpTypeEvidence}{\textbf{irrelevant}}; the writer may appear to misunderstand the prompt or provide no coherent stance\par - demonstrates poor or incoherent \textcolor{rpTypeWriting}{\textbf{organization}}; ideas are disconnected, illogical, or lack any meaningful progression; paragraphs may be absent, randomly structured, or fail to serve a functional purpose\par - provides inadequate, \textcolor{rpTypeEvidence}{\textbf{irrelevant}}, or absent exemplifications, explanations, or details to support claims; any examples given are unelaborated, nonsensical, unrelated, or cut off mid-thought with no discernible intent to develop them
\end{tcolorbox}
\caption{Pattern-focused view of the optimized rubric (ets3, qwen\_qwen3-next-80b-a3b-instruct, base\_expert\_True\_train100\_iteration5\_top3\_bs4-8-12\_mc4). Colored bold spans indicate regex-matched rubric cues. Color types: \textcolor{rpTypeRule}{\textbf{Rule Structure}} (Explicit decision logic for scoring: conditional branches, boundary tie-breakers, stepwise workflows, and numeric thresholds.); \textcolor{rpTypeEvidence}{\textbf{Evidence Handling}} (How evidence is validated and counted: specific-example requirements, repetition/non-double-count rules, and cap rules for weak evidence.); \textcolor{rpTypeWriting}{\textbf{Writing Quality}} (Language-quality criteria affecting score bands: organization/coherence/transition quality and grammar/mechanics severity.). Matched pattern categories: Conditional Gating (n=6); Specific Evidence Requirement (n=3); Off-Topic / Summary Cap (n=5); Organization / Coherence Signal (n=5); Grammar / Mechanics Signal (n=3).}
\label{fig:rubric_pattern_ets3_qwen_qwen3_next_80b_a3b_instruct_base_expert_True_train100_iteration5_top3_bs4_8_12_mc4}
\end{figure*}


\colorlet{rpTypeRule}{red!80!black}
\colorlet{rpTypeEvidence}{blue!80!black}
\colorlet{rpTypeWriting}{teal!80!black}
\begin{figure*}[t]
\centering
\begin{tcolorbox}[colback=white,colframe=black!25,title=Pattern Type Guide,fonttitle=\bfseries\small,fontupper=\scriptsize,boxsep=1pt,left=2pt,right=2pt,top=2pt,bottom=2pt]
\textcolor{rpTypeRule}{\textbf{Rule Structure}}: Explicit decision logic for scoring: conditional branches, boundary tie-breakers, stepwise workflows, and numeric thresholds.\par \textcolor{rpTypeEvidence}{\textbf{Evidence Handling}}: How evidence is validated and counted: specific-example requirements, repetition/non-double-count rules, and cap rules for weak evidence.\par \textcolor{rpTypeWriting}{\textbf{Writing Quality}}: Language-quality criteria affecting score bands: organization/coherence/transition quality and grammar/mechanics severity.
\end{tcolorbox}
\vspace{1mm}
\begin{tcolorbox}[colback=white,colframe=black!25,title=Detailed Pattern Notes,fonttitle=\bfseries\small,fontupper=\scriptsize,boxsep=1pt,left=2pt,right=2pt,top=2pt,bottom=2pt]
\textcolor{rpTypeRule}{\textbf{Rule Structure}}:\par \quad \textcolor{rpTypeRule}{\textbf{Conditional Gating}} [n=16] Captures explicit condition-based rules that switch decisions only when a stated condition is met. Typical cues: if, when.\par \textcolor{rpTypeEvidence}{\textbf{Evidence Handling}}:\par \quad \textcolor{rpTypeEvidence}{\textbf{Specific Evidence Requirement}} [n=7] Highlights demands for concrete examples and explicit evidence links instead of generic assertions. Typical cues: for example, e.g., specific example, illustration, anecdote, evidence.\par \quad \textcolor{rpTypeEvidence}{\textbf{Off-Topic / Summary Cap}} [n=2] Identifies cap rules that restrict scores when responses are off-topic, irrelevant, or dominated by summary-only content. Typical cues: off-topic, irrelevant, digression, summary-only, cap.\par \quad \textcolor{rpTypeEvidence}{\textbf{Repetition Non-Count Rule}} [n=1] Captures rules that treat repetition/restatement as non-distinct support and prevent double-counting. Typical cues: repetition, restatement, double-count, do not double-count.\par \textcolor{rpTypeWriting}{\textbf{Writing Quality}}:\par \quad \textcolor{rpTypeWriting}{\textbf{Organization / Coherence Signal}} [n=8] Detects explicit references to discourse structure and logical flow as scoring criteria. Typical cues: organization, coherence, logical flow, transition.\par \quad \textcolor{rpTypeWriting}{\textbf{Grammar / Mechanics Signal}} [n=10] Detects references to language-form quality, especially grammar, spelling, punctuation, and mechanics. Typical cues: grammar, mechanics, spelling, punctuation.
\end{tcolorbox}
\vspace{1mm}
\begin{tcolorbox}[colback=white,colframe=black!25,title=Optimized Rubric (Pattern-Highlighted),fonttitle=\bfseries\small,fontupper=\scriptsize]
\ttfamily
- addresses the topic and task thoroughly, with well-developed, specific, and relevant explanations, exemplifications, and details that directly support the position; examples are clearly explained, logically connected to the argument, and demonstrate insight rather than mere \textcolor{rpTypeEvidence}{\textbf{illustration}} - even \textcolor{rpTypeRule}{\textbf{if}} an example is unconventional or imperfectly factual, it is treated with analytical depth and purposefully tied to the claim; minor factual inaccuracies (\textcolor{rpTypeEvidence}{\textbf{e.g.}}, misnamed individuals, exaggerated statistics) are acceptable only \textcolor{rpTypeRule}{\textbf{if}} they are not central to the argument and do not undermine the credibility of the analysis\par - is clearly and logically organized, with smooth progression and strong \textcolor{rpTypeWriting}{\textbf{coherence}} between ideas; \textcolor{rpTypeWriting}{\textbf{transition}}s are natural, structure is purposeful, and there is no significant redundancy, \textcolor{rpTypeEvidence}{\textbf{digression}}, or unclear connections - minor \textcolor{rpTypeWriting}{\textbf{organization}}al imperfections (\textcolor{rpTypeEvidence}{\textbf{e.g.}}, repetitive phrasing or slightly awkward \textcolor{rpTypeWriting}{\textbf{transition}}s) are acceptable \textcolor{rpTypeRule}{\textbf{if}} the overall argument flows logically and the paragraphing supports the development of ideas\par - displays consistent and effective facility in the use of language, demonstrating varied and sophisticated sentence structures and a broad, precise vocabulary appropriate to the academic context; word choice enhances clarity and impact - occasional non-standard word forms, mis\textcolor{rpTypeWriting}{\textbf{spelling}}s, or idiomatic imprecisions are acceptable only \textcolor{rpTypeRule}{\textbf{if}} they are rare, isolated, and do not impede comprehension or undermine the writer's control; errors must not be systematic or frequent enough to require interpretive effort\par - contains only rare, minor errors in \textcolor{rpTypeWriting}{\textbf{grammar}}, word form, or idiomatic usage that are inconsequential to understanding and do not distract from the message; \textcolor{rpTypeWriting}{\textbf{spelling}} and \textcolor{rpTypeWriting}{\textbf{punctuation}} are consistently accurate - errors must not be predictable, recurring, or pervasive enough to suggest a lack of language proficiency; \textcolor{rpTypeRule}{\textbf{if}} errors appear frequently (\textcolor{rpTypeEvidence}{\textbf{e.g.}}, consistent mis\textcolor{rpTypeWriting}{\textbf{spelling}} of common words like "nowdays," "donot," "belive"), the essay cannot qualify for Score 3, regardless of content strength\par ... [3 lines omitted] ...\par - addresses the topic and task with sufficient development, but explanations, examples, or details may be somewhat general, inconsistently specific, or only partially developed; the argument is present but lacks depth, precision, or analytical insight in places; examples may be relevant but are not fully explained or are loosely tied to the claim - even \textcolor{rpTypeRule}{\textbf{if}} examples are flawed, fictional, or oversimplified, they are recognized as intended \textcolor{rpTypeEvidence}{\textbf{illustration}}s and are not dismissed as invalid \textcolor{rpTypeRule}{\textbf{if}} they serve a clear rhetorical purpose; factual inaccuracies or imprecise examples are acceptable \textcolor{rpTypeRule}{\textbf{if}} they do not dominate the argument or mislead the reader's understanding of the position\par - shows adequate \textcolor{rpTypeWriting}{\textbf{organization}} and logical progression, with clear overall structure, but may contain minor lapses in \textcolor{rpTypeWriting}{\textbf{coherence}}, \textcolor{rpTypeWriting}{\textbf{transition}}al awkwardness, or repetitive points that do not undermine the central argument; ideas are connected but not always with sophistication - redundancy or phrasing \textcolor{rpTypeEvidence}{\textbf{repetition}} is acceptable \textcolor{rpTypeRule}{\textbf{if}} the core logic remains intact and paragraphing provides discernible structure\par - demonstrates partial control of language: sentence structures are generally varied but may include occasional awkwardness, simplicity, or overuse of basic constructions; vocabulary is adequate and mostly appropriate, though sometimes repetitive, imprecise, or colloquial - frequent \textcolor{rpTypeWriting}{\textbf{spelling}}, \textcolor{rpTypeWriting}{\textbf{grammar}}, or word form errors are acceptable \textcolor{rpTypeRule}{\textbf{if}} they are predictable (\textcolor{rpTypeEvidence}{\textbf{e.g.}}, "nowdays," "donot," "belive," "Eddison," "popolation"), systematic but not severe, and do not consistently obscure meaning or require substantial reinterpretation; the writer's intent and position remain clearly discernible despite these errors\par - contains noticeable errors in \textcolor{rpTypeWriting}{\textbf{spelling}}, \textcolor{rpTypeWriting}{\textbf{punctuation}}, or syntax that affect tone or clarity but do not prevent the reader from understanding the writer's intent and position; errors must be frequent enough to be systematic but not so severe as to make comprehension difficult without contextual inference - \textcolor{rpTypeRule}{\textbf{if}} errors are pervasive (\textcolor{rpTypeEvidence}{\textbf{e.g.}}, recurring in nearly every sentence), the essay should not be rated higher than Score 2, even \textcolor{rpTypeRule}{\textbf{if}} the content is well-developed\par ... [3 lines omitted] ...\par - shows minimal or inadequate development of ideas in response to the task; explanations, examples, or details are missing, \textcolor{rpTypeEvidence}{\textbf{irrelevant}}, overly general, fabricated without purpose, or fail to meaningfully support the position; the response may misinterpret the prompt or avoid addressing it directly - fabricated examples are only acceptable \textcolor{rpTypeRule}{\textbf{if}} they are used deliberately and analyzed; otherwise, their presence without analysis or relevance signals Score 1\par - lacks clear \textcolor{rpTypeWriting}{\textbf{organization}} or logical progression; ideas are disjointed, poorly connected, or presented in a confusing sequence with no discernible structure; paragraphing is absent or arbitrary - the absence of paragraphing alone is not sufficient for Score 1 \textcolor{rpTypeRule}{\textbf{if}} ideas are otherwise logically grouped\par - contains frequent and severe errors in sentence structure, word form, \textcolor{rpTypeWriting}{\textbf{spelling}}, and usage that consistently obscure meaning and hinder communication; the language is often unintelligible or requires substantial interpretation to discern intent - errors must be pervasive and systemic enough to make comprehension difficult even with contextual inference\par - demonstrates a very limited or inappropriate use of vocabulary and syntactic structures, making the response difficult to follow even \textcolor{rpTypeRule}{\textbf{when}} content is partially understandable - vocabulary must be consistently primitive or misused, not merely imprecise\par - may include nonsensical examples, factual distortions, or persistent grammatical breakdowns that undermine credibility and prevent the reader from engaging with the argument - isolated factual inaccuracies are not grounds for Score 1 \textcolor{rpTypeRule}{\textbf{if}} the argument otherwise holds together
\end{tcolorbox}
\caption{Pattern-focused view of the optimized rubric (ets3, qwen\_qwen3-next-80b-a3b-instruct, base\_simplest\_True\_train100\_iteration5\_top3\_bs4-8-12\_mc4). Colored bold spans indicate regex-matched rubric cues. Color types: \textcolor{rpTypeRule}{\textbf{Rule Structure}} (Explicit decision logic for scoring: conditional branches, boundary tie-breakers, stepwise workflows, and numeric thresholds.); \textcolor{rpTypeEvidence}{\textbf{Evidence Handling}} (How evidence is validated and counted: specific-example requirements, repetition/non-double-count rules, and cap rules for weak evidence.); \textcolor{rpTypeWriting}{\textbf{Writing Quality}} (Language-quality criteria affecting score bands: organization/coherence/transition quality and grammar/mechanics severity.). Matched pattern categories: Conditional Gating (n=16); Specific Evidence Requirement (n=7); Off-Topic / Summary Cap (n=2); Organization / Coherence Signal (n=8); Grammar / Mechanics Signal (n=10); Repetition Non-Count Rule (n=1).}
\label{fig:rubric_pattern_ets3_qwen_qwen3_next_80b_a3b_instruct_base_simplest_True_train100_iteration5_top3_bs4_8_12_mc4}
\end{figure*}




\end{document}
