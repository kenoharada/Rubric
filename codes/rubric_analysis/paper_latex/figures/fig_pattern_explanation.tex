\colorlet{pcMatch}{red!75!black}
\begin{figure*}[t]
\centering
\begin{tcolorbox}[
colback=white,
colframe=black!25,
title=Pattern Explanations with Random Matched Snippets,
fonttitle=\bfseries\small,
fontupper=\scriptsize,
boxsep=1pt,left=2pt,right=2pt,top=2pt,bottom=2pt]
\textbf{1. Conditional Gating}\par \textit{What this pattern captures:} Condition-based branching rules (if / when / unless / provided that) that explicitly guide rater decisions under specific circumstances. Refined rubrics tend to add many conditional gates to reduce ambiguity in borderline situations.\par \textit{Typical cues:} if, when, unless, provided that\par \textit{Example 1:} ``... ge: Grammar and usage may prevent comprehension. - Use \textcolor{pcMatch}{\textbf{when}} the essay essentially fails to form an argument or pro ...'' \par \textit{Example 2:} ``... anics (E) and apply Mechanical-distortion penalty: - \textcolor{pcMatch}{\textbf{If}} E = Severe and errors frequently obscure meaning: redu ...'' \par\vspace{1.2mm}\par \textbf{2. Boundary / Tie-Break}\par \textit{What this pattern captures:} Rules for resolving borderline cases between adjacent score bands. Includes tie-break procedures, explicit threshold cutoffs, and 'N vs N' comparisons (e.g., '3 vs 4'). Refined rubrics often add detailed boundary-resolution instructions to improve inter-rater agreement.\par \textit{Typical cues:} tie-break, borderline, threshold, N vs N, between adjacent\par \textit{Example 1:} ``... on from MAX\_POSSIBLE, but treat Noticeable as negative \textcolor{pcMatch}{\textbf{tie-break}} er preventing upward movement: do not promote an essay ...'' \par \textit{Example 2:} ``... ization: clear paragraphs and transitions? (Yes raises \textcolor{pcMatch}{\textbf{borderline}} 3->4) - Is there counterargument, synthesis, or analyti ...'' \par\vspace{1.2mm}\par \textbf{3. Stepwise Workflow}\par \textit{What this pattern captures:} Ordered step-by-step procedures (Step 1, Step 2...) or checklists that structure the scoring process into a reproducible workflow. Optimization tends to transform free-form scoring guidance into structured, sequential procedures for raters to follow.\par \textit{Typical cues:} step N, checklist, workflow, procedure, in order\par \textit{Example 1:} ``... adjustment in Step 4. - If E = Minor: no reduction. \textcolor{pcMatch}{\textbf{Step 3}} - Assess Central Claim (A), Organization (C), Language ...'' \par \textit{Example 2:} ``... cing explanation), set MAX\_POSSIBLE = 6 and continue. \textcolor{pcMatch}{\textbf{Step 2}} - Assess Mechanics (E) and apply Mechanical-distortion ...'' \par\vspace{1.2mm}\par \textbf{4. Quantitative Threshold}\par \textit{What this pattern captures:} Numeric cutoffs and quantified criteria (e.g., 'at least 2 facts', '\textasciitilde{}30\% severe errors', '3 reasons') that replace vague qualitative descriptions with concrete numbers. Optimization frequently introduces numeric thresholds where the original rubric used imprecise terms like 'some' or 'several'.\par \textit{Typical cues:} at least, at most, <=, >=, N reasons/examples/sentences, N\%\par \textit{Example 1:} ``... Quantitative guidance: in a typical short essay (approx200- \textcolor{pcMatch}{\textbf{300 words}} ), the reader should need to re-read or infer meaning f ...'' \par \textit{Example 2:} ``... ion: at minimum 1) an explicit reason sentence, and 2) \textcolor{pcMatch}{\textbf{at least}} one additional sentence that explains relevance, conse ...'' \par\vspace{1.2mm}\par \textbf{5. Score Cap / Demotion}\par \textit{What this pattern captures:} Hard constraints that cap the maximum achievable score or forcibly demote ratings when specific conditions are unmet. Examples: 'cannot receive 4 or higher', 'do not award 5', 'downgrade to 2'. Refined rubrics add these guards to prevent systematic over-scoring of essays that superficially appear competent.\par \textit{Typical cues:} cannot be Score, must not receive, do not award, downgrade, demotion\par \textit{Example 1:} ``... Score 4-even with numerous grammatical errors. Do not \textcolor{pcMatch}{\textbf{downgrade}} for language if the argument's logic and evidence are ...'' \par \textit{Example 2:} ``... metimes obscure meaning (reader must infer), the essay \textcolor{pcMatch}{\textbf{cannot be Score}} 3; assign Score 2 or Score 1 depending on frequency an ...'' \par\vspace{1.2mm}\par \textbf{6. Concrete Exemplification}\par \textit{What this pattern captures:} Detects rubric text that uses illustrative examples (e.g., for example, for instance) to clarify scoring criteria. Refined rubrics frequently replace abstract descriptions with example-rich explanations, making this a strong indicator of rubric practicality improvement.\par \textit{Typical cues:} e.g., for example, for instance\par \textit{Example 1:} ``... e weak or uneven. - Uses occasional specific examples ( \textcolor{pcMatch}{\textbf{e.g.}} , "I talk to my cousin in Colombia") or anonymized plac ...'' \par \textit{Example 2:} ``... serve as identifiable, contextually grounded evidence ( \textcolor{pcMatch}{\textbf{e.g.}} , "@PERCENT1 say...", "@PERSON1, a researcher...")-even ...'' 
\end{tcolorbox}
\caption{Overview of rubric-refinement patterns and representative rubric snippets. For each pattern, we provide a short interpretation and randomly sampled matched spans from refined rubrics; highlighted words indicate the cue expressions that triggered each pattern. }
\label{fig:pattern_explanation}
\end{figure*}
